\documentclass[8pt,a4paper]{article}
\usepackage[utf8]{inputenc}
\usepackage[T2A]{fontenc}
\usepackage[russian]{babel}
\usepackage{amsmath}
\usepackage{graphicx}
\usepackage{listings}
\usepackage{color}
\usepackage{geometry}
\usepackage{array}
\usepackage{longtable}
\geometry{left=1cm, top=0.5cm, bottom=0.5cm, right=0.5cm}

\definecolor{dkgreen}{rgb}{0,0.6,0}
\definecolor{gray}{rgb}{0.5,0.5,0.5}
\definecolor{mauve}{rgb}{0.58,0,0.82}

\lstset{frame=tb,
    language=SQL,
    aboveskip=3mm,
    belowskip=3mm,
    showstringspaces=false,
    columns=flexible,
    basicstyle={\small\ttfamily},
    numbers=none,
    numberstyle=\tiny\color{gray},
    keywordstyle=\color{blue},
    commentstyle=\color{dkgreen},
    stringstyle=\color{mauve},
    breaklines=true,
    breakatwhitespace=true,
    tabsize=2
}


\title{Databases \& ORM}
\author{Mesenev}
\date{\today}

\begin{document}
    \subsection*{1 слайд.}
    Здравствуйте коллеги, выступает Месенев Павел.
    Представляю доклад по теме диссертации на соискание
    кандидата наук "оптимизационные методы решения обратных задач сложного теплообмена".


    \subsection*{Мотивация}
    Интерес к обратным задачам теплообмена вызван широкой применимостью рассматриваемых моделей в области инженерных приложений.
    Данные модели широко применяются при моделировании и оптимизации производства стекла, камер сгорания гозовых турбин, лазерной термотерапии и так далее. Иными словами, любой процесс, сопряженный с высокими температурами.

    как следствие имеется значительное число работ, посвященных теоретическому анализу моделей сложного теплообмена.
    \begin{itemize}
        \item Амосов, Laiitinen, Druet, Tiihonen -- исследовали разрешимость краевых и начально-краевых задач с полным уравнением переноса излучения.
        \item Rene Pinnau, Oliver Tse предоставили теоретический анализ квазистационарных моделей для $SP_1, SP_3$--приближений.
        \item В работах Чеботарёва, Ковтанюка Гренкина доказана однозначная разрешимость краевых задач на основе $P_1$ приближений.
        \item Также необходимо упомянуть серьёзный теоретический анализ обратных задач теплообмена в работах Сергея Григорьевича Пяткова и его научной группы.
    \end{itemize}

    Тем не менее ряд важных вопросов, связанных с анализом корректности стационарных, квазистационарных и квазилинейных моделей,
    построением и обоснованием сходимости оптимизационных алгоритмов решения обратных задач и задач с краевыми условиями Коши,
    разработкой программ для проведения численных экспериментов остаются открытыми.
    Данная работапосвящена решению указанных проблем.

    \vspace{10cm}
    \subsection*{Положения, выносимые на защиту}

    В работе рассматриваются диффузионные модели сложного теплообмена
    в рамках так называемого $Р_1$ приближения уравнения переноса излучения,
    где функция, описывающая тепловое излучение является интенсивностью излучения,
    усредненной по всем направлениям.

    Модели представляют собой нелинейные системы уравнений с частными производными.

    Граничные обратные задачи возникают в ситуациях,
    когда неизвестны отражающие свойства границы или ее части и требуется их найти,
    используя дополнительную информацию о температурном поле.
    Близкие к ним задачи с условиями Коши на границе для температуры
    (терминология акад.\ М.М.Лаврентьева) возникают когда нет инфомации
    о поведении интесивности излучения на границе.

    \subsection*{В области анализа рассмотренных математических моделей:}

    \vspace{15cm}

    \subsection*{Граничная обратная задача}
    Первая из рассматриваемых задач формулируется следующим образом:
    дана некая область омега,
    на части её границы неизвестен параметр среды $u$, характеризирующий отражающие свойства границы.
    Сам параметр выражается как $(\frac{\varepsilon}{2(2-\varepsilon)})$, где $\varepsilon$ меняется в диапазоне от 0 до 1,
    как следствие на сам параметр накладывается ограничение, такое что $u \in [0, 0.5]$.

    Степень черноты не зависит от направления и определяется формулой
    $\varepsilon_\nu(x) = \frac{I_{\nu,\text{исп}}(x)}{I_{b\nu}(T(x))}$, где
    $I_{\nu,\text{исп}}(x)$ -- интенсивность излучения, испускаемого
    поверхностью при температуре $T(x)$~\cite[53]{Ozisik1976}.
    Требуется отыскать тройку: температурное поле,
    поле излучения и неизвестный параметр
    по дополнительной информации о температуре на границе.
    Вопрос о корректности сформулированной обратной задачи является открытым
    (как её решать тоже вопрос открытый).
    Предлагается заменить обратную задачу на задачу
    оптимального управления, которая состоит в
    минимизации функционала~\eqref{eq:2_1:quality}.
    Решение данной экстремальной задачи называется квазирешением обратной задачи.
    Для нахождения квазирешения был разработан оптимизационный численный метод.

    Постоянные $a$ (температуропроводность $м^2/сек$), $b$(коэффициент радиационного переноса) и
    $\alpha$ (Эффективный коэффициент оптического взаимодействия)
    определяются следующим образом:
    \[
        a = \frac{k}{\rho c_v},\quad b = \frac{4\sigma n^2 T_{\max}^3}{\rho c_v},
        \quad \alpha=\frac{1}{3\kappa - A \kappa_s},
    \]
    где $k$ -- теплопроводность $(Vt/(m \cdot K)=Dj/(m \cdot c \cdot K))$, $c_v$ -- удельная теплоемкость,
    $\rho$ -- плотность,
    $\sigma$ -- постоянная Стефана-Больцмана $(5.67 \cdot 10^{-8}Vt/(m^2 \cdot K^4))$,
    $n$ -- показатель преломления,
    $T_{\max}$ -- максимальная температура в ненормализованной модели,
    $\kappa = \kappa_s + \kappa_a$ -- коэффициент
    полного взаимодействия,
    $\kappa_s$ -- коэффициент рассеяния, $\kappa_a$ -- коэффициент поглощения.

    \vspace{10cm}
    \subsection*{Нахождение квазирешения обратной задачи}
    Предложенный в работе алгоритм поиска квазирешения обратной задачи основан
    на выведенных условиях оптимальности
    (доказано, что квазирешение должно
    удовлетворять~\eqref{eq:2_1:weakOperational}--\eqref{eq:2_1:theorem_2_eq2}),
    куда входят сопряженные функции для температуры $p_1$ и излучения $p_2$,
    а также связь между сопряженным состоянием и искомым граничным управлением.
    Для компактной записи краевых задач, используется современная операторная форма.
    Система~\eqref{eq:2_1:weakOperational} является операторной записью краевой задачи,
    где $A_{1,2}$ описывают диффузионные члены модели, остальные моделируют граничные условия.
    Уравнения~\eqref{eq:2_1:theorem_2_eq1}--\eqref{eq:2_1:theorem_2_eq2}
    это сопряженная система,
    а вариационное неравенство~\eqref{eq:2_1:theorem_2_eq3}
    устанавливает связь с оптимальным управлением.

    Приведём алгоритм градиентного спуска с проекцией.
    Обратим внимание, что оператор проекции нужен
    из-за начальных ограничений на функцию управления
    (вызванных физичностью параметра, например).

    Отметим, что в силу невыпуклости экстремальной задачи градиентные алгоритмы не обладают
    свойством глобальной сходимости, что служит основой для их критики, зачастую заслуженной.

    Однако свойства диффузионных моделей сложного теплообмена представленные в диссертации
    и правильный выбор шага градиентного метода обеспечивают сходимость для
    рассматриваемых задач.
    Следующие примеры этот факт демонстрируют.

    \vspace{10cm}
    \subsection*{Модель управления температурным полем через граничный параметр}
    Положим параметры среды, соответствующие стеклу и зададим тестовую функцию управления
    как показано на слайде.
    Пластинка, у которого боковые стороны `обычные', верхняя грань - участок наблюдения,
    нижняя грань - участок под ``контролем''.

    \vspace{10cm}
    \subsection*{Модель управления температурным полем через граничный параметр}
    Интересный эффект "среднего значения".
    Большое количество итераций.
    Обратить внимание на функционал качества.
    Для получения представленных результатов, использовался разработанный мной комплекс программ,
    включающий решение прямой задачи, сопряженной системы и алгоритм градиентного спуска.

    \vspace{10cm}
    \section*{Обратная задача с условиями типа Коши}
    \subsection*{Задача без краевых условий для интенсивности излучения}
    Не задано $\varphi$!
    В основе разработанного алгоритма решения лежит анализ экстремальной задачи.

    Строго обосновано существование решения экстр задачи.
    Кроме того, и это принципиально важно, показана сходимость решений экстремальных задач
    к решению начально-краевой задачи без краевых условий для интенсивности излучения при $\lambda$ стремящемся к 0.

    \vspace{10cm}
    \subsection*{Пример 1}
    Обратите внимание на малость функционала качества.
    Сравним с тем, что получилось -- довольно близко, но не идеально.
    Уменьшение параметра регуляризации повышает точность решения,
    но и увеличивает вычислительные затраты.


    \vspace{10cm}
    \subsection*{Пример 2}
    "Честный" эксперимент - используем то, что полагается в исходной задаче.
    Функционал качества (его динамика) позволяет предположить аналогичный порядок
    близости (с предыдущим примером)
    точного и аппроксимированного решений.
    Обратите внимание на линейность $\theta_b$ по оси $z$.


    \vspace{10cm}
    \section*{Квазистационарная задача с данными Коши}
    Аналог стационарной задачи с небольшими сдвигами по времени.
    Для оптимизационного метода решения задачи требуются результаты анализа кв.стц. модели из гл. 1!
    параметр $u$ неизвестен.

    \vspace{10cm}
    \subsection*{Численное моделирование}
    Приведены изображения сравнения полученных результатов в рамках работы
    над диссертацией и коллег из Мюнхена (финальный момент времени)
    Параметры среды соответствуют воздуху при нормальном атмосферном давлении и температуре 400C.


    N.Botkin для расчетов использовал, разработанную в TUM программу, использующую
    эрмитов прямоугольный (конформный) элемент Богнера-Фокса-Шмидта и сведение задачи к нестационарной.
    Не ясно, что же такое случилось с пространством решений,
    что потребовались столь экзотические конечные элементы.

    Фактически ему пришлось решать краевую задачу для нелинейного уравнения 4 порядка.
    Предложенный в работе оптимизационный алгоритм является более простым и дает фактически те же результаты.
    Использовались конечные элементы Галёркина (Лагранжа-1).

    \vspace{10cm}
    \subsection*{Стационарная задача с условиями Коши на части границы}
     Рассмотрим случай если на всей границе известен поток, а параметр из граничного условия
    для $\varphi$ неизвестен.
    Мы дополняем "доступный" участок информацией о температуре: $\theta_b$.
    Если данные Коши заданы на части границы задача является ещё более сложной.
    Для точной постановки нет результатов по её корректности.
    Однако предлагаемый далее оптимизационный метод полностью теоретически обоснован и
    лежит в основе соответствующего программного комплекса для численного решения.

    \vspace{10cm}
    \subsection*{Постановка задачи оптимального управления}
    Используя замену, представленную на слайде, мы заменим исходную задачу, на краевую задачу
    с функциями $\theta$, $\psi$.
    Вместо системы двух нелинейных уравнений одно уравнение стало линейным (для $\Psi$)
    (!!!Где нелинейность в исходной задаче!!!)
    Два нелинейных заменили на нелинейное и линейное (пси - гармоническая, к тому же).
    Особо обратим внимание на параметр s, который пришлось добавить в данную постановку из-за
    того, что реализованный алгоритм не сходился. (Потеря точности решения.?)

    Расчёты выполнены при лямбда равной нулю.

    \vspace{10cm}
    \subsection*{Численное моделирование стационарной модели с условиями Коши на части границы}
    Квадрат с полостью внутри (недоступная среда).

    \vspace{10cm}
    \subsection*{Исследование устойчивости решений обратных задач с данными Коши}
    Также приведем результаты по исследованию устойчивости решений
    обратных задач с данными Коши.
    Для этого переопределим в уравнении~\eqref{eq:2_4:bc3} $a\partial_n \theta = q_b +\varepsilon \psi$,
    где $\psi = \psi(x), x \in \Gamma_1$ некоторая функция, моделирующая возмущение.
    Полученное таким образом решение задачи~\eqref{eq:2_4:eq2},~\eqref{eq:2_4:bc3}
    обозначим за $\theta^{\varepsilon}$.
    Следовательно, $\theta$ будет соответствовать случаю $\varepsilon = 0$.
    Для проведения численного моделирования область $\Omega$ определим
    как квадрат с единичной стороной, где $\Gamma_1$ соответствует стороне $y = 1$.
    Положим $\theta_b = (x + y) / 2$ и $q_b = a / 2$ соответственно.
    Выполним расчеты температурного поля
    для различных малых значений параметра возмущений $\varepsilon$
    из промежутка $[-0.1, 0.1]$ и вычислим $L^2$ норму отклонения возмущенного поля.


    Хорошо известно, что решение задачи с данными
    Коши на границе для одного эллиптического уравнения,
    напр.\ уравнения Лапласа, неустойчиво
    (знаменитый пример Адамара, когда малые изменения
    теплового потока на границе приводят к большим изменениям решения).
    Для рассматриваемой новой модели сложного теплообмена с данными Коши
    теоретический анализ устойчивости это открытая проблема.

    На первом этапе этот вопрос был исследован численно с использованием
    разработанного комплекса программ.

    Полученные численные результаты позволяют высказать гипотезу
    об устойчивости решения этой модели, которую в дальнейшем планируется обосновать аналитически.


    \vspace{10cm}
    \section*{Квазилинейные модели}
    \subsection*{Задачи оптимального управления с ограничениями на состояние системы и метод штрафа}
        Дана область, в ней две подобласти. Мы хотим в одной достичь определенного температурного режима,
    в другой хотим не допустить превышения заранее заданного ограничения.
    $P$ – максимальная мощность источника,
    $\alpha$ – коэффициент диффузии фотонов,
    $Hi$ есть характеристическая функция той части среды, в которой он расположен, деленная на его объём.
    $\beta$ – коэффициент поглощения, $k(\theta)$ является коэффициентом теплопроводности,
    $\sigma$ является произведением удельной теплоемкости и плотности среды,
    $u_1$ описывает мощность источника тепла, $u_2$ – мощность источника теплового излучения.

    Главная проблема здесь-наличие ограничения на температуру в области $G_2$.
    Для ее преодоления рассматривается задача со штрафом.
    Нарушение указанного ограничения штрафуется ростом функционала при малых значениях $\epsilon$.
    Обоснована сходимость предложенного штрафного алгоритма к решению задачи
    с ограничениями на температуру при $\epsilon\to+0$.


    \vspace{10cm}
    \subsection*{Моделирование влияния коэффициента $k(\theta)$ на динамику температурного поля}
        В рассмотренной модели коэффициент теплопроводности зависит от
    неизвестной температуры (квазилинейность уравнения).
    Это позволяет моделировать эффекты переноса энергии в областях с высокой температурой.
    Разработанный комплекс программ позволяет оценить
    влияние этого коэффициента на динамику темп поля.
    температурного поля

    Здесь показать анимацию.


\end{document}
