\documentclass[8pt,a4paper]{article}
\usepackage[utf8]{inputenc}
\usepackage[T2A]{fontenc}
\usepackage[russian]{babel}
\usepackage{amsmath}
\usepackage{graphicx}
\usepackage{listings}
\usepackage{color}
\usepackage{geometry}
\usepackage{array}
\usepackage{longtable}
\usepackage{amsmath,graphicx,latexsym,amssymb,exscale,relsize,caption,subfig,textcomp,tikz,stackrel,setspace,float}
\usepackage{xr-hyper}
\usepackage{hyperref}
\usepackage{amsthm}
\externaldocument[d1:]{presentation}% no .tex extension

\geometry{left=1cm, top=0.5cm, bottom=0.5cm, right=0.5cm}
\newtheorem*{remark}{Замечание}
\definecolor{dkgreen}{rgb}{0,0.6,0}
\definecolor{gray}{rgb}{0.5,0.5,0.5}
\definecolor{mauve}{rgb}{0.58,0,0.82}

\author{Mesenev}
\date{\today}

\begin{document}
    \subsection*{1 слайд.}
    Здравствуйте, уважаемые коллеги!
    Меня зовут Павел Месенев, я представляю доклад по своей диссертации,
    `оптимизационные методы решения обратных задач сложного теплообмена'.

    \subsection*{Актуальность}

    Теплообмен — ключевой процесс в самых разных отраслях: от авиации и энергетики до медицины и производства.
    В газовых турбинах критически важно контролировать температурные режимы,
    чтобы предотвратить перегрев и увеличить срок службы деталей.
    В лазерной термотерапии важно точно моделировать теплообмен в биологических тканях,
    чтобы не повредить окружающие области.
    В космической отрасли необходимо охлаждать летательные аппараты в условиях отсутствия атмосферы, и так далее.

    В таких задачах часто неизвестны краевые условия или распределение тепла внутри объекта.
    Это приводит к необходимости решать обратные задачи теплообмена,
    где нужно восстанавливать недостающие параметры по измеренным данным.

    В научном сообществе давно исследуются различные методы решения таких задач.
    На слайде приведены работы, которые легли в основу моего исследования:
    \begin{itemize}
        \item Андрей Авенирович Амосов, Laiitinen, Druet, Tiihonen -- исследовали разрешимость краевых и начально-краевых задач с полным уравнением переноса излучения.
        \item Rene Pinnau, Oliver Tse предоставили теоретический анализ квазистационарных моделей для $SP_1, SP_3$--приближений.
        \item В работах Чеботарёва, Ковтанюка, Гренкина доказана однозначная разрешимость краевых задач на основе $P_1$ приближений.
        \item Также необходимо упомянуть серьёзный теоретический анализ обратных задач теплообмена в работах Сергея Григорьевича Пяткова и его научной группы.
    \end{itemize}

    Тем не менее ряд важных вопросов, связанных с анализом корректности моделей,
    построением и обоснованием сходимости оптимизационных алгоритмов
    решения обратных задач и задач с краевыми условиями Коши,
    разработкой программ для проведения численных экспериментов остаются открытыми.
    Данная работа посвящена решению указанных проблем.

    Поясню, что понимается под обратными задачами сложного теплообмена.
    Если, например, неизвестны полностью или частично краевые условия или плотность внутренних источников,
    но ставится дополнительное условие, которое называется условие переопределения,
    то задача называется обратной.

    Представляют интерес также задачи с условиями Коши на границе, когда для интенсивности
    излучения неизвестны краевые условия, но задается дополнительное условие для температуры ---
    сама температура и тепловой поток на границе.

    \subsection*{Цели и задачи}
    Целью работы является разработка математических
    и численных методов решения обратных задач сложного теплообмена.
    Это включает в себя как теоретическое обоснование,
    так и разработку алгоритмов и программных комплексов,
    которые позволяют эффективно решать рассматриваемые задачи.
    \begin{itemize}
        \item Доказательство существования и однозначности решений для квазистационарных и квазилинейных моделей теплообмена.
        \item Исследование устойчивости решений, что особенно важно для задач с условиями типа Коши.
        \item Построение оптимизационных алгоритмов, которые гарантируют сходимость к физически осмысленным решениям.
        \item Анализ эффективности предложенных методов решения на тестовых примерах.
        \item Анализ свойств устойчивости и стабилизации процессов сложного теплообмена методами численного моделирования.
    \end{itemize}


    В диссертации предложены и проанализированы несколько постановок задач.
    Теоретическая строгость полученных результатов и обоснованность численных методов позволяют
    предположить, что разработанные подходы могут быть
    успешно применены для решения более широкого спектра задач, для которых пока отсутствуют
    адекватные теоретические или практические решения.

    \subsection*{Граничная обратная задача}
    В инженерных задачах теплообмена часто возникает ситуация, когда свойства границы объекта неизвестны.
    Например, при нагреве стекла или металла в печи отражательная способность поверхности изменяется,
    и для точного контроля температуры необходимо восстановить этот параметр.
    \begin{remark}
        Отражательная способность (а точнее, коэффициент излучения или эмиссивность) действительно может зависеть от температуры.
        При изменении температуры могут происходить изменения в микроструктуре поверхности,
        фазовых переходах или окислении, что влияет на оптические свойства материала.
        Например, для металлов при нагреве часто наблюдается повышение эмиссивности
        из-за образования оксидного слоя, а у некоторых керамических и полимерных материалов эмиссивность
        может меняться в силу изменения структуры материала.
    \end{remark}

    Имея информацию о граничных значениях температуры и теплового потока
    по этим данным пытаемся вычислить температурное поле, интенсивность излучения и коэффициент отражения.


    Первая из рассматриваемых задач формулируется следующим образом:
    дана некая область омега,
    на части её границы неизвестен параметр среды $u$, характеризующий отражающие свойства границы
    (цвет, шероховатость и так далее).
    \begin{remark}
        Сам параметр выражается как $(\frac{\varepsilon}{2(2-\varepsilon)})$,
        где $\varepsilon$ меняется в диапазоне от 0 до 1,
        как следствие на сам параметр накладывается ограничение, такое что $u \in [0, 0.5]$.
        Степень черноты не зависит от направления и определяется формулой
        $\varepsilon_\nu(x) = \frac{I_{\nu,\text{исп}}(x)}{I_{b\nu}(T(x))}$, где
        $I_{\nu,\text{исп}}(x)$ -- интенсивность излучения, испускаемого
        поверхностью при температуре $T(x)$~\cite[53]{Ozisik1976}.

    \end{remark}

    Требуется отыскать тройку: температурное поле,
    поле излучения и неизвестный параметр
    по дополнительной информации о температуре на границе.

    Прямая задача теплообмена, когда параметры известны, хорошо изучена.
    Однако в обратной задаче требуется восстановить неизвестный параметр границы,
    что не всегда возможно единственным образом.

    \textbf{Проблема корректности}
    Хотя существует доказательство однозначной разрешимости для исходной граничной задачи (Александр Юрьевич, 2015),
    по обратной задаче нет теоретических результатов о её корректности.

    \textbf{Способ решения}
    Мы заменяем обратную задачу на задачу оптимального управления, где ищем наилучшее
    приближение параметра границы путём минимизации функционала качества~\eqref{d1:eq:2_1:quality}.
    Это приближённое решение называется квазирешением исходной обратной задачи.
    Для нахождения квазирешения был разработан оптимизационный численный метод и теоретически доказана его корректность.

    Постоянные $a$ (температуропроводность $m^2/sec$), $b$(коэффициент радиационного переноса$[\frac{Dj}{sec \cdot K^2}]$) и
    $\alpha$ (Эффективный коэффициент оптического взаимодействия $m$)
    определяются следующим образом:
    \[
        a = \frac{k}{\rho c_v},\quad b = \frac{4\sigma n^2 T_{\max}^3}{\rho c_v},
        \quad \alpha=\frac{1}{3\kappa - A \kappa_s},
    \]
    где $k$ -- теплопроводность $(Vt/(m \cdot K)$, $c_v$ -- удельная теплоемкость,
    $\rho$ -- плотность $[kg/m^3]$,
    $\beta m/sec$ -- коэффициент теплоотдачи.
    $\sigma$ -- постоянная Стефана-Больцмана $(5.67 \cdot 10^{-8}Vt/(m^2 \cdot K^4))$,
    $n$ -- показатель преломления,
    $T_{\max}$ -- максимальная температура в ненормализованной модели,
    $\kappa = \kappa_s + \kappa_a$ -- коэффициент
    полного взаимодействия $[m^{-1}]$,
    $\kappa_{sparce}$ -- коэффициент рассеяния $[m^{-1}]$, $\kappa_{absorb}$ -- коэффициент поглощения $[m^{-1}]$.

    \subsection*{Нахождение квазирешения обратной задачи}
    Предложенный в работе алгоритм поиска квазирешения обратной задачи основан
    на выведенных условиях оптимальности
    (доказано, что квазирешение должно
    удовлетворять~\eqref{d1:eq:2_1:weakOperational}--\eqref{d1:eq:2_1:theorem_2_eq2}),
    куда входят сопряженные функции для температуры $p_1$ и излучения $p_2$,
    а также связь между сопряженным состоянием и искомым граничным управлением.
    Для компактной записи краевых задач, используется современная операторная форма.
    Система~\eqref{d1:eq:2_1:weakOperational} является операторной записью краевой задачи,
    где $A_{1,2}$ описывают диффузионные члены модели, остальные моделируют граничные условия.
    Уравнения~\eqref{d1:eq:2_1:theorem_2_eq1}--\eqref{d1:eq:2_1:theorem_2_eq2}
    это сопряженная система,
    а вариационное неравенство~\eqref{d1:eq:2_1:theorem_2_eq3}
    устанавливает связь с оптимальным управлением.

    Для нахождения квазирешения мы используем градиентный спуск с проекцией.
    \begin{itemize}
        \item Используя некоторое начально приближение для функции $u$ рассчитаем состояние решив краевую задачу
        \item Рассчитаем сопряженные переменные $p_1, p_2$
        \item Изменим функцию $u$ (управление) в сторону уменьшения градиента.
        \item Применяем оператор проекции, возвращая физические ограничения на $u$.
    \end{itemize}

    Метод градиента выбран потому, что:
    \begin{itemize}
        \item Позволяет учитывать физические ограничения на параметры (в нашем случае, $u$)
        \item Эффективен для диффузионных задач, где решения достаточно гладкие.
    \end{itemize}


    Отметим, что в силу невыпуклости экстремальной задачи градиентные алгоритмы не обладают
    свойством глобальной сходимости, что служит основой для их критики, зачастую заслуженной.
    Однако свойства рассматриваемых диффузионных моделей,
    правильный выбор шага градиентного спуска, а также начального приближения
    обеспечивают сходимость для рассматриваемых задач.

    Следующие примеры этот факт демонстрируют.

    \subsection*{Модель управления температурным полем через граничный параметр}
    Во многих задачах теплообмена важно контролировать температуру на поверхности объекта.
    Например, в производстве стекла требуется равномерный нагрев и охлаждение,
    чтобы избежать внутренних напряжений и трещин.

    Мы не можем изменить сам материал, но можем регулировать условия на границе
    (например, изменяя коэффициент теплопередачи или интенсивность излучения).
    Наша задача подобрать граничное управление $u$,
    чтобы полученное температурное поле $\theta$ соответствовало эталонному.

    Рассмотрим стеклянную пластину и зададим эталонную функцию управления как показано на слайде.
    Пластинка, у которого боковые стороны `обычные', верхняя грань - участок наблюдения,
    нижняя грань - участок под ``контролем''.

    \textbf{Результаты моделирования}
    Для получения представленных результатов, использовался разработанный комплекс программ,
    включающий решение прямой задачи, сопряженной системы и алгоритм градиентного спуска.


    Используя предложенный алгоритм рассчитаем тройку -- $\theta, \varphi, u$.
    Соответствие полученной функции управления эталонному можно видеть на верхних графиках.
    Близость полученного температурного поля эталонному определяется функционалом качества.
    Динамика функционала качества представлена на нижних графиках.
    Интересный эффект `среднего значения'.
    \begin{remark}
        В процессе оптимизации система стремится к усреднённому температурному профилю.
        Это объясняется свойствами диффузионных уравнений: тепло естественным образом распространяется,
        сглаживая резкие градиенты.
    \end{remark}
    В данной постановке требуется большое количество итераций.

    Оптимизационный метод успешно восстанавливает граничный параметр, температурное поле и поле излучения.
%    Оптимизационный алгоритм сходится медленно.


    \section*{Обратная задача с условиями типа Коши}

    \subsection*{Задача без краевых условий для интенсивности излучения}
    В реальных задачах теплообмена часто бывает, что нельзя измерить все параметры на границе.


    \begin{remark}
        Пример: в термографии при анализе нагрева объектов мы можем измерять температуру и тепловой поток, но интенсивность излучения остаётся неизвестной.
    \end{remark}
    Данная модель отличается от предыдущей следующим образом -- нет информации про поле излучения $\varphi$ на границе,
    но на границе известно температурное поле и тепловой поток.

    Требуется найти температурное поле и поле излучения.
    В основе разработанного алгоритма решения лежит анализ экстремальной задачи.

    Строго обосновано существование решения экстремальной задачи, выведены достаточные условия.
    Кроме того, и это принципиально важно, показана сходимость решений экстремальных задач
    к решению рассматриваемой начально-краевой задаче при $\lambda$ стремящемся к 0.
    Это есть обоснование предложенного численного метода решения задачи.

    \subsection*{Пример 1}
    Для параметров, соответствующих стеклу определим функции $r$ и $u$.
    Рассчитав из краевой задачи состояние, получим функции $\theta_b$ и $q_b$, которые будут
    тестироваться для апробации численного алгоритма.
    Для этого мы `забудем' $\hat u$ и восстановим его используя алгоритм градиентного спуска.
    Качество решения будет определяться соответствием полученного решения $\theta_\lambda$ начальным
    условиям~\eqref{d1:eq:2_2:bc2}.
    На рисунке представлен модуль относительного отклонения $\theta_\lambda$
    от $q_b$ на верхней грани куба, а также динамика
    функционала качества, определяющего квадрат нормы разности $\|\theta_\lambda\|^2_\Gamma$.
    На остальных гранях куба значение относительного отклонения имеют тот же порядок малости.

    Обратите внимание на малость функционала качества.
    Сравним с тем, что получилось -- довольно близко, но не идеально.
    Уменьшение параметра регуляризации повышает точность решения,
    но и увеличивает вычислительные затраты.

    \subsection*{Пример 2}
    `Честный' эксперимент - используем то, что полагается в исходной задаче не используя тестовые значения.
    Граничную температуру определим как линейную функцию по оси $z$.
    Нормальную производную определим положительной константой на верхней границе куба, и отрицательной -- на нижней.

    Значение функционала качества через сто итераций позволяет предположить аналогичный порядок
    близости (с предыдущим примером) точного и аппроксимированного решений.


    \section*{Квазистационарная задача с данными Коши}
    Близкие к обратным задачам, задачи с условиями Коши на границе для температуры
    (терминология академика Михаила Михайловича Лаврентьева) возникают когда нет информации
    о поведении интенсивности излучения на границе.

    Аналог стационарной задачи с изменениями по времени.
    В диссертационной работе удалось доказать однозначную разрешимость начально-краевой задачи,
    а также сходимость решений задачи оптимального управления к решению обратной задачи при $\lambda \to 0$
    (данный результат сформулирован в теореме 2.8).
    Для оптимизационного метода решения задачи требуются результаты анализа квазистационарной
    модели из первой главы диссертации, так как итеративный алгоритм требует
    решения начально-краевой задачи на каждой итерации.

    Параметр $u$ в граничном условии для излучения неизвестен.

    \begin{remark}
        Сама начально-краевая задача решается явной схемой - требует высокой дискретизации по времени,
        но численные эксперименты показывают хорошую устойчивость.

        Критерий Куранта-Фридрихса-Леви (устойчивость явной схемы для уравнений в частных производных):
        Для уравнения теплопроводности критерий CFL даёт ограничение на шаг по времени: $\tau \leq \frac{h^2}{2a}$.
        Он нарушается, т.к. в нашем случае:
        $\tau = 1/1200,\; h = 0.02. \; 0.000833 > \frac{0.02 ^ 2}{2 \cdot 0.6} \approx 0.00033$.

        Но в силу того, что это приближенный критерий,
        за счёт:
        1) нелинейные члены сглаживают нестабильность,
        2) небольшие градиенты температуры снижают вероятность неустойчивости.
        Расчёты показывают разницу в шестом знаке после запятой при изменении $\tau$ в 2--4 раза.
    \end{remark}

    \subsection*{Численное моделирование}
    Приведены изображения сравнения полученных результатов в рамках работы
    над диссертацией и коллег из Мюнхена (приведён финальный момент времени)
    Параметры среды соответствуют воздуху при нормальном атмосферном давлении и температуре 400C.


    Николай Боткин (профессор из Технического университета Мюнхена)
    для расчетов использовал разработанную в TUM программу, использующую
    эрмитов прямоугольный (конформный) элемент Богнера-Фокса-Шмидта и сведение задачи к нестационарной.
    Не ясно, что же такое случилось с пространством решений, что потребовались столь экзотические конечные элементы.

    Фактически ему пришлось решать краевую задачу для нелинейного уравнения 4 порядка.
    Предложенный в работе оптимизационный алгоритм является более простым и дает фактически те же результаты.
    Использовались конечные элементы Галёркина (Лагранжа-1).

    \subsection*{Стационарная задача с условиями Коши на части границы}
    Рассмотрим случай если на всей границе известен тепловой поток, а параметр из граничного условия
    для $\varphi$ неизвестен на части границы.
    Мы дополняем `доступный' участок информацией о температуре: $\theta_b$.
    Если данные Коши заданы на части границы задача является ещё более сложной.
    Для точной постановки нет результатов по её корректности.
    Однако предлагаемый далее оптимизационный метод полностью теоретически обоснован и
    лежит в основе соответствующего программного комплекса для численного решения.

    \subsection*{Постановка задачи оптимального управления}
    Используя замену, представленную на слайде, мы заменим исходную задачу, на краевую задачу
    с функциями $\theta$, $\psi$.
    Вместо системы двух нелинейных уравнений одно уравнение стало линейным (для $\Psi$)
    Два нелинейных заменили на нелинейное и линейное (пси - гармоническая, к тому же).
    Особо обратим внимание на параметр s, который пришлось добавить в данную постановку из-за
    того, что реализованный алгоритм не сходился (метод Ньютона для краевой задачи не сходится).

    Расчёты выполнены при лямбда равной нулю.

    \subsection*{Численное моделирование стационарной модели с условиями Коши на части границы}
    Рассматриваемая область является квадратом с полостью внутри, в которой граничный параметр $\gamma$ неизвестен.
    Параметры среды возьмём равными стеклу.
    Температуру на границе положим равной константе,
    тепловой поток на внешней границе равен 0.2, на внешней -0.2 [м/с].
    Результаты численного моделирования, а также начальное и конечное
    значение функционала качества приведены на слайде.

    \subsection*{Исследование устойчивости решений обратных задач с данными Коши}
    Хорошо известно, что решение задачи с данными
    Коши на границе для одного эллиптического уравнения,
    например, уравнения Лапласа, неустойчиво
    (пример Адамара, когда малые изменения теплового потока на границе приводят к большим изменениям решения).
    Для рассматриваемой новой модели сложного теплообмена с данными Коши
    теоретический анализ устойчивости это открытая проблема.


    Также приведем результаты по исследованию устойчивости решений
    обратных задач с данными Коши.
    Для этого переопределим в уравнении~\eqref{d1:eq:2_4:bc3} $a\partial_n \theta = q_b +\varepsilon \psi$,
    где $\psi = \psi(x), x \in \Gamma_1$ некоторая функция, моделирующая возмущение.
    Полученное таким образом решение задачи~\eqref{d1:eq:2_4:eq2},~\eqref{d1:eq:2_4:bc3}
    обозначим за $\theta^{\varepsilon}$.
    Следовательно, $\theta$ будет соответствовать случаю $\varepsilon = 0$.
    Для проведения численного моделирования область $\Omega$ определим
    как квадрат с единичной стороной, где $\Gamma_1$ соответствует стороне $y = 1$.
    Положим $\theta_b = (x + y) / 2$ и $q_b = a / 2$ соответственно.
    Выполним расчеты температурного поля
    для различных малых значений параметра возмущений $\varepsilon$
    из промежутка $[-0.1, 0.1]$ и вычислим $L^2$ норму отклонения возмущенного поля.


    На первом этапе этот вопрос был исследован численно с использованием
    разработанного комплекса программ.

    Полученные численные результаты позволяют высказать гипотезу
    об устойчивости решения этой модели, которую в дальнейшем планируется обосновать аналитически.


    \section*{Квазилинейные модели}

    \subsection*{Задачи оптимального управления с ограничениями на состояние системы и метод штрафа}
    Квазилинейная модель сложного теплообмена описывается двумя параболическими
    \textit{(производная по времени + две производные по пространству)} уравнениями.
    Рассмотрим область в которой содержится две подобласти -- они приведены на слайде.
    В области, отмеченной синим мы хотим достичь определенного температурного режима, который обозначается $\theta_b$.
    В красной мы хотим не допустить превышения заранее заданного ограничения.
    $P$ – максимальная мощность источника,
    $\alpha$ – коэффициент диффузии фотонов,
    $Hi$ есть характеристическая функция той части среды, в которой он расположен, деленная на его объём.
    $\beta$ – коэффициент поглощения, $k(\theta)$ является коэффициентом теплопроводности,
    $\sigma$ является произведением удельной теплоемкости и плотности среды,
    $u_1$ описывает мощность источника тепла,
    $u_2$ – мощность источника теплового излучения.

    Главная проблема здесь -- наличие ограничения на температуру в области $G_2$.
    Эта модель возникает при описании процесса ВВЛА и поэтому тут ограничения на темп в подобласти.

    Для ее преодоления рассматривается задача со штрафом.
    Нарушение указанного ограничения штрафуется ростом функционала при малых значениях $\epsilon$.
    Обоснована сходимость предложенного штрафного алгоритма к решению задачи
    с ограничениями на температуру при $\epsilon\to+0$.

    \subsection*{Моделирование влияния коэффициента $k(\theta)$ на динамику температурного поля}
    В рассмотренной модели коэффициент теплопроводности зависит от
    неизвестной температуры (отсюда квазилинейность уравнения).
    Это позволяет моделировать эффекты переноса энергии в
    областях с высокой температурой.
    Разработанный комплекс программ позволяет оценить
    влияние этого коэффициента на динамику темп поля.
    температурного поля

    \textit{Здесь показать анимацию.}

    \subsection*{Научная новизна}
    В работе получены новые априорные оценки решений
    начально-краевых задач для квазистационарных и квазилинейных
    уравнений сложного теплообмена и доказана их нелокальная
    однозначная разрешимость.

    Для рассмотренных моделей сложного теплообмена
    рассмотрены новые постановки граничных обратных задач,
    предложены оптимизационные методы их решения.

    Выполнен теоретический анализ возникающих новых экстремальных задач.

    Представлены априорные оценки решений регуляризованных задач и впервые
    обоснована сходимость их решений к точным решениям обратных задач.

    Для решения задач с фазовыми ограничениями, предложены алгоритмы,
    основанные на аппроксимации экстремальными задачами со штрафом.

    Разработаны и протестированы новые алгоритмы решения прямых,
    обратных и экстремальных задач для моделей сложного теплообмена.

    \subsection*{Положения, выносимые на защиту}
    Таким образом в диссертации присутствуют следующие результаты.

    \textbf{В области анализа математических моделей:}
    \begin{itemize}
        \item Доказано существование квазирешения обратной задачи с неизвестным коэффициентом отражения. Это означает, что в рассмотренной постановке возможно корректное восстановление параметров границы.
        \item Установлена однозначная разрешимость квазистационарных и квазилинейных моделей, что служит подтверждением их физическую состоятельность.
        \item Выведены условия, при которых оптимизационные задачи корректно аппроксимируют решения обратных задач.
        \item Построены системы оптимальности и доказано, что они не вырождаются, что важно для корректной работы численных методов.
    \end{itemize}

    \textbf{В области численных методов и комплексов программ:}
    \begin{itemize}
        \item Построены оптимизационные алгоритмы, позволяющие находить решения обратных задач с высокой точностью.
        \item Для задач с условиями Коши доказана сходимость решений при стремлении параметра регуляризации к нулю, что делает найденные методы стабильными.
        \item Разработаны и теоретически обоснованы методы решения экстремальных задач с температурными ограничениями,
        включая метод штрафов.
    \end{itemize}


    Сравнение разных моделей
    Для некорректных обратных задач увеличение точности модели переноса излучения не всегда приводит к улучшению восстановления параметров из-за усиления чувствительности к шуму.


    \section*{2. Почему в большинстве работ (и у тебя) его нет}

    Потому что используется серая аппроксимация (\textit{gray approximation}).

    \subsection*{Суть серой аппроксимации}

    \begin{itemize}

        \item
        спектральная зависимость интегрируется по частоте,




        \item
        коэффициенты считаются усреднёнными,




        \item
        остаётся только пространственно-угловая зависимость.


    \end{itemize}

    Формально:


    \section*{3. Физический смысл этого упрощения}

    Серая модель корректна, если:


    \begin{itemize}
        \item спектральные коэффициенты медленно меняются по частоте,
        \item нет выраженных спектральных линий,
        \item интересует тепловой баланс, а не спектр излучения.
    \end{itemize}
    Это типичная инженерная ситуация.
    В наиболее общей форме уравнение переноса излучения содержит спектральный параметр.
    В данной работе используется серая аппроксимация, при которой спектральная зависимость интегрируется по частоте,
    что позволяет существенно упростить модель и
    повысить устойчивость решения обратной задачи без потери корректности для рассматриваемых режимов.


    \section*{7. Если спросят жёстче: «А не теряется ли физика?»}
    Отвечай так:
    \textit{Для задач теплового баланса и восстановления интегральных теплофизических
    параметров спектральная детализация не является определяющей;
    её учет целесообразен для задач спектроскопии,
        но не для рассматриваемого класса обратных задач теплопереноса.}
    Очень важный момент:



    \textit{$P_1$ — это единственное приближение, которое одновременно имеет ясный физический смысл, замкнутую форму и разумную вычислительную сложность.}
    Для инженерных обратных задач важнее устойчивая интегральная информация, чем точное описание угловой структуры излучения.

    При переходе от P_1 к P_3 радиационная часть модели перестаёт быть скалярной и превращается в систему для векторных и тензорных величин, что увеличивает размерность задачи примерно на порядок.


    SP₃ учитывает первую существенную анизотропную поправку к диффузионной модели, не вводя векторные и тензорные неизвестные.
    Модель SP₃ действительно является разумным компромиссом между точностью и вычислительной сложностью. Однако целью работы было исследование обратных задач для минимальной по размерности физически корректной модели, что естественным образом привело к использованию
-приближения.

    Модель SP₃ действительно позволяет лучше учитывать анизотропию излучения, однако в контексте обратных задач её преимущества не всегда компенсируют увеличение размерности и усложнение структуры функционала. Поэтому в работе был осознанно выбран диффузионный-подход.
\end{document}
