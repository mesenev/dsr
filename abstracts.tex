1. Граничная обратная задача для уравнений сложного теплообмена

Рассмотрена граничная обратная задача нахождения отражающих
свойств участка границы для стационарных уравнений радиационно-кондуктивного
теплообмена в трёхмерной области.
Доказано существование квазирешения обратной задачи и получена система оптимальности.
Приведён алгоритм решения задачи, эффективность которого проиллюстрирована численными примерами.


2. Анализ оптимизационного метода решения задачи сложного теплообмена с граничными условиями типа коши

Предложен оптимизационный метод решения краевой задачи с условиями типа Коши
для уравнений радиационно-кондуктивного теплообмена в рамках $P_1$--приближения
уравнения переноса излучения. Выполнен теоретический анализ соответствующей задачи
граничного оптимального управления.
Показано, что последовательность решений экстремальных задач
сходится к решению краевой задачи с условиями типа Коши для температуры.
Результаты теоретического анализа проиллюстрированы численными примерами.
Библ. 32.


3. Граничная обратная задача сложного теплообмена

Рассмотрена граничная обратная задача для стационарных уравнений сложного теплообмена с незаданным краевым условием для интенсивности излучения на части границы и условием переопределения на другой части границы.
Предложен оптимизационный метод решения обратной задачи и представлен анализ соответствующей задачи
граничного оптимального управления.
Показано, что последовательность решений экстремальных задач
сходится к решению обратной задачи.
Эффективность алгоритма проиллюстрирована численными примерами.
Библ.
32.


4. Метод штрафов для решения задачи оптимального управления для квазилинейного параболического уравнения

Рассматривается задача оптимального управления для квазилинейного параболического уравнения,
моделирующего радиационный и кондуктивный теплообмен в ограниченной трехмерной области при
ограничениях на решение в заданной области.
Доказана разрешимость задачи оптимального управления.
Предложен алгоритм решения задачи, основанный на методе штрафов.


5. Optimal control with phase constraints for a quasilinear endovenous laser ablation model


Исследуется задача оптимального управления для квазилинейных уравнений радиационно-кондуктивного
теплообмена, моделирующая процесс внутривенной лазерной
абляции в ограниченной области с отражающей границей.
В данной подобласти требуется обеспечить заданное
распределение температуры, а в другой
данной подобласти температура не может превышать критического значения.
На основе оценок решения управляемой системы
доказывается разрешимость задачи управления.
Предложен алгоритм
решения задачи оптимального управления путем минимизации целевого функционала со штрафом.
Работоспособность алгоритма иллюстрируется численным примером.


6. Оптимизационный метод решения обратной задачи комплексного теплообмена

Предложен оптимизационный метод решения обратной задачи для стационарных уравнений сложного
теплообмена с незаданным граничным условием для интенсивности излучения на части
границы и условием переопределения на другой части границы.
Приведен анализ краевой задачи оптимального управления и показано,
что последовательность решений задач управления сходится к решению обратной задачи.
