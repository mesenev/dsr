    {\actuality}
    Под сложным теплообменом понимают
    процесс распространения тепла излучением, теплопроводностью и
    конвекцией.
    При этом важно, что радиационный перенос тепла дает
    существенный вклад в распределение температурного поля.
    Для многих инженерных приложений моделирование процессов теплопередачи имеет
    первостепенное значение (например, это относится к радиационному
    теплообмену в полупрозрачных материалах).
    Температура – важнейший параметр практически во всех производственных этапах.
    Следовательно, знание температурных полей необходимо для контроля производственных
    процессов и качества результата.

    С математической точки зрения процесс сложного теплообмена
    моделируется системой уравнений с частными производными, включающей
    уравнение теплопроводности, а также интегро-дифференциальное уравнение
    переноса теплового излучения~\cite{Ozisik1976, Sparrow1971, howell2010thermal, modest2013radiative}.
    Теоретический и численный анализ различных задач для такой системы является весьма
    затратным поскольку интенсивность излучения зависит не только от
    пространственной и временной переменных, но и от направления
    распространения излучения.
    Имеется ряд аппроксимаций уравнения переноса излучения, в том числе диффузионное
    $P_1$-приближение, в котором интенсивность излучения усредняется по направлениям.
    $P_1$-приближение является частным случаем метода сферических гармоник ($PN$-приближения)
    и упрощенного метода сферических гармоник ($SPN$-приближения).
    В данной работе рассмотрены модели, основанные на $P_1$-приближении.


    В классических прямых задачах сложного теплообмена задаются
    параметры системы, и по ним вычисляется состояние системы –
    температурное поле и интенсивность теплового излучения.
    Обратные задачи сложного теплообмена состоят в разыскании неизвестных параметров
    системы по некоторой дополнительной информации (условия переопределения)
    о температурном поле или интенсивности излучения.
    Методы, основанные на решении обратных задач, продолжают оставаться
    актуальным, быстро развивающимся направлением исследований.
    Отметим трудности, возникающие при решении обратных задач сложного
    теплообмена.


    Эти задачи математически классифицируются как некорректные в общем
    смысле из-за неустойчивости решений. Развитие методов оптимизации
    позволяет исправить проблемы некорректности исследуемых задач. В основе
    таких методов лежит идея замены исходной задачи на задачу оптимизации и
    фактически использует методику А.Н. Тихонова регуляризации задачи.

    Диссертация посвящена теоретическому и численному анализу одному из
    наиболее распространенных на практике типу обратных задач – граничных
    обратных задач сложного теплообмена в трёхмерной области в рамках
    $P_1$-приближения уравнения переноса излучения.
    При этом основное внимание уделяется разработке
    и обоснованию оптимизационных методов решения указанных задач.
    Исследование обратных задач для математических моделей
    радиационного теплопереноса, учитывающих одновременно вклад эффектов
    теплопроводности и излучения, даёт теоретическую основу для эффективных
    инженерных решений в различных областях, таких как производство стекла,
    лазерная термотерапия и другие.


    \textbf{Степень разработанности темы исследования.}
    Основополагающие работы А.Н. Тихонова~\cite{TikhonovSamarskii1972} и его коллег,
    которые исследовали уравнения математической физики, стали отправной точкой для
    разработки методов преодоления неустойчивостей в обратных задачах.
    О.М. Алифанов~\cite{Aliphanov2009} в своих работах, посвященных обратным задачам
    в исследовании сложного теплообмена, также сделал значительный вклад в развитие этой области.
    J.V.\ Beck ~\cite[]{Beck1985-fg} предложил новые подходы к решению обратных задач,
    основанных на методах оптимизации и статистическом анализе.


    Исследования, проведенные в работах~\cite{
        Tiihonen1997a, Tiihonen1997b, metzger1999existence, Amosov2005,
        Amosov2009, Philip2010, Amosov2016, Amosov2017,
        amosov2010stationary, druet2009weak, druet2010weak, laitinen2001conductive},
    позволили анализировать разрешимость моделей сложного теплообмена между телами,
    разделенными прозрачной средой.
    В рамках этих моделей было исследовано уравнение теплопроводности
    с нелинейным нелокальным краевым условием, имитирующим тепловое излучение
    границы области и теплообмен излучением между различными частями границы.

    В работах Tiihonen et al.~\cite{Tiihonen1997a, Tiihonen1997b} рассматриваются вопросы
    стационарного и нестационарного теплообмена с учетом радиационных и конвективных потоков.
    Metzger et al.~\cite{metzger1999existence} исследовали существование и единственность
    решений для уравнений теплообмена в двухфазных системах.


    В серии работ Amosov et
    al~\cite{Amosov2005, Amosov2009, Amosov2016, Amosov2017, amosov2010stationary}.
    авторы анализировали свойства существования и устойчивости стационарных
    и нестационарных решений в моделях теплообмена с нелокальными краевыми условиями,
    а также исследовали вопросы оптимального управления в таких системах.
    Результаты этих исследований позволили разработать новые методы и алгоритмы для определения
    оптимальных параметров управления и прогнозирования поведения системы в различных условиях.

    Philip et al.~\cite{Philip2010} также занимались анализом разрешимости обратных задач
    в моделях теплообмена, основанных на уравнениях конвекции-диффузии-реакции.
    Это позволило разработать методы восстановления неизвестных параметров системы,
    таких как коэффициенты теплопроводности или плотности источников тепла.


    Исследования, проведенные Druet et al.~\cite{druet2009weak, druet2010weak},
    посвящены слабым решениям стационарных и нестационарных моделей теплообмена
    с нелинейными нелокальными краевыми условиями.
    Это позволило разработать новые методы исследования структуры решений и их свойств,
    а также применить полученные результаты для анализа реальных физических систем.


    В работе Laitinen et al.~\cite{laitinen2001conductive} рассматривается проблема
    проводимости тепла в многослойных структурах с учетом радиационного и конвективного теплообмена.
    Результаты этого исследования позволили разработать методы оптимизации теплоизоляционных
    свойств материалов и технологий, используемых в различных отраслях промышленности.


    В целом, проведенные исследования в области обратных задач теплообмена
    с учетом радиационных и конвективных процессов позволили разработать
    новые методы и подходы к анализу и оптимизации тепловых систем,
    что нашло широкое применение в теплоэнергетике, авиации, космической технике,
    микроэлектронике и других областях науки и техники.

    В ряде исследований была изучена возможность решения моделей сложного теплообмена
    в полупрозрачной среде, где для описания радиационного теплообмена применяется
    уравнение переноса излучения.
    В работах~\cite{asllanaj2003existence, kelley1996existence}
    доказана единственность решений одномерных стационарных задач
    радиационно-кондуктивного теплообмена, в то время как
    в~\cite{ghattassi2018existence, Porzio2004, Thompson2004}
    была доказана единственность решений трехмерных задач.
    В исследовании~\cite{Thompson2004} рассматривается стационарная модель,
    в~\cite{Porzio2004} – нестационарная, а в~\cite{ghattassi2018existence} – квазистационарная модель.


    Квазистационарные модели сложного теплообмена представляют собой модели,
    которые включают нестационарное уравнение
    теплопроводности и стационарное уравнение переноса излучения.
    Эти модели позволяют учесть различные временные масштабы процессов,
    происходящих в системе, и использовать соответствующие численные методы для их решения.


    Обратим внимание на работы~\cite{
        asllanaj2004convergence, asllanaj2007transient, banoczi1999fast,
        ghattassi2016galerkin, klein2005transient, kovtanyuk2012},
    которые посвящены разработке численных методов для ранее упомянутых моделей сложного теплообмена.

    А.А. Амосов в своих работах~\cite{
        Amosov2016, Amosov2017, amosov2016unique, amosov2017unique}
    доказал единственность решений для стационарных и квазистационарных
    моделей сложного теплообмена в системе полупрозрачных тел.
    В этих моделях используется уравнение переноса излучения с краевыми условиями,
    которые моделируют отражение и преломление излучения на границах тел.
    Рассмотрена также зависимость интенсивности излучения и оптических свойств
    тел от частоты излучения.
    В работах~\cite{Amosov2016, Amosov2017}
    предполагались условия диффузного отражения и преломления излучения,
    в то время как в~\cite{amosov2016unique, amosov2017unique} рассматривались условия
    отражения и преломления излучения согласно законам Френеля.

    Эти исследования представляют собой значительный вклад в разработку и анализ
    численных методов для моделей сложного теплообмена в полупрозрачных средах.
    Результаты позволяют глубже понять особенности этих моделей и принять во внимание
    различные физические процессы, такие как отражение и преломление излучения,
    а также зависимость оптических свойств материалов от частоты излучения.

    В следующих работах акцент ставится на решении обратных задач в
    контексте моделей сложного теплообмена, основанных на полном уравнении переноса излучения.
    В статье~\cite{end2010optimization}, авторы разрабатывают численный алгоритм для решения
    задачи оптимального управления источниками тепла и излучения в стационарной модели сложного
    теплообмена, что позволяет управлять процессом с учетом радиационных эффектов.

    В статье~\cite{end2011analytical}, теоретический анализ задачи оптимального
    управления источниками тепла проводится в рамках квазистационарной модели сложного теплообмена,
    включая полное уравнение переноса излучения.
    Авторы доказывают однозначную разрешимость прямой задачи,
    разрешимость задачи управления и представляют условия оптимальности.
    Это позволяет лучше понять возможности и ограничения оптимального управления тепловыми
    источниками в сложных системах с учетом радиационного теплообмена.

    В работе~\cite{amosov2010stationary}, основной фокус направлен на теоретический
    и численный анализ обратной задачи восстановления начального распределения температуры,
    основываясь на известной зависимости температуры на границе области от времени.
    Это исследование проводится в рамках квазистационарной модели сложного теплообмена,
    что может облегчить понимание и оптимизацию процессов теплообмена с учетом радиационных эффектов.


    Работы~\cite{birgelis2003optimal, meyer2006optimal, meyer2009state, Philip2010}
    основываются на анализе задач оптимального управления для стационарных моделей
    сложного теплообмена в прозрачной среде.
    Они включают уравнение теплопроводности с нелинейным нелокальным краевым условием,
    моделирующим тепловое излучение границы области и теплообмен излучением между частями границы.
    Эти исследования разрабатывают методы для эффективного управления процессами теплообмена
    в прозрачных средах.

    В статье~\cite{belmiloudi2014nonlinear} представлен численный алгоритм для решения
    задачи оптимального управления граничными коэффициентами в одномерной нестационарной модели,
    которая включает уравнение теплопроводности с нелинейным краевым условием,
    описывающим тепловое излучение границ.
    Этот подход позволяет адаптировать граничные коэффициенты для улучшения процессов
    теплообмена в динамических условиях.

    Статьи~\cite{kovtanyuk2013iterative, Thommes2002, Pinnau2008, Siewert1991}
    посвящены численному моделированию в рамках диффузионных моделей сложного теплообмена.
    Они показывают различные подходы к аппроксимации и решению уравнений теплопроводности
    с учетом радиационных эффектов.

    В работе~\cite{gallouet2016analysis} исследуется схема метода конечных объемов
    для решения квазистационарной системы уравнений сложного теплообмена,
    основанной на $P_1$-приближении уравнения переноса излучения.
    Это исследование предлагает новый подход к численному решению квазистационарных задач,
    связанных с радиационным теплообменом.


    В статьях~\cite{modest2014elliptic, frank2011adaptive, kovtanyuk2012, Thommes2002, Larsen2002, Frank2007}
    проводится сравнение $P_1$-приближения с другими методами аппроксимации уравнения переноса излучения.
    Авторы исследуют различные подходы к приближению уравнения переноса излучения,
    такие как дискретные углы, сферические гармоники и другие методы,
    для определения наиболее точных и эффективных способов моделирования радиационного теплообмена.

    Статья~\cite{modest2014elliptic} представляет сравнение $P_1$-приближения
    с эллиптическим уравнением переноса излучения, оценивая их точность и вычислительные затраты.
    В работе~\cite{frank2011adaptive} рассматривается
    адаптивное изменение сетки (adaptive mesh refinement)
    для численного решения уравнения переноса излучения
    с использованием различных методов аппроксимации, включая $P_1$-приближение.

    В статье~\cite{kovtanyuk2012} представлено сравнение $P_1$-приближения с методом
    дискретных углов для решения уравнения переноса излучения в полупрозрачных средах.
    Работа~\cite{Thommes2002} сравнивает $P_1$-приближение с другими методами для решения
    уравнений радиационного переноса и теплопроводности.

    Статьи~\cite{Larsen2002} и~\cite{Frank2007} также оценивают эффективность и точность
    различных методов аппроксимации уравнения переноса излучения,
    включая $P_1$-приближение, для моделирования радиационного теплообмена в различных условиях.


    В статьях~\cite{Addam2015, olbrant2013asymptotic, frank2010optimal, frank2011adaptive, Frank2007}
    приведены вывод и численный анализ нестационарного $P_1$-приближения уравнения переноса излучения.
    Эти работы оценивают точность и эффективность данного подхода в моделировании радиационного теплообмена
    в различных условиях и сравнивают его с другими методами аппроксимации.

    Работы R. Pinnau и O. Tse ~\cite{Pinnau2007, Pinnau2013} проводят теоретический анализ квазистационарных
    моделей сложного теплообмена, основанных на $SP_1$ и $SP_3$-приближениях.
    Эти модели включают уравнение теплопроводности, стационарное $SP_N$-приближение,
    а также в~\cite{Pinnau2013} уравнения Навье – Стокса в приближении Буссинеска.

    В результате анализа авторы определяют преимущества и недостатки различных приближений,
    позволяющих точнее и эффективнее моделировать сложные процессы теплообмена, такие как радиационный перенос,
    теплопроводность и конвективный теплообмен, описываемый уравнениями Навье – Стокса.


    В работе~\cite{Pinnau2007} авторы доказывают существование, единственность и ограниченность
    решения задачи сложного теплообмена на основе $P_1$-приближения без источников тепла и излучения.
    В отличие от этого, в статье~\cite{Pinnau2013} авторы доказывают однозначную разрешимость задачи
    свободной конвекции с радиационным теплообменом на основе $SP_3$-приближения в двумерной области,
    где присутствуют источники тепла с ограниченной плотностью.

    В свою очередь, в работах А.Е. Ковтанюка и А.Ю. Чеботарева~\cite{Kovtanyuk2014, Kovtanyuk2016, Kovtanyuk2015}
    авторы исследуют стационарные модели сложного теплообмена на основе $P_1$-приближения.
    В этих статьях доказана однозначная разрешимость краевых задач
    и сходимость метода простой итерации для нахождения решения.
    Эти результаты важны для обоснования применимости и
    эффективности $P_1$-приближения в задачах сложного теплообмена.


    Стоит отметить, что численная реализация метода, предложенного в работах А.Е. Ковтанюка и А.Ю. Чеботарева,
    затруднена, поскольку на каждой итерации требуется решить нелинейное эллиптическое уравнение.
    В статье~\cite{Chebotarev2016Odnaznachnaya} авторы доказывают однозначную разрешимость
    сходной субдифференциальной краевой задачи с многозначной зависимостью коэффициента излучения
    границы от интенсивности излучения.

    В работах~\cite{astrakhantseva2017analysis, chebotarev2018inverse} получены результаты о существовании
    и единственности решений обратных задач для стационарной диффузионной модели сложного теплообмена.
    Эти задачи заключаются в нахождении неизвестной плотности источников тепла в виде линейной комбинации
    заданных функционалов при известных значениях этих функционалов на решении краевой задачи.

    Работы R. Pinnau и O. Tse~\cite{Pinnau2007, Pinnau2013} посвящены теоретическому анализу задач
    оптимального управления температурой на границе области в рамках квазистационарных моделей сложного
    теплообмена на основе $SP_N$-приближений.
    Авторы доказали разрешимость задач управления и нашли необходимые условия оптимальности,
    что является важным результатом для понимания и решения задач оптимального управления
    в рамках сложных теплообменных процессов.

    В работах~\cite{clever2012optimal, clever2014model, lang2005adaptive, frank2011adaptive, Pinnau2007b, Pinnau2013}
    были разработаны численные методы решения задач оптимального управления граничной температурой
    для квазистационарной модели сложного теплообмена на основе $P_1$-приближения.
    Особенностью методов, предложенных в~\cite{clever2012optimal, clever2014model, lang2005adaptive},
    является учет зависимости коэффициента поглощения от частоты излучения, что позволяет получать более
    точные результаты при моделировании сложных теплообменных процессов.
    Эти методы могут быть полезными для практических приложений
    в области управления и оптимизации теплообменных систем.


    В~\cite{frank2011adaptive, Pinnau2007b} авторы рассматривали задачу
    минимизации отклонения поля температуры от заданного.
    Для решения задачи оптимизации применялся метод Ньютона,
    который является эффективным итерационным методом для нахождения оптимального решения.

    В работе~\cite{lang2005adaptive} также рассматривалась задача минимизации
    отклонения поля температуры от заданного, но в данном случае использовался метод проекции градиента,
    который предлагает другой подход к оптимизации и может быть предпочтительным в некоторых ситуациях.

    В работах~\cite{clever2012optimal, Pinnau2013} авторы фокусировались на минимизации
    нормы градиента температуры и применяли метод проекции градиента для решения задачи оптимизации.
    Этот подход может быть полезен в задачах, где важна гладкость решения.

    В работе~\cite{clever2014model} решалась задача минимизации отклонения поля температуры
    от заданного на основе серии из трех моделей, аппроксимирующих уравнение переноса излучения с разной точностью.
    Авторы использовали оптимизационный метод второго порядка,
    что позволяет учитывать кривизну целевой функции и может привести к более быстрой сходимости и точности решения.

    Таким образом, в каждой из этих работ предложены различные методы
    оптимизации для решения задач минимизации отклонения температурного
    поля или нормы градиента температуры от заданных значений.
    Эти методы могут быть полезными для разработки и применения оптимальных
    стратегий управления температурой в различных теплообменных системах.


    В работе~\cite{Kovtanyuk2016Optimal} авторы провели теоретический анализ
    задачи оптимального управления температурой на границе области в рамках стационарной
    диффузионной модели сложного теплообмена.
    Для численного решения этой задачи управления был применен метод проекции
    градиента, который является эффективным итерационным методом оптимизации.

    В ряде работ А.Е. Ковтанюка, А.Ю. Чеботарева и
    других~\cite{Kovtanyuk2014, astrakhantseva2017design, Chebotarev2015, Kovtanyuk2014TheoreticalAnalysis},
    исследовались задачи оптимального управления коэффициентом
    излучения границы области в рамках стационарной модели сложного теплообмена на основе $P_1$-приближения.

    В~\cite{Kovtanyuk2014, Kovtanyuk2014TheoreticalAnalysis},
    авторы вывели необходимые условия оптимальности для задачи максимизации выходящей
    из среды энергии, доказали разрешимость задачи управления и получили
    достаточные условия регулярности системы оптимальности.
    Они также обнаружили, что эти условия выполняются при достаточно
    большой скорости движения среды и малых размерах области.


    В работах~\cite{end2011analytical, asllanaj2003existence}, авторы изучили
    оптимальное управление в задачах максимизации и минимизации
    полей температуры и излучения в области теплообмена.
    Они получили достаточные условия оптимальности для этих задач и доказали
    сходимость метода простой итерации для нахождения оптимального управления.
    Результаты этих исследований являются важным вкладом в разработку и применение
    оптимальных стратегий управления в теплообменных системах, особенно в случае,
    когда поле температуры или излучения должно быть максимизировано или минимизировано во всей области теплообмена.

    В последние годы исследования моделей сложного теплообмена стали особенно
    актуальными в связи с практическими приложениями, такими
    как высокотемпературные процессы и передовые технологии.
    Ниже приведены некоторые примеры исследований практической применимости рассматриваемых моделей:

    Производство стекла: работы~\cite{frank2010optimal, clever2012optimal} представляют
    собой примеры исследований, посвященных оптимальному
    управлению температурой в процессах производства стекла.
    В этих работах рассматриваются модели сложного теплообмена, которые позволяют управлять
    температурными полями и повышать эффективность производства.

    Лазерная интерстициальная термотерапия: в работах~\cite{Tse2012, Hubner2017}
    изучается применение моделей сложного теплообмена для
    управления процессами лазерной интерстициальной термотерапии.
    Этот метод используется для локального лечения опухолей и
    требует точного управления температурными полями во время процедуры.
    Модели сложного теплообмена могут помочь в определении оптимальных
    параметров управления для достижения желаемого терапевтического эффекта.

    Процедура внутривенной лазерной абляции (ВВЛА) является безопасной и эффективной
    в лечении варикозных вен~\cite{Endovenous_vandenBos2009}.
    Математическое моделирование лучевых и тепловых процессов при ВВЛА
    имеет решающее значение для определения оптимальных параметров
    излучения, обеспечивающих достаточно высокие
    температуры внутри вены для успешной облитерации,
    обеспечивая при этом сохранность окружающих тканей.
    Результаты численного моделирования для различных
    длин волн и диаметров жил обсуждались в ряде работ~\cite{
        Opticalthermal_vanRuijven2014, Some_Poluektova2014,
        Endovenous_Malskat2014, Mathematical_Mordon2006}.


    Эти примеры исследований показывают, что модели сложного теплообмена имеют
    большой потенциал для практического использования в различных отраслях промышленности и медицине.
    Они могут помочь в определении оптимальных стратегий
    управления для повышения эффективности процессов
    и обеспечения безопасности и точности в различных приложениях.

%Во время ВВЛА лазерный оптический волоконный световод вводится в поврежденную вену,
%и лазерное излучение передается через световод, который в это время вытаскивается из вены.
%Конец оптического световода обычно покрыт углеродистым слоем (наконечник оптического волокна).
%Углеродистый слой разделяет лазерную энергию на нагрев наконечника оптического волокна и излучение.
%Тепло от наконечника оптического волокна передается крови путем кондуктивного теплообмена,
%теплообмен значительно увеличивается за счет потока пузырьков,
%образующихся на нагретом наконечнике волокна.
%
%Излучение, попадающее в кровь и окружающую ткань,
%частично поглощается с выделением тепла.
%В результате образующаяся тепловая энергия вызывает значительный нагрев вены,
%что приводит к ее облитерации (закрытию вены).


    Эти исследования демонстрируют важность математического моделирования
    для выбора оптимальной мощности лазерного излучения и скорости отвода волокна.

    Наиболее перспективным подходом к выбору оптимальных параметров
    излучения является рассмотрение задачи оптимального управления
    для уравнений типа реакция-диффузия, описывающих процедуру ВВЛА\@.
    Различные подходы к анализу и оптимизации параметров для моделей реакция-диффузия,
    описывающих различные природные явления, можно найти в~\cite{
        Stability_Alekseev2016, Optimization_Brizitskii2018,
        chebotarev2018inverse, Theoretical_Maslovskaya2021}.

    Задачи оптимального управления для модели ВВЛА рассматриваются
    в~\cite{Optimal_Kovtanyuk2020, Inverse_Kovtanyuk2021}.
    В~\cite{Optimal_Kovtanyuk2020} поставлена задача оптимального управления для модели реакция-диффузия,
    описывающей процедуру ВВЛА, которая заключается в аппроксимации заданного
    температурного профиля в определенной точке области модели.
    В~\cite{Inverse_Kovtanyuk2021} изучается аналогичная задача оптимального управления,
    как в~\cite{Optimal_Kovtanyuk2020}.
    Здесь целевой функционал выбран таким образом, что его минимизация позволяет
    достичь заданного распределения температуры в различных частях области модели.
    Это позволяет обеспечить достаточно высокую температуру внутри
    вены для успешной облитерации и безопасную температуру в окружающей вену ткани.
    Доказана единственная разрешимость начально-краевой задачи,
    на основе которой показана разрешимость задачи оптимального управления.
    Предложен алгоритм поиска решения задачи оптимального управления.
    Эффективность алгоритма проиллюстрирована численным примером.


    Таким образом, ряд важных задач, относящихся к моделированию и
    оптимизации сложного теплообмена на основе
    диффузионного приближения, оставался нерешенным: исследование
    разрешимости нестационарной задачи сложного
    теплообмена с источниками тепла и излучения и нестационарной задачи
    свободной конвекции с радиационным теплообменом в трехмерной
    области, исследование устойчивости по Ляпунову стационарных решений,
    вывод диффузионной модели сложного теплообмена в многослойной среде,
    анализ сходимости метода Ньютона
    для уравнений сложного теплообмена, разработка численных
    методов решения задач оптимального управления
    коэффициентом излучения границы области в рамках нестационарных моделей
    сложного теплообмена, доказательство регулярности
    условий оптимальности для задачи оптимального управления коэффициентом
    излучения границы в рамках стационарной модели.

    В целом, исследования моделей сложного теплообмена в практических приложениях
    подчеркивают важность этого направления для развития новых технологий и применений.
    Моделирование и оптимизация тепловых процессов в различных областях может привести
    к более эффективным и безопасным методам производства, лечения и управления температурой.
    С развитием вычислительных технологий и улучшением численных методов, можно ожидать
    дальнейшего прогресса в этой области исследований.


        {\aim} данной работы является является теоретический и численный анализ граничных
    обратных задач, включая задачи с условиями типа Коши на границе области,
    и задач оптимального управления для моделей сложного теплообмена на
    основе $P_1$-приближения уравнения переноса излучения, а именно:
    \begin{itemize}[leftmargin=5.5mm]
        \renewcommand\labelitemi{--}
        \item исследование разрешимости краевых и начально-краевых задач для
        квазистационарных и квазилинейных моделей сложного теплообмена;
        \item разработка оптимизационных алгоритмов решения обратных задач и задач
        с краевыми условиями Коши, теоретический анализ и обоснование их
        сходимости;
        \item разработка комплекса программ для проведения вычислительных
        экспериментов и тестирования предложенных алгоритмов.
    \end{itemize}
    \textit{Для достижения целей работы были сформулированы следующие задачи:}
    \begin{itemize}[leftmargin=5.5mm]
        \renewcommand\labelitemi{--}
        \item доказать существование и единственность решения начально-краевой
        задачи для квазистационарной и квазилинейной моделей сложного теплообмена,
        разработать итерационный алгоритм нахождения решения и обосновать его
        сходимость;
        \item получить условия существования квазирешения обратной задачи с
        неизвестным коэффициентом отражения на границе, вывести условия
        оптимальности первого порядка и получить численный алгоритм;
        \item выполнить анализ оптимизационных методов решения задач сложного
        теплообмена с условиями Коши на границе для стационарной и
        квазистационарной моделей, исследовать разрешимость возникающих
        регуляризованных экстремальных задач, получить условия оптимальности,
        обосновать сходимость решений регуляризованных задач к решению задач с
        условиями Коши на границе при стремлении параметра регуляризации к
        нулю;
        \item провести теоретический анализ обратных экстремальных задач с фазовыми
        ограничениями для квазилинейной модели сложного теплообмена,
        рассмотреть аппроксимации задачами с штрафными функционалами и
        доказать сходимость их решений при увеличении штрафа;
        \item предложить численные методы решения рассматриваемых краевых и
        оптимизационных задач сложного теплообмена, адаптировать метод
        Ньютона и градиентные методы для их решения, реализовать полученные
        методы в виде программных систем, осуществить тестирование
        предложенных алгоритмов.
    \end{itemize}


    \textbf{Научная новизна.}
    В работе получены новые априорные оценки решений
    начально-краевых задач для квазистационарных и квазилинейных уравнений
    сложного теплообмена и доказана их нелокальная однозначная
    разрешимость.
    Для рассмотренных моделей сложного теплообмена
    рассмотрены новые постановки граничных обратных задач, предложены
    оптимизационные методы их решения.
    Выполнен теоретический анализ возникающих новых экстремальных задач.
    Представлены априорные оценки решений регуляризованных задач и впервые
    обоснована сходимость их решений к точным решениям обратных задач.
    Для решения задач с фазовыми ограничениями, предложены алгоритмы,
    основанные на аппроксимации экстремальными задачами со штрафом.
    Разработаны и протестированы новые алгоритмы решения прямых,
    обратных и экстремальных задач для моделей сложного теплообмена.


    \textbf{Теоретическая и практическая значимость.}
    Научная значимость результатов диссертации основана, с одной
    стороны, на решении открытых задач, связанных с корректностью моделей
    теплообмена, учитывающих тепловое излучение, теоретическим и
    численным анализом граничных обратных задач и задач оптимального
    управления.
    С другой стороны, развитие новых оптимизационных методов
    решения рассматриваемых нелинейных обратных задач является основой для
    анализа прикладных задач и задач проектирования систем с заданными
    экстремальными свойствами.
    Решение граничных обратных задач имеет практическое значение при
    выборе оптимальных параметров границы области, в которой происходит
    процесс сложного теплообмена.
    Задачи оптимизации имеют важное
    практическое применение при выборе параметров системы для получения
    необходимого распределения температуры или уровня теплового излучения.
    Необходимость выбора параметров системы возникает при проектировании
    инженерных установок, в которых присутствуют процессы сложного
    теплообмена.
    Теоретическая значимость работы обусловлена также тем, что
    результаты, связанные с корректностью рассматриваемых задач,
    сходимостью предлагаемых численных алгоритмов имеют нелокальный
    характер и не содержат нефизичных ограничений типа малости
    определенных параметров.
    Представленный комплекс программ имеет открытый характер и может
    дополняться для решения прикладных обратных задач сложного
    теплообмена.
    С помощью разработанного комплекса программ проведено численное
    моделирование процесса внутривенной лазерной абляции, изучено влияние
    различных факторов на распределения полей температуры и интенсивности
    излучения.


    \textbf{Методология и методы исследования.}
    В работе широко использовались методы математического и функционального анализа,
    теории дифференциальных уравнений в частных производных, теории экстремальных задач.
    Для разработки численных алгоритмов решения применялись методы
    вычислительной математики, объектно-ориентированное и функциональное
    программирование, методы оптимизации и другие.
    Методология исследования обратных задач, рассмотренных в работе,
    заключается в следующем.
    Предварительно выполняется теоретический анализ краевых или начально-краевых задач, моделирующих
    рассматриваемый процесс.
    Для построения оптимизационного алгоритма ставится экстремальная задача или задача
    оптимального управления для рассматриваемой модели сложного теплообмена и проводится ее
    теоретический анализ, включающий нахождение условий разрешимости и
    вывод условий оптимальности.
    Для регуляризованных задач обосновывается
    сходимость последовательности их решений.
    Далее, на основе полученных условий оптимальности
    строится градиентный метод решения обратной задачи.
    Полученные алгоритмы реализуются в виде комплекса программ и
    теоретические результаты тестируются на численных примерах,
    иллюстрирующих эффективность предложенных методов.


    \textbf{Положения, выносимые на защиту.}
    \textit{В области математического моделирования:}
    \newcounter{nameOfYourChoice}
    \begin{enumerate}[leftmargin=5.5mm]
        \item Доказательство однозначной разрешимости начально-краевой задачи,
        моделирующей квазистационарный сложный теплообмен в трехмерной
        области.
        \item Доказательство однозначной разрешимости квазилинейной начально-
        краевой задачи, моделирующей сложный теплообмен с нелинейной
        зависимостью коэффициента теплопроводности от температуры.
        \item Обоснование существования квазирешения обратной задачи с неизвестным
        коэффициентом отражения на части границы и условием переопределения на
        другой части.\ Вывод необходимых условий оптимальности.
        \item Вывод условий разрешимости экстремальных задач, аппроксимирующих
        решения граничных обратных задач (задач с условиями Коши на границе
        области) для стационарной и квазистационарной моделей сложного
        теплообмена.
        \item Построение систем оптимальности и доказательство их невырожденности
        для задач оптимального управления стационарными, квазистационарными и
        квазилинейными уравнениями сложного теплообмена.
        \setcounter{nameOfYourChoice}{\value{enumi}}
    \end{enumerate}
    \textit{В области численных методов:}
    \begin{enumerate}[leftmargin=5.5mm]
        \setcounter{enumi}{\value{nameOfYourChoice}}
        \item Разработка численного алгоритма решения квазилинейной начально-
        краевой задачи, моделирующей сложный теплообмен и доказательство его
        сходимости.
        \item Обоснование сходимости последовательности решений задач
        оптимального управления к решениям задач с условиями Коши на границе
        при стремлении параметра регуляризации к нулю.
        \item Обоснование сходимости алгоритма решения экстремальных обратных
        задач с ограничениями температурных полей методом штрафа.
        \setcounter{nameOfYourChoice}{\value{enumi}}
    \end{enumerate}
    \textit{В области комплексов программ:}
    \begin{enumerate}[leftmargin=5.5mm]
        \setcounter{enumi}{\value{nameOfYourChoice}}
        \item Разработка программ, реализующих численное моделирование процессов
        сложного теплообмена на основе метода конечных элементов.\ Реализация и
        тестирование оптимизационных алгоритмов решения граничных обратных
        задач для стационарных, квазистационарных и квазилинейных моделей.
    \end{enumerate}


    \textbf{Степень достоверности и апробация результатов.}
    Достоверность полученных в диссертации теоретических результатов основывается на
    использовании методов функционального анализа, дифференциальных
    уравнений, теории оптимального управления распределенными системами
    задач и строгих математических доказательствах.
    Достоверность результатов численного моделирования обеспечивается
    доказательством сходимости предложенных итерационных процессов
    и тестированием разработанного комплекса программ.

    Основные результаты диссертации докладывались и обсуждались на
    научных семинарах департамента математического и компьютерного моделирования
    ДВФУ, института прикладной математики ДВО РАН, института автоматики
    и процессов управления ДВО РАН и на следующих научных конференциях:
    \begin{itemize}[leftmargin=5.5mm]
        \item Региональная научно-практическая конференция студентов, аспирантов
        и молодых учёных по естественным наукам (Владивосток, 2018, 2019);
        \item Workshop on Computing Technologies and Applied Mathematics
        (Вычислительные технологии и прикладная математика, Владивосток, 2022);
        \item International Conference DAYS on DIFFRACTION (Санкт-Петербург, 2021, 2023);
        \item International Workshop on Mathematical Modeling and Scientific Computing (Мюнхен, 2020, 2022).
    \end{itemize}

    {\publications}
    Результаты диссертации опубликованы в 10 статьях, из них 4
    статьи~\cite{mesenev_23_opt, mesenev_22_penalty, mesenev_20_alg, mesenev_18_boundary}
    в «Дальневосточном математическом журнале», индексируемом
    в ядре РИНЦ и в международных базах научного цитирования (MathSciNet,
    zbMATH), 1 статья~\cite{mesenev_20_opt_proc} в трудах CEUR (Scopus),
    1 статья~\cite{mesenev_23_math}
    в «Journal of Physics: Conference Series» (Scopus),
    2 статьи~\cite{mesenev_21_optimal_proc, mesenev_23_inv_proc}
    в Proceedings of the International Conference Days on Diffraction (Web of Science, Scopus),
    2 статьи~\cite{mesenev_23_problem, Mesenev_22_analysis}
    в «Журнале вычислительной математики и математической физики»
    (Web of Science, Scopus, MathSciNet, zbMATH).

    Получено 3 свидетельства о регистрации программ для ЭВМ~\cite{progbib1, progbib2, progbib3}.

    \textbf{Личный вклад автора.}
    Результаты в области математического моделирования получены совместно с научным руководителем.
    Результаты в области численных методов и комплексов программ получены автором самостоятельно.
