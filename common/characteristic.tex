{\actuality}
Под сложным теплообменом понимают процесс распространения тепла,
в котором участвуют несколько видов переноса тепла – радиационный, кондуктивный, конвективный.
При чём в данном процессе радиационный перенос тепла занимает существенную роль при высоких температурах.
С математической точки зрения процесс сложного теплообмена моделируется системой из
дифференциального уравнения теплопроводности, а также интегро-дифференциального уравнения переноса излучения.


Решение уравнения переноса излучения является трудно вычислимой задачей из-за того,
что помимо временной и пространственной переменной также задействовано
векторное поле, задающее направление излучения.
В связи с этим для уравнения переноса излучения применяют ряд аппроксимаций,
в том числе диффузионное $P_1$ приближение, которое использует
усреднённая интенсивность излучения по всем направлениям.
Широко используемое $P_1$ приближение является частным случаем метода сферических
гармоник ($P_N$-приближения) и упрощенного метода сферических гармоник
($SP_N$-приближения, $SP_1$ эквивалентно $P_1$).


В классических прямых задачах сложного теплообмена задаются параметры системы, и по ним вычисляется
состояние системы – температурное поле и интенсивность теплового излучения.
Обратные задачи сложного теплообмена состоят в разыскании исходных параметров системы по некоторым
известным сведениям о температурном поле или интенсивности излучения.
Например, обратные задачи, связанные с теплопроводностью, обычно связаны с
оценкой неизвестного граничного теплового потока при известной температуре.


Отметим трудности, возникающие при решении обратных задач сложного теплообмена.


Эти задачи математически классифицируются как некорректные в общем смысле, из-за высокой нестабильности решений.
Как следствие, обратные задачи теплообмена долгое время не представляли физического интереса.
Появление в 50-х годах эвристических методов и в 60–70е годы методов оптимизации
позволило исправить проблемы некорректности исследуемых задач.
В основе таких методов лежит идея замены исходной задачи на задачу оптимизации
с использованием регуляризации, которая и позволяет преодолеть проблемы нестабильности решений.
Приведём некоторые пионерские работы А.Н. Тихонова, О.М. Олифанова, J.V.Beck и других, нашедших различные способы
преодоления неустойчивостей обратных задач.


Диссертация посвящена теоретическому анализу обратных стационарных
задач сложного теплообмена в трёхмерной области в рамках $P_1$-приближения
уравнения переноса излучения.
Теоретические результаты проиллюстрированы численными примерами.


Исследование математических моделей радиационного теплопереноса учитывающих одновременно
вклад эффектов теплопроводности и излучения даёт теоретическую основу для инженерных решений в различных областях,
таких как производство стекла~\cite{glass}, лазерная интерстициальная термотерапия~\cite{therapy}, и другие.
Главной особенностью данных процессов является существенное влияние излучения на теплообмен при высоких температурах.


\textbf{Степень разработанности темы исследования.}

Приведём работы [3, 4, 5, 6, 7, 8, 9, 10, 11, 12, 13], которые посвящены анализу разрешимости моделей сложного
теплообмена между телами, разделенными прозрачной средой.
Эти модели включают уравнение теплопроводности с нелинейным нелокальным краевым условием,
моделирующим тепловое излучение границы области и теплообмен излучением между частями границы,
в [4, 6, 7, 9, 11, 12, 13] рассмотрены стационарные модели, в [3, 5, 8, 9, 10, 11, 12] — нестационарные.

В следующих работах исследована разрешимость моделей сложного теплообмена в полупрозрачной среде,
в которых для моделирования радиационного теплообмена используется полное уравнение переноса излучения.
В [14, 15] доказана однозначная разрешимость одномерных стационарных задач радиационно-кондуктивного
теплообмена, в [16, 17, 18] доказана однозначная разрешимость трехмерных задач: в [18]
исследована стационарная модель, в [17] — нестационарная, в [16] — квазистационарная.
Под квазистационарными моделями сложного теплообмена понимаются модели, включающие нестационарное
уравнение теплопроводности и стационарное уравнение переноса излучения.

Отметим работы [19, 20, 21, 22, 23, 24], посвященные разработке численных методов для указанных моделей.
В работах А.А. Амосова [25, 26, 27, 28] доказана однозначная разрешимость стационарных и квазистационарных
моделей сложного теплообмена в системе полупрозрачных тел, где для описания распространения излучения используется
уравнение переноса излучения с краевыми условиями, моделирующими отражение и преломление излучения
на границах тел, также учитывается зависимость интенсивности излучения и оптических свойств тел от частоты
излучения: в [25, 26] ставились условия диффузного отражения и преломления излучения, в [27, 28] — условия отражения
и преломления излучения по законам Френеля.


Отметим следующие работы, посвященные решению обратных задач в
рамках моделей сложного теплообмена с полным уравнением переноса излучения.
В [29] проведен теоретический анализ задачи оптимального управления
источниками тепла в рамках квазистационарной модели сложного теплообмена, включающей полное уравнение
переноса излучения: доказана однозначная разрешимость прямой задачи,
доказана разрешимость задачи управления, и получены условия оптимальности.
В [30] разработан численный алгоритм решения задачи оптимального управления источниками тепла и излучения в рамках
стационарной модели сложного теплообмена с полным уравнением переноса излучения.
Работа [31] посвящена теоретическому и численному анализу обратной
задачи восстановления начального распределения температуры по известной
зависимости температуры на границе области от времени в рамках квазистационарной модели сложного теплообмена.


Отметим также работы [32, 33, 34, 11], посвященные анализу задач оптимального управления для стационарных
моделей сложного теплообмена в прозрачной среде, включающих уравнение теплопроводности с нелинейным
нелокальным краевым условием, моделирующим тепловое излучение границы
области и теплообмен излучением между частями границы, и работу [35], в которой построен численный
алгоритм решения задачи оптимального управления граничными коэффициентами в одномерной нестационарной модели.
Которая включает уравнение теплопроводности с нелинейным краевым условием, описывающее тепловое излучение границ.
Работы [36, 37, 38, 39] посвящены численному моделированию в рамках диффузионных моделей сложного теплообмена.
В [40] исследована схема метода конечных объемов для решения квазистационарной системы
уравнений сложного теплообмена на основе $P_1$-приближения уравнения переноса излучения.


Сравнение $P_1$-приближения с другими методами аппроксимации уравнения переноса излучения проводилось
в [41, 42, 24, 37, 43, 44].
Вывод и численный анализ нестационарного $P_1$-приближения выполнен в [45, 46, 47, 42, 44].
В работах R. Pinnau, O. Tse [48, 49] проведен теоретический
анализ квазистационарных моделей сложного теплообмена на основе $SP_1$ и $SP_3$-приближений.
Эти модели включают уравнение теплопроводности, стационарное $SP_N$ -приближение,
а также в [49] уравнения Навье – Стокса в приближении Буссинеска.


В [48] доказаны существование, единственность и ограниченность решения задачи сложного
теплообмена на основе $P_1$-приближения без источников
тепла и излучения, в [49] доказана однозначная разрешимость задачи свободной
конвекции с радиационным теплообменом на основе $SP_3$-приближения в двумерной области,
в этой модели присутствуют источники тепла с ограниченной плотностью.
В работах А.Е. Ковтанюка, А.Ю.Чеботарева [50, 51, 52] доказана
однозначная разрешимость краевых задач для стационарных моделей сложного теплообмена
на основе $P_1$-приближения, доказана сходимость метода простой итерации нахождения решения.


Отметим, что численная реализация данного метода затруднена, поскольку на каждой итерации необходимо решить
нелинейное эллиптическое уравнение.
В [53] доказана однозначная разрешимость сходной субдифференциальной краевой задачи с многозначной
зависимостью коэффициента излучения границы от интенсивности излучения.
В [54, 55] получены результаты о существовании и единственности решений обратных задач для стационарной диффузионной
модели сложного теплообмена, которые заключаются в нахождении неизвестной плотности источников тепла в виде линейной
комбинации заданных функционалов при известных значениях этих функционалов на решении краевой задачи.
Работы R. Pinnau, O. Tse [48, 49] посвящены теоретическому анализу задач оптимального управления температурой на границе
области в рамках квазистационарных моделей сложного теплообмена на основе
$SP_N$ - приближений: доказана разрешимость задач управления, найдены необходимые условия оптимальности.
В [56, 57, 47, 58, 59, 60] разработаны численные методы решения задач оптимального управления граничной температурой
для квазистационарной модели сложного теплообмена на основе $P_1$-приближения,
при этом в [56, 57, 58] использовалась модель с учетом зависимости коэффициента поглощения от частоты излучения:
в [47, 59] минимизировалось отклонение поля температуры от заданного и для решения задачи оптимизации применялся метод Ньютона,
в [58] для решения задачи минимизации отклонения поля температуры от заданного применялся метод проекции градиента, в [56,
60] минимизировалась норма градиента температуры и для решения задачи
оптимизации применялся метод проекции градиента.
В [57] решалась задача минимизации отклонения поля температуры от заданного на основе серии из
трех моделей, аппроксимирующих уравнение переноса излучения с разной точностью,
использовался оптимизационный метод второго порядка


В [61] проведен теоретический анализ задачи оптимального управления
температурой на границе области в рамках стационарной диффузионной модели
сложного теплообмена, для численно горешения задачи управления применен
метод проекции градиента.
В работах А.Е.Ковтанюка, А.Ю.Чеботарева и др. [50, 62, 63, 64] исследованы задачи оптимального управления коэффициентом
излучения границы области в рамках стационарной модели сложного теплообмена на основе $P_1$-приближения.
В [50, 64] выведены необходимые условия оптимальности для задачи максимизации выходящей из среды энергии,
доказана разрешимость задачи управления и получены достаточные условия регулярности системы оптимальности,
которые выполняются при достаточно большой скорости движения среды и малых размерах области.
В [62, 63] получены достаточные условия оптимальности для задач максимизации и минимизации полей температуры и излучения
во всей области теплообмена, доказана сходимость метода простой итерации нахождения оптимального управления,
эти исследования были выполнены позже аналогичной работы автора для нестационарной модели.


Таким образом, ряд важных задач, относящихся к моделированию и оптимизации сложного теплообмена на основе
диффузионного приближения, оставался нерешенным: исследование разрешимости нестационарной задачи сложного
теплообмена с источниками тепла и излучения и нестационарной задачи свободной конвекции с радиационным
теплообменом в трехмерной области, исследование устойчивости по Ляпунову стационарных решений,
вывод диффузионной модели сложного теплообмена в многослойной среде, анализ сходимости метода Ньютона
для уравнений сложного теплообмена, разработка численных методов решения задач оптимального управления
коэффициентом излучения границы области в рамках нестационарных
моделей сложного теплообмена, доказательство регулярности условий оптимальности для задачи
оптимального управления коэффициентом излучения границы в рамках стационарной модели.


\textbf{Цели и задачи диссертационной работы.}
Цели работы - теоретическое исследование разрешимости обратных стационарных задач сложной теплопроводности,
разработка численных методов решения исследуемых краевых задач, а также задач оптимального управления.
Разработка вычислительных программ для постановки численных экспериментов и демонстрации результатов расчётов.
% TODO: !!!
Перед началом работы были поставлены следующие задачи:
\begin{itemize}
  \item[--] Исследовать разрешимость задачи по нахождению коэффициента отражения участка границы для
  стационарной модели, по дополнительной информации о температурном поле.
  \item[--] Разработать численный метод по нахождению решения для соответствующей экстремальной задачи.
  \item[--] Исследовать стационарную задачу оптимального управления для уравнений радиационно-кондуктивного
  теплообмена в трехмерной области в рамках $P_1$–приближения уравнения переноса излучения.
  \item[--] Результаты теоретического анализа проиллюстрировать численными примерами.
\end{itemize}


\textbf{Научная новизна.}
Результатом работы является теоретический анализ
разрешимости обратных задач сложного теплообмена.
Доказано существование квазирешения для первой рассматриваемой задачи.
Реализован алгоритм градиентного спуска для решения экстремальной задачи и представлены результаты
численных экспериментов.
Далее показано, что последовательность решений экстремальных задач сходится к решению
краевой задачи с условиями типа Коши для температуры.
Результаты теоретического анализа также проиллюстрированы численными примерами.

\textbf{Теоретическая и практическая значимость.}
Исследование однозначной разрешимости экстремальных задач, а также задач оптимального
управления крайне важно при реализации численных алгоритмов и позволяет судить об адекватности полученных решений.

Задачи оптимизации имеют крайне важное практическое применение при выборе параметров системы для
получения желаемой температуры или теплового излучения.
Необходимость выбора параметров системы возникает при проектировании инженерных установок
в которых присутствуют процессы сложного теплообмена.

Разработанные комплексы программ служат практическим подтверждением теоретических результатов,
а также могут быть использованы в качестве примеров для решения подобных задач.


Научная значимость данной работы состоит в теоретическом вкладе в исследования корректности и
разрешимости задач сложного теплообмена.
Реализация конкретных методов решения проблем оптимального управления, в свою очередь, имеет высокую
значимость для решения прикладных инженерных задач по проектированию тепловых установок с
заданными температурными свойствами.


\textbf{Методология и методы исследования.}
В работе широко использовались методы математического и функционального анализа, теории дифференциальных
уравнений в частных производных, теории экстремальных задач.
Для разработки численных алгоритмов решения применялись методы вычислительной математики,
объектно-ориентированное и функциональное программирование, методы оптимизации и другие.


\textbf{Положения, выносимые на защиту.}
В области математического моделирования
% TODO: !!!
\begin{itemize}
    \item Разрешимость экстремальной задачи для стандартной модели радиационно-диффузионного теплообмена
\end{itemize}


\textbf{Степень достоверности и апробация результатов.}
Теоретические результаты, представленные в диссертации получены
использованием методов функционального анализа, теорий
дифференциальных уравнений и экстремальных задач.
Теоремы имеют строгие математические доказательства.
Достоверность численных экспериментов обеспечивается согласованностью
с теоретическими результатами, доказательством
сходимости итерационных процессов и тестированием разработанного программного обеспечения.

\textbf{Публикации.}
Результаты диссертационного исследования опубликованы в пяти статьях [65, 66] в изданиях, рекомендованных ВАК.


\textbf{Личный вклад автора.}
Результаты в области математического моделирования получены совместно с научным руководителем.
В области численных методов и комплексов программ результаты получены автором самостоятельно.
