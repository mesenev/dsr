    {\actuality}
    Под сложным теплообменом понимают
    процесс распространения тепла излучением, теплопроводностью и
    конвекцией.
    При этом важно, что радиационный перенос тепла дает
    существенный вклад в распределение температурного поля.
    Для многих инженерных приложений моделирование процессов теплопередачи имеет
    первостепенное значение (например, это относится к радиационному
    теплообмену в полупрозрачных материалах).
    Температура – важнейший параметр практически во всех производственных этапах.
    Следовательно, знание температурных полей необходимо для контроля производственных
    процессов и качества результата.

    С математической точки зрения процесс сложного теплообмена
    моделируется системой уравнений с частными производными, включающей
    уравнение теплопроводности, а также интегро-дифференциальное уравнение
    переноса теплового излучения~\cite{Ozisik1976, Sparrow1971, howell2010thermal, modest2013radiative}.
    Теоретический и численный анализ различных задач для такой системы является весьма
    затратным поскольку интенсивность излучения зависит не только от
    пространственной и временной переменных, но и от направления
    распространения излучения.
    Имеется ряд аппроксимаций уравнения переноса излучения, в том числе диффузионное
    $P_1$-приближение, в котором интенсивность излучения усредняется по направлениям.
    $P_1$-приближение является частным случаем метода сферических гармоник ($PN$-приближения)
    и упрощенного метода сферических гармоник ($SPN$-приближения).
    В данной работе рассмотрены модели, основанные на $P_1$-приближении.


    В классических прямых задачах сложного теплообмена задаются
    параметры системы, и по ним вычисляется состояние системы –
    температурное поле и интенсивность теплового излучения.
    Обратные задачи сложного теплообмена состоят в разыскании неизвестных параметров
    системы по некоторой дополнительной информации (условия переопределения)
    о температурном поле или интенсивности излучения.
    Методы, основанные на решении обратных задач, продолжают оставаться
    актуальным, быстро развивающимся направлением исследований.
    Отметим трудности, возникающие при решении обратных задач сложного
    теплообмена.


    Эти задачи математически классифицируются как некорректные в общем
    смысле из-за неустойчивости решений.
    Развитие методов оптимизации позволяет исправить проблемы некорректности исследуемых задач.
    В основе таких методов лежит идея замены исходной задачи на задачу оптимизации и
    фактически использует методику А.Н. Тихонова регуляризации задачи.

    Диссертация посвящена теоретическому и численному анализу одному из
    наиболее распространенных на практике типу обратных задач – граничных
    обратных задач сложного теплообмена в трёхмерной области в рамках
    $P_1$-приближения уравнения переноса излучения.
    При этом основное внимание уделяется разработке
    и обоснованию оптимизационных методов решения указанных задач.
    Исследование обратных задач для математических моделей
    радиационного теплопереноса, учитывающих одновременно вклад эффектов
    теплопроводности и излучения, даёт теоретическую основу для эффективных
    инженерных решений в различных областях, таких как производство стекла,
    лазерная термотерапия и другие.


    \textbf{Степень разработанности темы исследования.}
    Имеется значительное число работ, посвященных теоретическому анализу моделей сложного
    теплообмена.
    Результаты А.\ А.\ Амосова, M.\ Laitinen, T.\ Tiihonen, P.-E.\ Druet и др.~\cite{
        Amosov2005,
        Amosov2009,
        Amosov2010nonstationary,
        amosov2010stationary,
        druet2009weak,
        druet2010weak,
        laitinen2001conductive,
        metzger1999existence,
        Philip2010,
        Tiihonen1997a,
        Tiihonen1997b,
        amosov17,
        amosov18,
        Amosov20,
        Amosov21-1,
        Amosov2021-2,
        Amosov23} %[Гренкин 1, 2, 30, 31, 56, 57, 80, 86, 95, 119, 120 Добавить свежие статьи Амосова]
    посвящены анализу разрешимости моделей сложного теплообмена,
    которые включают уравнение теплопроводности с нелинейным нелокальным
    краевым условием, моделирующим тепловое излучение границы области и
    теплообмен излучением между частями границы,
    в~\cite{Amosov2009, amosov2010stationary, druet2009weak, laitinen2001conductive,
        Philip2010, Tiihonen1997a, Tiihonen1997b} %[2, 31, 56, 80, 95, 119, 120]
    рассмотрены стационарные модели,
    в~\cite{
        Amosov2005, Amosov2010nonstationary,
        druet2010weak, laitinen2001conductive,
        metzger1999existence,
        Philip2010,
        Tiihonen1997a} — нестационарные.%[1, 30, 57, 80, 86, 95, 119]

    Анализ моделей сложного теплообмена в полупрозрачной среде, в
    которых для моделирования радиационного теплообмена используется
    полное уравнение переноса излучения представлен в следующих статьях.
    В~\cite{asllanaj2003existence, kelley1996existence} доказана
    однозначная разрешимость одномерных стационарных
    задач радиационно-кондуктивного теплообмена,
    в~\cite{ghattassi2018existence, Porzio2004, Thompson2004} доказана
    однозначная разрешимость трехмерных задач: в~\cite{Thompson2004} исследована
    стационарная модель, в~\cite{Porzio2004} -- нестационарная,
    в~\cite{ghattassi2018existence} -- квазистационарная.
    Под квазистационарными моделями сложного теплообмена понимаются
    модели, включающие нестационарное уравнение теплопроводности и
    стационарное уравнение переноса излучения.
    В работах R.\ Pinnau, O.\ Tse~\cite{Pinnau2007, Pinnau2013}
    проведен теоретический анализ квазистационарных моделей сложного
    теплообмена на основе $SP_1$- и $SP_3$-приближений.
    Эти модели включают уравнение теплопроводности, стационарное $SP_N$-приближение, а также
    в~\cite{Pinnau2013} уравнения Навье–Стокса в приближении Буссинеска.
    В работах Г.В.\ Гренкина, А.Е.\ Ковтанюка,
    А.\ Ю.\ Чеботарева~\cite{Kovtanyuk2014, Kovtanyuk2016, Chebotarev22, Pak23}
    доказана однозначная разрешимость краевых задач для моделей
    сложного теплообмена на основе $P_1$-приближения, доказана сходимость
    метода простой итерации нахождения решения.
    В~\cite{Chebotarev2016Odnaznachnaya} доказана однозначная
    разрешимость субдифференциальной краевой задачи с многозначной
    зависимостью коэффициента излучения границы от интенсивности
    излучения.

    Подробный обзор различных численных методов решения задач сложного
    теплообмена можно найти в~\cite{modest2013radiative}.
    Большинство методов можно разделить на четыре класса:
    \begin{itemize}
        \item Методы диффузионной аппроксимации.
        Типичным примером является приближение Розеланда~\cite{farina2011mathematical, Siedow2011}.
        \item Методы, в которых зависимость теплового излучения от направления
        выражается рядом специальных базисных функций.
        Представителями методов являются $PN$ и $SPN$-аппроксимации.
        \item Методы, использующие дискретизацию угловой зависимости, например
        метод дискретных ординат, который является наиболее популярным.
        \item Методы Монте-Карло.
    \end{itemize}


    Статьи~\cite{kovtanyuk2013iterative, Thommes2002, Pinnau2008, Siewert1991}
    посвящены численному моделированию в рамках
    диффузионных моделей сложного теплообмена, в~\cite{gallouet2016analysis} исследована схема
    метода конечных объемов для решения квазистационарной системы
    уравнений сложного теплообмена на основе
    $P_1$-приближения уравнения переноса излучения.
    Сравнение $P_1$-приближения с другими методами
    аппроксимации уравнения переноса излучения проводилось
    в~\cite{modest2014elliptic, frank2011adaptive, kovtanyuk2012, Thommes2002, Larsen2002, Frank2007}.
    Численный анализ нестационарного $P_1$-приближения выполнен
    в~\cite{Addam2015, olbrant2013asymptotic, frank2010optimal, frank2011adaptive, Frank2007}.

    Основополагающие работы А.Н. Тихонова~\cite{TikhonovSamarskii1972} и его коллег, которые
    исследовали обратные задачи для уравнений математической физики, стали
    отправной точкой для разработки методов преодоления неустойчивостей в обратных задачах.
    О.\ М.\ Алифанов~\cite{Aliphanov77, Aliphanov99, Aliphanov2009}
    в своих работах, посвященных обратным задачам сложного теплообмена, также
    сделал значительный вклад в развитие этой области.
    J.\ V.\ Beck~\cite{Beck1985-fg} предложил
    новые подходы к решению обратных задач, основанных на методах
    оптимизации и статистическом анализе.
    Отметим также серьезный теоретический анализ обратных задач тепломассопереноса
    представленный в работах С.\ Г.\ Пяткова и его учеников~\cite{Pyatkov19, Pyatkov22, Pyatkov23}.


    Решение обратных задач оптимизационным методом можно
    рассматривать как соответствующую задачу оптимального управления
    системой, моделируемой уравнениями сложного теплообмена.
    Обратно, каждую задачу оптимального управления можно интерпретировать как
    обратную экстремальную задачу.
    Отметим следующие работы, посвященные
    решению обратных задач в рамках моделей сложного теплообмена с полным
    уравнением переноса излучения.
    В~\cite{end2011analytical} проведен теоретический анализ
    задачи оптимального управления источниками тепла в рамках
    квазистационарной модели сложного теплообмена, включающей полное
    уравнение переноса излучения.
    В~\cite{end2010optimization} разработан численный алгоритм
    решения задачи оптимального управления источниками тепла и излучения в
    рамках стационарной модели сложного теплообмена с полным уравнением
    переноса излучения.
    Работа~\cite{Pereverzyev2008} посвящена теоретическому и численному
    анализу обратной задачи восстановления начального распределения
    температуры по известной зависимости температуры на границе области от
    времени в рамках квазистационарной модели сложного теплообмена.
    Отметим также работы~\cite{birgelis2003optimal, meyer2006optimal, meyer2009state, Philip2010},
%    [48, 87, 88, 95],
    посвященные анализу задач оптимального управления для стационарных
    моделей сложного теплообмена в прозрачной среде, включающих уравнение теплопроводности с
    нелинейным нелокальным краевым условием, моделирующим тепловое
    излучение границы области и теплообмен излучением между частями
    границы, и работу~\cite{belmiloudi2014nonlinear},
    в которой построен численный алгоритм решения
    задачи оптимального управления граничными коэффициентами
    в одномерной нестационарной модели, включающей уравнение
    теплопроводности с нелинейным краевым условием, которое описывает
    тепловое излучение границ.
    В~\cite{clever2012optimal, clever2014model, frank2010optimal, lang2005adaptive, Pinnau2007b, Pinnau2004}
%    [50, 51, 64, 81, 98, 100]
    разработаны численные
    методы решения задач оптимального управления граничной температурой
    для квазистационарной модели сложного теплообмена на основе $P_1$-приближения,
    при этом в~\cite{clever2012optimal, clever2014model, lang2005adaptive} использовалась
    модель с учетом зависимости коэффициента поглощения от частоты излучения:
    в~\cite{frank2010optimal, Pinnau2007b}
    минимизировалось отклонение поля температуры от заданного и для
    решения задачи оптимизации применялся метод Ньютона, в~\cite{lang2005adaptive} для решения
    задачи минимизации отклонения поля температуры от заданного применялся
    метод проекции градиента, в~\cite{clever2012optimal, Pinnau2004} минимизировалась норма градиента
    температуры и для решения задачи оптимизации применялся метод проекции
    градиента, в~\cite{clever2014model} решалась задача минимизации отклонения поля
    температуры от заданного на основе серии из трех моделей,
    аппроксимирующих уравнение переноса излучения с разной точностью,
    использовался оптимизационный метод второго порядка.
    В~\cite{Pinnau2004} проведен
    теоретический анализ задачи оптимального управления температурой на
    границе области в рамках стационарной диффузионной модели сложного
    теплообмена, для численного решения задачи управления применен метод
    проекции градиента.
    В работах А.\ Е.\ Ковтанюка, А.\ Ю.\ Чеботарева и др.\ \cite{Kovtanyuk2014,
        Mesenev2018, astrakhantseva2017design, Chebotarev2015,
        Kovtanyuk2014TheoreticalAnalysis, Chebotarev23nonhomogeneus, Chebotarev23optimal}.
    исследованы задачи оптимального
    управления коэффициентом излучения границы области в рамках
    стационарной модели сложного теплообмена на основе $P_1$-приближения.


    В последние годы исследования моделей сложного теплообмена стали
    особенно актуальными в связи с практическими приложениями.
    Ниже приведены некоторые примеры исследований практической применимости
    рассматриваемых моделей.
    В работах~\cite{clever2012optimal, clever2014model,
        frank2010optimal, klar2005, lang2005adaptive, farina2011mathematical,
        Thommes2002, Pinnau2007b, Pinnau2004, Seaid2007, Larsen2002}
    модели сложного теплообмена на основе -приближений
    применялись для моделирования и оптимизации сложного теплообмена при
    производстве стекла, в~\cite{frank2004comparison, seaid2004efficient, Seaid2005}
    моделировался сложный теплообмен в
    камерах сгорания газовых турбин, в~\cite{Backofen2004} диффузионные модели
    использовались для моделирования переноса теплового излучения в
    растущем кристалле.
    Также $P_1$-приближение применялось в составе моделей
    лазерной термотерапии~\cite{Dombrovskii2015, Tse2012, Hubner2017}.
    Процедура внутривенной лазерной
    абляции (ВВЛА) является безопасной и эффективной в лечении варикозных
    вен~\cite{Endovenous_vandenBos2009}.
    Математическое моделирование лучевых и тепловых процессов
    при ВВЛА имеет решающее значение для определения оптимальных
    параметров излучения, обеспечивающих достаточно высокие температуры
    внутри вены для успешной облитерации, обеспечивая при этом сохранность
    окружающих тканей.
    Результаты численного моделирования для различных
    длин волн и диаметров вен обсуждались в ряде
    работ~\cite{van2014optical, Some_Poluektova2014, Endovenous_Malskat2014, Mathematical_Mordon2006}.
    Задачи оптимального управления для модели ВВЛА рассматриваются
    в~\cite{Optimal_Kovtanyuk2020, Inverse_Kovtanyuk2021}.
    В~\cite{Optimal_Kovtanyuk2020} поставлена задача оптимального
    управления для модели реакция-диффузия,
    описывающей процедуру ВВЛА, которая заключается в
    аппроксимации заданного температурного профиля в определенной точке
    области модели.
    В~\cite{Inverse_Kovtanyuk2021} изучается аналогичная задача оптимального управления.
    Здесь целевой функционал выбран таким образом, что его
    минимизация позволяет достичь заданного распределения температуры в
    различных частях области модели.


    Несмотря на представленный в обзоре значительный объем исследований,
    включающий анализ обратных и обратных экстремальных задач, ряд важных
    задач, связанных с анализом корректности стационарных,
    квазистационарных и квазилинейных \textit{\textbf{моделей сложного теплообмена}},
    корректности постановок поиска квазирешений граничных обратных задач,
    \textit{\textbf{построением и обоснованием сходимости}} оптимизационных алгоритмов
    решения обратных задач и задач с краевыми условиями Коши, а также с
    \textit{\textbf{разработкой комплекса программ для проведения вычислительных экспериментов}}
    и тестирования предложенных алгоритмов оставался нерешенным.
    Настоящая работа посвящена решению указанных проблем.


    \textbf{Цели и задачи диссертационной работы.}
    \textit{Целью работы} является является теоретический и численный анализ граничных
    обратных задач, включая задачи с условиями типа Коши на границе области,
    и задач оптимального управления для моделей сложного теплообмена на
    основе $P_1$-приближения уравнения переноса излучения, а именно:
    \begin{itemize}[leftmargin=5.5mm]
        \renewcommand\labelitemi{--}
        \item исследование разрешимости краевых и начально-краевых задач для
        квазистационарных и квазилинейных моделей сложного теплообмена;
        \item разработка оптимизационных алгоритмов решения обратных задач и задач
        с краевыми условиями Коши, теоретический анализ и обоснование их
        сходимости;
        \item разработка комплекса программ для проведения вычислительных
        экспериментов и тестирования предложенных алгоритмов.
    \end{itemize}
    \textit{Для достижения целей работы были сформулированы следующие задачи:}
    \begin{itemize}[leftmargin=5.5mm]
        \renewcommand\labelitemi{--}
        \item доказать существование и единственность решения начально-краевой
        задачи для квазистационарной и квазилинейной моделей сложного теплообмена,
        разработать итерационный алгоритм нахождения решения и обосновать его
        сходимость;
        \item получить условия существования квазирешения обратной задачи с
        неизвестным коэффициентом отражения на границе, вывести условия
        оптимальности первого порядка и получить численный алгоритм;
        \item выполнить анализ оптимизационных методов решения задач сложного
        теплообмена с условиями Коши на границе для стационарной и
        квазистационарной моделей, исследовать разрешимость возникающих
        регуляризованных экстремальных задач, получить условия оптимальности,
        обосновать сходимость решений регуляризованных задач к решению задач с
        условиями Коши на границе при стремлении параметра регуляризации к
        нулю;
        \item провести теоретический анализ обратных экстремальных задач с фазовыми
        ограничениями для квазилинейной модели сложного теплообмена,
        рассмотреть аппроксимации задачами с штрафными функционалами и
        доказать сходимость их решений при увеличении штрафа;
        \item предложить численные методы решения рассматриваемых краевых и
        оптимизационных задач сложного теплообмена, адаптировать метод
        Ньютона и градиентные методы для их решения, реализовать полученные
        методы в виде программных систем, осуществить тестирование
        предложенных алгоритмов используя данные реальных сред и материалов.
    \end{itemize}


    \textbf{Научная новизна.}
    В работе получены новые априорные оценки решений
    начально-краевых задач для квазистационарных и квазилинейных уравнений
    сложного теплообмена и доказана их нелокальная однозначная
    разрешимость.
    Для рассмотренных моделей сложного теплообмена
    рассмотрены новые постановки граничных обратных задач, предложены
    оптимизационные методы их решения.
    Выполнен теоретический анализ возникающих новых экстремальных задач.
    Представлены априорные оценки решений регуляризованных задач и впервые
    обоснована сходимость их решений к точным решениям обратных задач.
    Для решения задач с фазовыми ограничениями, предложены алгоритмы,
    основанные на аппроксимации экстремальными задачами со штрафом.
    Разработаны и протестированы новые алгоритмы решения прямых,
    обратных и экстремальных задач для моделей сложного теплообмена.


    \textbf{Теоретическая и практическая значимость.}
    Научная значимость результатов диссертации основана, с одной
    стороны, на решении открытых задач, связанных с корректностью моделей
    теплообмена, учитывающих тепловое излучение, теоретическим и
    численным анализом граничных обратных задач и задач оптимального
    управления.
    С другой стороны, развитие новых оптимизационных методов
    решения рассматриваемых нелинейных обратных задач является основой для
    анализа прикладных задач и задач проектирования систем с заданными
    экстремальными свойствами.
    Решение граничных обратных задач имеет практическое значение при
    выборе оптимальных параметров границы области, в которой происходит
    процесс сложного теплообмена.
    Задачи оптимизации имеют важное
    практическое применение при выборе параметров системы для получения
    необходимого распределения температуры или уровня теплового излучения.
    Необходимость выбора параметров системы возникает при проектировании
    инженерных установок, в которых присутствуют процессы сложного
    теплообмена.
    Теоретическая значимость работы обусловлена также тем, что
    результаты, связанные с корректностью рассматриваемых задач,
    сходимостью предлагаемых численных алгоритмов имеют нелокальный
    характер и не содержат нефизичных ограничений типа малости
    определенных параметров.
    Представленный комплекс программ имеет открытый характер и может
    дополняться для решения прикладных обратных задач сложного
    теплообмена.
    С помощью разработанного комплекса программ проведено численное
    моделирование процесса внутривенной лазерной абляции, изучено влияние
    различных факторов на распределения полей температуры и интенсивности
    излучения.


    \textbf{Методология и методы исследования.}
    В работе широко использовались методы математического и функционального анализа,
    теории дифференциальных уравнений в частных производных, теории экстремальных задач.
    Для разработки численных алгоритмов решения применялись методы
    вычислительной математики, объектно-ориентированное и функциональное
    программирование, методы оптимизации и другие.
    Методология исследования обратных задач, рассмотренных в работе,
    заключается в следующем.
    Предварительно выполняется теоретический анализ краевых или начально-краевых задач, моделирующих
    рассматриваемый процесс.
    Для построения оптимизационного алгоритма ставится экстремальная задача или задача
    оптимального управления для рассматриваемой модели сложного теплообмена и проводится ее
    теоретический анализ, включающий нахождение условий разрешимости и
    вывод условий оптимальности.
    Для регуляризованных задач обосновывается
    сходимость последовательности их решений.
    Далее, на основе полученных условий оптимальности
    строится градиентный метод решения обратной задачи.
    Полученные алгоритмы реализуются в виде комплекса программ и
    теоретические результаты тестируются на численных примерах,
    иллюстрирующих эффективность предложенных методов.


    \textbf{Положения, выносимые на защиту.}

    \textit{В области математического моделирования:}

    \newcounter{nameOfYourChoice}
    \begin{enumerate}[leftmargin=5.5mm]
        \item Доказательство однозначной разрешимости начально-краевой задачи,
        моделирующей квазистационарный сложный теплообмен в трехмерной
        области.
        \item Доказательство однозначной разрешимости квазилинейной начально-
        краевой задачи, моделирующей сложный теплообмен с нелинейной
        зависимостью коэффициента теплопроводности от температуры.
        \item Обоснование существования квазирешения обратной задачи с неизвестным
        коэффициентом отражения на части границы и условием переопределения на
        другой части.\ Вывод необходимых условий оптимальности.
        \item Вывод условий разрешимости экстремальных задач, аппроксимирующих
        решения граничных обратных задач (задач с условиями Коши на границе
        области) для стационарной и квазистационарной моделей сложного
        теплообмена.
        \item Построение систем оптимальности и доказательство их невырожденности
        для задач оптимального управления стационарными, квазистационарными и
        квазилинейными уравнениями сложного теплообмена.
        \setcounter{nameOfYourChoice}{\value{enumi}}
    \end{enumerate}

    \textit{В области численных методов:}

    \begin{enumerate}[leftmargin=5.5mm]
        \setcounter{enumi}{\value{nameOfYourChoice}}
        \item Разработка численного алгоритма решения квазилинейной начально-
        краевой задачи, моделирующей сложный теплообмен и доказательство его
        сходимости.
        \item Обоснование сходимости последовательности решений задач
        оптимального управления к решениям задач с условиями Коши на границе
        при стремлении параметра регуляризации к нулю.
        \item Обоснование сходимости алгоритма решения экстремальных обратных
        задач с ограничениями температурных полей методом штрафа.
        \setcounter{nameOfYourChoice}{\value{enumi}}
    \end{enumerate}

    \textit{В области комплексов программ:}

    \begin{enumerate}[leftmargin=5.5mm]
        \setcounter{enumi}{\value{nameOfYourChoice}}
        \item Разработка программ, реализующих численное моделирование процессов
        сложного теплообмена на основе метода конечных элементов.\ Реализация и
        тестирование оптимизационных алгоритмов решения граничных обратных
        задач для стационарных, квазистационарных и квазилинейных моделей.
    \end{enumerate}


    \textbf{Степень достоверности и апробация результатов.}
    Достоверность полученных в диссертации теоретических результатов основывается на
    использовании методов функционального анализа, дифференциальных
    уравнений, теории оптимального управления распределенными системами
    задач и строгих математических доказательствах.
    Достоверность результатов численного моделирования обеспечивается
    доказательством сходимости предложенных итерационных процессов
    и тестированием разработанного комплекса программ.

    Основные результаты диссертации докладывались и обсуждались на
    научных семинарах департамента математического и компьютерного моделирования
    ДВФУ, института прикладной математики ДВО РАН, института автоматики
    и процессов управления ДВО РАН и на следующих научных конференциях:
    \begin{itemize}[leftmargin=5.5mm]
        \item Региональная научно-практическая конференция студентов, аспирантов
        и молодых учёных по естественным наукам (Владивосток, 2018, 2019);
        \item Workshop on Computing Technologies and Applied Mathematics
        (Вычислительные технологии и прикладная математика, Владивосток, 2022);
        \item International Conference DAYS on DIFFRACTION (Санкт-Петербург, 2021, 2023);
        \item International Workshop on Mathematical Modeling and Scientific Computing (Мюнхен, 2020, 2022).
    \end{itemize}

    {\publications}
    Результаты диссертации опубликованы в 10 статьях, из них 4
    статьи~\cite{mesenev_23_opt, mesenev_22_penalty, mesenev_20_alg, mesenev_18_boundary}
    в «Дальневосточном математическом журнале», индексируемом
    в ядре РИНЦ и в международных базах научного цитирования (MathSciNet,
    zbMATH), 1 статья~\cite{mesenev_20_opt_proc} в трудах CEUR (Scopus),
    1 статья~\cite{mesenev_23_math}
    в «Journal of Physics: Conference Series» (Scopus),
    2 статьи~\cite{mesenev_21_optimal_proc, mesenev_23_inv_proc}
    в Proceedings of the International Conference Days on Diffraction (Web of Science, Scopus),
    2 статьи~\cite{mesenev_23_problem, Mesenev_22_analysis}
    в «Журнале вычислительной математики и математической физики»
    (Web of Science, Scopus, MathSciNet, zbMATH).

    Получено 3 свидетельства о регистрации программ для ЭВМ~\cite{progbib1, progbib2, progbib3}.

    \textbf{Личный вклад автора.}
    Результаты в области математического моделирования получены совместно с научным руководителем.
    Результаты в области численных методов и комплексов программ получены автором самостоятельно.
