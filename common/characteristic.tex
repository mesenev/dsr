{\actuality}
Под сложным теплообменом понимают процесс распространения тепла,
в котором участвуют несколько видов переноса тепла – радиационный, кондуктивный, конвективный.
При чём в данном процессе радиационный перенос тепла занимает существенную роль при высоких температурах.
С математической точки зрения процесс сложного теплообмена моделируется системой из
дифференциального уравнения теплопроводности, а также интегро-дифференциального
уравнения переноса излучения.


Решение уравнения переноса излучения является трудно вычислимой задачей из-за того,
что помимо временной и пространственной переменной также задействовано
векторное поле, задающее направление излучения.
В связи с этим для уравнения переноса излучения применяют ряд аппроксимаций,
в том числе диффузионное $P_1$ приближение, которое использует
усреднённую интенсивность излучения по всем направлениям.
Широко используемое $P_1$ приближение является частным случаем метода сферических
гармоник ($P_N$-приближения) и упрощенного метода сферических гармоник
($SP_N$-приближения, $SP_1$ эквивалентно $P_1$).


В классических прямых задачах сложного теплообмена задаются параметры системы, и по ним вычисляется
состояние системы – температурное поле и интенсивность теплового излучения.
Обратные задачи сложного теплообмена состоят в разыскании исходных параметров системы по некоторым
известным сведениям о температурном поле или интенсивности излучения.
Например, обратные задачи, связанные с теплопроводностью, обычно связаны с
оценкой неизвестного граничного теплового потока при известной температуре.


Отметим трудности, возникающие при решении обратных задач сложного теплообмена.


Эти задачи математически классифицируются как некорректные в общем смысле,
из-за высокой нестабильности решений.
Как следствие, обратные задачи теплообмена долгое время не представляли физического интереса.
Появление в 50-х годах эвристических методов и в 60–70е годы методов оптимизации
позволило исправить проблемы некорректности исследуемых задач.
В основе таких методов лежит идея замены исходной задачи на задачу оптимизации
с использованием регуляризации, которая и позволяет преодолеть проблемы нестабильности решений.



Диссертация посвящена теоретическому анализу обратных стационарных
задач сложного теплообмена в трёхмерной области в рамках $P_1$-приближения
уравнения переноса излучения.
Теоретические результаты проиллюстрированы численными примерами.


Исследование математических моделей радиационного теплопереноса учитывающих одновременно
вклад эффектов теплопроводности и излучения даёт теоретическую основу для инженерных
решений в различных областях,
таких как производство стекла,
лазерная интерстициальная термотерапия, и другие.
Главной особенностью данных процессов является существенное влияние излучения
на теплообмен при высоких температурах.




\textbf{Степень разработанности темы исследования.}
Основополагающие работы А.Н. Тихонова~\cite{TikhonovSamarskii1972} и его коллег,
которые исследовали уравнения математической физики, стали отправной точкой для
разработки методов преодоления неустойчивостей в обратных задачах.
О.М. Алифанов~\cite{Aliphanov2009} в своих работах, посвященных обратным задачам
в исследовании сложного теплообмена, также сделал значительный вклад в развитие этой области.
J.V.\ Beck ~\cite[]{Beck1985-fg} предложил новые подходы к решению обратных задач,
основанных на методах оптимизации и статистическом анализе.


Исследования, проведенные в работах~\cite{
    Tiihonen1997a, Tiihonen1997b, metzger1999existence, Amosov2005,
    Amosov2009, Philip2010, Amosov2016, Amosov2017,
    amosov2010stationary, druet2009weak, druet2010weak, laitinen2001conductive,
}, позволили анализировать разрешимость моделей сложного теплообмена между телами,
разделенными прозрачной средой.
В рамках этих моделей было исследовано уравнение теплопроводности
с нелинейным нелокальным краевым условием, имитирующим тепловое излучение
границы области и теплообмен излучением между различными частями границы.

В работах Tiihonen et al.~\cite{Tiihonen1997a, Tiihonen1997b} рассматриваются вопросы
стационарного и нестационарного теплообмена с учетом радиационных и конвективных потоков.
Metzger et al.~\cite{metzger1999existence} исследовали существование и единственность
решений для уравнений теплообмена в двухфазных системах.


В серии работ Amosov et al.~\cite{
    Amosov2005, Amosov2009, Amosov2016, Amosov2017, amosov2010stationary,
} авторы анализировали свойства существования и устойчивости стационарных
и нестационарных решений в моделях теплообмена с нелокальными краевыми условиями,
а также исследовали вопросы оптимального управления в таких системах.
Результаты этих исследований позволили разработать новые методы и алгоритмы для определения
оптимальных параметров управления и прогнозирования поведения системы в различных условиях.

Philip et al.~\cite{Philip2010} также занимались анализом разрешимости обратных задач
в моделях теплообмена, основанных на уравнениях конвекции-диффузии-реакции.
Это позволило разработать методы восстановления неизвестных параметров системы,
таких как коэффициенты теплопроводности или плотности источников тепла.


Исследования, проведенные Druet et al.~\cite{druet2009weak, druet2010weak},
посвящены слабым решениям стационарных и нестационарных моделей теплообмена
с нелинейными нелокальными краевыми условиями.
Это позволило разработать новые методы исследования структуры решений и их свойств,
а также применить полученные результаты для анализа реальных физических систем.


В работе Laitinen et al.~\cite{laitinen2001conductive} рассматривается проблема
проводимости тепла в многослойных структурах с учетом радиационного и конвективного теплообмена.
Результаты этого исследования позволили разработать методы оптимизации теплоизоляционных
свойств материалов и технологий, используемых в различных отраслях промышленности.


В целом, проведенные исследования в области обратных задач теплообмена
с учетом радиационных и конвективных процессов позволили разработать
новые методы и подходы к анализу и оптимизации тепловых систем,
что нашло широкое применение в теплоэнергетике, авиации, космической технике,
микроэлектронике и других областях науки и техники.

В ряде исследований была изучена возможность решения моделей сложного теплообмена
в полупрозрачной среде, где для описания радиационного теплообмена применяется
уравнение переноса излучения.
В работах~\cite{
    asllanaj2003existence,
    kelley1996existence,
} доказана единственность решений одномерных стационарных задач
радиационно-кондуктивного теплообмена, в то время как
в~\cite{
    ghattassi2018existence, Porzio2004, Thompson2004,
} была доказана единственность решений трехмерных задач.
В исследовании~\cite{Thompson2004} рассматривается стационарная модель,
в~\cite{Porzio2004} – нестационарная, а в~\cite{ghattassi2018existence} – квазистационарная модель.


Квазистационарные модели сложного теплообмена представляют собой модели,
которые включают нестационарное уравнение
теплопроводности и стационарное уравнение переноса излучения.
Эти модели позволяют учесть различные временные масштабы процессов,
происходящих в системе, и использовать соответствующие численные методы для их решения.


Обратим внимание на работы~\cite{
    asllanaj2004convergence, asllanaj2007transient, banoczi1999fast,
    ghattassi2016galerkin, klein2005transient, kovtanyuk2012,
}, которые посвящены разработке численных методов для ранее упомянутых моделей сложного теплообмена.

А.А. Амосов в своих работах~\cite{
    Amosov2016, Amosov2017, amosov2016unique, amosov2017unique,
} доказал единственность решений для стационарных и квазистационарных
моделей сложного теплообмена в системе полупрозрачных тел.
В этих моделях используется уравнение переноса излучения с краевыми условиями,
которые моделируют отражение и преломление излучения на границах тел.
Рассмотрена также зависимость интенсивности излучения и оптических свойств
тел от частоты излучения.
В работах~\cite{
    Amosov2016, Amosov2017,
} предполагались условия диффузного отражения и преломления излучения,
в то время как в~\cite{amosov2016unique, amosov2017unique} рассматривались условия
отражения и преломления излучения согласно законам Френеля.

Эти исследования представляют собой значительный вклад в разработку и анализ
численных методов для моделей сложного теплообмена в полупрозрачных средах.
Результаты позволяют глубже понять особенности этих моделей и принять во внимание
различные физические процессы, такие как отражение и преломление излучения,
а также зависимость оптических свойств материалов от частоты излучения.

В следующих работах акцент ставится на решении обратных задач в
контексте моделей сложного теплообмена, основанных на полном уравнении переноса излучения.
В статье~\cite{end2010optimization}, авторы разрабатывают численный алгоритм для решения
задачи оптимального управления источниками тепла и излучения в стационарной модели сложного
теплообмена, что позволяет управлять процессом с учетом радиационных эффектов.

В статье~\cite{end2011analytical}, теоретический анализ задачи оптимального
управления источниками тепла проводится в рамках квазистационарной модели сложного теплообмена,
включая полное уравнение переноса излучения.
Авторы доказывают однозначную разрешимость прямой задачи,
разрешимость задачи управления и представляют условия оптимальности.
Это позволяет лучше понять возможности и ограничения оптимального управления тепловыми
источниками в сложных системах с учетом радиационного теплообмена.

В работе~\cite{amosov2010stationary}, основной фокус направлен на теоретический
и численный анализ обратной задачи восстановления начального распределения температуры,
основываясь на известной зависимости температуры на границе области от времени.
Это исследование проводится в рамках квазистационарной модели сложного теплообмена,
что может облегчить понимание и оптимизацию процессов теплообмена с учетом радиационных эффектов.



Работы~\cite{birgelis2003optimal, meyer2006optimal, meyer2009state, Philip2010}
основываются на анализе задач оптимального управления для стационарных моделей
сложного теплообмена в прозрачной среде.
Они включают уравнение теплопроводности с нелинейным нелокальным краевым условием,
моделирующим тепловое излучение границы области и теплообмен излучением между частями границы.
Эти исследования разрабатывают методы для эффективного управления процессами теплообмена
в прозрачных средах.

В статье~\cite{belmiloudi2014nonlinear} представлен численный алгоритм для решения
задачи оптимального управления граничными коэффициентами в одномерной нестационарной модели,
которая включает уравнение теплопроводности с нелинейным краевым условием,
описывающим тепловое излучение границ.
Этот подход позволяет адаптировать граничные коэффициенты для улучшения процессов
теплообмена в динамических условиях.

Статьи~\cite{kovtanyuk2013iterative, Thommes2002, Pinnau2008, Siewert1991}
посвящены численному моделированию в рамках диффузионных моделей сложного теплообмена.
Они показывают различные подходы к аппроксимации и решению уравнений теплопроводности
с учетом радиационных эффектов.

В работе~\cite{gallouet2016analysis} исследуется схема метода конечных объемов
для решения квазистационарной системы уравнений сложного теплообмена,
основанной на $P_1$-приближении уравнения переноса излучения.
Это исследование предлагает новый подход к численному решению квазистационарных задач,
связанных с радиационным теплообменом.



В статьях~\cite{
    modest2014elliptic, frank2011adaptive, kovtanyuk2012, Thommes2002, Larsen2002, Frank2007,
} проводится сравнение $P_1$-приближения с другими методами аппроксимации уравнения переноса излучения.
Авторы исследуют различные подходы к приближению уравнения переноса излучения,
такие как дискретные углы, сферические гармоники и другие методы,
для определения наиболее точных и эффективных способов моделирования радиационного теплообмена.

Статья~\cite{modest2014elliptic} представляет сравнение $P_1$-приближения
с эллиптическим уравнением переноса излучения, оценивая их точность и вычислительные затраты.
В работе~\cite{frank2011adaptive} рассматривается
адаптивное изменение сетки (adaptive mesh refinement)
для численного решения уравнения переноса излучения
с использованием различных методов аппроксимации, включая $P_1$-приближение.

В статье~\cite{kovtanyuk2012} представлено сравнение $P_1$-приближения с методом
дискретных углов для решения уравнения переноса излучения в полупрозрачных средах.
Работа~\cite{Thommes2002} сравнивает $P_1$-приближение с другими методами для решения
уравнений радиационного переноса и теплопроводности.

Статьи~\cite{Larsen2002} и~\cite{Frank2007} также оценивают эффективность и точность
различных методов аппроксимации уравнения переноса излучения,
включая $P_1$-приближение, для моделирования радиационного теплообмена в различных условиях.



В статьях~\cite{Addam2015, olbrant2013asymptotic, frank2010optimal, frank2011adaptive, Frank2007}
приведены вывод и численный анализ нестационарного $P_1$-приближения уравнения переноса излучения.
Эти работы оценивают точность и эффективность данного подхода в моделировании радиационного теплообмена
в различных условиях и сравнивают его с другими методами аппроксимации.

Работы R. Pinnau и O. Tse ~\cite{Pinnau2007, Pinnau2013} проводят теоретический анализ квазистационарных
моделей сложного теплообмена, основанных на $SP_1$ и $SP_3$-приближениях.
Эти модели включают уравнение теплопроводности, стационарное $SP_N$-приближение,
а также в~\cite{Pinnau2013} уравнения Навье – Стокса в приближении Буссинеска.

В результате анализа авторы определяют преимущества и недостатки различных приближений,
позволяющих точнее и эффективнее моделировать сложные процессы теплообмена, такие как радиационный перенос,
теплопроводность и конвективный теплообмен, описываемый уравнениями Навье – Стокса.


В работе~\cite{Pinnau2007} авторы доказывают существование, единственность и ограниченность
решения задачи сложного теплообмена на основе $P_1$-приближения без источников тепла и излучения.
В отличие от этого, в статье~\cite{Pinnau2013} авторы доказывают однозначную разрешимость задачи
свободной конвекции с радиационным теплообменом на основе $SP_3$-приближения в двумерной области,
где присутствуют источники тепла с ограниченной плотностью.

В свою очередь, в работах А.Е. Ковтанюка и А.Ю. Чеботарева~\cite{Kovtanyuk2014, Kovtanyuk2016, Kovtanyuk2015}
авторы исследуют стационарные модели сложного теплообмена на основе $P_1$-приближения.
В этих статьях доказана однозначная разрешимость краевых задач
и сходимость метода простой итерации для нахождения решения.
Эти результаты важны для обоснования применимости и
эффективности $P_1$-приближения в задачах сложного теплообмена.


Стоит отметить, что численная реализация метода, предложенного в работах А.Е. Ковтанюка и А.Ю. Чеботарева,
затруднена, поскольку на каждой итерации требуется решить нелинейное эллиптическое уравнение.
В статье~\cite{Chebotarev2016Odnaznachnaya} авторы доказывают однозначную разрешимость
сходной субдифференциальной краевой задачи с многозначной зависимостью коэффициента излучения
границы от интенсивности излучения.

В работах~\cite{astrakhantseva2017analysis, chebotarev2018inverse} получены результаты о существовании
и единственности решений обратных задач для стационарной диффузионной модели сложного теплообмена.
Эти задачи заключаются в нахождении неизвестной плотности источников тепла в виде линейной комбинации
заданных функционалов при известных значениях этих функционалов на решении краевой задачи.

Работы R. Pinnau и O. Tse~\cite{Pinnau2007, Pinnau2013} посвящены теоретическому анализу задач
оптимального управления температурой на границе области в рамках квазистационарных моделей сложного
теплообмена на основе $SP_N$-приближений.
Авторы доказали разрешимость задач управления и нашли необходимые условия оптимальности,
что является важным результатом для понимания и решения задач оптимального управления
в рамках сложных теплообменных процессов.

В работах~\cite{
    clever2012optimal, clever2014model, lang2005adaptive, frank2011adaptive, Pinnau2007b, Pinnau2013,
} были разработаны численные методы решения задач оптимального управления граничной температурой
для квазистационарной модели сложного теплообмена на основе $P_1$-приближения.
Особенностью методов, предложенных в~\cite{clever2012optimal, clever2014model, lang2005adaptive},
является учет зависимости коэффициента поглощения от частоты излучения, что позволяет получать более
точные результаты при моделировании сложных теплообменных процессов.
Эти методы могут быть полезными для практических приложений
в области управления и оптимизации теплообменных систем.


В~\cite{frank2011adaptive, Pinnau2007b} авторы рассматривали задачу
минимизации отклонения поля температуры от заданного.
Для решения задачи оптимизации применялся метод Ньютона,
который является эффективным итерационным методом для нахождения оптимального решения.

В работе~\cite{lang2005adaptive} также рассматривалась задача минимизации
отклонения поля температуры от заданного, но в данном случае использовался метод проекции градиента,
который предлагает другой подход к оптимизации и может быть предпочтительным в некоторых ситуациях.

В работах~\cite{clever2012optimal, Pinnau2013} авторы фокусировались на минимизации
нормы градиента температуры и применяли метод проекции градиента для решения задачи оптимизации.
Этот подход может быть полезен в задачах, где важна гладкость решения.

В работе~\cite{clever2014model} решалась задача минимизации отклонения поля температуры
от заданного на основе серии из трех моделей, аппроксимирующих уравнение переноса излучения с разной точностью.
Авторы использовали оптимизационный метод второго порядка,
что позволяет учитывать кривизну целевой функции и может привести к более быстрой сходимости и точности решения.

Таким образом, в каждой из этих работ предложены различные методы
оптимизации для решения задач минимизации отклонения температурного
поля или нормы градиента температуры от заданных значений.
Эти методы могут быть полезными для разработки и применения оптимальных
стратегий управления температурой в различных теплообменных системах.


В работе~\cite{Kovtanyuk2016Optimal} авторы провели теоретический анализ
задачи оптимального управления температурой на границе области в рамках стационарной
диффузионной модели сложного теплообмена.
Для численного решения этой задачи управления был применен метод проекции
градиента, который является эффективным итерационным методом оптимизации.

В ряде работ А.Е. Ковтанюка, А.Ю. Чеботарева и других~\cite{
Kovtanyuk2014,
astrakhantseva2017design,
Chebotarev2015,
Kovtanyuk2014TheoreticalAnalysis,
}, исследовались задачи оптимального управления коэффициентом
излучения границы области в рамках стационарной модели сложного теплообмена на основе $P_1$-приближения.

В~\cite{Kovtanyuk2014, Kovtanyuk2014TheoreticalAnalysis},
авторы вывели необходимые условия оптимальности для задачи максимизации выходящей
из среды энергии, доказали разрешимость задачи управления и получили
достаточные условия регулярности системы оптимальности.
Они также обнаружили, что эти условия выполняются при достаточно
большой скорости движения среды и малых размерах области.




В работах~\cite{end2011analytical, asllanaj2003existence}, авторы изучили
оптимальное управление в задачах максимизации и минимизации
полей температуры и излучения в области теплообмена.
Они получили достаточные условия оптимальности для этих задач и доказали
сходимость метода простой итерации для нахождения оптимального управления.
Результаты этих исследований являются важным вкладом в разработку и применение
оптимальных стратегий управления в теплообменных системах, особенно в случае,
когда поле температуры или излучения должно быть максимизировано или минимизировано во всей области теплообмена.

В последние годы исследования моделей сложного теплообмена стали особенно
актуальными в связи с практическими приложениями, такими
как высокотемпературные процессы и передовые технологии.
Ниже приведены некоторые примеры исследований практической применимости рассматриваемых моделей:

Производство стекла: работы~\cite{frank2010optimal, clever2012optimal} представляют
собой примеры исследований, посвященных оптимальному
управлению температурой в процессах производства стекла.
В этих работах рассматриваются модели сложного теплообмена, которые позволяют управлять
температурными полями и повышать эффективность производства.

Лазерная интерстициальная термотерапия: в работах~\cite{Tse2012, Hubner2017}
изучается применение моделей сложного теплообмена для
управления процессами лазерной интерстициальной термотерапии.
Этот метод используется для локального лечения опухолей и
требует точного управления температурными полями во время процедуры.
Модели сложного теплообмена могут помочь в определении оптимальных
параметров управления для достижения желаемого терапевтического эффекта.

Процедура внутривенной лазерной абляции (ВВЛА) является безопасной и эффективной
в лечении варикозных вен~\cite{Endovenous_vandenBos2009}.
Математическое моделирование лучевых и тепловых процессов при ВВЛА
имеет решающее значение для определения оптимальных параметров
излучения, обеспечивающих достаточно высокие
температуры внутри вены для успешной облитерации,
обеспечивая при этом сохранность окружающих тканей.
Результаты численного моделирования для различных
длин волн и диаметров жил обсуждались в ряде работ~\cite{
    Opticalthermal_vanRuijven2014, Some_Poluektova2014,
    Endovenous_Malskat2014, Mathematical_Mordon2006,
}.


Эти примеры исследований показывают, что модели сложного теплообмена имеют
большой потенциал для практического использования в различных отраслях промышленности и медицине.
Они могут помочь в определении оптимальных стратегий
управления для повышения эффективности процессов
и обеспечения безопасности и точности в различных приложениях.

%Во время ВВЛА лазерный оптический волоконный световод вводится в поврежденную вену,
%и лазерное излучение передается через световод, который в это время вытаскивается из вены.
%Конец оптического световода обычно покрыт углеродистым слоем (наконечник оптического волокна).
%Углеродистый слой разделяет лазерную энергию на нагрев наконечника оптического волокна и излучение.
%Тепло от наконечника оптического волокна передается крови путем кондуктивного теплообмена,
%теплообмен значительно увеличивается за счет потока пузырьков,
%образующихся на нагретом наконечнике волокна.
%
%Излучение, попадающее в кровь и окружающую ткань,
%частично поглощается с выделением тепла.
%В результате образующаяся тепловая энергия вызывает значительный нагрев вены,
%что приводит к ее облитерации (закрытию вены).


Эти исследования демонстрируют важность математического моделирования
для выбора оптимальной мощности лазерного излучения и скорости отвода волокна.

Наиболее перспективным подходом к выбору оптимальных параметров
излучения является рассмотрение задачи оптимального управления
для уравнений типа реакция-диффузия, описывающих процедуру ВВЛА\@.
Различные подходы к анализу и оптимизации параметров для моделей реакция-диффузия,
описывающих различные природные явления, можно найти в~\cite{
Stability_Alekseev2016, Optimization_Brizitskii2018,
chebotarev2018inverse, Theoretical_Maslovskaya2021,
}.

Задачи оптимального управления для модели ВВЛА рассматриваются в~\cite{
    Optimal_Kovtanyuk2020, Inverse_Kovtanyuk2021}.
В~\cite{Optimal_Kovtanyuk2020} поставлена задача оптимального управления для модели реакция-диффузия,
описывающей процедуру ВВЛА, которая заключается в аппроксимации заданного
температурного профиля в определенной точке области модели.
В~\cite{Inverse_Kovtanyuk2021} изучается аналогичная задача оптимального управления,
как в~\cite{Optimal_Kovtanyuk2020}.
Здесь целевой функционал выбран таким образом, что его минимизация позволяет
достичь заданного распределения температуры в различных частях области модели.
Это позволяет обеспечить достаточно высокую температуру внутри
вены для успешной облитерации и безопасную температуру в окружающей вену ткани.
Доказана единственная разрешимость начально-краевой задачи,
на основе которой показана разрешимость задачи оптимального управления.
Предложен алгоритм поиска решения задачи оптимального управления.
Эффективность алгоритма проиллюстрирована численным примером.



Таким образом, ряд важных задач, относящихся к моделированию и
оптимизации сложного теплообмена на основе
диффузионного приближения, оставался нерешенным: исследование
разрешимости нестационарной задачи сложного
теплообмена с источниками тепла и излучения и нестационарной задачи
свободной конвекции с радиационным теплообменом в трехмерной
области, исследование устойчивости по Ляпунову стационарных решений,
вывод диффузионной модели сложного теплообмена в многослойной среде,
анализ сходимости метода Ньютона
для уравнений сложного теплообмена, разработка численных
методов решения задач оптимального управления
коэффициентом излучения границы области в рамках нестационарных моделей
сложного теплообмена, доказательство регулярности
условий оптимальности для задачи оптимального управления коэффициентом
излучения границы в рамках стационарной модели.

В целом, исследования моделей сложного теплообмена в практических приложениях
подчеркивают важность этого направления для развития новых технологий и применений.
Моделирование и оптимизация тепловых процессов в различных областях может привести
к более эффективным и безопасным методам производства, лечения и управления температурой.
С развитием вычислительных технологий и улучшением численных методов, можно ожидать
дальнейшего прогресса в этой области исследований.


\textbf{Цели и задачи диссертационной работы.}
Цели работы - теоретическое исследование разрешимости обратных стационарных задач
сложной теплопроводности.
Разработка численных методов решения исследуемых краевых задач,
а также задач оптимального управления.
Разработка вычислительных программ для постановки численных экспериментов
и демонстрации результатов расчётов.
% TODO: !!!
Перед началом работы были поставлены следующие задачи:
\begin{itemize}
  \item[--] Исследовать разрешимость задачи по нахождению коэффициента отражения участка границы для
  стационарной модели, по дополнительной информации о температурном поле.
  \item[--] Разработать численный метод по нахождению решения для соответствующей экстремальной задачи.
  \item[--] Исследовать стационарную задачу оптимального управления для уравнений радиационно-кондуктивного
  теплообмена в трехмерной области в рамках $P_1$–приближения уравнения переноса излучения.
  \item[--] Результаты теоретического анализа проиллюстрировать численными примерами.
\end{itemize}


\textbf{Научная новизна.}
Результатом работы является теоретический анализ
разрешимости обратных задач сложного теплообмена.
Доказано существование квазирешения для первой рассматриваемой задачи.
Реализован алгоритм градиентного спуска для решения экстремальной задачи и представлены результаты
численных экспериментов.
Далее показано, что последовательность решений экстремальных задач сходится к решению
краевой задачи с условиями типа Коши для температуры.
Результаты теоретического анализа также проиллюстрированы численными примерами.

\textbf{Теоретическая и практическая значимость.}
Исследование однозначной разрешимости экстремальных задач, а также задач оптимального
управления крайне важно при реализации численных алгоритмов и позволяет
судить об адекватности полученных решений.

Задачи оптимизации имеют крайне важное практическое применение при выборе параметров системы для
получения желаемой температуры или теплового излучения.
Необходимость выбора параметров системы возникает при проектировании инженерных установок
в которых присутствуют процессы сложного теплообмена.

Разработанные комплексы программ служат практическим подтверждением теоретических результатов,
а также могут быть использованы в качестве примеров для решения подобных задач.


Научная значимость данной работы состоит в теоретическом вкладе
в исследования корректности и разрешимости задач сложного теплообмена.
Реализация конкретных методов решения проблем оптимального управления,
в свою очередь, имеет высокую
значимость для решения прикладных инженерных задач по проектированию
тепловых установок с заданными температурными свойствами.


Исследования в области моделирования сложного теплообмена и оптимального управления
температурой продолжают расширять наше знание и понимание фундаментальных процессов,
лежащих в основе разнообразных приложений.

\textbf{Методология и методы исследования.}
В работе широко использовались методы математического и
функционального анализа, теории дифференциальных
уравнений в частных производных, теории экстремальных задач.
Для разработки численных алгоритмов решения
применялись методы вычислительной математики,
объектно-ориентированное и функциональное программирование,
методы оптимизации и другие.


\textbf{Положения, выносимые на защиту.}
В области математического моделирования
% TODO: !!!
\begin{itemize}
    \item Разрешимость экстремальной задачи для стандартной модели радиационно-диффузионного теплообмена
\end{itemize}


\textbf{Степень достоверности и апробация результатов.}
Теоретические результаты, представленные в диссертации получены
использованием методов функционального анализа, теорий
дифференциальных уравнений и экстремальных задач.
Теоремы имеют строгие математические доказательства.
Достоверность численных экспериментов обеспечивается согласованностью
с теоретическими результатами, доказательством
сходимости итерационных процессов и тестированием разработанного программного обеспечения.

\textbf{Публикации.}
Результаты диссертационного исследования опубликованы в пяти статьях
[65, 66] в изданиях, рекомендованных ВАК.


\textbf{Личный вклад автора.}
Результаты в области математического моделирования получены совместно с научным руководителем.
В области численных методов и комплексов программ результаты получены автором самостоятельно.
