%% Согласно ГОСТ Р 7.0.11-2011:
%% 5.3.3 В заключении диссертации излагают итоги выполненного исследования, рекомендации, перспективы дальнейшей разработки темы.
%% 9.2.3 В заключении автореферата диссертации излагают итоги данного исследования, рекомендации и перспективы дальнейшей разработки темы.
В диссертации доказано существование квазирешения задачи нахождения
коэффициента отражения участка границы для стационарной модели,
по дополнительной информации о температурном поле.
Экспериментально поверена устойчивость получаемых
решений методом градиентного спуска.
Таким образом, получены важные с теоретической точки зрения результаты,
которые могут быть полезны при дальнейшем использовании
стационарных моделей сложного теплообмена и анализе обратных
задач в рамках нестационарных моделей сложного теплообмена.
Развитые методы исследования начально-краевых задач могут
применяться для изучения различных моделей, описываемых нелинейными
уравнениями со сходной структурой.

Разработанный комплекс программ для постановки численных экспериментов
показал свою надёжность и может в дальнейшем быть использован как
пример для решения подобных задач.

Разработан комплекс программ для проведения вычислительных экспериментов.
Для презентации результатов расчётов, помимо самих солверов,
был разработан программный комплекс для отрисовки
полученных расчётов в трёхмерных областях.

Исследование нестационарных моделей сложного теплообмена и соответствующих им
обратных задач является крайне перспективной областью математического моделирования
и в то же время достаточно сложной для теоретического
анализа и реализации численных решений.
Более широкий класс процессов может быть покрыт задачами на оптимальное
управление многими переменными среды, что позволяет более точно находить
решения для инженерных задач.
