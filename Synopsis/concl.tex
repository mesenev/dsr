%% Согласно ГОСТ Р 7.0.11-2011:
%% 5.3.3 В заключении диссертации излагают итоги выполненного исследования, рекомендации, перспективы дальнейшей разработки темы.
%% 9.2.3 В заключении автореферата диссертации излагают итоги данного исследования, рекомендации и перспективы дальнейшей разработки темы.
в диссертации, в соответствии с паспортом специальности 1.2.2, представлен
математический анализ диффузионных моделей сложного теплообмена,
предложены новые постановки обратных задач, разработаны
оптимизационные методы решения обратных задач, основанные на понятии
квазирешения и сведения рассмотренных задач к задачам оптимального
управления.
Разработаны и программно реализованы новые алгоритмы
решения прямых, обратных и экстремальных задач для моделей сложного
теплообмена.


Кроме того, представлены новые априорные оценки решений начально-краевых задач для
квазистационарных и квазилинейных уравнений сложного теплообмена и
доказана их нелокальная однозначная разрешимость.
Выполнен теоретический анализ возникающих новых экстремальных задач.
Получены априорные оценки решений регуляризованных задач и
обоснована сходимость их решений к точным решениям обратных задач.
Для решения задач с фазовыми ограничениями, предложены алгоритмы,
основанные на аппроксимации экстремальными задачами со штрафом.


Указанные результаты могут быть полезны при дальнейшем использовании
моделей сложного теплообмена и анализе обратных задач сложного
теплообмена.
Развитые методы исследования краевых, начально-краевых и
экстремальных задач могут применяться для изучения различных моделей,
описываемых нелинейными уравнениями типа реакции-диффузии.
Численные алгоритмы решения задач оптимизации сложного теплообмена
могут использоваться для выбора оптимальных характеристик процессов
теплообмена.


Все рассмотренные в работе типы задач логически связаны следующим образом.
Теоретический анализ математических моделей сложного
теплообмена, представленный в первой главе, является основой для
исследования оптимизационных методов решения обратных задач во второй
и третьей главах.
Соответственно, полученные там условия оптимальности
дают возможность представить численные алгоритмы решения
сформулированных задач и численно реализовать их в главе~4.


Конечно же, автору не удалось рассмотреть все важные вопросы в теории и
методах решения обратных задач сложного теплообмена.
Ряд постановок, которые можно будет исследовать на основе
предложенной методики, ожидает своего решения, в том числе и в связи с
вопросами нахождения наиболее эффективных механизмов и способов
управления теплофизическими полями.
