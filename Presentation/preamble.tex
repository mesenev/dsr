\begin{frame}[noframenumbering,plain]
    \setcounter{framenumber}{1}
    \maketitle
\end{frame}


\begin{frame}
    \frametitle{Мотивация}
    Интерес к изучению задач сложного теплообмена
    (одновременно учитываются радиационный, конвективный и кондуктивный факторы)
    обусловлен важностью для многих
    инженерных применений, связанных с высокими температурами: оценки
    эффективности систем охлаждения, моделирования теплопередачи в деталях
    газотурбинных двигателей, космической техники, летательных аппаратов,
    контроль тепловых процессов при производстве стекла и др.

    \includegraphics[width=3.cm,height=2.5cm]{Ex1}
    \hfill
    \includegraphics[width=3cm,height=2.5cm]{ex2}
    \hfill
    \includegraphics[width=3cm,height=2.5cm]{ex4}
       \begin{itemize}
        \item Теоретическое исследование моделей сложного теплообмена с
        полным уравнением переноса излучения
        (\textit{А. А. Амосов, C. T. Kelley, M. Laitinen, T. Tiihonen, P.-E. Druet и др.})
        \item Однозначная разрешимость различных задач радиационно-кондуктивного теплообмена
        (\textit{F. Asllanaj, M. Ghattassi, M. Porzio, M. Thompson и др.})
        \item Анализ квазистационарных моделей сложного теплообмена на основе $SP_1$ и $SP_3$-приближений
        (\textit{R. Pinnau, O. Tse})
        \item Однозначная разрешимость краевых задач для моделей сложного теплообмена
        (\textit{А. Ю. Чеботарев, А. Е. Ковтанюк, Г. В. Гренкин})
        \item Теоретический анализ обратных задач (\textit{С. Г. Пятков и др.})
    \end{itemize}
\end{frame}


\begin{frame}
    \frametitle{Цели и задачи работы}
    \textit{Целью работы} является теоретический и численный анализ граничных обратных задач, включая
    задачи с условиями типа Коши на границе области, и задач оптимального
    управления для моделей сложного теплообмена на основе $P_1$-приближения
    уравнения переноса излучения


    \begin{itemize}
        \item исследование разрешимости краевых и начально-краевых задач для
        квазистационарных и квазилинейных моделей сложного теплообмена;
        \item разработка оптимизационных алгоритмов решения обратных задач и
        задач с краевыми условиями Коши, теоретический анализ и обоснование их сходимости;
        \item разработка комплекса программ для проведения вычислительных
        экспериментов и тестирования предложенных алгоритмов.
    \end{itemize}
\end{frame}

\begin{frame}
    \frametitle{Положения, выносимые на защиту}
    \textit{В области математического моделирования:}
    \begin{itemize}
        \item Доказательство однозначной разрешимости начально-краевой задачи,
        моделирующей квазистационарный сложный теплообмен в трехмерной
        области.
        \item Доказательство однозначной разрешимости квазилинейной начально-краевой задачи,
        моделирующей сложный теплообмен с нелинейной
        зависимостью коэффициента теплопроводности от температуры.
        \item Обоснование существования квазирешения обратной задачи с неизвестным
        коэффициентом отражения на части границы и условием
        переопределения на другой части.
        \item Вывод условий разрешимости экстремальных задач, аппроксимирую
        щих решения граничных обратных задач (задач с условиями Коши на границе области)
        для стационарной и квазистационарной моделей сложного теплообмена.
        \item Построение систем оптимальности и доказательство их невырожденности
        для задач оптимального управления стационарными, квазистационарными
        и квазилинейными уравнениями сложного теплообмена.
    \end{itemize}
\end{frame}


\begin{frame}
    \frametitle{Положения, выносимые на защиту}
    \textit{В области численных методов:}
    \begin{itemize}
        \item Разработка численного алгоритма решения квазилинейной начально-краевой задачи,
        моделирующей сложный теплообмен и доказательство
        его сходимости.
        \item Обоснование сходимости последовательности решений задач оптимального
        управления к решениям задач с условиями Коши на границе при
        стремлении параметра регуляризации к нулю.
        \item Обоснование сходимости алгоритма решения экстремальных обратных
        задач с ограничениями температурных полей методом штрафа.
    \end{itemize}

    \textit{В области комплексов программ:}
    \begin{itemize}
        \item Разработка программ, реализующих численное моделирование процессов
        сложного теплообмена на основе метода конечных элементов.
        Реализация и тестирование оптимизационных алгоритмов решения
        граничных обратных задач для стационарных, квазистационарных и
        квазилинейных моделей.
    \end{itemize}
\end{frame}

\begin{frame}
    \frametitle{Содержание}
    \tableofcontents
\end{frame}
\note{
    Работа состоит из четырёх глав.

    \medskip
    В первой главе \dots

    Во второй главе \dots

    Третья глава посвящена \dots

    В четвёртой главе \dots
}
