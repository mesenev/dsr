\begin{frame}[noframenumbering,plain]
    \setcounter{framenumber}{1}
    \maketitle
\end{frame}


\begin{frame}
    \frametitle{Мотивация}

\end{frame}


\begin{frame}
    \frametitle{Обзор}

\end{frame}


\begin{frame}
    \frametitle{Положения, выносимые на защиту}
    \textit{В области математического моделирования:}
    \begin{itemize}
        \item Доказательство однозначной разрешимости начально-краевой задачи,
        моделирующей квазистационарный сложный теплообмен в трехмерной
        области.
        \item Доказательство однозначной разрешимости квазилинейной начально-краевой задачи,
        моделирующей сложный теплообмен с нелинейной
        зависимостью коэффициента теплопроводности от температуры.
        \item Обоснование существования квазирешения обратной задачи с неизвестным
        коэффициентом отражения на части границы и условием
        переопределения на другой части.
        \item Вывод условий разрешимости экстремальных задач, аппроксимирую
        щих решения граничных обратных задач (задач с условиями Коши на границе области)
        для стационарной и квазистационарной моделей сложного теплообмена.
        \item Построение систем оптимальности и доказательство их невырожденности для задач оптимального управления стационарными, квазистационарными
        и квазилинейными уравнениями сложного теплообмена.
    \end{itemize}
\end{frame}
\note{
    Проговариваются вслух положения, выносимые на защиту
}


\begin{frame}
    \frametitle{Положения, выносимые на защиту}
    \textit{В области численных методов:}
    \begin{itemize}
        \item Разработка численного алгоритма решения квазилинейной начально-краевой задачи,
        моделирующей сложный теплообмен и доказательство
        его сходимости.
        \item Обоснование сходимости последовательности решений задач оптимального
        управления к решениям задач с условиями Коши на границе при
        стремлении параметра регуляризации к нулю.
        \item Обоснование сходимости алгоритма решения экстремальных обратных
        задач с ограничениями температурных полей методом штрафа.
    \end{itemize}

    \textit{В области комплексов программ:}
    \begin{itemize}
        \item Разработка программ, реализующих численное моделирование процессов
        сложного теплообмена на основе метода конечных элементов.
        Реализация и тестирование оптимизационных алгоритмов решения
        граничных обратных задач для стационарных, квазистационарных и
        квазилинейных моделей.
    \end{itemize}
\end{frame}

\begin{frame}
    \frametitle{Содержание}
    \tableofcontents
\end{frame}
\note{
    Работа состоит из четырёх глав.

    \medskip
    В первой главе \dots

    Во второй главе \dots

    Третья глава посвящена \dots

    В четвёртой главе \dots
}
