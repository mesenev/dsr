\section{Модели сложного теплообмена}\label{sec:$p_1$}

\subsection{Стационарная модель сложного теплообмена}\label{subsec:st}
\begin{frame}
    \frametitle{Стационарная модель}
    Область $\Omega \subset \mathbb{R}^3$, граница  $\Gamma = \partial \Omega$.
    \begin{gather}
        -a \Delta \theta + + b \kappa_a \theta^4 =  b \kappa_a \varphi,
        \quad
        - \alpha \Delta \varphi + \kappa_a \varphi = \kappa_a \theta^4,  \label{eq:pres:1}\\
        a \frac{\partial \theta}{\partial \mathbf{n}}
        +\left.\beta\left(\theta-\theta_{b}\right)\right|_{\Gamma}=0,
        \quad
        \alpha \frac{\partial \varphi}{\partial \mathbf{n}} + \gamma
        (\varphi-\theta_b^4)|_{\Gamma} = 0. \label{eq:pres:2}
    \end{gather}
    $\Omega$ -- липшицева ограниченная область, $\Gamma$ состоит из конечного числа гладких кусков,
    исходные данные удовлетворяют условиям:
    \begin{itemize}
        \item (i) $\theta_{0}, \beta, \gamma \in L^{\infty}(\Gamma),
        0 \leqslant \theta_{0} \leqslant M, \beta \geqslant \beta_{0}>0, \gamma \geqslant \gamma_{0}>0$;
    \end{itemize}
    Здесь $M, \beta_{0}, \gamma_{0}$, и $c_{0}$ положительные постоянные.
\end{frame}
\note{
    Стационарная нормализованная диффузионная модель, описывающая
    радиационный, кондуктивный и конвективный теплообмен в
    ограниченной области $\Omega \subset \mathbb{R}^3$,
    имеет следующий вид
    ...
    Здесь $\theta$ -- нормализованная температура, $\varphi$ --
    нормализованная интенсивность излучения, усредненная по всем
    направлениям, $\textbf{v}$ -- заданное поле скоростей, $\kappa_a$ --
    коэффициент поглощения.
    Постоянные $a$, $b$ и $\alpha$
    определяются следующим образом:
    \[
        a = \frac{k}{\rho c_v},\quad b = \frac{4\sigma n^2 T_{\max}^3}{\rho c_v},
        \quad \alpha=\frac{1}{3\kappa - A \kappa_s},
    \]
    где $k$ -- теплопроводность, $c_v$ -- удельная теплоемкость, $\rho$ --
    плотность, $\sigma$ -- постоянная Стефана-Больцмана, $n$ --
    показатель преломления, $T_{\max}$ -- максимальная температура в
    ненормализованной модели, $\kappa = \kappa_s + \kappa_a$ -- коэффициент
    полного взаимодействия, $\kappa_s$ -- коэффициент рассеяния.
    Коэффициент $A \in [-1, 1]$ описывает анизотропию рассеяния, случай
    $A=0$ соответствует изотропному рассеянию.

    Будем предполагать, что функции $\theta$ и $\varphi$, описывающие
    процесс сложного теплообмена, удовлетворяют следующим условиям на
    границе $\Gamma = \partial \Omega$: ...
}

\begin{frame}
    \begin{definition}
        Пара
        $\{\theta, \varphi\} \in H^1(\Omega) \times H^1(\Omega)$ называется слабым решением задачи,
        если для любых $\eta, \psi \in H^1(\Omega)$ выполняются равенства:
        \begin{gather*}
            a(\nabla \theta, \nabla \eta)
            + \left(b \kappa_{a}\left(|\theta| \theta^{3} - \varphi\right), \eta\right)
            + \int_{\Gamma} \beta\left(\theta - \theta_{0}\right) \eta d \Gamma=0, \\
            \alpha(\nabla \varphi, \nabla \psi)+\kappa_{a}\left(\varphi-|\theta| \theta^{3},
            \psi\right)+\int_{\Gamma} \gamma\left(\varphi-\theta_{0}^{4}\right) \psi d \Gamma=0.
        \end{gather*}
    \end{definition}
    \begin{theorem}
        Пусть выполняются условия (i).
        Тогда существует единственное слабое
        решение задачи~\eqref{eq:pres:1}--\eqref{eq:pres:2},
        удовлетворяющее неравенствам:
        \begin{align}
            & a\|\nabla \theta\|^{2} \leqslant b \kappa_{a} M^{5}|\Omega|
            + \|\gamma\|_{L^{\infty}(\Gamma)} M^{2}|\Gamma|,\\
            & \alpha\|\nabla \varphi\|^{2} \leqslant \kappa_{a} M^{8}|\Omega|
            + \|\beta\|_{L^{\infty}(\Gamma)} M^{8}|\Gamma|,\\
            & 0 \leqslant \theta \leqslant M, \quad 0 \leqslant \varphi \leqslant M^{4}.
        \end{align}
    \end{theorem}
\end{frame}

\subsection{Квазистационарная модель сложного теплообмена}\label{subsec:qst}
\begin{frame}
    \frametitle{Квазистационарная модель}
    \begin{align}
        \frac{\partial\theta}{\partial t} - a\Delta\theta
        + b\kappa_a (|\theta|\theta^3 - \phi) &= 0, \label{eq:1_5:1}\\
        - \alpha\Delta\phi + \kappa_a (\phi - |\theta|\theta^3 ) &= 0,
        \quad x \in \Omega, \quad 0 < t < T ; \label{eq:1_5:1+} \\
        a \frac{\partial \theta}{\partial n}
        +\left.\beta\left(\theta-\theta_{b}\right)\right|_{\Gamma}&=0,
        \quad \alpha \frac{\partial \varphi}{\partial \mathbf{n}} + \gamma
        (\varphi-\theta_b^4)|_{\Gamma} = 0 \text{ на } \Gamma; \label{eq:1_5:2} \\
        \theta|_{t=0} &= \theta_0. \label{eq:1_5:3}
    \end{align}
    \begin {itemize}
        \item (j) $a, b, \alpha, \kappa_{a} =$ Const $>0$,
        \item (jj) $\theta_{b}, q_{b}, u=\theta^4_b \in U, r
        =a\left(\theta_{b}+q_{b}\right) \in L^{5}(\Sigma), \; \theta_{0} \in L^{5}(\Omega)$.
    \end{itemize}

    Здесь через $U$ обозначено пространство $L^{2}(\Sigma)$ с нормой
    \[
        \|u\|_{\Sigma}=\left(\int_{\Sigma} u^{2} d \Gamma d t\right)^{1/2}.
    \]
\end{frame}

\begin{frame}
    Определим операторы $A: V \rightarrow V^{\prime}, B: U \rightarrow V^{\prime}$,
    которые выполняются для любых $y, z \in V, w \in L^{2}(\Gamma)$:
    \[
        (A y, z)=(\nabla y, \nabla z)+\int_{\Gamma} y z d \Gamma, \quad(B w, z)=\int_{\Gamma} w z d \Gamma.
    \]

    \begin{definition}
        Пара $\theta \in W, \varphi \in L^{2}(0, T ; V)$
        называется слабым решением задачи~\eqref{eq:1_5:1}--\eqref{eq:1_5:3}
        если
        \begin{equation}
            \label{eq:1_5:weak}
            \theta^{\prime}+a A \theta+b \kappa_{a}\left([\theta]^{4}-\varphi\right)=B r,
            \quad \theta(0)=\theta_{0}, \quad \alpha A \varphi+\kappa_{a}\left(\varphi-[\theta]^{4}\right)=B u.
        \end{equation}
    \end{definition}
    Здесь и далее будем обозначать через
    $[\theta]^s \coloneqq |\theta|^s \mathrm{sign}\theta,\quad s \in \mathbb{R}$.
    \begin{lemma}[1.20]
        Пусть выполняются условия (j), (jj).
        Тогда существует единственное слабое решение задачи~\eqref{eq:1_5:1}--\eqref{eq:1_5:3} и справедливо
        \[
            \psi=[\theta]^{5 / 2} \in L^{\infty}(0, T ; H) \cap L^{2}(0, T ; V),
            \quad[\theta]^{4} \in L^{2}(0, T ; H).
        \]
    \end{lemma}

\end{frame}


\section{Граничные обратные задачи и задачи с данными Коши}\label{sec:-------}
\begin{frame}[plain, noframenumbering]
    \begin{center}
        \Huge
        Графика
    \end{center}
\end{frame}


\begin{frame}
    \frametitle{Одиночное изображение}
    \centering
%    \includegraphics[width=0.8\linewidth]{} % окружение figure не требуется
\end{frame}

\begin{frame}

\end{frame}

\subsection{Расположение}\label{subsec:}

\begin{frame}

\end{frame}


\section{Задачи оптимального управления для квазилинейных моделей}\label{sec:-----}


\section{Численные методы и комплексы программ}\label{sec:----}
