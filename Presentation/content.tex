\section{Модели сложного теплообмена}\label{sec:$p_1$}

\subsection{Стационарная модель}\label{subsec:st}
\begin{frame}
    \frametitle{Стационарная модель}
    Область $\Omega \subset \mathbb{R}^3$, граница  $\Gamma = \partial \Omega$.
    \begin{gather}
        -a \Delta \theta + + b \kappa_a \theta^4 =  b \kappa_a \varphi,
        \quad
        - \alpha \Delta \varphi + \kappa_a \varphi = \kappa_a \theta^4,  \label{eq:pres:1}\\
        a \frac{\partial \theta}{\partial \mathbf{n}}
        +\left.\beta\left(\theta-\theta_{b}\right)\right|_{\Gamma}=0,
        \quad
        \alpha \frac{\partial \varphi}{\partial \mathbf{n}} + \gamma
        (\varphi-\theta_b^4)|_{\Gamma} = 0. \label{eq:pres:2}
    \end{gather}
    $\Omega$ -- липшицева ограниченная область, $\Gamma$ состоит из конечного числа гладких кусков,
    исходные данные удовлетворяют условиям:
    \begin{itemize}
        \item (i) $\theta_{0}, \beta, \gamma \in L^{\infty}(\Gamma),
        0 \leqslant \theta_{0} \leqslant M, \beta \geqslant \beta_{0}>0, \gamma \geqslant \gamma_{0}>0$;
    \end{itemize}
    Здесь $M, \beta_{0}, \gamma_{0}$, и $c_{0}$ положительные постоянные.
\end{frame}
%\note{
%    Стационарная нормализованная диффузионная модель, описывающая
%    радиационный, кондуктивный и конвективный теплообмен в
%    ограниченной области $\Omega \subset \mathbb{R}^3$,
%    имеет следующий вид
%    ...
%    Здесь $\theta$ -- нормализованная температура, $\varphi$ --
%    нормализованная интенсивность излучения, усредненная по всем
%    направлениям, $\textbf{v}$ -- заданное поле скоростей, $\kappa_a$ --
%    коэффициент поглощения.
%    Постоянные $a$, $b$ и $\alpha$
%    определяются следующим образом:
%    \[
%        a = \frac{k}{\rho c_v},\quad b = \frac{4\sigma n^2 T_{\max}^3}{\rho c_v},
%        \quad \alpha=\frac{1}{3\kappa - A \kappa_s},
%    \]
%    где $k$ -- теплопроводность, $c_v$ -- удельная теплоемкость, $\rho$ --
%    плотность, $\sigma$ -- постоянная Стефана-Больцмана, $n$ --
%    показатель преломления, $T_{\max}$ -- максимальная температура в
%    ненормализованной модели, $\kappa = \kappa_s + \kappa_a$ -- коэффициент
%    полного взаимодействия, $\kappa_s$ -- коэффициент рассеяния.
%    Коэффициент $A \in [-1, 1]$ описывает анизотропию рассеяния, случай
%    $A=0$ соответствует изотропному рассеянию.
%
%    Будем предполагать, что функции $\theta$ и $\varphi$, описывающие
%    процесс сложного теплообмена, удовлетворяют следующим условиям на
%    границе $\Gamma = \partial \Omega$: ...
%}


\begin{frame}
    \begin{definition}
        Пара
        $\{\theta, \varphi\} \in H^1(\Omega) \times H^1(\Omega)$ называется слабым решением задачи,
        если для любых $\eta, \psi \in H^1(\Omega)$ выполняются равенства:
        \begin{gather*}
            a(\nabla \theta, \nabla \eta)
            + \left(b \kappa_{a}\left(|\theta| \theta^{3} - \varphi\right), \eta\right)
            + \int_{\Gamma} \beta\left(\theta - \theta_{0}\right) \eta d \Gamma=0, \\
            \alpha(\nabla \varphi, \nabla \psi)+\kappa_{a}\left(\varphi-|\theta| \theta^{3},
            \psi\right)+\int_{\Gamma} \gamma\left(\varphi-\theta_{0}^{4}\right) \psi d \Gamma=0.
        \end{gather*}
    \end{definition}
    \begin{theorem}[Chebotarev, 2015]
        Пусть выполняются условия (i).
        Тогда существует единственное слабое
        решение задачи~\eqref{eq:pres:1}--\eqref{eq:pres:2},
        удовлетворяющее неравенствам:
        \begin{align}
            & a\|\nabla \theta\|^{2} \leqslant b \kappa_{a} M^{5}|\Omega|
            + \|\gamma\|_{L^{\infty}(\Gamma)} M^{2}|\Gamma|,\\
            & \alpha\|\nabla \varphi\|^{2} \leqslant \kappa_{a} M^{8}|\Omega|
            + \|\beta\|_{L^{\infty}(\Gamma)} M^{8}|\Gamma|,\\
            & 0 \leqslant \theta \leqslant M, \quad 0 \leqslant \varphi \leqslant M^{4}.
        \end{align}
    \end{theorem}
\end{frame}

\subsection{Квазистационарная модель}\label{subsec:qst}
\begin{frame}
    \frametitle{Квазистационарная модель}
    \begin{align}
        \frac{\partial\theta}{\partial t} - a\Delta\theta
        + b\kappa_a (|\theta|\theta^3 - \phi) &= 0, \label{eq:1_5:1}\\
        - \alpha\Delta\phi + \kappa_a (\phi - |\theta|\theta^3 ) &= 0,
        \quad x \in \Omega, \quad 0 < t < T ; \label{eq:1_5:1+} \\
        a \frac{\partial \theta}{\partial n}
        +\left.\beta\left(\theta-\theta_{b}\right)\right|_{\Gamma}&=0,
        \quad \alpha \frac{\partial \varphi}{\partial \mathbf{n}} + \gamma
        (\varphi-\theta_b^4)|_{\Gamma} = 0 \text{ на } \Gamma; \label{eq:1_5:2} \\
        \theta|_{t=0} &= \theta_0. \label{eq:1_5:3}
    \end{align}
    Предполагаем, что
    \begin {itemize}
        \item (j) $a, b, \alpha, \kappa_{a} =$ Const $>0$,
        \item (jj) $\theta_{b}, q_{b}, u=\theta^4_b \in U, r
        =a\left(\theta_{b}+q_{b}\right) \in L^{5}(\Sigma), \; \theta_{0} \in L^{5}(\Omega)$.
    \end{itemize}

    Здесь $\Sigma = \Gamma \times (0, T)$, $U$ -- пространство $L^{2}(\Sigma)$ с нормой
    \[
        \|u\|_{\Sigma}=\left(\int_{\Sigma} u^{2} d \Gamma d t\right)^{1/2}.
    \]
\end{frame}


\begin{frame}
    Определим операторы $A: V \rightarrow V^{\prime}, B: U \rightarrow V^{\prime}$,
    которые выполняются для любых $y, z \in V, w \in L^{2}(\Gamma)$:
    \[
        (A y, z)=(\nabla y, \nabla z)+\int_{\Gamma} y z d \Gamma, \quad(B w, z)=\int_{\Gamma} w z d \Gamma.
    \]

    \begin{definition}
        Пара $\theta \in W, \varphi \in L^{2}(0, T ; V)$
        называется слабым решением задачи~\eqref{eq:1_5:1}--\eqref{eq:1_5:3}
        если
        \begin{equation*}
            \theta^{\prime}+a A \theta+b \kappa_{a}\left([\theta]^{4}-\varphi\right)=B r,
            \quad \theta(0)=\theta_{0}, \quad \alpha A \varphi+\kappa_{a}\left(\varphi-[\theta]^{4}\right)=B u.
        \end{equation*}
    \end{definition}
    Здесь и далее будем обозначать через
    $[\theta]^s \coloneqq |\theta|^s \mathrm{sign}\theta,\quad s \in \mathbb{R}$.
    \begin{lemma}[1.20]
        Пусть выполняются условия (j), (jj).
        Тогда существует единственное слабое решение задачи~\eqref{eq:1_5:1}--\eqref{eq:1_5:3} и справедливо
        \[
            \psi=[\theta]^{5 / 2} \in L^{\infty}(0, T ; H) \cap L^{2}(0, T ; V),
            \quad[\theta]^{4} \in L^{2}(0, T ; H).
        \]
    \end{lemma}
\end{frame}

\subsection{Квазилинейная модель}\label{subsec:ql}
\begin{frame}
    \frametitle{Квазилинейная модель}
    \begin{gather}
        \sigma \partial \theta / \partial t
        -\operatorname{div}(k(\theta) \nabla \theta)
        +b\left(\theta^{3}|\theta|-\varphi\right)=f, \label{eq:1_6:1}\\
        -\operatorname{div}(\alpha \nabla \varphi)
        +\beta\left(\varphi-\theta^{3}|\theta|\right)=g, x \in \Omega, 0<t<T, \label{eq:1_6:2}\\
        k(\theta) \partial_{n} \theta+\left.p\left(\theta-\theta_{b}\right)\right|_{\Gamma}=0,
        \alpha \partial_{n} \varphi
        +\left.\gamma\left(\varphi-\theta_{b}^{4}\right)\right|_{\Gamma}=0,
        \left.\quad \theta\right|_{t=0}=\theta_{in}.\label{eq:1_6:3}
    \end{gather}

    Предполагаем, что:
    \begin{itemize}
        \item (k1) $\alpha, \beta, \sigma \in L^{\infty}(\Omega),
        \quad b=r \beta, r = Const > 0; \alpha \geq \alpha_{0}, \beta \geq \beta_{0},
        \sigma \geq \sigma_{0}, \alpha_{0}, \beta_{0}, \sigma_{0}=$ Const $>0$.

        \item (k2) $0<k_{0} \leq k(s) \leq k_{1},\left|k^{\prime}(s)\right| \leq k_{2},
        s \in \mathbb{R}, \quad k_{j}=$ Const.

        \item (k3) $0 \leq \theta_{b} \in L^{\infty}(\Sigma), 0 \leq \theta_{\text{in}}
        \in L^{\infty}(\Omega)$; $\gamma_{0} \leq \gamma \in L^{\infty}(\Gamma), p_{0}
        \leq p \in L^{\infty}(\Gamma), \gamma_{0}, p_{0}=$ Const $>0$.

        \item (k4) $0 \leq f, g \in L^{\infty}(Q).$
    \end{itemize}

\end{frame}


\begin{frame}
    Определим операторы $A_{1}: V \rightarrow V_{0}^{\prime}$ и $A_{2}: V \rightarrow V^{\prime}$
    такие, что для всех $\theta, \varphi, v \in V$ выполняются следующие равенства:
    \[
        \begin{gathered}
            \left(A_{1}(\theta), v\right)=(k(\theta) \nabla \theta, \nabla v)
            +\int_{\Gamma} p \theta v d \Gamma=(\nabla h(\theta), \nabla v)
            +\int_{\Gamma} p \theta v d \Gamma, \\
            \left(A_{2} \varphi, v\right)=(\alpha \nabla \varphi, \nabla v)
            +\int_{\Gamma} \gamma \varphi v d \Gamma,
        \end{gathered}
    \]
    где $ h(s)=\int_{0}^{s} k(r) d r$.


    \begin{definition}
        Пару $\theta \in W, \varphi \in L^{2}(0, T ; V)$ будем называть слабым
        решением задачи~\eqref{eq:1_6:1}--\eqref{eq:1_6:3}, если
        \begin{equation}
            \label{eq:1_6:4}
            \sigma \theta^{\prime}+A_{1}(\theta)+b\left([\theta]^{4}-\varphi\right)=f_{b}+f
            \quad \text { п. в. в }(0, T), \quad \theta(0)=\theta_{\text{in}},
        \end{equation}
        \begin{equation}
            \label{eq:1_6:5}
            A_{2} \varphi+\beta\left(\varphi-[\theta]^{4}\right)
            =g_{b}+g \quad \text { п. в. в }(0, T).
        \end{equation}
    \end{definition}
    Здесь $f_{b}, g_{b} \in L^{2}\left(0, T ; V^{\prime}\right)$ и
    \[
        \left(f_{b}, v\right)=\int_{\Gamma} p \theta_{b} v d \Gamma,
        \quad\left(g_{b}, v\right)=
        \int_{\Gamma} \gamma \theta_{b}^{4} v d \Gamma \quad \forall v \in V.
    \]
\end{frame}

\begin{frame}
    Рекурсивно определим последовательность
    $\theta_{m} \in W, \quad \varphi_{m} \in L^{2}(0, T ; V)$ такую, что
    \begin{equation}
        \label{eq:1_6:10}
        \theta_{m}=F_{2}\left(\theta_{m-1}, \varphi_{m-1}\right),
        \quad \varphi_{m}=F_{1}\left(\theta_{m}\right), \quad m=1,2, \ldots
    \end{equation}
    Здесь операторы $F_{1}: L^{\infty}(\Omega) \rightarrow V$ и
    $F_{2}: L^{\infty}(Q) \times L^{2}(0, T ; V) \rightarrow W$ определены следующим образом.
    Пусть $\varphi=F_{1}(\theta)$, если
    \begin{equation}
        \label{eq:1_6:6}
        A_{2} \varphi+\beta\left(\varphi-[\theta]^{4}\right)=g_{b}+g,
    \end{equation}
    и $\theta=F_{2}(\zeta, \varphi)$, если
    \begin{equation}
        \label{eq:1_6:7}
        \sigma \theta^{\prime}+A(\zeta, \theta)
        +b\left([\theta]^{4}-\varphi\right)=f_{b}+f
        \quad \text { п. в. в }(0, T), \quad \theta(0)=\theta_{i n}.
    \end{equation}
    Здесь
    $ (A(\zeta, \theta), v)=(k(\zeta) \nabla \theta, \nabla v) +\int_{\Gamma} p \theta v d \Gamma \quad \forall v \in V$,
    \[
        W=\left\{y \in L^{2}(0, T ; V): \sigma y^{\prime}=\sigma d y / d t \in L^{2}
        \left(0, T, V^{\prime}\right)\right\}.
    \]

\end{frame}

\begin{frame}
    \frametitle{Существование решения квазилинейной задачи}
    \begin{lemma}
        \label{lm:1_6:3}
        Если выполнены условия (k1)--(k4), то существует константа $C>0$,
        не зависящая от $m$, такая, что
        \begin{gather*}
            \left\|\varphi_{m}\right\|_{L^{2}(0, T ; V)} \leq C,
            \quad\left\|\theta_{m}\right\|_{L^{2}(0, T ; V)} \leq C, \\
            \int_{0}^{T-\delta}\left\|\theta_{m}(s+\delta)
            -\theta_{m}(s)\right\|^{2} d s \leq C \delta.
        \end{gather*}
    \end{lemma}

    \begin{equation*}
        \begin{aligned}
            & \theta_{m} \rightarrow \widehat{\theta} \text { слабо в } L^{2}(0, T ; V),
            \text { сильно в } L^{2}(0, T ; H), \\
            & \varphi_{m} \rightarrow \widehat{\varphi} \text { слабо в } L^{2}(0, T ; V).
        \end{aligned}
    \end{equation*}

    \begin{theorem}
        \label{th:1_6:1}
        Если выполнены условия (k1)--(k4), то существует хотя бы одно
        решение задачи~\eqref{eq:1_6:1}--\eqref{eq:1_6:3}.
    \end{theorem}

\end{frame}

\begin{frame}
    \frametitle{Теорема единственности и сходимость итеративного метода}
    \begin{theorem}
        Если выполнены условия (k1)--(k4) и $\theta_{*}, \varphi_{*}$ является
        решением задачи~\eqref{eq:1_6:1}--\eqref{eq:1_6:3}
        так, что $\theta_{*}, \nabla \theta_{*} \in L^{\infty}(Q)$,
        то других ограниченных решений этой задачи нет.
    \end{theorem}


    \begin{theorem}
        Если выполнены условия (k1)--(k4) и $\theta_{*}, \varphi_{*}$ является
        решением задачи~\eqref{eq:1_6:1}--\eqref{eq:1_6:3}
        так, что $\theta_{*}, \nabla \theta_{*} \in L^{\infty}(Q)$.
        Тогда для последовательностей~\eqref{eq:1_6:10} справедливы следующие сходимости:
        \[
            \theta_{m} \rightarrow \theta_{*} \quad \text { в } L^{2}(0, T ; V),
            \quad \varphi_{m} \rightarrow \varphi_{*} \quad \text { в } L^{2}(0, T ; V).
        \]
    \end{theorem}

\end{frame}


\section{Граничные обратные задачи и задачи с данными Коши}\label{sec:rev}

\subsection{Граничная обратная задача}\label{subsec:rev}
\begin{frame}
    \frametitle{Граничная обратная задача}
    Модель имеет следующий вид
    \begin{equation}
        \label{eq:2_1:initial}
        - a \Delta \theta + b \kappa_a(\theta ^ 3 | \theta | - \varphi) = 0,  \quad
        - \alpha \Delta \varphi + \kappa_a (\varphi - \theta ^3 | \theta |) = 0,
    \end{equation}
    и дополняется граничными условиями на
    $\Gamma \coloneqq \partial \Omega =\overline{\Gamma}_0 \cup \overline{\Gamma}_1 \cup \overline{\Gamma}_2$,
    где части границы $\Gamma_0, \Gamma_1, \Gamma_2$ не имеют пересечений:

    \begin{equation}
        \label{eq:2_1:initial-boundary}
        \begin{aligned}
            \Gamma &: \; a \partial_n \theta + \beta (\theta - \theta _b) = 0, \\
            \Gamma_0 \cup \Gamma_2 &: \; \alpha \partial_n \varphi
            + \gamma(\varphi - \theta_b ^4 ) = 0, \\
            \Gamma_1 &: \; \alpha \partial_n \varphi + u(\varphi - \theta_b ^4 ) = 0. \\
        \end{aligned}
    \end{equation}

    Функции $\gamma, \theta_b, \beta$ известны.
    Неизвестная функция $u$ характеризует отражающие свойства участка границы $\Gamma_1$.
    Предполагается, что $0 < u_1 \leq u \leq u_2$.

    \textbf{Обратная задача} заключается в отыскании тройки $\theta, \varphi, u$
    по дополнительному условию $\theta|_{\Gamma_2} = \theta_0$.

    \textbf{Экстремальная задача} заключается в минимизации функционала
    \[ J(\theta) = \frac{1}{2} \int_{\Gamma_2} (\theta - \theta_0)^2 d\Gamma. \]
\end{frame}


\begin{frame}
    \frametitle{Существование решения и условия оптимальности}
    Будем предполагать что исходные данные удовлетворяют условию
    \begin{itemize}
        \item $\text{(i)}\;\beta\in L^\infty(\Gamma); \gamma \in L^\infty(\Gamma_0\cup\Gamma_2);$
        $u_1, u_2 \in L^\infty(\Gamma_1);$
        $0 < \beta_0 \le \beta; 0 < \gamma_0 \le \gamma;\; \beta_0,\gamma_0=Const,$
        $0 \le u_1 \le u_2$.
    \end{itemize}
    \begin{theorem}
        Пусть выполняется условие (i).
        Тогда существует хотя бы одно решение задачи оптимального управления.
    \end{theorem}

    \begin{theorem}
        \label{th:2_1:2}
        Пусть $\hat{y}=\{\hat{\theta},\hat{\varphi} \} \in Y, \hat{u} \in U_{ad}$
        --- решение экстремальной задачи.
        Тогда существует пара $p = (p_1, p_2)$, $p \in Y$
        такая, что тройка $(\hat{y}, \hat{u}, p)$, удовлетворяет следующим условиям:
        \begin{gather*}
            A_1 p_1 + 4 |\hat{\theta}|^3 \kappa_a(b p_1 - p_2) = f_c,
            \;\; (f_c,v) = - \int_{\Gamma_2} (\hat{\theta} - \theta_0) v d\Gamma, \\
            A_2 p_2 + \kappa_a (p_2-b p_1) = g_c(p_2, \hat{u}),
            \;(g_c(p_2, \hat{u}), v) = -\int_{\Gamma_1} \hat{u} p_2 v d\Gamma, \\
            \int_{\Gamma_1} p_2 (\hat{\varphi} - \theta_b^4)(u-w) d\Gamma
            \leq 0 \quad \forall w \in U_{ad}.
        \end{gather*}
    \end{theorem}
\end{frame}

\subsection{Обратная задача с условиями типа Коши}\label{subsec:rev_koshi}
\begin{frame}
    \frametitle{Задача без краевых условий для интенсивности излучения}
    \begin{equation}
        \label{eq:2_2:eq1}
        - a \Delta \theta + b \kappa_a(\theta ^ 3 | \theta | - \varphi) = 0,  \quad
        - \alpha \Delta \varphi + \kappa_a (\varphi - \theta ^3 | \theta |) = 0,
    \end{equation}
    На $\Gamma$ известно температурное поле и тепловой поток:
    \begin{equation}
        \label{eq:2_2:bc2} \theta = \theta_b, \quad \partial_n\theta = q_b.
    \end{equation}
    Заменяем на <<искусственные>> краевые условия
    \begin{equation}
        \label{eq:2_2:bc3}
        a(\partial_n\theta+\theta) = r,\;\;
        \alpha(\partial_n\varphi+\varphi) = u \text{ на }\Gamma.
    \end{equation}
    Функция $r(x),\, x\in\Gamma$ является заданной, а неизвестная функция $u(x),\, x\in\Gamma$
    играет роль управления.

    \textbf{Экстремальная задача} заключается в отыскании тройки
    $\{\theta_\lambda,\varphi_\lambda,u_\lambda\}$ такой, что
    \begin{equation}
        \label{eq:2_2:cost}
        J_\lambda(\theta, u) = \frac{1}{2}\int\limits_\Gamma (\theta - \theta_b)^2 d\Gamma
        + \frac{\lambda}{2}\int\limits_\Gamma u^2 d\Gamma \rightarrow\inf
    \end{equation}
    на решениях краевой задачи.
\end{frame}

\begin{frame}
    Будем предполагать, что
    \begin{itemize}
        \item $(j) \;\; a,b,\alpha,\kappa_a, \lambda ={\textrm Const}> 0,$
        \item $(jj) \;\, \theta_b, \,q_b \in U,\;\; r=a(\theta_b+q_b)$.
    \end{itemize}
    Определим оператор ограничений $F(\theta, \varphi, u) : V \times V \times U \rightarrow V' \times V'$,
    \[
        F(\theta, \varphi, u) = \{ aA\theta + b \kappa_a ( [\theta]^4- \varphi) - Br,\;
        \alpha A \varphi + \kappa_a (\varphi -[\theta]^4) - Bu\}.
    \]


    \textbf{Задача $CP$.} Найти тройку $\{\theta, \varphi, u \} \in V \times V \times U$ такую, что
    \begin{equation}
        \label{eq:2_2:cp}
        J_\lambda(\theta, u) \equiv \frac{1}{2}\|\theta -\theta_b\|^2_\Gamma
        + \frac{\lambda}{2}\|u\|^2_\Gamma \rightarrow \inf,\;\; F(\theta, \varphi, u)=0.
    \end{equation}
\end{frame}

\begin{frame}
    \frametitle{Разрешимость задачи $CP$ и условия оптимальности}

    \begin{theorem}
        \label{th:2_2:1}
        Пусть выполняются условия $(j), (jj)$.
        Тогда существует решение задачи $CP$.
    \end{theorem}
    \begin{theorem}
        \label{th:2_2:2}
        Пусть выполняются условия (j),(jj).
        Если $\{\hat{\theta}, \hat{\varphi}, \hat{u}\}$ -- решение задачи $CP$,
        то существует единственная пара $\{p_1, p_2 \} \in V\times V$ такая, что
        \begin{equation}
            \label{eq:2_2:as}
            aAp_1 +4|\hat{\theta}|^3 \kappa_a(bp_1 - p_2) = B(\theta_b - \hat{\theta}), \;\;
            \alpha A p_2 + \kappa_a (p_2 - b p_1)=0
        \end{equation}
        и при этом $\lambda\hat{u} = p_2$.
    \end{theorem}
\end{frame}


\begin{frame}
    \frametitle{Аппроксимация задачи с условиями типа Коши}
    \begin{theorem}
        \label{th:2_2:3}
        Пусть выполняются условия (j),(jj) и существует решение
        задачи~\eqref{eq:2_2:eq1}--\eqref{eq:2_2:bc2}.
        Если $\{\theta_\lambda,\varphi_\lambda,u_\lambda\}$ -- решение
        задачи $CP$ для $\lambda>0$, то существует последовательность $\lambda\to +0$
        такая, что
        \[
            \theta_\lambda\rightarrow\theta_*, \;\; \varphi_\lambda\rightarrow\varphi_*
            \text{ слабо в }V,\text{ сильно в }H,
        \]
        где $\theta_*,\varphi_*$ -- решение задачи~\eqref{eq:2_2:eq1}--\eqref{eq:2_2:bc2}.
    \end{theorem}


    Из ограниченности последовательности $u_\lambda$
    в пространстве $U$ следует
    ее слабая относительная компактность и существование последовательности
    (возможно не единственной) $\lambda\to+0$ такой, что
    $u_\lambda \rightarrow u_*$ слабо в $U$.

%    Для практического решения задачи~\eqref{eq:2_2:eq1}--\eqref{eq:2_2:bc2} важно то,
%    что \textit{для любой последовательности} $\lambda\to+0$ справедлива оценка
%    $\|\theta_\lambda -\theta_b\|^2_\Gamma\leq C\lambda$,
%    а поскольку $\partial_n\theta_\lambda=\theta_b+q_b-\theta_\lambda$,
%    то также $\|\partial_n\theta_\lambda-q_b\|^2_\Gamma\leq C\lambda$.
%    Указанные неравенства гарантируют, что граничные значения
%    $\theta_\lambda,\,\partial_n\theta_\lambda$ при малых $\lambda$
%    аппроксимируют краевые условия задачи~\eqref{eq:2_2:eq1}--\eqref{eq:2_2:bc2}.
\end{frame}

\subsection{Квазистационарная задача с данными Коши}\label{subsec:qst_koshi}
\begin{frame}
    \frametitle{Квазистационарная модель с данными Коши}
    \begin{equation}
        \label{eq:2_3:1}
        \begin{split}
            & \frac{\partial \theta}{\partial t} - a \Delta \theta
            + b \kappa_{a} \left(|\theta| \theta^{3}-\varphi\right) = 0,\\
            & - \alpha \Delta \varphi
            + \kappa_{a} \left(\varphi-|\theta| \theta^{3}\right) = 0,
            \quad x \in \Omega, \quad 0 < t < T;
        \end{split}
    \end{equation}
    \begin{align}
        a \left(\partial_{n} \theta+\theta\right)=r,
        & \quad \alpha\left(\partial_{n} \varphi
        + \varphi\right) = u \text { на } \Gamma;  \label{eq:2_3:2}\\
        & \left.\theta\right|_{t=0} = \theta_{0}. \label{eq:2_3:3}
    \end{align}


    \textbf{Экстремальная задача} состоит в том, чтобы найти тройку
    $\left\{\theta_{\lambda}, \varphi_{\lambda}, u_{\lambda}\right\}$ такую, что
    \begin{equation}
        \label{eq:2_3:4}
        J_{\lambda}(\theta, u)=\frac{1}{2} \int_{0}^{T}
        \int_{\Gamma}\left(\theta-\theta_{b}\right)^{2} d \Gamma d t+\frac{\lambda}{2}
        \int_{0}^{T} \int_{\Gamma} u^{2} d \Gamma d t \rightarrow \inf
    \end{equation}
    на решениях задачи~\eqref{eq:2_3:1}--\eqref{eq:2_3:3}.
\end{frame}

\begin{frame}
    \frametitle{Задача оптимального управления $OC$}
    Будем считать, что
    \begin{itemize}
        \item $(k)\; a, b, \alpha, \kappa_{a}, \lambda=$ Const $>0$,
        \item $(kk)\; \theta_{b}, q_{b} \in U, r=a\left(\theta_{b}+q_{b}\right)
        \in L^{5}(\Sigma), \; \theta_{0} \in L^{5}(\Omega)$.
    \end{itemize}

    Задача оптимального управления $OC$ заключается в отыскании тройки
    $\{\theta, \varphi, u\} \in W \times L^{2}(0, T ; V) \times U$ такой, что
    \[
        J_{\lambda}(\theta, u) \equiv \frac{1}{2}\left\|\theta-
        \theta_{b}\right\|_{\Sigma}^{2}+
        \frac{\lambda}{2}\|u\|_{\Sigma}^{2}
        \rightarrow \inf, \quad F(\theta, \varphi, u)=0.
    \]
    \begin{theorem}
        \label{th:2_3:1}
        Пусть выполняются условия $(k), (kk)$.
        Тогда существует решение задачи $OC$.
    \end{theorem}

\end{frame}

\begin{frame}
    \frametitle{Условия оптимальности и аппроксимация обратной задачи}

    \begin{theorem}
        \label{th:2_3:2}
        Пусть выполнены условия $(k), (kk)$.
        Если $\{\widehat{\theta}, \widehat{\varphi}, \widehat{u}\}$ — решение задачи $OC$,
        то существует единственная пара $\left\{p_ {1}, p_{2}\right\} \in W \times W$ такая, что
        \begin{equation}
            \label{eq:2_3:15}
            \begin{aligned}
                -p_{1}^{\prime}+a A p_{1}+4|\widehat{\theta}|^{3} \kappa_{a}\left(b p_{1}
                -p_{2}\right)&=B\left(\theta_{b}-\widehat{\theta}\right),
                p_{1}(T)=0, \\
                \alpha A p_{2}+\kappa_{a}\left(p_{2}-b p_{1}\right)&=0,
            \end{aligned}
        \end{equation}
        а также $\lambda \widehat{u}=\left.p_{2}\right|_{\Sigma}$.
    \end{theorem}

    \begin{theorem}
        \label{th:2_3:3}
        Пусть выполняются условия $(k), (kk)$ и существует решение
        $\theta, \varphi \in$ $L^{2}\left(0, T ; H^{2}(\Omega) \right)$
        задачи~\eqref{eq:2_3:16},~\eqref{eq:2_3:17}.
        Если $\left\{\theta_{\lambda}, \varphi_{\lambda}, u_{\lambda}\right\}$
        — решение задачи $OC$ при $\lambda>0$, то при $\lambda\rightarrow+0$
        \[
            \begin{gathered}
                \theta_{\lambda} \rightarrow \theta \text { слабо в } L^{2}(0, T ; V),
                \text { сильно в } L^{2}(Q), \\
                \varphi_{\lambda} \rightarrow \varphi \text { слабо в } L^{2}(0, T ; V).
            \end{gathered}
        \]
    \end{theorem}
\end{frame}


\section{Задачи оптимального управления для квазилинейных моделей}\label{sec:opt}
\begin{frame}

\end{frame}


\section{Численные методы и комплексы программ}\label{sec:prog}
\begin{frame}

\end{frame}
