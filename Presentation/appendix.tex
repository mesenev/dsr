\begin{frame}
    \frametitle{Ответы на замечания ведущей организации ТОИ им. Ильичёва ДВО РАН}
\begin{enumerate}
        \item В разделе, посвящённом численному моделированию, целесообразно было бы привести
        более детальное сравнение вычислительной эффективности предложенных алгоритмов
        с другими известными методами.\\
        \textbf{Ответ:} С замечанием согласен.
        Сравнение с альтернативными методами может быть предметом дальнейших исследований.

        \item Недостаточно освещён вопрос о чувствительности предложенных методов
        к уровню шума во входных данных для обратных задач.\\
        \textbf{Ответ:} С замечанием согласен.

        \item В тексте автореферата присутствуют незначительные опечатки,
        не искажающие смысла изложения.\\
        \textbf{Ответ:} С замечанием согласен.
    \end{enumerate}
\end{frame}

\begin{frame}
    \frametitle{Ответы на замечания оф.\ оппонента к.ф.-м.н. Максимовой\,Н.\,Н.}
    \begin{enumerate}
        \item В постановке задач работы и в разделе «Научная новизна» используются неудачные формулировки
        ``получить численный алгоритм'', ``предложить численные методы решения'', и т.д.
        \textbf{Ответ:} С замечанием согласен.
%
        \item Диффузионное $P_1$ приближение уравнения переноса излучения при
        выводе уравнения (1.17) из (1.9) написано, что пренебрегают слагаемым $\frac{1}{c} \frac{\partial I*}{\partial t}$
        Исходя из каких соображений это сделано?\\
        \textbf{Ответ:}Из-за множителя $\frac{1}{c}$ данное слагаемое
        на несколько порядков меньше транспортного и поглощающего членов,
        что позволяет им пренебречь в рамках рассматриваемых моделей.
        Данное приближение стандартно для диффузионных моделей переноса излучения.

        \item В разных разделах по-разному нумеруются условия на исходные данные.
        \textbf{Ответ:} C замечанием согласен.
%
        \item По тексту диссертации по-разному обозначается производная по направлению внешней нормали: $\frac{\partial \theta}{\partial n}$ и $\partial_n \theta$. \textbf{Ответ:} С замечанием согласен.
%%
        \item В п.\ 1.5 ``Квазистационарная модель сложного теплообмена''
        рассматриваемая модель (1.31)–(1.34) не содержит источников (в отличии, например, от модели (1.41)–(1.43)
        в п.\ 1.6 ``Квазилинейная модель сложного теплообмена'').
        Можно ли рассматривать соответствующие задачи с источниками?
        Если да, то в чём будет состоять существенное отличие исследования таких задач?
        \\ \textbf{Ответ:}
        Включение источников тепла принципиально возможно,
        однако приводит к усложнению структуры обратной задачи.
        Предложенные численные методы могут быть обобщены на данный случай,
        но это требует отдельного исследования.
    \end{enumerate}
\end{frame}
%

\begin{frame}
    \frametitle{Ответы на замечания оф.\ оппонента к.ф.-м.н. Максимовой\,Н.\,Н.}
    \begin{enumerate}
        \setcounter{enumi}{5}
        \item Какой смысл несут обозначения задач управления: Задача $CP$, Задача $OP$, $P_\lambda$, $P$, $P_{\varepsilon}$, $CPP$? \\
        \textbf{Ответ:} Используемые обозначения отражают тип рассматриваемых задач и соответствуют принятой в работе классификации.


        \item В п.\ 4.1.2 ``Примеры численного решения краевой задачи (4.1)–(4.3)''
        стоило привести более подробное описание вычислительного эксперимента.
        Какие конкретно программные средства и серверы использовались?
        На какие конечные элементы разбивалась область, какой шаг сетки?
        Указано, что представлена визуализация состояний за 3 (в двумерном случае) и за 6 (в трёхмерном случае) итераций,
        но не приведён критерий останова итерационного процесса.\  \\
        \textbf{Ответ.} С замечанием согласен.
        В разделе 4.1.2 основное внимание уделялось демонстрации работоспособности предложенного численного подхода,
        в связи с чем описание вычислительного эксперимента носит сжатый характер.

        \item Замечания по визуализации результатов. \textbf{Ответ:} с замечанием согласен.

        \item При реализации метода градиентного спуска
        значение параметра сглаживания $\mu$ выбиралось согласованно со значением градиента
        таким образом, чтобы его изменение определяло значимую поправку для $\mu_k$.
        При этом для различных экспериментов приведены различные значения данного параметра,
        подобранные эмпирически.
        Отмечается, что целесообразно рассмотреть выбор шага по правилам нисходящего спуска,
        что может ускорить вычислительный процесс при отсутствии проблем с реализацией
        и дополнительных вычислительных затрат. \textbf{Ответ:} с замечанием согласен.
        Выбор шага градиентного спуска следовало осветить подробнее.
    \end{enumerate}
\end{frame}
\note{
\textit{Метод наискорейшего спуска с линейным поиском ориентирован
прежде всего на ускорение сходимости. В данной работе приоритетом являлась
устойчивость решения и демонстрация работоспособности метода,
    поэтому использовалась более консервативная стратегия выбора шага..}


}

\begin{frame}
    \frametitle{Ответы на замечания оф.\ оппонента д.ф.-м.н. Шишленина\,М.\,А.}
    Отмечена необходимость более чёткого пояснения физического смысла
    отдельных параметров в квазилинейных моделях и расширения обсуждения области
    применимости предложенных численных алгоритмов.\\
    \textbf{Ответ:} С замечанием согласен.
    \begin{enumerate}
        \item Пункт 1.3, содержащий определения и основы функционального анализа,
        является излишним.\\
        \textbf{Ответ:} С замечанием согласен.
        Раздел добавлен для удобства читателей.

        \item Постановки обратных задач сводятся к задачам минимизации функционалов.
        Как связаны решения дифференциальной и оптимизационной постановок обратных задач?
        \textbf{Ответ:} показано (например, в п.п. 2.3), что при стремлении параметра регуляризации к нулю, решение оптимизационной задачи сходится к решению обратной задачи.

        \item В чём заключается некорректность постановки обратной задачи (2.1)–(2.4)?\\
        \textbf{Ответ:} Некорректность обусловлена возможностью поставить начальные параметры, для которых не существует решение.
        Также неизвестен результат о единственности решений.

        \item Доказана единственность решения соответствующих обратных задач.
        Исследовался ли теоретически вопрос устойчивости?\\
        \textbf{Ответ:} Теоретический анализ устойчивости для рассматриваемых задач -- открытая проблема.

        \item В численных расчётах используются конкретные параметры среды.
        Насколько они соответствуют реальным материалам?
        Могут ли параметры среды быть функциями пространственных координат?\\
        \textbf{Ответ:} Используемые параметры соответствуют значениям стекла, воздуха.
        Параметры среды могут зависеть от пространственных координат.
    \end{enumerate}
\end{frame}

\begin{frame}
    \frametitle{Ответы на замечания оф.\ оппонента д.ф.-м.н. Шишленина\,М.\,А.}
    \begin{enumerate}
        \setcounter{enumi}{5}
        \item В пункте 4.2.2 применяется градиентный метод с проекцией для минимизации целевого
        функционала.
        Как полученное решение связано с решением дифференциальной
        постановки?
        Как выбирается параметр регуляризации в функционале Тихонова?
        Почему из сходимости по функционалу следует сильная сходимость градиентного
        метода?
        \textbf{Ответ:} В случае существования решения дифференциальной задачи,
        удовлетворяющей указанным ограничениям на неизвестную функцию $u$,
        то решение задачи оптимального управления аппроксимирует решение обратной задачи.
        Параметр регуляризации выбирается эмпирически, и, что важно, в приведённых
        результатах он мал.
        Теория сходимости градиентных методов для рассматриваемого класса задач в настоящее время отсутствует.

        \item Графики убывания функционала не являются наглядными.
        Было бы лучше, если бы были приведены графики убывания функционала в логарифмической шкале.\\
        \textbf{Ответ:} С замечанием согласен.

        \item Правильно было приводить двумерные сечения точного и приближенного решений
обратной задачи вместо трехмерной картинки.
        \textbf{Ответ:} С замечанием согласен.

        \item Не хватает более подробных описаний численных расчетов: размер сетки, условие
Куранта, какая схема использовалась для решения прямой и сопряженной задачи,
параметры компьютера, на котором проводился расчет.
         \textbf{Ответ:} С замечанием согласен.

        \item Не очень понятно, как численно исследовался вопрос устойчивости разработанных
алгоритмов.

        \textbf{Ответ:} Численный анализ устойчивости основан на тестировании алгоритмов для разных типов возмущений,
        включающие быстро осциллирующие.
        Тестирование показало устойчивость решения по крайне мере на указанных типах возмущений.
    \end{enumerate}
\end{frame}
