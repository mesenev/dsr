\documentclass[12pt]{article}

\usepackage[T2A]{fontenc}
\usepackage[utf8]{inputenc}
\usepackage[russian]{babel}
\usepackage{amsmath,amssymb}
\usepackage{geometry}
\geometry{a4paper, margin=2.5cm}

\begin{document}

    \begin{center}
        \Large
        \textbf{Ответы на замечания официального оппонента}
    \end{center}

    \vspace{1em}

    \noindent
    Благодарю официального оппонента за внимательное рассмотрение диссертационной работы и высказанные замечания. Все замечания были внимательно изучены и способствовали улучшению качества представления результатов.
    Ниже приведены ответы по каждому пункту.

    \vspace{1em}


    \section*{Ответ на замечание 1}

    Оппонент указывает на неудачные формулировки в постановке задач и в разделе «Научная новизна», в частности использование выражений вида «получить численный алгоритм», «предложить численные методы решения», «представлены априорные оценки».

    Замечание является справедливым.
    Указанные формулировки действительно носят процессуальный и описательный характер.

    \vspace{1em}


    \section*{Ответ на замечание 2}
    \begin{equation}
        \label{eq:1_1:9}
        \begin{aligned}
            &\frac{1}{c} \frac{\partial I^{*}(x, \omega, t)}{\partial t}
            +\omega \cdot \nabla_{x} I^{*}(x, \omega, t) +\kappa I^{*}(x, \omega, t)=\\
            &= \frac{\kappa_{s}}{4 \pi} \int_{S} P\left(\omega, \omega^{\prime}\right) I^{*}
            \left(x, \omega^{\prime}, t\right) d \omega^{\prime}+\kappa_{a} \theta^{4}(x, t),
        \end{aligned}
    \end{equation}
    \begin{equation}
        \label{eq:1_1:10}
        \begin{aligned}
            & \frac{\partial \theta(x, t)}{\partial t}
            - a \Delta \theta(x, t) = \\
            & = - b \kappa_{a}\left(\theta^{4}(x, t)-\frac{1}{4 \pi}
                                  \int_{S} I^{*}(x, \omega, t) d \omega\right)
            + \frac{b}{4 \pi c} \frac{\partial}{\partial t}
            \int_{S} I^{*}(x, \omega, t) d \boldsymbol{\omega},
        \end{aligned}
    \end{equation}
    В пункте 1.2 при выводе уравнения (1.17) из (1.9) указано, что пренебрегают слагаемым
    \[
        \frac{1}{c}\frac{\partial I^*}{\partial t},
    \]
    однако не приведено обоснование данного допущения.


    Данное допущение может быть обосновано оценкой порядка величины указанного члена.
    Введём характерный линейный размер области $L$ и характерное время изменения
    температурного поля $\tau_T$.
    Тогда имеем оценки
    \[
        \frac{\partial I^*}{\partial t} \sim \frac{I_0}{\tau_T},
        \qquad
        \nabla_x I^* \sim \frac{I_0}{L},
    \]
    где $I_0$ — характерная величина интенсивности излучения.

    Отношение временного члена к конвективному оценивается как
    \[
        \frac{
            \left| \frac{1}{c}\frac{\partial I^*}{\partial t} \right|
        }{
            \left| \omega \cdot \nabla_x I^* \right|
        }
        \sim
        \frac{L}{c\,\tau_T}.
    \]

    Для задач сложного теплообмена характерные значения параметров составляют
    $L = 10^{-2}\text{–}10^{0}$ м и $\tau_T = 10^{-1}\text{–}10^{3}$ с,
    в то время как скорость света $c \approx 3\cdot 10^{8}$ м/с.
    В этом случае
    \[
        \frac{L}{c\,\tau_T} = 10^{-6}\text{–}10^{-9},
    \]
    то есть временной член в уравнении переноса излучения
    на несколько порядков меньше конвективного и поглощающего членов.


    Таким образом, в рамках рассматриваемых режимов теплообмена
    использование квазистационарного приближения является обоснованным,
    а слагаемым
    $\frac{1}{c}\frac{\partial I^*}{\partial t}$
    можно пренебречь без потери точности модели.

    \vspace{1em}


    \section*{Ответ на замечание 3}

    Оппонент отмечает различную нумерацию условий на исходные данные в разных разделах работы.

    Замечание справедливое.
    Различия в нумерации условий носят исключительно редакционный характер.

    \vspace{1em}


    \section*{Ответ на замечание 4}

    В тексте диссертации используются различные обозначения производной по направлению внешней нормали:
    \[
        \frac{\partial \theta}{\partial n} \quad \text{и} \quad \partial_n \theta.
    \]

    Замечание справедливое.
    Действительно, в работе применяются эквивалентные обозначения одной и той же величины.

    \vspace{1em}


    \section*{Ответ на замечание 5}

    В пункте 1.5 рассматривается квазистационарная модель сложного теплообмена без внутренних источников тепла, в отличие от модели, приведённой в пункте 1.6. Оппонент задаётся вопросом о возможности рассмотрения соответствующих задач с источниками.

    В разделе 1.5 рассматривалась модель без внутренних источников тепла
    с целью сосредоточиться на анализе граничных обратных задач и их численного решения.
    Включение источников тепла принципиально возможно,
    однако приводит к усложнению структуры обратной задачи,
    появлению дополнительных неизвестных функций и усилению её некорректности.

    Исследование таких моделей представляет самостоятельное направление
    которое может быть рассмотрено как направление работы в дальнейшем.

    \vspace{1em}


    \section*{Ответ на замечание 6}

    Оппонент указывает на использование различных обозначений параметров
    регуляризации и шага градиентного метода ( $\lambda$, $\eta$, $\varepsilon$ )
    в разных разделах работы.

    Замечание принято.
    Следовало унифицировать обозначения параметров.

    \vspace{1em}


    \section*{Ответ на замечание 7}

    Оппонент задаёт вопрос о смысле используемых обозначений задач
    управления и обратных задач (CP, OP, $P_\lambda$, $P$, $P_\varepsilon$, CPP).

    Используемые обозначения отражают тип рассматриваемых задач и
    соответствуют принятой в работе классификации
    (краевая задача Коши, оптимизационная постановка,
    регуляризованная задача, аппроксимированная задача и т.д.).

    Данные обозначения целесообразно было более
    явно пояснить при их первом введении.

    \vspace{1em}


    \section*{Ответ на замечание 8}

    В пункте 4.1.2 приведены примеры численного решения краевой задачи,
    однако описание вычислительного эксперимента,
    по мнению оппонента, является недостаточно подробным.

    Замечание справедливое.
    В разделе 4.1.2 основное внимание уделялось демонстрации работоспособности
    предложенного численного подхода,
    в связи с чем описание вычислительного эксперимента носит сжатый характер.

    \vspace{1em}


    \section*{Ответ на замечание 9}
    Оппонент отмечает следующие замечания по визуализации результатов:\\
    на рис.~4.3 отсутствует шкала значений;\\
    на рис.~4.5 по оси абсцисс отмечен параметр $k(\theta)$, однако остаётся неясным,
    каково его влияние на норму температурного поля;\\
    на рис.~4.7 на рис.~4.7 использовано неудачное обозначение величин,
    приведённых в подписи к рисунку.

    Замечания по визуализации результатов являются справедливыми.
    Визуальная составляющая представленных численных экспериментов
    действительно могла быть усилена и уточнена.

    \vspace{1em}


    \section*{Ответ на замечание 10}
    Оппонент указывает, что при реализации метода градиентного спуска
    значение параметра сглаживания $\mu$ выбиралось согласованно со значением градиента
    таким образом, чтобы его изменение определяло значимую поправку для $\mu_k$.
    При этом для различных экспериментов приведены различные значения данного параметра,
    подобранные эмпирически.
    Отмечается, что целесообразно рассмотреть выбор шага по правилам нисходящего спуска,
    что может ускорить вычислительный процесс при отсутствии проблем с реализацией
    и дополнительных вычислительных затрат.

    Замечание является справедливым.
    Выбор шага градиентного метода и параметра сглаживания
    действительно следовало описать более подробно для каждого численного эксперимента.
    \vspace{1em}

    \noindent
    Автор выражает благодарность официальному оппоненту за конструктивные замечания,
    которые способствовали улучшению качества изложения диссертационной работы.

\end{document}
