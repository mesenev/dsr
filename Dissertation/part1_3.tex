\section{Математический аппарат моделирования сложного теплообмена}
\label{sec:ch1/sec6}
Предполагается, что $\Omega \subset \mathbb{R}^3$ — ограниченная
область с липшицевой границей $\Gamma$~\cite[с.~232]{Zeidler1986},
$Q = \Omega \times (0, T)$, $\Sigma = \Gamma \times (0, T)$.
Вектор внешней нормали к границе области обозначается через $\mathbf{n}$.

Введем следующие обозначения: $|\Omega|$ — объем области $\Omega$,
$|\Gamma|$ — площадь границы $\Gamma$, $\mu(E)$ — мера множества $E$;
$L^p(\Omega)$, $L^p(\Gamma)$, $1 \le p \le \infty$ — пространства Лебега;
$H^s(\Omega)$ — пространство Соболева $W^{s}_{2}(\Omega)$,
$H^1_0(\Omega) = \{v \in H^1(\Omega) : v|_\Gamma = 0\}$,
$C^\infty_0(\Omega)$ — пространство функций класса $C^\infty$
с компактным носителем в $\Omega$; аналогично определяются пространства
вектор-функций $L^p(\Omega)$, $H^s(\Omega)$, $H^1_0(\Omega)$,
$C^\infty_0(\Omega)$; $L^p(0, T; X)$ — пространство Лебега функций
со значениями в банаховом пространстве $X$, $C([0, T]; X)$ — пространство
функций, непрерывных на $[0, T]$, со значениями в $X$, $X'$ — пространство,
сопряженное с пространством $X$.
Если $y \in L^p(0, T; X)$, то обозначаем $y' = \frac{dy}{dt}$.

Обозначим $H = L^2(\Omega)$, $V = H^1(\Omega)$.
Пространство $H$ отождествляем с пространством $H'$,
так что $V \subset H = H' \subset V'$.
Через $\| \cdot \|$ будем обозначать норму в $H$,
а через $(f, v)$ — значение функционала $f \in V'$ на элементе
$v \in V$, совпадающее со скалярным произведением в $H$,
если $f \in H$.
Определим пространство $W = \{ y \in L^2(0, T; V):
y' \in L^2(0, T, V') \}$.

Скалярное произведение в $H$ и нормы в $H$,
$V$, $V'$ определяются формулами:
\begin{align*}
(f, g)
    &= \int_\Omega f(x)g(x) dx, \| f \|^2 = (f, f), \\
    \| f \|^2_{V} &= \| f \|^2 + \| \nabla f \|^2, \\
    \| f \|_{V'} &= \sup \{ (f, v):
    v \in V, \|v\|_{V} = 1 \}.
\end{align*}


\begin{lemma}[неравенство Гёльдера]
    \label{lemma:hoelder}~\cite[с.~35]{Zeidler1990b}
    Пусть $p, q \in [1, \infty]$, $\frac{1}{p} + \frac{1}{q} = 1$.
    Тогда для любых $u \in L^p(\Omega)$, $v \in L^q(\Omega)$
    выполнено неравенство
    \[
        \left| \int_\Omega uv \, dx \right|
        \le \|u\|_{L^p(\Omega)} \|v\|_{L^q(\Omega)}.
    \]
\end{lemma}

\begin{lemma}[неравенство Юнга]
    \label{lemma:young}~\cite[с.~38]{Zeidler1990a}
    Пусть $p, q > 1$, $\frac{1}{p} + \frac{1}{q} = 1$.
    Тогда для любых $a, b \ge 0$, $\epsilon > 0$
    справедливо неравенство
    \[
        ab \le \frac{\epsilon^p a^p}{p} + \frac{b^q}{\epsilon^q}.
    \]
\end{lemma}

\begin{lemma}
    \label{lemma:reflexive}~\cite[с.~254]{Zeidler1990a}
    Всякое гильбертово пространство рефлексивно.
\end{lemma}

\begin{lemma}
    \label{lemma:reflexive_weak}~\cite[с.~255]{Zeidler1990a}
    Пусть $X$ — рефлексивное банахово пространство.
    Если последовательность $x_n \in X$ ограничена,
    то в ней существует слабо сходящаяся подпоследовательность.
\end{lemma}

\begin{lemma}
    \label{lemma:separable_star_weak}~\cite[260]{Zeidler1990a}
    Пусть $X$ — сепарабельное банахово пространство.
    Если последовательность $f_n \in X'$ ограничена,
    то в ней существует *-слабо сходящаяся подпоследовательность.
\end{lemma}

\begin{lemma}
    \label{lemma:weak_limit}~\cite[258]{Zeidler1990a}
    Пусть $X$ — банахово пространство.
    Если $x_n \in X$, $x_n \rightharpoonup x$ слабо в $X$,
    то последовательность $x_n$ ограничена и
    $\|x\| \le \lim\limits_{n \to \infty} \|x_n\|$.
\end{lemma}

\begin{lemma}
    \label{lemma:convex_closed}~\cite[47]{Troeltzsch2010}
    Пусть $U$ — выпуклое, замкнутое множество в банаховом пространстве
    $X$, $x_n \in U$, $x_n \rightharpoonup x$ слабо в $X$.
    Тогда $x \in U$.
\end{lemma}

\begin{definition}
    \label{def:compact_set}~\cite[с.~48]{Lusternik1982}
    Множество $M$ банахова пространства $X$ называется компактным,
    если из всякого бесконечного подмножества множества $M$ можно
    выделить подпоследовательность,
    сходящуюся к некоторой точке этого множества.
    Множество называется относительно компактным,
    если его замыкание компактно.
\end{definition}

\begin{definition}
    \label{def:compact_operator}~\cite[с.~190]{Lusternik1982}
    Пусть $X, Y$ — банаховы пространства.
    Оператор $A : M \subset X \to Y$ называется компактным,
    если он переводит всякое ограниченное подмножество множества
    $M$ в относительно компактное множество пространства $Y$.
    Если, кроме того, оператор $A$ непрерывен,
    то он называется вполне непрерывным.
\end{definition}

\begin{theorem}[Принцип Шаудера]
    \label{thm:schauder}~\cite[с.~193]{Lusternik1982}
    Пусть $X$ — банахово пространство, $M \subset X$ — ограниченное выпуклое
    замкнутое множество, $A : M \to M$ — вполне непрерывный оператор.
    Тогда оператор $A$ имеет неподвижную точку $x \in M$, т. е. $Ax = x$.
\end{theorem}


\begin{theorem}
    \label{thm:compactness}~\cite[теорема 3]{Simon1986}
    Пусть $F$ — ограниченное множество в $L^2(0, T; V)$, и
    \[
        \int_0^{T-h} \|f(t+h) - f(t)\|^2 dt \to 0
    \]
    при $h \to 0$ равномерно относительно $f \in F$.
    Тогда $F$ относительно компактно в $L^2(0, T; H)$.
\end{theorem}

\begin{theorem}[Гильберт, Шмидт]
    \label{thm:hilbert-schmidt}~\cite[с. 263]{Kolmogorov2004}
    Пусть $A$ — линейный, самосопряженный, компактный оператор
    в гильбертовом пространстве $X$.
    Тогда множество собственных элементов оператора $A$ образует
    ортогональный базис в $X$.
\end{theorem}

\begin{theorem}[Лакс, Мильграм]
    \label{thm:lax-milgram}~\cite[с. 40]{Oben1977}
    Пусть $X$ — гильбертово пространство,
    $B : X \times X \to \mathbb{R}$ — непрерывная билинейная форма,
    $B(x, x) \geq C\|x\|^2_X$, $C > 0$.
    Тогда для любого $f \in X'$ задача
    $B(x, z) = (f, z) \quad \forall z \in X$
    имеет единственное решение $x \in X$.
\end{theorem}

\begin{definition}
    \label{def:weakly_semicontinuous}~\cite[с. 47]{Troeltzsch2010}
    Пусть $X$ — банахово пространство.
    Функционал $J : X \to \mathbb{R}$ называется слабо полунепрерывным
    снизу, если для любой последовательности $x_n \in X$ такой,
    что $x_n \rightharpoonup x$ слабо, выполняется неравенство
    $J(x) \leq \lim_{n \to \infty} J(x_n)$.
\end{definition}

\begin{lemma}
    \label{lem:weakly_semicontinuous_functional}~\cite[с. 47]{Troeltzsch2010}
    Пусть $X$ — банахово пространство.
    Если функционал $J : X \to \mathbb{R}$ непрерывный и выпуклый,
    то он слабо полунепрерывен снизу.
\end{lemma}

\begin{corollary}
    \label{cor:weakly_semicontinuous_norm}
    Пусть $X$ — банахово пространство, $a \in X$.
    Тогда функционал $J : X \to \mathbb{R}$, $J(x) = \|x - a\|^2_X$
    слабо полунепрерывен снизу.
\end{corollary}

\begin{lemma}
    \label{lem:embedding_Lp_Ls}~\cite[с. 37]{Zeidler1990a}
    Вложение $L^p(\Omega) \subset L^s(\Omega)$ непрерывно при
    $1 \leq s \leq p \leq \infty$.
\end{lemma}

\begin{theorem}[Лебег]
    \label{th:lebeg}~\cite[321]{Kolmogorov2004}
    Пусть $\Omega \subset \mathbb{R}^N$, $f_n \to f$ п.\ в. в $\Omega$,
    $\forall n : |f_n(x)| \leq \varphi(x)$ п.\ в. в $\Omega$,
    где $\varphi \in L^1(\Omega)$.
    Тогда $f \in L^1(\Omega)$, $\int_{\Omega} f_n(x) \, dx \to \int_{\Omega} f(x) \, dx$.
\end{theorem}


\begin{lemma}
    \label{lm:1_7:15}\cite[47]{Ziemer1989}
    Пусть $u \in H^{1}(\Omega)$, $\nabla u = 0$ n. в. в $\Omega$.
    Тогда $u = \text{const} \text{ в } \Omega$.
\end{lemma}

\begin{lemma}
    \label{lm:1_7:embedding}\cite[1026]{Zeidler1990b}
    Вложение $H^{1}(\Omega) \subset L^{p}(\Omega)$ непрерывно
    при $1 \leq p \leq 6$ и компактно при $1 \leq p<6$.
\end{lemma}

%TODO: refactor further (limit reached)

\begin{lemma}
    \label{lm:1_7:19}\cite[239]{Zeidler1990a}
    Оператор следа $\gamma: H^{1}(\Omega) \rightarrow L^{2}(\Gamma)$ непрерывен.
\end{lemma}

\begin{lemma}
    \label{lm:1_7:20}\cite[4]{girault1979finite}
    Образ оператора следа $\gamma: H^{1}(\Omega) \rightarrow L^{2}(\Gamma)$
    -- плотное подпространство пространства $L^{2}(\Gamma)$.
\end{lemma}

\begin{lemma}
    \label{lm:1_7:23}\cite[41]{grisvard1985elliptic}
    Для любого $\varepsilon>0$ существует постоянная $C_{\varepsilon}>0$ такая,
    что для любой функции $u \in H^{1}(\Omega)$ выполняется неравенство
    \[
        \|u\|_{L^{2}(\Gamma)}^{2} \leq \varepsilon\|\nabla u\|^{2}
        +C_{\varepsilon}\|u\|^{2}.
    \]
\end{lemma}

\begin{lemma}[Гронуолл]
    \label{lm:1_7:24}\cite[191]{Gaevskii1978}
    Пусть $f:[0, T] \rightarrow \mathbb{R}$ -- непрерывная функция $b \geq 0$.
    Если
    \[
        f(t) \leq a+b \int_{0}^{t} f(\tau) d \tau \quad \forall t \in[0, T],
    \]
    то $f(t) \leq a e^{bt} \; \forall \; t \in[0, T]$.
\end{lemma}

\begin{lemma}
    \label{lm:1_7:28}~\cite[411]{Zeidler1990a}
    Если $X$ -- рефлексивное и сепарабельное банахово пространство,
    то $L^{p}(0, T ; X), 1<p<\infty$ -- рефлексивное
    и сепарабельное банахово пространство.
\end{lemma}

\begin{lemma}
    \label{lm:1_7:29}\cite[449]{Zeidler1990a}
    Если $X$ -- рефлексивное и сепарабельное банахово пространство,
    то $L^{1}(0, T ; X)$ -- сепарабельное банахово пространство.
\end{lemma}

\begin{lemma}
    \label{lm:1_7:31}\cite[423]{Zeidler1990a}
    Для любой функции и $\in W$ справедливо равенство
    \[
        \|u(t)\|^{2}-\|u(0)\|^{2}=2 \int_{0}^{t}\left(u^{\prime}(\tau),
        u(\tau)\right) d \tau, \quad 0 \leq t \leq T.
    \]
\end{lemma}

\begin{lemma}
    \label{lm:1_7:32}\cite[356]{Kolmogorov2004}
    Для любой функции и $\in W$ справедливо равенство
    \[
        \frac{d\|u(t)\|^{2}}{d t}=2\left(u^{\prime}(t),
        u(t)\right) \, \text { п.\ в.\ нa }(0, T).
    \]
\end{lemma}
