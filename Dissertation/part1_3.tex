\section{Стационарная модель сложного теплообмена}\label{sec:ch1/sec3}
%TODO: rework it!!!

\subsection{Постановка краевой задачи}\label{subsec:ch1/sec3/state}

Стационарная нормализованная диффузионная модель, описывающая
радиационный, кондуктивный и конвективный теплообмен в
ограниченной области $G\subset \mathbb{R}^3$,
имеет следующий вид~\cite{modest2013radiative}:

\begin{equation}
    \label{eq:1_3:4-1}
    -a \Delta \theta + \textbf{v} \cdot \nabla \theta
    + b \mu_a \theta^4 =  b \mu_a \varphi,
\end{equation}

\begin{equation}
    \label{eq:1_3:4-2}
    - \alpha \Delta \varphi + \mu_a \varphi = \mu_a \theta^4.
\end{equation}

Здесь $\theta$ -- нормализованная температура, $\varphi$ --
нормализованная интенсивность излучения, усредненная по всем
направлениям, $\textbf{v}$ -- заданное поле скоростей, $\mu_a$ --
коэффициент поглощения.
Постоянные $a$, $b$ и $\alpha$
определяются следующим образом:
\[
    a=\frac{k}{\rho c_v},\quad b = \frac{4\sigma n^2 T_{\max}^3}{\rho c_v},
    \quad \alpha=\frac{1}{3\mu - A \mu_s},
\]
где $k$ -- теплопроводность, $c_v$ -- удельная теплоемкость, $\rho$ --
плотность, $\sigma$ -- постоянная Стефана-Больцмана, $n$ --
показатель преломления, $T_{\max}$ -- максимальная температура в
ненормализованной модели, $\mu = \mu_s + \mu_a$ -- коэффициент
полного взаимодействия, $\mu_s$ -- коэффициент рассеяния.
Коэффициент $A \in [-1, 1]$ описывает анизотропию рассеяния, случай
$A=0$ соответствует изотропному рассеянию.

Будем предполагать, что функции $\theta$ и $\varphi$, описывающие
процесс сложного теплообмена, удовлетворяют следующим условиям на
границе $\Gamma = \partial G$:
\begin{equation}
    \label{eq:1_3:4-3}
    \theta|_{\Gamma} = \Theta_0,
\end{equation}
\begin{equation}
    \label{eq:1_3:4-4}
    \alpha \frac{\partial \varphi}{\partial \mathbf{n}} + \beta
    (\varphi-\Theta_0^4)|_{\Gamma} = 0.
\end{equation}
Здесь через $\partial/\partial \mathbf{n}$ обозначаем производную
в направлении внешней нормали.
Неотрицательная функция
$\Theta_{0}$, определенная на $\Gamma$,  и функция $\beta$,
описывающая, в частности, отражающие свойства границы $\Gamma$,
являются заданными.


Основные результаты, представленные в данном параграфе, состоят в
получении новых априорных оценок решения
краевой задачи~\eqref{eq:1_3:4-1}--\eqref{eq:1_3:4-4},
на основе которых доказана разрешимость
задачи и выведены достаточные условия единственности решения.
Кроме того, найдены условия на параметры модели, геометрию области
$G$ и поле скоростей $\textbf{v}$, гарантирующие однозначную
разрешимость в случае равномерного потока среды.
Результаты
теоретического анализа иллюстрируются примерами численного
моделирования температурных полей в канале прямоугольной формы.

\subsection{Слабое решение краевой задачи}\label{subsec:ch1/sec3/weak}

Пусть $G$ -- липшицева ограниченная область, граница $\Gamma$
которой состоит из конечного числа гладких
кусков, а исходные данные удовлетворяют условиям: \\
$(i) \;\;\;\; \textbf{v} \in H^1(G) \cap L^{\infty}(G), \quad
\nabla \cdot \textbf{v} = 0;$ \\
$(ii) \;\; \Theta_0 \in L^{\infty}(\Gamma), \; 0 \leq m \leq
\Theta_0 \leq M; \; \exists \;\widetilde{\theta} \in H^1(G), \;
\widetilde{\theta}|_{\Gamma} = \Theta_0, \; m \leq
\widetilde{\theta} \leq M;$ \\
$(iii) \; \beta \in L^{\infty}(\Gamma), \; \beta \geq \beta_0>0.$ \\

Здесь и далее через $L^p$, $1 \leq p \leq \infty$, обозначаем
пространство Лебега, а через $H^s$ -- пространство Соболева
$W^s_2$.
Через $(\cdot,\cdot)$ обозначаем скалярное произведение в $L^2(G)$,
\[
    (f,g) = \int_G f(r)g(r)dr, \quad \|f\|^2=(f,f).
\]

Кроме этого будем использовать пространство
\[
    H^1_0(G) = \{ \eta \in H^1(G): \; \eta|_{\Gamma}=0\}
\]
с нормой $\|\eta\|_{H^1_0(G)}=(\nabla \eta, \nabla \eta)^{1/2}$.

\begin{definition}
    Пара $\{\theta, \varphi\} \in H^1(G) \times H^1(G)$ называется
    слабым решением задачи $(\ref{eq:1_3:4-1})$-$(\ref{eq:1_3:4-4})$, если
    \begin{equation}
        \label{eq:1_3:4-5}
        a(\nabla\theta, \nabla\eta) + (\mathbf{v}\cdot\nabla\theta +
        b\mu_a(\theta^4-\varphi), \eta) = 0 \quad \forall \eta \in
        H^1_0(G),
    \end{equation}
    \begin{equation}
        \label{eq:1_3:4-6}
        \alpha (\nabla\varphi, \nabla \psi) + \mu_a(\varphi - \theta^4,
        \psi)+ \int \limits_{\Gamma} \beta (\varphi-\Theta_0^4)\psi
        d\Gamma = 0 \quad \forall \psi \in H^1(G)
    \end{equation}
    и при этом $\theta|_{\Gamma} = \Theta_0$.
\end{definition}

Отметим, что в силу вложения $H^1(G) \subset L^6(G)$ выражение
$(\theta^4, \eta)$  имеет смысл для любой функции $\eta \in
H^1(G)$.

\begin{theorem}
    \label{th:1_3:weakExist}
    Пусть выполняются условия (i)-(iii).
    Тогда существует слабое
    решение задачи $(\ref{eq:1_3:4-1})$-$(\ref{eq:1_3:4-4})$, удовлетворяющее
    условиям
    \begin{equation}
        \label{eq:1_3:4-7}
        m \leq \theta \leq M, \quad m^4 \leq \varphi < M^4.
    \end{equation}
\end{theorem}
Доказательство теоремы~\ref{th:1_3:weakExist} основано на построении
операторного уравнения, которое определяет слабое решение
задачи~\eqref{eq:1_3:4-1}--\eqref{eq:1_3:4-4},
обосновании слабого принципа максимума и
применении принципа Лере-Шаудера.


%\subsection{Построение операторного уравнения}

Рассмотрим пространство $V = H^1_0(G) \times H^1(G)$.
Скалярное произведение в $V$ удобно выбрать следующим образом:
\[
    ((y,z)) = a(\nabla \zeta, \nabla \eta) + \alpha(\nabla \varphi,
    \nabla \psi)+\int \limits_{\Gamma} \beta \varphi\psi d\Gamma,
\]
где $y = \{\zeta, \varphi\} \in V, z= \{\eta, \psi\} \in V.$
Отметим, что норма пространства $V$, соответствующая выбранному
скалярному произведению, эквивалентна норме пространства
$H^1(G) \times H^1(G)$.
Через $\widetilde{y}$ обозначим элемент пространства $V$ такой, что
\[
    ((\widetilde{y}, z)) = a(\nabla\widetilde{\theta}, \nabla
    \eta)-\int \limits_{\Gamma} \beta \Theta_0^4\psi d\Gamma \quad
    \forall z = \{\eta, \psi\} \in V,
\]
где $\widetilde{\theta}$ -- функция из условия $(ii)$.

Определим нелинейный оператор $F: V \to V$, используя равенство
\begin{equation}
    \label{eq:1_3:4-8}
    ((F(y),z))=(\textbf{v}\cdot\nabla\theta, \eta) + \mu_a b
    \left(\theta^4-\varphi, \eta)+\mu_a(\varphi-\theta^4, \psi \right),
\end{equation}
справедливое для всех $y=\{\zeta, \varphi\}, z=\{\eta, \psi\} \in V$.
Здесь $\theta = \widetilde{\theta} + \zeta$.

Из определения скалярного произведения в пространстве $V$ и
соотношения~\eqref{eq:1_3:4-8} следует утверждение.

\begin{lemma}
    \label{lm:1_3:weak}
    Пара $\{\theta, \varphi\} \in H^1(G) \times H^1(G)$ является
    слабым решением задачи $(\ref{eq:1_3:4-1})$-$(\ref{eq:1_3:4-4})$, если и только
    если элемент $y=\{\theta-\widetilde{\theta}, \varphi\} \in V$
    удовлетворяет в пространстве $V$ уравнению
    \begin{equation}
        \label{eq:1_3:4-9}
        y + \widetilde{y} + F(y) = 0.
    \end{equation}
\end{lemma}

\subsection{Разрешимость краевой задачи}\label{subsec:ch1/sec3/solvability}

Для доказательства разрешимости уравнения~\eqref{eq:1_3:4-9}
предварительно рассмотрим в пространстве $V$ уравнение
\begin{equation}
    \label{eq:1_3:4-10}
    y + \widetilde{y} + F_M(y) = 0.
\end{equation}
Здесь оператор $F_M: V \to V$ определяется равенством
\[
    ((F_M(y),z))=(\textbf{v}\cdot\nabla\theta, \eta) + \mu_a b
    (|\theta|\theta^3-g(\varphi),
    \eta)+\mu_a(\varphi-|\theta|\theta^3, \psi)
\]
для всех $y=\{\zeta, \varphi\}, z=\{\eta, \psi\} \in V$, где
$\theta = \widetilde{\theta}+ \zeta$,
\[
    g(t) = \left\{
    \begin{array}{lll}
        m^4, & \text{ при } \quad t < 0,           \\
        t,   & \text{ при } \quad t \in [0, M^4], \\
        M^4, & \text{ при } \quad t > M^4.
    \end{array}
    \right.
\]
Отметим, что если $y=\{\zeta, \varphi\}$ есть решение~\eqref{eq:1_3:4-10},
удовлетворяющее условиям $m\leq \theta\leq M$,
$m^4 \leq \varphi \leq M^4$,
то $y$ является решением уравнения~\eqref{eq:1_3:4-9},
и поэтому пара $\{\theta, \varphi \}$ будет слабым
решением задачи~\eqref{eq:1_3:4-1}--\eqref{eq:1_3:4-4}.

\begin{lemma}
    \label{lemma:4-2}
    Оператор $F_M: V \to V$ вполне непрерывен.
\end{lemma}

\begin{proof}
    Пусть $y_1=\{\zeta_1, \varphi_1\}
    \in V$, $y_2=\{\zeta_2, \varphi_2\} \in V$,
    $\|\zeta_{1,2}\|_{H^1(G)} \leq \chi$, $\zeta = \zeta_1 - \zeta_2$,
    $\varphi = \varphi_1 - \varphi_2$, $\theta_{1,2} =
    \widetilde{\theta} + \zeta_{1,2}$, $z = \{\eta, \psi\} \in V$.
    Оценим разность
    \begin{gather*}
        ((F_M(y_1)-F_M(y_2), z)) = (\textbf{v}\cdot\nabla\zeta, \eta) +
        \mu_a
        b(|\theta_1|\theta_1^3-|\theta_2|\theta_2^3+g(\varphi_2)-g(\varphi_1),
        \eta)+\\
        \mu_a(\varphi+|\theta_2|\theta_2^3-|\theta_1|\theta_1^3, \psi)
        \leq \|\textbf{v}\|_{L^{\infty}(G)} \|\nabla\zeta\| \|\eta\| +\\
        2\mu_a b\left(\|\theta_1\|^3_{L^6(G)} +
        \|\theta_2\|^3_{L^6(G)}\right) \|\zeta\|_{L^4(G)}
        \|\eta\|_{L^4(G)} + \mu_a
        b\|\varphi\|\|\eta\|+\mu_a\|\varphi\|\|\psi\|+\\
    \end{gather*}
    \begin{equation}
        \label{eq:1_3:4-11}
        2\mu_a
        \left(\|\theta_1\|^3_{L^6(G)}
        + \|\theta_2\|^3_{L^6(G)}\right)\|\zeta\|_{L^4(G)}\|\psi\|_{L^4(G)}.
    \end{equation}
    Обозначим $h=F_M(y_1)-F_M(y_2)=\{h_1, h_2\} \in V$, и в
    неравенстве (\ref{eq:1_3:4-11}) положим $z=h$.
    Учтем непрерывность
    операторов вложения $H^1(G)$ в $L^s(G)$, $1 \leq s \leq 6$.
    Отметим, что указанные операторы компактны, если $1 \leq s < 6$.
    Тогда
    \begin{equation}
        \label{eq:1_3:4-12}
        \|h\|^2_V \leq C \left( \|\zeta\|\|\nabla h_1\| +
        \|\zeta\|_{L^4(G)}\|h\|_V\right).
    \end{equation}
    Здесь через $C>0$ обозначена постоянная, зависящая только от $a$,
    $b$, $\alpha$, $\beta$, $\chi$, $\|\textbf{v}\|_{L^{\infty}(G)}$ и
    норм соответствующих операторов вложения.
    Следствием оценки~\eqref{eq:1_3:4-12} является
    непрерывность оператора $F_M$, а также, в
    силу компактности вложения $H^1(G)$ в $L^2(G)$ и $L^4(G)$,
    компактность оператора $F_M$.
    Лемма доказана.
\end{proof}

Рассмотрим далее операторное уравнение с параметром
$\lambda \in (0,1]$:
\begin{equation}
    \label{eq:1_3:4-13}
    y_{\lambda} + \widetilde{y} + \lambda F_M(y_{\lambda}) = 0,
\end{equation}
для решения которого получим априорные оценки, равномерные по $\lambda$.

\begin{lemma}
    \label{lemma:4-3}
    Пусть $y_{\lambda}=\{\zeta_{\lambda}, \varphi_{\lambda}\} \in V$
    удовлетворяют $(\ref{eq:1_3:4-13})$, $\lambda \in (0,1]$.
    Тогда для
    функций $\theta_{\lambda} = \widetilde{\theta} + \zeta_{\lambda}
    \in H^1(G)$, $\varphi_{\lambda} \in H^1(G)$ справедливы оценки
    \begin{equation}
        \label{eq:1_3:4-14}
        m \leq \theta_{\lambda}(r) \leq M, \quad m^4 \leq
        \varphi_{\lambda}(r) \leq M^4, \quad r \in G.
    \end{equation}
\end{lemma}

\begin{proof}
    Умножим (\ref{eq:1_3:4-13}) скалярно в
    $V$ на элемент $z=\{\eta_{\lambda}, 0\} \in V$, где
    $\eta_{\lambda} = \max(\theta_{\lambda}-M, 0) \in H^1_0(G)$.
    Тогда
    \[
        a(\nabla\theta_{\lambda}, \nabla \eta_{\lambda}) +
        \lambda(\textbf{v}\cdot \nabla\theta_{\lambda}, \eta_{\lambda}) +
        \lambda\mu_a
        b(|\theta_{\lambda}|\theta_{\lambda}^3-g(\varphi_{\lambda}),
        \eta_{\lambda})=0.
    \]
    Учтем, что
    \begin{gather*}
    (\nabla\theta_{\lambda}, \nabla \eta_{\lambda})
        =
        \|\nabla\eta_{\lambda}\|^2, \quad (\textbf{v}\cdot
        \nabla\theta_{\lambda}, \eta_{\lambda})=(\textbf{v}\cdot
        \nabla\eta_{\lambda}, \eta_{\lambda})=\frac{1}{2}(\textbf{v},
        \nabla(\eta_{\lambda}^2))=0,\\
        (|\theta_{\lambda}|\theta_{\lambda}^3-g(\varphi_{\lambda}),
        \eta_{\lambda}) = \int \limits_{\theta_{\lambda}>M}
        (|\theta_{\lambda}|\theta_{\lambda}^3-g(\varphi_{\lambda}))(\theta_{\lambda}-M)dr
        \geq 0.\\
    \end{gather*}
    Поэтому $\eta_{\lambda}=0$ и соответственно $\theta_{\lambda} \leq
    M$ в области $G$.


    Далее, умножая~\eqref{eq:1_3:4-13} скалярно в $V$ на
    $z=\{0,\psi_{\lambda}\}$, где $\psi_{\lambda} = \max
    (\varphi_{\lambda}-M^4,0)$, получаем
    \[
        \alpha \|\nabla\psi_{\lambda}\|^2 + \lambda\mu_a \int
        \limits_{\varphi_{\lambda}>M^4}(\varphi_{\lambda}-|\theta_{\lambda}|\theta_{\lambda}^3)
        \psi_{\lambda}dr + \int \limits_{\Gamma}\beta
        (\varphi_{\lambda}-\Theta_0^4)\psi_{\lambda}d\Gamma =0.
    \]
    Из неотрицательности каждого слагаемого следует, что
    $\psi_{\lambda}=0$ и соответственно $\varphi_{\lambda} \leq M^4$ в
    области $G$.

    Аналогичным образом, умножая~\eqref{eq:1_3:4-13} скалярно в $V$ сначала
    на $z=\{\min(\theta_{\lambda}-m, 0), 0\} \in V$, затем на
    $z=\{0,\min(\varphi_{\lambda}-m^4,0)\} \in V$, приходим к
    неравенствам: $\theta_{\lambda} \geq m$, $\varphi_{\lambda} \geq
    m^4$.
\end{proof}


Полученные оценки позволяют доказать равномерную по $\lambda \in
(0,1]$ ограниченность решений уравнения~\eqref{eq:1_3:4-13} в
пространстве $V$.
Действительно, умножая~\eqref{eq:1_3:4-13} скалярно в
$V$ на $y_{\lambda} = \{\zeta_{\lambda}, \varphi_{\lambda}\} \in V$
и учитывая~\eqref{eq:1_3:4-14}, получаем
\begin{equation}
    \label{eq:1_3:4-15}
    \|y_{\lambda}\|^2_V + ((\widetilde{y}, y_{\lambda})) + \lambda
    (\textbf{v}\cdot \nabla\theta_{\lambda}, \zeta_{\lambda}) +
    \lambda\mu_a b(\theta_{\lambda}^4-\varphi_{\lambda},
    \theta_{\lambda}-\widetilde{\theta}) +\lambda \mu_a
    (\varphi_{\lambda}-\theta_{\lambda}^4, \varphi_{\lambda}) = 0.
\end{equation}
Заметим, что
\begin{gather*}
    |(\textbf{v} \cdot \nabla\theta_{\lambda}, \zeta_{\lambda})|=
    |(\textbf{v} \cdot \nabla\zeta_{\lambda}, \theta_{\lambda})| \leq
    \|\textbf{v}\|_{L^{\infty}(G)} M \|\nabla\zeta_{\lambda}\| \leq \\
    \leq
    \frac{M}{\sqrt{a}}\|\textbf{v}\|_{L^{\infty}(G)}\|y_{\lambda}\|_V
    \leq \frac{M^2}{a}\|\textbf{v}\|^2_{L^{\infty}(G)}
    + \frac{1}{4}\|y_{\lambda}\|^2_V.
\end{gather*}
С учетом~\eqref{eq:1_3:4-14} из~\eqref{eq:1_3:4-15} следует оценка
\[
    \|y_{\lambda}\|^2_V \leq \frac{1}{2} \|y_{\lambda}\|^2 +
    \frac{1}{2} \|\widetilde{y}\|^2_V +
    \frac{M^2}{a}\|\textbf{v}\|^2_{L^{\infty}(G)} +
    \frac{1}{4}\|y_{\lambda}\|^2_V + \mu_a b M^5 mes G + \mu_a M^8 mes
    G.
\]
Таким образом,
\begin{equation}
    \label{eq:1_3:4-16}
    \|y_{\lambda}\|^2_V \leq C,
\end{equation}
где
\[
    C= 2 \|\widetilde{y}\|^2_V +
    \frac{4M^2}{a}\|\textbf{v}\|^2_{L^{\infty}(G)} + 4 \mu_a M^5 mes
    G(b+M^3).
\]
Здесь через $mes G$ обозначен объем $G$.
Поскольку оператор $F_M:
V \to V$ вполне непрерывен, оценка~\eqref{eq:1_3:4-16} гарантирует по
теореме Лере-Шаудера разрешимость уравнения (\ref{eq:1_3:4-10}). На
основании леммы~\ref{lemma:4-3}, если $y=\{\zeta, \varphi\}$ --
решение~\eqref{eq:1_3:4-10}, то $m \leq \theta \leq M$,
$m^4 \leq \varphi \leq M^4$, где $\theta = \widetilde{\theta} + \zeta$.
Поэтому
$F_M(y) = F(y)$ и в силу леммы~\ref{lm:1_3:weak} пара
$\{\theta, \varphi\}$ является слабым решением задачи
(\ref{eq:1_3:4-1})-(\ref{eq:1_3:4-4}), что доказывает теорему~\ref{th:1_3:weakExist}

\subsection{Достаточные условия единственности решения}\label{subsec:ch1/sec3/uniqueness}

Получим условия, гарантирующие однозначную разрешимость задачи
(\ref{eq:1_3:4-1})--(\ref{eq:1_3:4-4}) в классе функций из $H^1(G)$,
удовлетворяющих ограничениям (\ref{eq:1_3:4-7}). Пусть  $\{\theta_1,
\varphi_1\}$ и $\{\theta_2, \varphi_2\}$ -- слабые решения задачи
(\ref{eq:1_3:4-1})--(\ref{eq:1_3:4-4}) такие, что $m \leq \theta_{1,2} \leq M$ в
области $G$.
Положим $\theta = \theta_1 - \theta_2$, $\varphi = \varphi_1 - \varphi_2$.
Из определения слабого решения вытекают равенства
\begin{equation}
    \label{eq:1_3:4-17}
    a(\nabla\theta, \nabla\eta) + (\textbf{v}\cdot\nabla\theta +
    b\mu_a(f\theta-\varphi), \eta) = 0 \quad \forall \eta \in
    H^1_0(G),
\end{equation}
\begin{equation}
    \label{eq:1_3:4-18}
    \alpha (\nabla\varphi, \nabla \psi) + \mu_a(\varphi - f\theta,
    \psi)+ \int \limits_{\Gamma} \beta \varphi\psi d\Gamma \quad
    \forall \psi \in H^1(G).
\end{equation}
Здесь $f=(\theta_1+\theta_2)(\theta_1^2+\theta_2^2)$.

Положим $\eta=\theta$, $\psi=\varphi$ в (\ref{eq:1_3:4-17}), (\ref{eq:1_3:4-18})
и учтем, что  $(\textbf{v}\cdot\nabla\theta, \theta)=0$ и $4m^3
\leq f \leq 4M^3$, тогда
\[
    \|\nabla\theta\|^2 \geq \gamma_1 \|\theta\|^2, \quad \alpha
    \|\nabla\varphi\|^2 + \int \limits_{\Gamma} \beta\varphi^2 d\Gamma
    \geq \gamma_2\|\varphi\|^2,
\]
где
\begin{gather*}
    \gamma_1 = \inf\{\|\nabla\zeta\|^2: \zeta \in H^1_0(G), \|\zeta\| = 1\},\\
    \gamma_2 = \inf\{\alpha\|\nabla\psi\|^2
    + \int \limits_{\Gamma}\beta\psi^2 d\Gamma: \psi \in H^1(G), \|\psi\| = 1\}.
\end{gather*}
Отсюда следуют неравенства
\begin{equation}
    \label{eq:1_3:4-19}
    (a\gamma_1+4m^3\mu_a b)\|\theta\| \leq \mu_a b \|\varphi\|, \quad
    (\gamma_2+\mu_a)\|\varphi\| \leq  4M^3\mu_a\|\theta\|.
\end{equation}
Таким образом, если выполняется условие
\begin{equation}
    \label{eq:1_3:4-20}
    4M^3\mu_a^2 b < (a\gamma_1 + 4m^3\mu_a b) (\gamma_2 + \mu_a),
\end{equation}
то из (\ref{eq:1_3:4-19}) следует, что $\theta=0$, $\varphi=0$.

\begin{theorem}
    \label{thm:4-2}
    Пусть выполняются условия $($i$)$-$($iii$)$ и условие
    $(\ref{eq:1_3:4-20})$.
    Тогда задача $(\ref{eq:1_3:4-1})$-$(\ref{eq:1_3:4-4})$
    однозначно разрешима в классе слабых решений, удовлетворяющих
    $(\ref{eq:1_3:4-7})$.
\end{theorem}

Приведем теперь пример условий единственности решения задачи
сложного теплообмена, учитывающих скорость движения среды в канале
длиной $L$, поперечное сечение которого является квадратом со
стороной $d$.
В этом случае
\begin{equation}
    \label{eq:1_3:4-21}
    G= \{r=(x_1, x_2, x_3): \; 0 < x_1 < L, \; 0 < x_{2,3} < d \}.
\end{equation}
Пусть среда движется с постоянной скоростью $\textbf{v} = (v, 0,
0)$.
В равенствах (\ref{eq:1_3:4-17}) и (\ref{eq:1_3:4-18}), определяющих
разности $\theta$ и $\varphi$ двух возможных решений, полагаем
\begin{gather*}
    \theta = (\gamma- e^{-sx_1})q_1, \quad \eta = (\gamma-
    e^{-sx_1})^{-1}q_1,\\
    \varphi = (\gamma- e^{-sx_1})q_2, \quad \psi = (\gamma-
    e^{-sx_1})^{-1}q_2.\\
\end{gather*}
Здесь $\gamma > 1$, $s > 0$.
Тогда
\begin{gather*}
    a\|\nabla q_1\|^2 + \int \limits_G \left(\frac{vs}{\gamma
    e^{sx_1}-1} - \frac{as^2}{(\gamma e^{sx_1}-1)^2} + b\mu_a
    f\right)q_1^2 dr + b\mu_a(q_1, q_2)=0,\\
    \alpha\|\nabla q_2\|^2 + \int \limits_G \left(\mu_a - \frac{\alpha
    s^2}{(\gamma e^{sx_1}-1)^2}\right)q_2^2 dr + \int \limits_{\Gamma}
    \beta q_2^2 d\Gamma - (f q_1, q_2)=0.\\
\end{gather*}
Таким образом, аналогично тому, как выводилась оценка~\eqref{eq:1_3:4-19}, получаем
\begin{gather*}
    \left(a\gamma_1+4m^3\mu_a b + \frac{vs}{\gamma e^{sL}-1} -
    \frac{as^2}{(\gamma-1)^2}\right)\|q_1\| \leq b\mu_a\|q_2\|,\\
    \left(\gamma_2+\mu_a - \frac{\alpha
    s^2}{(\gamma-1)^2}\right)\|q_2\| \leq 4M^3\mu_a\|q_1\|.\\
\end{gather*}
Положим здесь, например, $s=1/L$,
$\gamma=1+s\sqrt{2\alpha/\mu_a}$.
Тогда
\begin{gather*}
    \left(a\gamma_1+4m^3\mu_a b +
    \frac{v}{L(e-1)+e\sqrt{2\alpha/\mu_a}} -
    \frac{a\mu_a}{2\alpha}\right)\|q_1\| \leq b\mu_a\|q_2\|,\\
    \left(\gamma_2+ \frac{\mu_a}{2}\right)\|q_2\| \leq
    4M^3\mu_a\|q_1\|.\\
\end{gather*}
Следовательно, условие
\begin{equation}
    \label{eq:1_3:4-22}
    4M^3\mu_a^2 b < \left(a\gamma_1 + 4m^3\mu_a b +
    \frac{v}{L(e-1)+e\sqrt{2\alpha/\mu_a}}-
    \frac{a\mu_a}{2\alpha}\right)\left(\gamma_2+\frac{\mu_a}{2}\right)
\end{equation}
обеспечивает единственность в классе ограниченных решений.
Для
данной геометрии канала нетрудно вычислить, что
\[
    \gamma_1 = \pi^2\left(\frac{2}{d^2} + \frac{1}{L^2}\right), \quad
    \gamma_2 \geq
    \left(2L\max\left(\frac{1}{\beta_0},\frac{2L}{\alpha}\right)\right)^{-1}.
\]


Таким образом, выполнение условия единственности~\eqref{eq:1_3:4-22}
можно обеспечить как за счет малости размеров канала ($d$ или $L$),
так и за счет выбора достаточно большой скорости движения среды $v$.


В следующем разделе приводятся результаты численного моделирования
сложного теплообмена с параметрами задачи, удовлетворяющими
условию~\eqref{eq:1_3:4-22}.
