\section{Стационарная модель сложного теплообмена}\label{sec:ch1/sec3}

\subsection{Постановка краевой задачи}
\label{subsec:ch1/sec3/state}

Стационарная нормализованная диффузионная модель, описывающая
радиационный, кондуктивный и конвективный теплообмен в
ограниченной области $G\subset \mathbb{R}^3$,
имеет следующий вид~\cite{modest2013radiative}:

\begin{equation}
    \label{eq:1_3:4-1}
    -a \Delta \theta + \textbf{v} \cdot \nabla \theta
    + b \mu_a \theta^4 =  b \mu_a \varphi,
\end{equation}

\begin{equation}
    \label{eq:1_3:4-2}
    - \alpha \Delta \varphi + \mu_a \varphi = \mu_a \theta^4.
\end{equation}

Здесь $\theta$ -- нормализованная температура, $\varphi$ --
нормализованная интенсивность излучения, усредненная по всем
направлениям, $\textbf{v}$ -- заданное поле скоростей, $\mu_a$ --
коэффициент поглощения.
Постоянные $a$, $b$ и $\alpha$
определяются следующим образом:
\[
    a=\frac{k}{\rho c_v},\quad b = \frac{4\sigma n^2 T_{\max}^3}{\rho c_v},
    \quad \alpha=\frac{1}{3\mu - A \mu_s},
\]
где $k$ -- теплопроводность, $c_v$ -- удельная теплоемкость, $\rho$ --
плотность, $\sigma$ -- постоянная Стефана-Больцмана, $n$ --
показатель преломления, $T_{\max}$ -- максимальная температура в
ненормализованной модели, $\mu = \mu_s + \mu_a$ -- коэффициент
полного взаимодействия, $\mu_s$ -- коэффициент рассеяния.
Коэффициент $A \in [-1, 1]$ описывает анизотропию рассеяния, случай
$A=0$ соответствует изотропному рассеянию.

Будем предполагать, что функции $\theta$ и $\varphi$, описывающие
процесс сложного теплообмена, удовлетворяют следующим условиям на
границе $\Gamma = \partial G$:
\begin{equation}
    \label{eq:1_3:4-3}
    \theta|_{\Gamma} = \Theta_0,
\end{equation}
\begin{equation}
    \label{eq:1_3:4-4}
    \alpha \frac{\partial \varphi}{\partial \mathbf{n}} + \beta
    (\varphi-\Theta_0^4)|_{\Gamma} = 0.
\end{equation}
Здесь через $\partial/\partial \mathbf{n}$ обозначаем производную
в направлении внешней нормали.
Неотрицательная функция
$\Theta_{0}$, определенная на $\Gamma$,  и функция $\beta$,
описывающая, в частности, отражающие свойства границы $\Gamma$,
являются заданными.


Основные результаты, представленные в данном параграфе, состоят в
получении новых априорных оценок решения
краевой задачи~\eqref{eq:1_3:4-1}--\eqref{eq:1_3:4-4},
на основе которых доказана разрешимость
задачи и выведены достаточные условия единственности решения.
Кроме того, найдены условия на параметры модели, геометрию области
$G$ и поле скоростей $\textbf{v}$, гарантирующие однозначную
разрешимость в случае равномерного потока среды.
Результаты теоретического анализа иллюстрируются примерами численного
моделирования температурных полей в канале прямоугольной формы.

\subsection{Слабое решение краевой задачи}\label{subsec:ch1/sec3/weak}

Пусть $G$ -- липшицева ограниченная область, граница $\Gamma$
которой состоит из конечного числа гладких
кусков, а исходные данные удовлетворяют условиям: \\
$(i) \;\;\;\; \textbf{v} \in H^1(G) \cap L^{\infty}(G), \quad
\nabla \cdot \textbf{v} = 0;$ \\
$(ii) \;\; \Theta_0 \in L^{\infty}(\Gamma), \; 0 \leq m \leq
\Theta_0 \leq M; \; \exists \;\widetilde{\theta} \in H^1(G), \;
\widetilde{\theta}|_{\Gamma} = \Theta_0, \; m \leq
\widetilde{\theta} \leq M;$ \\
$(iii) \; \beta \in L^{\infty}(\Gamma), \; \beta \geq \beta_0>0.$ \\

Здесь и далее через $L^p$, $1 \leq p \leq \infty$, обозначаем
пространство Лебега, а через $H^s$ -- пространство Соболева
$W^s_2$.
Через $(\cdot,\cdot)$ обозначаем скалярное произведение в $L^2(G)$,
\[
    (f,g) = \int_G f(r)g(r)dr, \quad \|f\|^2=(f,f).
\]

Кроме этого будем использовать пространство
\[
    H^1_0(G) = \{ \eta \in H^1(G): \; \eta|_{\Gamma}=0\}
\]
с нормой $\|\eta\|_{H^1_0(G)}=(\nabla \eta, \nabla \eta)^{1/2}$.

\begin{definition}
    Пара $\{\theta, \varphi\} \in H^1(G) \times H^1(G)$ называется
    слабым решением задачи $(\ref{eq:1_3:4-1})$-$(\ref{eq:1_3:4-4})$, если
    \begin{equation}
        \label{eq:1_3:4-5}
        a(\nabla\theta, \nabla\eta) + (\mathbf{v}\cdot\nabla\theta +
        b\mu_a(\theta^4-\varphi), \eta) = 0 \quad \forall \eta \in
        H^1_0(G),
    \end{equation}
    \begin{equation}
        \label{eq:1_3:4-6}
        \alpha (\nabla\varphi, \nabla \psi) + \mu_a(\varphi - \theta^4,
        \psi)+ \int \limits_{\Gamma} \beta (\varphi-\Theta_0^4)\psi
        d\Gamma = 0 \quad \forall \psi \in H^1(G)
    \end{equation}
    и при этом $\theta|_{\Gamma} = \Theta_0$.
\end{definition}

Отметим, что в силу вложения $H^1(G) \subset L^6(G)$ выражение
$(\theta^4, \eta)$  имеет смысл для любой функции $\eta \in
H^1(G)$.

\begin{theorem}
    \label{th:1_3:weakExist}
    Пусть выполняются условия (i)-(iii).
    Тогда существует слабое
    решение задачи $(\ref{eq:1_3:4-1})$-$(\ref{eq:1_3:4-4})$, удовлетворяющее
    условиям
    \begin{equation}
        \label{eq:1_3:4-7}
        m \leq \theta \leq M, \quad m^4 \leq \varphi < M^4.
    \end{equation}
\end{theorem}
Доказательство теоремы~\ref{th:1_3:weakExist} основано на построении
операторного уравнения, которое определяет слабое решение
задачи~\eqref{eq:1_3:4-1}--\eqref{eq:1_3:4-4},
обосновании слабого принципа максимума и
применении принципа Лере-Шаудера.


%\subsection{Построение операторного уравнения}

Рассмотрим пространство $V = H^1_0(G) \times H^1(G)$.
Скалярное произведение в $V$ удобно выбрать следующим образом:
\[
    ((y,z)) = a(\nabla \zeta, \nabla \eta) + \alpha(\nabla \varphi,
    \nabla \psi)+\int \limits_{\Gamma} \beta \varphi\psi d\Gamma,
\]
где $y = \{\zeta, \varphi\} \in V, z= \{\eta, \psi\} \in V.$
Отметим, что норма пространства $V$, соответствующая выбранному
скалярному произведению, эквивалентна норме пространства
$H^1(G) \times H^1(G)$.
Через $\widetilde{y}$ обозначим элемент пространства $V$ такой, что
\[
    ((\widetilde{y}, z)) = a(\nabla\widetilde{\theta}, \nabla
    \eta)-\int \limits_{\Gamma} \beta \Theta_0^4\psi d\Gamma \quad
    \forall z = \{\eta, \psi\} \in V,
\]
где $\widetilde{\theta}$ -- функция из условия $(ii)$.

Определим нелинейный оператор $F: V \to V$, используя равенство
\begin{equation}
    \label{eq:1_3:4-8}
    ((F(y),z))=(\textbf{v}\cdot\nabla\theta, \eta) + \mu_a b
    \left(\theta^4-\varphi, \eta)+\mu_a(\varphi-\theta^4, \psi \right),
\end{equation}
справедливое для всех $y=\{\zeta, \varphi\}, z=\{\eta, \psi\} \in V$.
Здесь $\theta = \widetilde{\theta} + \zeta$.

Из определения скалярного произведения в пространстве $V$ и
соотношения~\eqref{eq:1_3:4-8} следует утверждение

\begin{lemma}
    \label{lm:1_3:weak}
    Пара $\{\theta, \varphi\} \in H^1(G) \times H^1(G)$ является
    слабым решением задачи
    $(\ref{eq:1_3:4-1})$-$(\ref{eq:1_3:4-4})$, если и только
    если элемент $y=\{\theta-\widetilde{\theta}, \varphi\} \in V$
    удовлетворяет в пространстве $V$ уравнению
    \begin{equation}
        \label{eq:1_3:4-9}
        y + \widetilde{y} + F(y) = 0.
    \end{equation}
\end{lemma}
