\section{Корректность начально-краевой задачи для квазилинейной модели}
\label{sec:ch3:sec1}
%6_Chebotarev.pdf
%Mathematical modeling of complex heat transfer in
%the context of the endovenous laser ablation

\subsection{Постановка начально-краевой задачи}
\label{subsec:ch3/sec1/model}

Рассмотрим следующую начально-краевую задачу
в ограниченной трехмерной области $\Omega$ с
отражающей границей $\Gamma=\partial \Omega$:
\begin{gather}
    \sigma \partial \theta / \partial t
    -\operatorname{div}(k(\theta) \nabla \theta)
    +b\left(\theta^{3}|\theta|
    -\varphi\right)=f, \label{eq:3_1:1} \\
    -\operatorname{div}(\alpha \nabla \varphi)
    + \beta\left(\varphi-\theta^{3}|\theta|\right)=g,
    x \in \Omega, 0<t<T \label{eq:3_1:2}\\
    k(\theta) \partial_{n} \theta
    + p\left(\theta-\theta_{b}\right)|_{\Gamma}=0,
    \quad \alpha \partial_{n} \varphi
    + \gamma\left(\varphi-\theta_{b}^{4}\right)|_{\Gamma}=0,
    \quad \theta|_{t=0}=\theta_{i n}. \label{eq:3_1:3}
\end{gather}

Здесь $\theta$ — нормированная температура, $\varphi$ — нормированная интенсивность излучения,
усредненная по всем направлениям.
Нормирующими множителями для получения из
$\theta$ и $\varphi$ абсолютной температуры и средней интенсивности излучения
являются $\mathcal{M}_{\theta}$ и $\mathcal{M}_{\varphi}$ соответственно.


Положительные параметры $b, \alpha, \beta, \gamma, p$ описывают радиационные
и теплофизические свойства среды~\cite{Chebotarev2017}, $\sigma(x, t)$ - произведение
удельной теплоемкости на объем плотность, $k(\theta)$ — коэффициент теплопроводности,
$f$ и $g$ описывают вклад источников тепла и излучения соответственно.
Предположим, что $\Omega$ — липшицева ограниченная область,
$\Gamma=\partial \Omega, Q=\Omega \times(0, T), \Sigma=\Gamma \times(0, T)$.

Будем далее предполагать, что исходные данные удовлетворяют следующим условиям:

$(i)\;\alpha, \beta, \sigma \in L^{\infty}(\Omega), b=r \beta,\; r= \text{const} >0;\;
\alpha \geq \alpha_{0}, \beta \geq \beta_{0},
\sigma \geq \sigma_{0}, \alpha_{0}, \beta_{0},\; \sigma_{0}=\text{const}>0$.

$(ii)\;0<k_{0} \leq k(s) \leq k_{1},\left|k^{\prime}(s)\right| \leq k_{2},\; s \in \mathbb{R},
\quad k_{j} = \text{const}.$

$(iii)\; 0 \leq \theta_{b} \in L^{\infty}(\Sigma),\; 0 \leq \theta_{i n} \in L^{\infty}(\Omega);
\gamma_{0} \leq \gamma \in L^{\infty}(\Gamma),
p_{0} \leq p \in L^{\infty}(\Gamma),\; \gamma_{0}, p_{0}= \text{const} >0$.

$(iv)\; 0 \leq f,\; g \in L^{\infty}(Q).$

Пусть
$
W=\left\{y \in L^{2}(0, T ; V): \sigma y^{\prime}=\sigma d y / d t
\in L^{2}\left(0, T, V^{\prime}\right)\right\}.
$

Определим операторы $A_{1}: V \rightarrow V_{0}^{\prime}$ и $A_{2}: V \rightarrow V^{\prime}$ такие,
что для всех $\theta, \varphi v$ справедливы следующие равенства:
\[
    \begin{gathered}
        \left(A_{1}(\theta), v\right)=(k(\theta) \nabla \theta, \nabla v)
        +\int_{\Gamma} p \theta v d \Gamma=(\nabla h(\theta), \nabla v)+\int_{\Gamma} p \theta v d \Gamma, \\
        \left(A_{2} \varphi, v\right)=(\alpha \nabla \varphi, \nabla v)
        +\int_{\Gamma} \gamma \varphi v d \Gamma,
    \end{gathered}
\]
где $h(s)=\int_{0}^{s} k(r) d r$.

\begin{definition}
    Пара $\theta \in W, \varphi \in L^{2}(0, T ; V)$ называется слабым решением
    задачи~\eqref{eq:3_1:1}--\eqref{eq:3_1:3}, если
    \begin{gather}
        \sigma \theta^{\prime}+A_{1}(\theta)
        +b\left([\theta]^{4}-\varphi\right)=f_{b}+f
        \quad \text { п. в. на }(0, T),
        \quad \theta(0)=\theta_{i n}, \label{eq:3_1:4}\\
        A_{2} \varphi+\beta\left(\varphi-[\theta]^{4}\right)=
        g_{b}+g \quad \text { п. в. на }(0, T). \label{eq:3_1:5}
    \end{gather}
\end{definition}

Здесь, $f_{b}, g_{b} \in L^{2}\left(0, T ; V^{\prime}\right)$ и

\[
    \left(f_{b}, v\right)=\int_{\Gamma} p \theta_{b} v d \Gamma,
    \quad\left(g_{b}, v\right)=\int_{\Gamma} \gamma \theta_{b}^{4} v d \Gamma \quad \forall v \in V.
\]

\begin{remark}
    Так как $\theta \in W$, то $\theta \in C([0, T] ; V)$.
    Следовательно, начальное условие имеет смысл.
\end{remark}

\subsection{Итерационный метод}
\label{subsec:ch3/sec1/iterative}
Определим операторы
$F_{1}: L^{\infty}(\Omega) \rightarrow V$ и $F_{2}: L^{\infty}(Q) \times L^{2}(0, T ;V) \rightarrow W$
следующим образом.
Пусть $\varphi=F_{1}(\theta)$, если

\begin{equation}
    \label{eq:3_1:6}
    A_{2} \varphi+\beta\left(\varphi-[\theta]^{4}\right)=g_{b}+g,
\end{equation}
и $\theta=F_{2}(\zeta, \varphi)$, если
\begin{equation}
    \label{eq:3_1:7}
    \sigma \theta^{\prime}+A(\zeta, \theta)+b\left([\theta]^{4}-\varphi\right)=f_{b}
    +f \quad \text { п. в. на }(0, T), \quad \theta(0)=\theta_{i n}.
\end{equation}
Здесь
\[
    (A(\zeta, \theta), v)=(k(\zeta) \nabla \theta, \nabla v)
    +\int_{\Gamma} p \theta v d \Gamma \quad \forall v \in V
\]
Пусть $w(t)=M_{0}+M_{1} t, t \in[0, T]$, где
\[
    \begin{gathered}
        M_{0}=\max \left\{\left\|\theta_{b}\right\|_{L^{\infty}(\Sigma)},
        \left\|\theta_{i n}\right\|_{L^{\infty}(\Omega)}\right\} \\
        M_{1}=\sigma_{0}^{-1}\left(\|f\|_{L^{\infty}(Q)}+\max b M_{2}\right),
        \quad M_{2}=\beta_{0}^{-1}\|g\|_{L^{\infty}(Q)}.
    \end{gathered}
\]

\begin{lemma}
    \label{lm:3_1:1}
    Пусть выполнены условия $(i) - (iv)$, $0 \leq \theta \leq w(t), \varphi=F_{1}(\theta)$.
    В таком случае
    \begin{equation}
        \label{eq:3_1:8}
        0 \leq \varphi \leq w^{4}(t)+M_{2}.
    \end{equation}
\end{lemma}
\begin{proof}
    Умножая~\eqref{eq:3_1:6} в смысле внутреннего произведения $H$ на
    $\psi=\max \left\{\varphi-M_{2}-w^{4}, 0\right\} \in$ $L^{2}(0, T ; V)$, получаем
    \[
        \left(A_{2} \varphi-g_{b}, \psi\right)+\left(\beta\left(\varphi-M_{2}-[\theta]^{4}\right),
        \psi\right)=\left(g-\beta M_{2}, \psi\right) \leq 0.
    \]
    Заметим, что с учетом ограничений на $\theta$ выполняются следующие неравенства:
    \[
        \begin{gathered}
            \left(A_{2} \varphi-g_{b}, \psi\right)=(\alpha \nabla \psi, \nabla \psi)+
            \int_{\Gamma} \gamma\left(\varphi-\theta_{b}^{4}\right) d \Gamma
            \geq (\alpha \nabla \psi, \nabla \psi), \\
            \left(\beta\left(\varphi-M_{2}-[\theta]^{4}\right), \psi\right)=
            (\beta \psi, \psi)+\left(\beta\left(w^{4}-[\theta]^{4}\right), \psi\right)
            \geq(\beta \psi, \psi).
        \end{gathered}
    \]
    Таким образом, $\psi=0$ и $\varphi \leq w^{4}+M_{2}$.

    Далее, умножая~\eqref{eq:3_1:6} в смысле скалярного произведения $H$
    на $\xi=\min \{\varphi, 0\} \in$ $L^{2}(0, T ; V) $ аналогично получаем,
    что $\xi=0$.
    Таким образом, $\varphi\geq 0$.
\end{proof}

\begin{lemma}
    \label{lm:3_1:2}
    Пусть выполняются условия $(i)-(iv)$, $0 \leq \varphi \leq w^{4}(t)+M_{2},
    \theta=F_{2}(\zeta, \varphi), \zeta \in L^{\infty}(Q)$.
    Тогда $0 \leq \theta \leq w(t)$.
\end{lemma}

\begin{proof}
    Пусть $\widehat{\theta}=\theta-w$.
    Перепишем уравнение~\eqref{eq:3_1:7} следующим образом
    \begin{equation}
        \label{eq:3_1:9}
        \sigma \widehat{\theta}^{\prime}+A(\zeta, \theta)-f_{b} +b\left([\widehat{\theta}
        +w]^{4}-\left(\varphi-M_{2}\right)\right)=f-\sigma M_{1}+b M_{2} \leq 0.
    \end{equation}

    Умножая~\eqref{eq:3_1:9} в смысле скалярного
    произведения $H$ на $\eta=\max \{\widehat{\theta}, 0\} \in W$.

    Заметим, что значение правой части неположительно, а также
    \[
        \begin{gathered}
            \left(\sigma \widehat{\theta}^{\prime}, \eta\right)=
            \left(\sigma \eta^{\prime}, \eta\right)=\frac{d}{2 d t}(\sigma \eta, \eta), \\
            \left(A(\zeta, \theta)-f_{b}, \eta\right)=(k(\zeta) \nabla \eta, \nabla \eta)
            + \int_{\Gamma} p\left(\widehat{\theta}+w-\theta_{b}\right) \eta d \Gamma \geq 0, \\
            \left([\widehat{\theta}+w]^{4}-w^{4}\right) \max
            \{\widehat{\theta}, 0\} \geq 0, \quad\left(w^{4}
            + M_{2} - \varphi\right) \eta \geq 0.
        \end{gathered}
    \]
    Тогда
    \[ \frac{d}{d t}(\sigma \eta, \eta) \leq 0,\left.\quad \eta\right|_{t=0}=0. \]
    Таким образом, $\eta=0, \widehat{\theta} \leq 0, \theta \leq w$.
    Аналогично, умножая~\eqref{eq:3_1:9} в смысле скалярного произведения
    $H$ на $\eta=\min \{\theta, 0\} \in W$, получаем, что $\eta=0,\, \theta \geq 0$.

    Пусть $\theta_{0}=\theta_{i n}, \quad \varphi_{0}=F_{1}\left(\theta_{0}\right)$.
    Определим рекурсивно последовательности
    $\theta_{m} \in W, \varphi_{m} \in L^{2}(0, T ; V)$ таким образом, что
    \begin{equation}
        \label{eq:3_1:10}
        \theta_{m}=F_{2}\left(\theta_{m-1}, \varphi_{m-1}\right),
        \quad \varphi_{m}=F_{1}\left(\theta_{m}\right), \quad m=1,2, \ldots
    \end{equation}


    Из лемм~\ref{lm:3_1:1} и~\ref{lm:3_1:2} следуют оценки
    \begin{equation}
        \label{eq:3_1:11}
        0 \leq \varphi_{m} \leq w^{4}(t)+M_{2},
        \quad 0 \leq \theta_{m} \leq w(t), \quad m=1,2, \ldots
    \end{equation}
    Лемма доказана.
\end{proof}
\begin{lemma}
    \label{lm:3_1:3}
    Пусть выполнены условия $(i)-(iv)$.
    Тогда существует константа $C > 0$, не зависящая от $m$, такая, что
    \begin{gather}
        \left\|\varphi_{m}\right\|_{L^{2}(0, T ; V)} \leq C,
        \quad\left\|\theta_{m}\right\|_{L^{2}(0, T ; V)} \leq C, \label{eq:3_1:12}\\
        \int_{0}^{T-\delta}\left\|\theta_{m}(s+\delta)
        -\theta_{m}(s)\right\|^{2} d s \leq C \delta.\label{eq:3_1:13}
    \end{gather}
\end{lemma}

\begin{proof}
    Из определения последовательностей $\varphi_{m}, \theta_{m}$ следуют равенства
    \begin{equation}
        \label{eq:3_1:14}
        A_{2} \varphi_{m}+\beta\left(\varphi_{m}
        -\left[\theta_{m}\right]^{4}\right)=g_{b}+g.
    \end{equation}
    \begin{equation}
        \label{eq:3_1:15}
        \begin{gathered}
            \sigma \theta_{m}^{\prime}+A\left(\theta_{m-1},
            \theta_{m}\right)+b\left(\left[\theta_{m}\right]^{4}-\varphi_{m-1}\right)=f_{b}+f
            \text { п. в. в }(0, T), \\ \theta_{m}(0)=\theta_{i n}.
        \end{gathered}
    \end{equation}
    Оценки~\eqref{eq:3_1:12} выводятся стандартным образом
    из уравнений~\eqref{eq:3_1:14} и~\eqref{eq:3_1:15}
    и с учетом~\eqref{eq:3_1:11}, т.е. ограниченности
    последовательностей в $L^{\infty}(Q)$.

    Получим оценку, гарантирующую компактность последовательности
    $\theta_{m}$ в $L^{2}(Q)$.
    Перепишем~\eqref{eq:3_1:15} как

    \begin{equation}
        \label{eq:3_1:16}
        \sigma \theta_{m}^{\prime}=\chi_{m}
        \text { п. в. в }(0, T), \quad \theta_{m}(0)=\theta_{i n},
    \end{equation}
    где
    \[
        -\chi_{m}=A\left(\theta_{m-1},
        \theta_{m}\right)+b\left(\left[\theta_{m}\right]^{4}
        -\varphi_{m-1}\right)-f_{b}-f.
    \]
    Отметим, что с учетом полученных оценок последовательность
    $\chi_{m}$ ограничена в $L^{2}\left(0, T ; V^{\prime}\right)$.
    Умножим~\eqref{eq:3_1:16} в смысле скалярного произведения произведения
    $H$ на $\theta_{m}(t)-\theta_{m}(s)$
    и проинтегрируем по $t$ на интервале $(s, s+\delta)$ и
    над $s$ на $(0, T-\delta)$, предполагая, что $\delta>0$ достаточно мало.
    В результате получим
    \[
        \frac{1}{2} \int_{0}^{T-\delta}\left\|\sqrt{\sigma}\left(\theta_{m}(s+\delta)
        - \theta_{m}(s)\right)\right\|^{2} d s
        = \int_{0}^{T-\delta} \int_{s}^{s+\delta} c_{m}(t, s) d t d s
    \]
    где
    \[
        c_{m}(t, s)=\left(\chi_{m}(t), \theta_{m}(t)-\theta_{m}(s)\right)
        \leq\left\|\chi_{m}(t)\right\|_{V^{\prime}}^{2}
        +\frac{1}{2}\left\|\theta_{m}(t)\right\|_{V}^{2}
        +\frac{1}{2}\left\|\theta_{m}(s)\right\|_{V}^{2}.
    \]

    Для оценки интегралов от слагаемых, зависящих от $t$,
    достаточно изменить порядок интегрирования.
    Используя ограниченность последовательностей
    $\theta_{m}$ в $L^{2}(0, T ; V)$ и $\chi_{m}$ в $L^{2}\left(0, T ; V^{\prime}\right)$,
    получаем оценку равностепенной непрерывности~\eqref{eq:3_1:13}.

    Полученные оценки~\eqref{eq:3_1:12},\eqref{eq:3_1:13}
    позволяют утверждать,
    переходя при необходимости к подпоследовательностям,
    что существуют функции $\theta_{*}, \varphi_{*}$ такие, что
    \begin{equation}
        \label{eq:3_1:17}
        \begin{aligned}
            \theta_{m} \rightarrow \widehat{\theta}
            \text { слабо в } L^{2}(0, T ; V),
            \text { сильно в } L^{2}(0, T ; H), \\
            \varphi_{m} \rightarrow \widehat{\varphi}
            \text { слабо в } L^{2}(0, T ; V).
        \end{aligned}
    \end{equation}
    Сходимости~\eqref{eq:3_1:17} достаточно, для перехода к пределу
    при $m \rightarrow \infty$
    в равенствах~\eqref{eq:3_1:14},~\eqref{eq:3_1:15}
    и доказательства, что предельные функции $\widehat{\theta},
    \widehat{\varphi } \in L^{2}(0, T ; V)$ таковы,
    что $\sigma \widehat{\theta}^{\prime} \in L^{2}\left(0, T ; V^{ \prime}\right)$
    и для них выполняются равенства~\eqref{eq:3_1:4}~\eqref{eq:3_1:5}.
\end{proof}

\begin{theorem}
    \label{th:3_1:1}
    Пусть выполнены условия $(i)-(iv)$.
    Тогда задача~\eqref{eq:3_1:1}-\eqref{eq:3_1:3} имеет хотя бы одно решение.
\end{theorem}

\subsection{Теорема единственности и сходимость итерационного метода}
\label{subsec:ch3/sec1/uniqeness_convergense}
Покажем, что в классе функций с ограниченным градиентом температуры решение единственно.
Это позволяет доказать сходимость итерационного алгоритма.

\begin{theorem}
    \label{th:3_1:2}
    Пусть выполнены условия $(i)-(iv)$.
    Если $\theta_{*}, \varphi_{*}$ — решение
    задачи~\eqref{eq:3_1:1}--\eqref{eq:3_1:3} такое,
    что $\theta_{*}, \nabla \theta_{*} \in L^{\infty }(Q)$,
    то других ограниченных решений этой задачи нет.
\end{theorem}

\begin{proof}
    Пусть $\theta_{1}, \varphi_{1}$ -- другое решение
    задачи~\eqref{eq:3_1:1}--\eqref{eq:3_1:3},
    $\theta=\theta_{1}-\theta_{*}, \varphi=\varphi_{ 1}-\varphi_{*}$.
    В таком случае
    \begin{equation}
        \label{eq:3_1:18}
        \begin{gathered}
            \sigma \theta^{\prime}+A_{1}\left(\theta_{1}\right)
            -A_{1}\left(\theta_{*}\right)+b\left(\left[\theta_{1}\right]^{4}-
            \left[\theta_{*}\right]^{4}-\varphi\right)=0
            \text { п. в. в }(0, T), \theta(0)=0.
        \end{gathered}
    \end{equation}
    \begin{equation}
        \label{eq:3_1:19}
        A_{2} \varphi+\beta\left(\varphi-\left(\left[\theta_{1}\right]^{4}
        -\left[\theta_{*}\right]^{4}\right)\right)=0
        \text { п. в. в }(0, T).
    \end{equation}

    Умножим~\eqref{eq:3_1:18} в смысле скалярного произведения
    $H$ на $\theta$ и проинтегрируем по времени.
    Как результат
    \begin{gather*}
        \frac{1}{2}\|\sqrt{\sigma} \theta\|^{2}+
        \int_{0}^{t}\left(\left(k\left(\theta_{1}\right) \nabla \theta,
        \nabla \theta\right)+\int_{\Gamma} p \theta^{2}(s) d \Gamma\right) d s=\\
        -\int_{0}^{t}\left(b\left(\left[\theta_{1}\right]^{4}-
        \left[\theta_{*}\right]^{4}-\varphi\right),
        \theta\right) d s-\int_{0}^{t}\left(\left(k\left(\theta_{1}\right)
        -k\left(\theta_{*}\right)\right)
        \nabla \theta_{*}, \nabla \theta\right) d s.\\
    \end{gather*}
    Пусть $\left|\theta_{1}\right| \leq M,\left|\theta_{*}\right| \leq M$.
    С учетом ограничения на функцию $k$ получаем неравенство
    \begin{equation}
        \label{eq:3_1:20}
        \begin{aligned}
            \frac{\sigma_{0}}{2}\|\theta\|^{2}+k_{0}
            & \int_{0}^{t}\|\nabla \theta\|^{2} d s \leq \\
            & \int_{0}^{t}\left(4 M \max b\|\theta\|^{2}
            +\|\varphi\|\|\theta\|\right) d s
            +k_{2}\left\|\nabla \theta_{*}\right\|_{L^{\infty}(Q)}
            \int_{0}^{t}\|\theta\|\|\nabla \theta\| d s.
        \end{aligned}
    \end{equation}
    Принимая во внимание, что
    $\|\theta\|\|\nabla \theta\| \leq \varepsilon\|\nabla \theta\|^{2}+
    \frac{1}{4 \varepsilon}\|\theta\|^{2}$
    и предполагая
    \[
        \varepsilon=\frac{k_{0}}{k_{2}\left\|\nabla \theta_{*}\right\|_{L^{\infty}(Q)}}
    \]


    из~\eqref{eq:3_1:20} получаем

    \begin{equation}
        \label{eq:3_1:21}
        \frac{\sigma_{0}}{2}\|\theta\|^{2} \leq \int_{0}^{t}
        \left(4 M \max b\|\theta\|^{2}+\|\varphi\|\|\theta\|\right) d s
        +\frac{1}{4 \varepsilon} k_{2}\left\|\nabla \theta_{*}\right\|_{L^{\infty}(Q)}
        \int_{0}^{t}\|\theta\|^{2} d s.
    \end{equation}
Умножим~\eqref{eq:3_1:19} на $\varphi$ в смысле скалярного произведения $H$.
В результате получим
    \[
        \left(A_{2} \varphi, \varphi\right)+(\beta \varphi, \varphi)=
        \left(\beta\left(\left[\theta_{1}\right]^{4}-
        \left[\theta_{2}\right]^{4}\right), \varphi\right)
        \leq 4 \max \beta M^{3}\|\theta\|\|\varphi\|.
    \]


    Следовательно, $\|\varphi\| \leq 4 \beta_{0}^{-1} \max \beta M^{3}\|\theta\|$.
    Из~\eqref{eq:3_1:21} и неравенства Гронуолла следует, что $\theta=0, \theta_{1}$
    совпадает с $\theta_{*}$ и, соответственно, $\varphi_{1}$ совпадает с $\varphi_{*}$.
\end{proof}

\begin{theorem}
    \label{th:3_1:3}
    Пусть выполнены условия $(i)-(iv)$.
    Если $\theta_{*}, \varphi_{*}$ — решение задачи~\eqref{eq:3_1:1}--\eqref{eq:3_1:3}
    такое, что $\theta_{*}, \nabla \theta_{*} \in L^{\infty }(Q)$,
    то для последовательностей~\eqref{eq:3_1:10} справедливо следующее:
    \[
        \theta_{m} \rightarrow \theta_{*} \text { in } L^{2}(0, T ; V),
        \quad \varphi_{m} \rightarrow \varphi_{*} \text { in } L^{2}(0, T ; V).
    \]
\end{theorem}

\begin{proof}

    Сначала покажем, что $\theta_{m} \rightarrow \theta_{*}$ в $L^{2}(0, T ; H)$.
    Предполагая противное, заключаем, что существуют $\varepsilon_{0}>0$
    и подпоследовательность $\theta_{m^{\prime}}$ такая, что
    $\left\|\theta_{m^{\prime}} -\theta_{*}\right\|_{L^{2}(0, T ; H)}
    \geq \varepsilon_{0}$.
    Оценки~\eqref{eq:3_1:12}~\eqref{eq:3_1:13} позволяют утверждать,
    переходя при необходимости к подпоследовательностям,
    что справедливы результаты сходимости~\eqref{eq:3_1:17},
    где пара $\widehat{\theta}, \widehat{\varphi}$
    также является решением задачи ~\eqref{eq:3_1:1}--\eqref{eq:3_1:3}.
    Следовательно, $\left\|\widehat{\theta}-\theta_{*}\right\|_{L^{2}(0, T ; H)}
    \geq \varepsilon_{0}$, что противоречит теореме~\ref{th:3_1:2} о единственности решения.
    Из уравнений~\eqref{eq:3_1:14} и~\eqref{eq:3_1:15} с учетом~\eqref{eq:3_1:11},
    т.е. ограниченности последовательностей в $L^{\infty}(Q)$,
    а также доказанной сходимости $\theta_{m}$ в $L^{2}(0, T; H)$
    следует сходимость $\theta_{m} \rightarrow \theta_{*}, \varphi_{m}
    \rightarrow \varphi_{*}$ в $L^ {2}(0, T; V)$.
\end{proof}
