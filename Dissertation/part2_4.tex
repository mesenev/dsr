\section{Задача сложного теплообмена с условиями Коши для температуры на части границы}
\label{sec:ch2/sec4}

\subsection{Постановка обратной задачи}\label{subsec:ch2/sec4/state}

Рассмотрим следующую систему полулинейных эллиптических уравнений, которая
моделирует радиационный и диффузионный (сложный) теплообмен в
ограниченной липшицевой области $\Omega\subset \mathbb{R}^3$ с границей
$\Gamma=\partial\Omega$ ~\cite{Pinnau2007, Kovtanyuk2014a}.
\begin{equation}
    \label{eq:2_4:eq1}
    - a\Delta\theta + b\kappa_a(|\theta|\theta^3- \varphi)=0,   \quad
    -\alpha \Delta \varphi + \kappa_a(\varphi-|\theta|\theta^3)=0,\; x\in\Omega.
\end{equation}

Пусть граница области состоит из двух участков,
$\Gamma \coloneqq \partial \Omega =\overline{\Gamma}_1 \cup \overline{\Gamma}_2$,
так что $\Gamma_1 \cap \Gamma_2 =  \emptyset$.
На всей границе $\Gamma$ задается тепловой поток $q_b$,
\begin{equation}
    \label{eq:2_4:bc1}
    a\partial_n\theta = q_b, \quad x\in \Gamma.
\end{equation}

Для задания краевого условия для интенсивности излучения
требуется знать функцию $\gamma$, описывающую отражающие свойства границы~\cite{JVM-14}.
В случае, если эта функция неизвестна на части границы $\Gamma_2$,
краевое условие для интенсивности излучения на $\Gamma_2$ не ставится, а в качестве условия
переопределения на $\Gamma_1$, в дополнение к условию на
$\varphi$, задается температурное поле $\theta_b$,
\begin{equation}
    \label{eq:2_4:bc2}
    \alpha\partial_n\varphi + \gamma (\varphi - \theta_b ^4 ) = 0,\;
    \theta=\theta_b\quad x\in \Gamma_1.
\end{equation}
Здесь через $\partial_n$ обозначаем производную в направлении
внешней нормали $\mathbf n$.

Для постановки задачи управления введем новую неизвестную функцию
$\psi= a\theta + \alpha b \varphi$.
Складывая первое уравнение в~\eqref{eq:2_4:eq1} со вторым, умноженным на $b$,
заключаем, что $\psi$ -- гармоническая функция.
Исключая $\varphi$ из первого уравнения в~\eqref{eq:2_4:eq1} и используя краевые
условия~\eqref{eq:2_4:bc1},\eqref{eq:2_4:bc2},
получаем краевую задачу
\begin{equation}
    \label{eq:2_4:eq2}
    - a \Delta \theta + g (\theta) = \frac{\kappa_a}{\alpha}\psi, \quad
    \Delta \psi = 0, \; x \in \Omega,
\end{equation}
\begin{equation}
    \label{eq:2_4:bc3}
    a \partial_n \theta = q_b, \; \text{ на }\Gamma, \;\;
    \alpha \partial_n \psi + \gamma \psi  =  r,\;\;
    \theta = \theta_b  \text{ на }\Gamma_1.
\end{equation}
Здесь $g(\theta) = b \kappa_a|\theta|\theta^3 + \frac{a\kappa_a}{\alpha}\theta$, $r=\alpha b \gamma \theta_b^4+ \alpha q_b + a \gamma \theta_b$.

Сформулируем задачу оптимального управления, которая аппроксимирует
задачу~\eqref{eq:2_4:eq2},\eqref{eq:2_4:bc3}.
Задача состоит в отыскании тройки $\{\theta_\lambda,\psi_\lambda,u_\lambda\}$ такой, что
\begin{gather}
    \label{eq:2_4:cost}
    J_\lambda(\theta, u) =
    \frac{1}{2} \int \limits_{\Gamma_1} (\theta - \theta_b)^2 d \Gamma
    + \frac{\lambda}{2}\int\limits_{\Gamma_2} u^2 d\Gamma \rightarrow \inf, \\
    - a \Delta \theta + g (\theta) = \frac{\kappa}{\alpha}\psi, \quad
    \Delta \psi = 0, \; x \in \Omega, \\
    a \partial_n \theta + s \theta = q_b + s \theta_b,
    \; \alpha \partial_n \psi + \gamma \psi = r
    \text{ на } \Gamma_1,\\
    a \partial_n \theta = q_b, \;
    \alpha \partial_n \psi = u \text{ на } \Gamma_2.
\end{gather}
Здесь $\lambda, s > 0$ -- регуляризирующие параметры.

\subsection{Разрешимость задачи оптимального управления}
\label{subsec:ch2/sec4/solvability}
Рассмотрим пространства $H = L^2(\Omega), \; V = H^1(\Omega)=W^1_2(\Omega)$, $V'$
-- пространство, сопряженное с $V$.
Пространство $H$ отождествляем с пространством $H'$ так, что $V \subset H = H' \subset V'$.
Через $U$ обозначаем пространство управлений $L^2(\Gamma_2)$.
Стандартную норму в $H$ обозначаем $\|\cdot\|$,
$(f,v)$ -- значение функционала $f\in V'$ на элементе $v\in V$,
совпадающее со скалярным произведением в $H$, если $f\in H$.


Будем предполагать, что исходные данные удовлетворяют условиям:

$(l) \; a,b,\alpha,\kappa_a, \lambda, s ={\textrm Const}> 0 ,$

$(ll) \; 0<\gamma_0\leq \gamma \in L^\infty(\Gamma_1),
\; \theta_b, r \in L^2(\Gamma_1),\; q_b\in L^2(\Gamma).$


Определим операторы $A_{1,2}\colon V \to V'$, $B_1\colon L^2(\Gamma_1)\to V'$,
$B_2\colon U\to V'$ используя
равенства, справедливые для любых $y,z \in V$, $f,v\in L^2(\Gamma_1)$,
$h,w\in U$:
\begin{gather*}
(A_1 y,z)
    =a (\nabla y, \nabla z) +
    s\int\limits_{\Gamma_1}yz d\Gamma, \;\;
    (A_2 y,z) =\alpha (\nabla y, \nabla z)
    + \int\limits_{\Gamma_1}\gamma yz d\Gamma, \\
    (B_1 f, v) = \int\limits_{\Gamma_1}fv d\Gamma,\;\;
    (B_2 h, w) = \int\limits_{\Gamma_2}hw d\Gamma.
\end{gather*}

Заметим, что билинейная форма $(A_1 y,z)$ определяет скалярное произведение
в пространстве $V$ и норма $\|z\|_V=\sqrt{(A_1 z,z)}$ эквивалентна
стандартной норме $V$.
Кроме того, определены непрерывные обратные операторы
$A_{1,2}^{-1}:\,V'\mapsto V$.
Для $y\in V$, $f\in L^2(\Gamma_1)$, $h\in V'$ справедливы неравенства
\begin{equation}
    \label{eq:2_4:e}
    \|y\| \leq K_0\|y\|_V,\; \; \|B_1 f\|_{V'}
    \leq K_1\|w\|_{L^2(\Gamma_1)},\;\;
    \|B_2 h\|_{V'}\leq K_2\|h\|_{U}.
\end{equation}
Здесь постоянные $K_j>0$ зависят только от области $\Omega$.

Используя введенные операторы, слабую формулировку краевой задачи,
на решениях которой минимизируется функционал~\eqref{eq:2_4:cost},
нетрудно записать в виде
\begin{equation}
    \label{eq:2_4:cs}
    A_1 \theta + g(\theta) = \frac{\kappa_a}{\alpha}\psi
    + f_1,\;\;\; A_2\psi = f_2 + B_2 u,
\end{equation}
где $f_1 = B_1 (q_b + s \theta_b) + B_2 q_b$, $f_2 = B_1 r$.

Для формализации задачи оптимального управления определим оператор
ограничений $F(\theta, \psi, u) : V \times V \times U \rightarrow V' \times V'$,
\[
    F(\theta, \psi, u) = \left\{A_1\theta + g(\theta) - \frac{\kappa_a}{\alpha}\psi - f_1,\;
    A_2 \psi - f_2 - B_2 u \right\}.
\]

\textbf{Задача $(P_\lambda)$.} Найти тройку
$\{\theta_\lambda, \psi_\lambda, u_\lambda \} \in V \times V \times U$
такую, что
\begin{equation}
    \label{eq:2_4:cp}
    J_\lambda(\theta, u) = \frac{1}{2}\|\theta -\theta_b\|^2_{L^2(\Gamma_1)}
    + \frac{\lambda}{2}\|u\|^2_U \rightarrow \inf,\;\; F(\theta, \psi, u) = 0.
\end{equation}


\begin{lemma}
    \label{lm:2_4:1}
    Пусть выполняются условия $(l), (ll), u \in U$.
    Тогда существует единственное решение системы~\eqref{eq:2_4:cs} и при этом
    \begin{equation}
        \label{eq:2_4:e1}
        \begin{aligned}
            \|\theta\|_V \leq
            \frac{K_0^2\kappa_a}{\alpha}\|\psi\|_V+\|f_1\|_{V'}, \,
            \|\psi\|_V\leq \|A_2^{-1}\| \left(\|f_2\|_{V'}+K_2\|u\|_U\right).
        \end{aligned}
    \end{equation}
\end{lemma}

\begin{proof}
    Из второго уравнения в~\eqref{eq:2_4:cs} следует, что $\psi = A_2^{-1}(f_2+B_2 u)$
    и поэтому
    \[
        \|\psi\|_V\leq \|A_2^{-1}\|\left(\|f_2\|_{V'}+K_2\|u\|_U\right).
    \]
    Однозначная разрешимость первого уравнения~\eqref{eq:2_4:cs} с монотонной
    нелинейностью хорошо известна (см.\ например~\cite{Kufner}).
    Умножим скалярно это уравнение на на $\theta$,
    отбросим неотрицательное слагаемое $(g(\theta),\theta)$ и оценим правую часть,
    используя неравенство Коши-Буняковского.
    Тогда, с учетом неравенств~\eqref{eq:2_4:e}, получаем
    \[
        \|\theta\|^2_V \leq \frac{\kappa_a}{\alpha}\|\psi\|\|\theta\|
        +\|f_1\|_{V'}\|\theta\|_V \leq
        \frac{K_0^2\kappa_a}{\alpha}\|\psi\|_V\|\theta\|_V
        +\|f_1\|_{V'}\|\theta\|_V.
    \]
    В результате получаем оценки~\eqref{eq:2_4:e1}.
\end{proof}


\begin{theorem}
    \label{th:2_4:1}
    При выполнении условий $(l), (ll)$ существует решение задачи $(P_\lambda)$.
\end{theorem}

\begin{proof}
    Обозначим через
    $j_\lambda $ точную нижнюю грань целевого функционала $J_\lambda$
    на множестве $u \in U$, $F(\theta, \psi, u)=0$ и рассмотрим
    последовательности такие, что
    $u_m \in U, \; \theta_m \in V, \;\psi_m\in V$, $J_\lambda(\theta_m, u_m)
    \rightarrow j_\lambda,$
    \begin{equation}
        \label{eq:2_4:ms}
        A_1\theta_m+g(\theta_m) = \frac{\kappa_a}{\alpha}\psi_m
        + f_1,\;\;\; A_2\psi_m = f_2 + B_2 u_m.
    \end{equation}
    Из ограниченности последовательности $u_m$ в
    пространстве $U$ следуют, на основании леммы~\ref{lm:2_4:1}, оценки
    \[
        \|\theta_m\|_V \leq C,\;\;
        \|\psi_m\|_V \leq C,\;\;\|\theta_m\|_{L^6(\Omega)} \leq C.
    \]
    Здесь через $C>0$ обозначена наибольшая из постоянных,
    ограничивающих соответствующие нормы и не зависящих от $m$.
    Переходя при необходимости к подпоследовательностям, заключаем, что
    существует тройка $\{ \hat{u}, \hat{\theta}, \hat{\psi} \} \in U \times V \times V,$
    \begin{equation}
        \label{eq:2_4:l}
        u_m \rightarrow \hat{u} \text{  слабо в } U, \;\;
        \theta_m, \psi_m \rightarrow \hat{\theta}, \hat{\psi} \text{ слабо в } V,
        \text{ сильно в } L^4(\Omega).
    \end{equation}
    Заметим также, что $\forall v \in V$ имеем
    \begin{equation}
        \label{eq:2_4:l1}
        |( [\theta_m]^4 - [\hat{\theta}]^4, v)|
        \leq 2 \| \theta_m - \hat{\theta}\|_{L^4(\Omega)} \|v\|_{L^4(\Omega)}
        \left( \| \theta_m \|^3_{L_6(\Omega)} + \| \hat{\theta} \|^3_{L_6(\Omega)}\right).
    \end{equation}
    Результаты о сходимости~\eqref{eq:2_4:l} и неравенство~\eqref{eq:2_4:l1} позволяют перейти
    к пределу в~\eqref{eq:2_4:ms}.
    В результате получим
    \begin{equation}
        \label{eq:2_4:w1}
        A_1 \hat{\theta} + g(\hat{\theta}) =
        \frac{\kappa_a}{\alpha}\hat{\psi}+f_1,\, A_2\hat{\psi}=f_2+B_2\hat{u}.
    \end{equation}
    Поскольку целевой функционал слабо полунепрерывен снизу, то
    $j_\lambda \leq J_\lambda(\hat{\theta}, \hat{u})
    \leq \varliminf J_\lambda(\theta_m, u_m) = j_\lambda$ и поэтому
    тройка $\{\hat{\theta}, \hat{\psi}, \hat{u} \}$ есть
    решение задачи $(P_\lambda).$
\end{proof}

\subsection{Условия оптимальности первого порядка}
\label{subsec:ch2/sec4/optimality}

Воспользуемся принципом Лагранжа для
гладко-выпуклых экстремальных задач~\cite{11,10}.
Невырожденность условий оптимальности гарантируется условием, что образ
производной оператора ограничений $F(y, u)$, где $y=\{\theta, \psi\} \in V\times V$,
совпадает с пространством $V' \times V'$.
Последнее означает, что линейная система
\[
    A_1\xi + g'(\theta)\xi - \frac{\kappa_a}{\alpha}\eta = q_1, \quad
    A_2\eta = q_2
\]
разрешима для всех $\theta\in V$, $q_1,q_2\in V'$.
Здесь $g'(\theta)=4b\kappa_a|\theta|^3+\frac{\kappa_a}{\alpha}$.
Из второго уравнения получаем $\eta = A_2^{-1}q_2$.
Разрешимость первого уравнения при известном $\eta\in V$
очевидным образом следует из леммы Лакса--Мильграма.
Отметим, что справедливость остальных условий принципа Лагранжа также очевидна.

Функция Лагранжа задачи $(P_\lambda)$
имеет вид
\[
    L (\theta, \psi, u, p_1, p_2) = J_\lambda(\theta, u)
    + \left( A_1 \theta + g (\theta)
    - \frac{\kappa_a}{\alpha}\psi - f_1, p_1 \right)
    + (A_2 \psi - f_2 - B_2 u, p_2).
\]
Здесь $p=\{p_1,p_2\} \in V \times V$ -- сопряженное состояние.

Пусть $\{\hat{\theta}, \hat{\varphi}, \hat{u} \}$ -- решение задачи $(P_\lambda)$.
Вычислив производные Гато функции Лагранжа по $\theta,\,\psi$ и $u$, получаем
в силу принципа Лагранжа~\cite[Гл. 2, теорема 1.5]{10} следующие равенства
$\forall v\in V,\, w \in U$
\begin{equation}
    \label{eq:2_4:oc1}
    (B_1(\hat{\theta} -\theta_b), v) + (A_1 v + g'(\hat{\theta})v, p_1)=0,\;
    -\frac{\kappa_a}{\alpha}(v ,p_1)+ (A_2 v,p_2), = 0,
\end{equation}
\begin{equation}
    \label{eq:2_4:oc2}
    \lambda(B_2\hat{u},w) - (B_2 w, p_2) = 0.
\end{equation}

Из условий~\eqref{eq:2_4:oc1},\eqref{eq:2_4:oc2}
вытекают уравнения для сопряженного состояния,
которые вместе с уравнениями~\eqref{eq:2_4:w1}
для оптимальной тройки дают систему оптимальности задачи $(P_\lambda)$.

\begin{theorem}
    Пусть выполняются условия $(l), (ll)$.
    Если $\{\hat{\theta}, \hat{\psi}, \hat{u}\}$ -- решение
    задачи $(P_\lambda)$, то существует единственная пара
    $\{p_1, p_2 \} \in V\times V$ такая, что
    \begin{equation}
        \label{eq:2_4:as}
        A_1 p_1+g'(\hat{\theta}) p_1=-B_1(\hat{\theta} -\theta_b),\;\;
        A_2 p_2=\frac{\kappa_a}{\alpha}p_1,\;\;
        \lambda\hat{u}=p_2|_{\Gamma_2}.
    \end{equation}
\end{theorem}

\begin{remark}
    Если рассмотреть приведенный целевой функционал
    $\tilde J_\lambda(u)=J_\lambda(\theta(u), u)$, где $\theta(u)$ компонента
    решения задачи~\eqref{eq:2_4:cs} соответствующая управлению $u\in U$,
    то градиент функционала $\tilde J_\lambda(u)$ равен
    $ \tilde J'_\lambda (u) = \lambda u - p_2$.
    Здесь $p_2$ -- компонента решения сопряженной системы~\eqref{eq:2_4:as},
    где $\hat{\theta}\coloneqq\theta(u)$.
\end{remark}

\subsection{Аппроксимация решения обратной задачи}
\label{subsec:ch2/sec4/approximation}

Покажем, что если существует пара
$\{\theta,\varphi\}\in V\times V$ -- решение обратной
задачи~\eqref{eq:2_4:eq1}-\eqref{eq:2_4:bc2} и при этом
$q=a\partial_n\varphi|_{\Gamma_2}\in L^2(\Gamma_2)$, то
решения задачи $(P_\lambda)$ при $\lambda \to + 0$
аппроксимируют решение задачи~\eqref{eq:2_4:eq1}-\eqref{eq:2_4:bc2}.

Предварительно заметим, что указанная пара
для всех $ v \in V$ удовлетворяет равенствам
\begin{equation}
    \label{eq:2_4:ip1}
    a(\nabla\theta, \nabla v)
    + b\kappa_a(|\theta|\theta^3 - \varphi, v)
    = \int\limits_\Gamma q_b v d \Gamma,
\end{equation}
\begin{equation}
    \label{eq:2_4:ip2}
    \alpha (\nabla \varphi,\nabla v)
    + \int\limits_{\Gamma_1}\gamma\varphi vd\Gamma
    + \kappa_a(\varphi - |\theta|\theta^3,v) =
    \int\limits_{\Gamma_1}\gamma\theta_b^4 v d\Gamma
    +\int\limits_{\Gamma_2} q v d\Gamma
\end{equation}
и при этом $\theta|_{\Gamma_1}=\theta_b$.

\begin{theorem}
    Пусть выполняются условия $(l), (ll)$ и существует решение
    задачи~\eqref{eq:2_4:eq1}-\eqref{eq:2_4:bc2},
    удовлетворяющее равенствам~\eqref{eq:2_4:ip1},~\eqref{eq:2_4:ip2},
    Если $\{\theta_\lambda,\psi_\lambda,u_\lambda\}$ -- решение
    задачи $(P_\lambda)$ для $\lambda>0$, то существует последовательность
    $\lambda\to +0$
    такая, что
    \[
        \theta_\lambda\rightarrow\theta_*, \;\;
        \frac{1}{\alpha b}(\psi_\lambda-a\theta_\lambda)\rightarrow\varphi_*
        \text{ \textit{ слабо в} }V,\text{ \textit{ сильно в} }H,
    \]
    где $\theta_*,\varphi_*$ -- решение задачи~\eqref{eq:2_4:eq1}-\eqref{eq:2_4:bc2}.
\end{theorem}


\begin{proof}
    Умножим равенство~\eqref{eq:2_4:ip1} на $\alpha$,~\eqref{eq:2_4:ip2}
    на $\alpha b$ и сложим равенства.
    Тогда, полагая $\psi=a\theta+\alpha b\varphi$,
    $u = \alpha b q + \alpha q_b|_{\Gamma_2}$, получаем
    \[
        \alpha(\nabla\psi,\nabla v) + \int\limits_{\Gamma_1}\gamma\psi vd\Gamma =
        \int\limits_{\Gamma_1}r v d\Gamma + \int\limits_{\Gamma_2}u vd\Gamma.
    \]
    Здесь $r=\alpha b \gamma \theta_b^4+ \alpha q_b + a \gamma \theta_b$.
    Поэтому $A_2 \psi = f_2 + B_2 u$.

    Из~\eqref{eq:2_4:ip1}, с учетом условия $\theta|_{\Gamma_1} = \theta_b$
    выводим равенство $A_1\theta + g(\theta) = \frac{\kappa_a}{\alpha} \psi + f_1$.
    Таким образом, тройка $\{\theta, \psi, u\} \in V \times V \times U$
    является допустимой для задачи $(P_\lambda)$ и следовательно
    \[
        J_\lambda(\theta_\lambda, u_\lambda) =
        \frac{1}{2}\|\theta_\lambda -\theta_b\|^2_{L^2(\Gamma_1)}
        + \frac{\lambda}{2}\|u_\lambda\|^2_U
        \leq J_\lambda(\theta, u) = \frac{\lambda}{2}\|u\|^2_U.
    \]
    Тогда
    \[
        \|u_\lambda\|^2_U\leq \|u\|^2_U, \;
        \|\theta_\lambda -\theta_b\|^2_{L^2(\Gamma_1)} \to 0,\;
        \lambda\to +0.
    \]
    Из ограниченности последовательности $u_\lambda$ в пространстве $U$ следуют, на основании
    леммы~\ref{lm:2_4:1}, оценки
    \[ \|\theta_\lambda\|_V \leq C,\; \|\psi_\lambda\|_V \leq C, \]
    где постоянная $C > 0$ не зависит от $\lambda$.
    Поэтому можно выбрать последовательность $\lambda \to + 0$ такую, что
    \begin{equation}
        \label{eq:2_4:ll}
        \begin{aligned}
            u_\lambda \rightarrow u_* \text{ \textit{  слабо в} } U, \;\;
            &\theta_\lambda, \psi_\lambda \rightarrow \theta_*,\psi_* \text{
                \textit{ слабо в} } V, \text{
                \textit{ сильно в} } H,L^4(\Omega),\\
            &\theta_\lambda|_{\Gamma_1}\rightarrow
            \theta_*|_{\Gamma_1} \text{ \textit{ сильно в } }L^2(\Gamma_1).
        \end{aligned}
    \end{equation}
    Результаты~\eqref{eq:2_4:ll} позволяют перейти к пределу при
    $\lambda \to + 0$ в уравнениях для
    $\theta_\lambda, \psi_\lambda, u_\lambda$ и тогда
    \begin{equation}
        \label{eq:cc}
        A_1 \theta_* + g(\theta_*) = \frac{\kappa_a}{\alpha} \psi_* + f_1, \quad
        A_2 \psi_* = f_2 + B_2 u_*, \quad \theta_*|_{\Gamma_1} = \theta_b.
    \end{equation}
    Полагая $\varphi_*= \frac{1}{\alpha b}(\psi_*-a\theta_*)$, заключаем, что
    пара $\theta_*,\varphi_*$ -- решение задачи~\eqref{eq:2_4:eq1}-\eqref{eq:2_4:bc2}.
\end{proof}
