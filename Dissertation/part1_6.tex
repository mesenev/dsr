\section{Квазилинейная модель сложного теплообмена}
\label{sec:ch1/sec5}
%JPCS ВВЛА 2023

В данном параграфе представлен анализ модели сложного теплообмена, в котором учитывается зависимость
коэффициента теплопроводности от температуры.
Результаты анализа используются в разделе~\ref{sec:ch4/sec3}.

\subsection{Формулировка задачи}
\label{subsec:ch1/sec5/subsec1}

Рассмотрим следующую начально-краевую задачу в ограниченной трехмерной
области $\Omega$ с отражающей границей $\Gamma=\partial \Omega$:

\begin{equation}
    \label{eq:1_6:1}
    \sigma \partial \theta / \partial t
    -\operatorname{div}(k(\theta) \nabla \theta)
    +b\left(\theta^{3}|\theta|-\varphi\right)=f,
\end{equation}
\begin{equation}
    \label{eq:1_6:2}
    -\operatorname{div}(\alpha \nabla \varphi)
    +\beta\left(\varphi-\theta^{3}|\theta|\right)=g, x \in \Omega, 0<t<T,
\end{equation}
\begin{equation}
    \label{eq:1_6:3}
    k(\theta) \partial_{n} \theta+\left.p\left(\theta-\theta_{b}\right)\right|_{\Gamma}=0,
    \alpha \partial_{n} \varphi
    +\left.\gamma\left(\varphi-\theta_{b}^{4}\right)\right|_{\Gamma}=0,
    \left.\quad \theta\right|_{t=0}=\theta_{in}.
\end{equation}


Здесь $\theta$ -- нормированная температура, $\varphi$ -- нормированная интенсивность излучения.
Нормирующими коэффициентами для получения из $\theta$ и $\varphi$ абсолютной температуры
и средней интенсивности излучения являются
$\mathcal{M}_{\theta}$ и $\mathcal{M}_{\varphi}$ соответственно~\cite{Kovtanyuk2014a}.
Положительные параметры $b, \alpha, \beta, \gamma, p$ описывают радиационные
и теплофизические свойства среды~\cite{ESAIM}, $\sigma(x, t)$ -- произведение удельной
теплоемкости на объемную плотность, $k(\theta)$ -- коэффициент теплопроводности,
$f$ и $g$ описывают вклад источников тепла и излучения соответственно.
Символом $\partial_{n}$ обозначена производная по направлению
внешней нормали $\mathbf{n}$ к границе $\Gamma$.

Как и ранее предположим, что $\Omega$ -- липшицева ограниченная область,
$\Gamma=\partial \Omega, Q=\Omega \times(0, T), \Sigma=\Gamma \times(0, T)$.
Обозначим через $L^{p}, 1 \leq p \leq \infty$ пространство Лебега,
через $H^{1}$ пространство Соболева $W_{2}^{1}$ и через $L^{p}(0, T ; X)$
пространство Лебега функций из $L^{p}$, определенных на $(0, T)$,
со значениями в банаховом пространстве $X$.
Пусть $H=L^{2}(\Omega), V=H^{1}(\Omega)$,
а пространство $V^{\prime}$ двойственно к $V$.

Тогда мы отождествим $H$ с его двойственным пространством $H^{\prime}$
таким, что $V \subset H=H^{\prime} \subset V^{\prime}$,
и обозначим через $\|\cdot\|$ норму в $H$, а через $(h, v)$ значение
функционала $h \in V^{\prime}$ на элементе $v \in V$, совпадающее
со скалярным произведением в $H$, если $h \in H$.
Предположим, что исходные данные удовлетворяют следующим условиям:

(k1) $\alpha, \beta, \sigma \in L^{\infty}(\Omega),
\quad b=r \beta, r=$ Const $>0 ; \alpha \geq \alpha_{0}, \beta \geq \beta_{0},
\sigma \geq \sigma_{0}, \alpha_{0}, \beta_{0}, \sigma_{0}=$ Const $>0$.

(k2) $0<k_{0} \leq k(s) \leq k_{1},\left|k^{\prime}(s)\right| \leq k_{2},
s \in \mathbb{R}, \quad k_{j}=$ Const.

(k3) $0 \leq \theta_{b} \in L^{\infty}(\Sigma), 0 \leq \theta_{\text{in}}
\in L^{\infty}(\Omega)$; $\gamma_{0} \leq \gamma \in L^{\infty}(\Gamma), p_{0}
\leq p \in L^{\infty}(\Gamma), \gamma_{0}, p_{0}=$ Const $>0$.

(k4) $0 \leq f, g \in L^{\infty}(Q).$

Пусть
\[
    W=\left\{y \in L^{2}(0, T ; V): \sigma y^{\prime}=\sigma d y / d t \in L^{2}
    \left(0, T, V^{\prime}\right)\right\}.
\]
Определим операторы $A_{1}: V \rightarrow V_{0}^{\prime}$ и $A_{2}: V \rightarrow V^{\prime}$
такие, что для всех $\theta, \varphi, v \in V$ выполняются следующие равенства:
\[
    \begin{gathered}
        \left(A_{1}(\theta), v\right)=(k(\theta) \nabla \theta, \nabla v)
        +\int_{\Gamma} p \theta v d \Gamma=(\nabla h(\theta), \nabla v)
        +\int_{\Gamma} p \theta v d \Gamma, \\
        \left(A_{2} \varphi, v\right)=(\alpha \nabla \varphi, \nabla v)
        +\int_{\Gamma} \gamma \varphi v d \Gamma,
    \end{gathered}
\]
где
\[
    h(s)=\int_{0}^{s} k(r) d r.
\]

\begin{definition}
    Пару $\theta \in W, \varphi \in L^{2}(0, T ; V)$ будем называть слабым
    решением задачи~\eqref{eq:1_6:1}--\eqref{eq:1_6:3}, если
    \begin{equation}
        \label{eq:1_6:4}
        \sigma \theta^{\prime}+A_{1}(\theta)+b\left([\theta]^{4}-\varphi\right)=f_{b}+f
        \quad \text { п. в. в }(0, T), \quad \theta(0)=\theta_{\text {in }},
    \end{equation}
    \begin{equation}
        \label{eq:1_6:5}
        A_{2} \varphi+\beta\left(\varphi-[\theta]^{4}\right)
        =g_{b}+g \quad \text { п. в. в }(0, T).
    \end{equation}
\end{definition}
Здесь $f_{b}, g_{b} \in L^{2}\left(0, T ; V^{\prime}\right)$ и
\[
    \left(f_{b}, v\right)=\int_{\Gamma} p \theta_{b} v d \Gamma,
    \quad\left(g_{b}, v\right)=
    \int_{\Gamma} \gamma \theta_{b}^{4} v d \Gamma \quad \forall v \in V.
\]

\begin{remark}
    Так как $\theta \in W$, следовательно $\theta \in C([0, T] ; H)$.
    Таким образом, начальное условие имеет смысл.
\end{remark}

\subsection{Расщепление задачи}
\label{subsec:ch1/sec5/subsec2}

Определим операторы $F_{1}: L^{\infty}(\Omega) \rightarrow V$ и
$F_{2}: L^{\infty}(Q) \times L^{2}(0, T ; V) \rightarrow W$ следующим образом.
Пусть $\varphi=F_{1}(\theta)$, если
\begin{equation}
    \label{eq:1_6:6}
    A_{2} \varphi+\beta\left(\varphi-[\theta]^{4}\right)=g_{b}+g,
\end{equation}
и $\theta=F_{2}(\zeta, \varphi)$, если
\begin{equation}
    \label{eq:1_6:7}
    \sigma \theta^{\prime}+A(\zeta, \theta)
    +b\left([\theta]^{4}-\varphi\right)=f_{b}+f
    \quad \text { п. в. в }(0, T), \quad \theta(0)=\theta_{i n}.
\end{equation}
Здесь
$ (A(\zeta, \theta), v)=(k(\zeta) \nabla \theta, \nabla v)
    +\int_{\Gamma} p \theta v d \Gamma \quad \forall v \in V$.

Пусть $w(t)=M_{0}+M_{1} t, \quad t \in[0, T]$, где
\[
    \begin{gathered}
        M_{0}=\max \left\{\left\|\theta_{b}\right\|_{L^{\infty}(\Sigma)},
        \left\|\theta_{i n}\right\|_{L^{\infty}(\Omega)}\right\}, \\
        M_{1}=\sigma_{0}^{-1}\left(\|f\|_{L^{\infty}(Q)}+\max b M_{2}\right),
        \quad M_{2}=\beta_{0}^{-1}\|g\|_{L^{\infty}(Q)}.
    \end{gathered}
\]

\begin{lemma}
    \label{lm:1_6:1}
    Пусть выполняются условия (k1)--(k4), $0 \leq \theta \leq w(t), \varphi=F_{1}(\theta)$.
    В таком случае
    \begin{equation}
        \label{eq:1_6:8}
        0 \leq \varphi \leq w^{4}(t)+M_2.
    \end{equation}
\end{lemma}

\begin{proof}
    Умножая скалярно~\eqref{eq:1_6:6} на
    $\psi=\max \left\{\varphi-M_{2}-w^{4}, 0\right\} \in L^{2}(0, T ; V)$,
    получаем
    \[
        \left(A_{2} \varphi-g_{b}, \psi\right)+\left(\beta\left(\varphi-M_{2}
        -[\theta]^{4}\right), \psi\right)=\left(g-\beta M_{2}, \psi\right) \leq 0.
    \]

    Заметим, что с учетом ограничений на $\theta$ выполняются следующие неравенства:
    \[
        \begin{gathered}
            \left(A_{2} \varphi-g_{b}, \psi\right)=(\alpha \nabla \psi, \nabla \psi)+\int_{\Gamma}
            \gamma\left(\varphi-\theta_{b}^{4}\right) d \Gamma \geq(\alpha \nabla \psi, \nabla \psi), \\
            \left(\beta\left(\varphi-M_{2}-[\theta]^{4}\right),
            \psi\right)=(\beta \psi, \psi)+\left(\beta\left(w^{4}
            -[\theta]^{4}\right), \psi\right) \geq(\beta \psi, \psi).
        \end{gathered}
    \]
    Таким образом, $\psi=0$ и $\varphi \leq w^{4}+M_{2}$.

    Далее, умножая~\eqref{eq:1_6:6} на $\xi=\min \{\varphi, 0\} \in L^{2}(0, T ; V)$,
    мы получаем $\xi=0$.
    Следовательно, $\varphi \geq 0$.
\end{proof}

\begin{lemma}
    \label{lm:1_6:2}
    Пусть выполняются условия (k1)--(k4),
    $0 \leq \varphi \leq w^{4}(t)+M_{2}, \theta=F_{2}(\zeta, \varphi)$,
    $\zeta \in L^{\infty}(Q)$, тогда $0 \leq \theta \leq w(t)$.
\end{lemma}

\begin{proof}
    Пусть $\widehat{\theta}=\theta-w$.
    Перепишем уравнение~\eqref{eq:1_6:7} следующим образом:
    \begin{equation}
        \label{eq:1_6:9}
        \sigma \widehat{\theta}^{\prime}+A(\zeta, \theta)-f_{b}
        +b\left([\widehat{\theta}+w]^{4}
        -\left(\varphi-M_{2}\right)\right)=f-\sigma M_{1}+b M_{2} \leq 0.
    \end{equation}

    Умножим скалярно~\eqref{eq:1_6:9} на $\eta=\max \{\widehat{\theta}, 0\} \in W$.
    Учтём, что правая часть не положительна, а также то, что
    \[
        \begin{gathered}
            \left(\sigma \widehat{\theta}^{\prime}, \eta\right)
            =\left(\sigma \eta^{\prime}, \eta\right)=\frac{d}{2 d t}(\sigma \eta, \eta), \\
            \left(A(\zeta, \theta)-f_{b}, \eta\right)
            =(k(\zeta) \nabla \eta, \nabla \eta)
            +\int_{\Gamma} p\left(\widehat{\theta}
            +w-\theta_{b}\right) \eta d \Gamma \geq 0, \\
            \left([\widehat{\theta}+w]^{4}-w^{4}\right) \max \{\widehat{\theta}, 0\} \geq 0,
            \quad\left(w^{4}+M_{2}-\varphi\right) \eta \geq 0.
        \end{gathered}
    \]
    Тогда
    \[
        \frac{d}{d t}(\sigma \eta, \eta) \leq 0,\left.\quad \eta\right|_{t=0}=0.
    \]


    Таким образом, $\eta=0, \widehat{\theta} \leq 0, \theta \leq w$.
    Аналогично, умножив~\eqref{eq:1_6:9} на $\eta=\min \{\theta, 0\} \in W$,
    мы получим $\eta=0, \, \theta \geq 0$.
\end{proof}

Пусть $\theta_{0}=\theta_{\text{in}}, \, \varphi_{0}=F_{1}\left(\theta_{0}\right)$.
Рекурсивно определим последовательность
$\theta_{m} \in W, \quad \varphi_{m} \in L^{2}(0, T ; V)$ такую, что
\begin{equation}
    \label{eq:1_6:10}
    \theta_{m}=F_{2}\left(\theta_{m-1}, \varphi_{m-1}\right),
    \quad \varphi_{m}=F_{1}\left(\theta_{m}\right), \quad m=1,2, \ldots
\end{equation}

Из лемм~\ref{lm:1_6:1},~\ref{lm:1_6:2} следуют оценки:

\begin{equation}
    \label{eq:1_6:11}
    0 \leq \varphi_{m} \leq w^{4}(t)+M_{2},
    \quad 0 \leq \theta_{m} \leq w(t), \quad m=1,2, \ldots
\end{equation}

\begin{lemma}
    \label{lm:1_6:3}
    Если выполнены условия (k1)--(k4), то существует константа $C>0$,
    не зависящая от $m$, такая, что
    \begin{equation}
        \label{eq:1_6:12}
        \left\|\varphi_{m}\right\|_{L^{2}(0, T ; V)} \leq C,
        \quad\left\|\theta_{m}\right\|_{L^{2}(0, T ; V)} \leq C,
    \end{equation}
    \begin{equation}
        \label{eq:1_6:13}
        \int_{0}^{T-\delta}\left\|\theta_{m}(s+\delta)
        -\theta_{m}(s)\right\|^{2} d s \leq C \delta.
    \end{equation}
\end{lemma}

\begin{proof}
    Из определения последовательностей $\varphi_{m}, \theta_{m}$ следуют равенства:
    \begin{equation}
        \label{eq:1_6:14}
        A_{2} \varphi_{m}+\beta\left(\varphi_{m}-\left[\theta_{m}\right]^{4}\right)=g_{b}+g,
    \end{equation}
    \begin{equation}
        \label{eq:1_6:15}
        \sigma \theta_{m}^{\prime}+A\left(\theta_{m-1}, \theta_{m}\right)
        +b\left(\left[\theta_{m}\right]^{4}-\varphi_{m-1}\right)=f_{b}+f
        \, \text{ п. в. в }(0, T), \, \theta_{m}(0)=\theta_{in}.
    \end{equation}
    Оценки~\eqref{eq:1_6:12} выводятся стандартным образом
    из уравнений~\eqref{eq:1_6:14} и~\eqref{eq:1_6:15}
    и с учетом~\eqref{eq:1_6:11},
    т. е. ограниченности последовательностей в $L^{\infty}(Q)$.
    Получим оценку, гарантирующую компактность последовательности $\theta_{m}$ в $L^{2}(Q)$.
    Перепишем~\eqref{eq:1_6:15} как
    \begin{equation}
        \label{eq:1_6:16}
        \sigma \theta_{m}^{\prime}=\chi_{m} \, \text{ п. в. в }(0, T),
        \, \theta_{m}(0)=\theta_{in},
    \end{equation}
    где
    \[
        -\chi_{m}=A\left(\theta_{m-1}, \theta_{m}\right)
        +b\left(\left[\theta_{m}\right]^{4}-\varphi_{m-1}\right)-f_{b}-f.
    \]
    Заметим, что с учетом полученных оценок последовательность $\chi_{m}$
    ограничена в $L^{2}\left(0, T ; V^{\prime}\right)$.
    Умножим~\eqref{eq:1_6:16} на $\theta_{m}(t)-\theta_{m}(s)$ и проинтегрируем по $t$
    на интервале $(s, s+\delta)$ и по $s$ на $(0, T-\delta)$,
    предполагая, что $\delta>0$ достаточно мало.
    В результате получим
    \[
        \frac{1}{2} \int_{0}^{T-\delta}\left\|\sqrt{\sigma}\left(\theta_{m}(s+\delta)
        -\theta_{m}(s)\right)\right\|^{2} d s
        =\int_{0}^{T-\delta} \int_{s}^{s+\delta} c_{m}(t, s) d t d s,
    \]
    где
    \[
        c_{m}(t, s)=\left(\chi_{m}(t), \theta_{m}(t)-\theta_{m}(s)\right)
        \leq\left\|\chi_{m}(t)\right\|_{V^{\prime}}^{2}
        +\frac{1}{2}\left\|\theta_{m}(t)\right\|_{V}^{2}
        +\frac{1}{2}\left\|\theta_{m}(s)\right\|_{V}^{2}.
    \]


    Для оценки интегралов от слагаемых, зависящих от $t$,
    достаточно изменить порядок интегрирования.
    Используя ограниченность последовательностей $\theta_{m}$ в $L^{2}(0, T ; V)$
    и $\chi_{m}$ в $L^{2}\left(0, T ; V^{\prime}\right)$,
    получаем оценку равностепенной непрерывности~\eqref{eq:1_6:13}.
\end{proof}

Полученные оценки~\eqref{eq:1_6:12},~\eqref{eq:1_6:13} позволяют утверждать,
переходя при необходимости к подпоследовательностям,
что существуют функции $\widehat{\theta}, \widehat{\varphi}$ такие, что
\begin{equation}
    \label{eq:1_6:17}
    \begin{aligned}
        & \theta_{m} \rightarrow \widehat{\theta} \text { слабо в } L^{2}(0, T ; V),
        \text { сильно в } L^{2}(0, T ; H), \\
        & \varphi_{m} \rightarrow \widehat{\varphi} \text { слабо в } L^{2}(0, T ; V).
    \end{aligned}
\end{equation}

Результатов о сходимости~\eqref{eq:1_6:17} достаточно, чтобы перейти к пределу
при $m \rightarrow \infty$ в равенствах~\eqref{eq:1_6:14},~\eqref{eq:1_6:15} и доказать,
что предельные функции $\widehat{\theta}, \widehat{\varphi} \in L^{2}(0, T ; V)$
таковы, что $\sigma \widehat{\theta}^{\prime} \in L^{2}\left(0, T ; V^{\prime} \right)$
и для них выполняются равенства~\eqref{eq:1_6:4},~\eqref{eq:1_6:5}.

\begin{theorem}
    \label{th:1_6:1}
    Если выполнены условия (k1)--(k4), то существует хотя бы одно
    решение задачи~\eqref{eq:1_6:1}--\eqref{eq:1_6:3}.
\end{theorem}

\subsection{Теорема единственности и сходимость итеративного метода}
\label{subsec:ch1/sec5/subsec3}
Покажем, что в классе функций с ограниченным градиентом температуры решение единственно.
Это позволит доказать сходимость итерационной процедуры.

\begin{theorem}
    \label{th:1_6:2}
    Если выполнены условия (k1)--(k4) и $\theta_{*}, \varphi_{*}$ является
    решением задачи~\eqref{eq:1_6:1}--\eqref{eq:1_6:3}
    так, что $\theta_{*}, \nabla \theta_{*} \in L^{\infty}(Q)$,
    то других ограниченных решений этой задачи нет.
\end{theorem}

\begin{proof}
    Пусть $\theta_{1}, \varphi_{1}$ — другое решение задачи~\eqref{eq:1_6:1}--\eqref{eq:1_6:3},
    $\theta=\theta_{1}-\theta_{*}, \varphi=\varphi_{1}-\varphi_{*}$.
    Тогда
    \begin{equation}
        \label{eq:1_6:18}
        \sigma \theta^{\prime}+A_{1}\left(\theta_{1}\right)-A_{1}\left(\theta_{*}\right)
        +b\left(\left[\theta_{1}\right]^{4}-\left[\theta_{*}\right]^{4}-\varphi\right)
        =0 \, \text { п. в. в }(0, T), \, \theta(0)=0.
    \end{equation}
    \begin{equation}
        \label{eq:1_6:19}
        A_{2} \varphi+\beta\left(\varphi
        -\left(\left[\theta_{1}\right]^{4}
        -\left[\theta_{*}\right]^{4}\right)\right)
        =0 \, \text{ п. в. в }(0, T).
    \end{equation}

    Умножим~\eqref{eq:1_6:18} на $\theta$ и проинтегрируем по времени.
    Получим:
    \[
        \begin{gathered}
            \frac{1}{2}\|\sqrt{\sigma} \theta\|^{2}
            +\int_{0}^{t}\left(\left(k\left(\theta_{1}\right) \nabla \theta,
            \nabla \theta\right)+\int_{\Gamma} p \theta^{2}(s) d \Gamma\right) ds= \\
            -\int_{0}^{t}\left(b\left(\left[\theta_{1}\right]^{4}
            -\left[\theta_{*}\right]^{4}-\varphi\right), \theta\right) ds
            -\int_{0}^{t}\left(\left(k\left(\theta_{1}\right)
            -k\left(\theta_{*}\right)\right) \nabla \theta_{*}, \nabla \theta\right) ds.
        \end{gathered}
    \]


    Пусть $\left|\theta_{1}\right| \leq M,\left|\theta_{*}\right| \leq M$.
    С учетом ограничения на функцию $k$ получаем неравенство
    \begin{equation}
        \label{eq:1_6:20}
        \begin{aligned}
            &\frac{\sigma_{0}}{2}\|\theta\|^{2}+k_{0} \int_{0}^{t}\|\nabla \theta\|^{2} d s \leq \\
            & \int_{0}^{t}\left(4 M \max b\|\theta\|^{2}+\|\varphi\|\|\theta\|\right) d s
            +k_{2}\left\|\nabla \theta_{*}\right\|_{L^{\infty}(Q)} \int_{0}^{t}\|\theta\|\|\nabla \theta\| d s.
        \end{aligned}
    \end{equation}

    Принимая во внимание,
    что $\|\theta\|\|\nabla \theta\| \leq \varepsilon\|\nabla \theta\|^{2}
    +\frac{1}{4 \varepsilon}\|\theta\|^{2}$ и полагая
    \[
        \varepsilon=\frac{k_{0}}{k_{2}\left\|\nabla \theta_{*}\right\|_{L^{\infty}(Q)}},
    \]
    из~\eqref{eq:1_6:20} получаем оценку
    \begin{equation}
        \label{eq:1_6:21}
        \frac{\sigma_{0}}{2}\|\theta\|^{2} \leq \int_{0}^{t}\left(4 M \max b\|\theta\|^{2}
        +\|\varphi\|\|\theta\|\right) d s
        +\frac{1}{4 \varepsilon} k_{2}\left\|\nabla
        \theta_{*}\right\|_{L^{\infty}(Q)} \int_{0}^{t}\|\theta\|^{2} d s.
    \end{equation}

    Умножим~\eqref{eq:1_6:19} скалярно на $\varphi$, в результате получим:
    \[
        \left(A_{2} \varphi, \varphi\right)+(\beta \varphi, \varphi)
        =\left(\beta\left(\left[\theta_{1}\right]^{4}
        -\left[\theta_{2}\right]^{4}\right), \varphi\right)
        \leq 4 \max \beta M^{3}\|\theta\|\|\varphi\|.
    \]


    Следовательно, $\|\varphi\| \leq 4 \beta_{0}^{-1} \max \beta M^{3}\|\theta\|$.
    Тогда из~\eqref{eq:1_6:21} и неравенства Гронуолла следует,
    что $\theta=0$, $\theta_{1}$ совпадает с $\theta_{*}$ и,
    соответственно, $\varphi_{1}$ совпадает с $\varphi_{*}$.
\end{proof}

\begin{theorem}
    \label{th:1_6:3}
    Если выполнены условия (k1)--(k4) и $\theta_{*}, \varphi_{*}$ является
    решением задачи~\eqref{eq:1_6:1}--\eqref{eq:1_6:3}
    так, что $\theta_{*}, \nabla \theta_{*} \in L^{\infty}(Q)$.
    Тогда для последовательностей~\eqref{eq:1_6:10} справедливы следующие сходимости:
    \[
        \theta_{m} \rightarrow \theta_{*} \quad \text { в } L^{2}(0, T ; V),
        \quad \varphi_{m} \rightarrow \varphi_{*} \quad \text { в } L^{2}(0, T ; V).
    \]
\end{theorem}

\begin{proof}
    Сначала покажем, что $\theta_{m} \rightarrow \theta_{*}$ в $L^{2}(0, T ; H)$.
    Предполагая противное, заключаем, что существуют $\varepsilon_{0}>0$ и подпоследовательность
    $\theta_{m^{\prime}}$ такие,
    что $\left\|\theta_{m^{\prime}}-\theta_{*}\right\|_{L^{2}(0, T ; H)} \geq \varepsilon_{0}$.
    Оценки~\eqref{eq:1_6:12},~\eqref{eq:1_6:13} позволяют утверждать,
    переходя при необходимости к подпоследовательностям,
    что справедливы результаты сходимости~\eqref{eq:1_6:17}, где $\widehat{\theta}, \widehat{\varphi}$
    также является решением задачи~\eqref{eq:1_6:1}--\eqref{eq:1_6:3}.
    Следовательно, $\left\|\widehat{\theta}-\theta_{*}\right\|_{L^{2}(0, T ; H)} \geq \varepsilon_{0}$,
    что противоречит теореме~\ref{th:1_6:2} о единственности решения.
    Из уравнений~\eqref{eq:1_6:14} и~\eqref{eq:1_6:15}, с учетом~\eqref{eq:1_6:11},
    т. е. ограниченности последовательностей
    в $L^{\infty}(Q)$, а также доказанной
    сходимости $\theta_{m}$ в $L^{2}(0, T ; H)$,
    следуют сходимости $\theta_{m} \rightarrow \theta_{*},
    \varphi_{m} \rightarrow \varphi_ {*}$ в $L^{2}(0, T ; V)$.
\end{proof}
