\chapter{Модели сложного теплообмена}\label{ch:ch1}


\section{Уравнение переноса теплового излучения}\label{sec:ch1/sec1}


Уравнение переноса излучения описывает поле интенсивности излучения
при взаимодействии теплового излучения с поглощающей, излучающей и рассеивающей средой
(radiatively participating medium).
Будем предполагать, что среда имеет постоянный показатель преломления $n$, является неполяризующей,
находится в состоянии покоя (по сравнению со скоростью света) и в локальном
термодинамическом равновесии [67, с. 280].


Спектральной интенсивностью излучения $I_\nu (x \omega, t)$ $[\text{Вт}/(м^2 \cdot \text{стер} \cdot \text{Гц})]$
называется количество энергии излучения, проходящего через единичную
площадку, перпендикулярную направлению распространения $\omega$, внутри единичного телесного угла,
осью которого является направление $\omega$, в единичном
интервале частот, включающем частоту $\nu$, и в единицу времени.
Считаем, что направления излучения $\omega$ связаны с точками единичной
сферы $S = \{\omega \in R^3: \| \omega\| = 1\}$.


Рассмотрим пучок излучения интенсивностью $I_\nu (x \omega, t)$, распространяющегося в поглощающей,
излучающей и рассеивающей среде в заданном направлении.
Энергия излучения будет уменьшаться вследствие поглощения
излучения веществом и отклонения части его от первоначальной траектории в
результате рассеяния во всех направлениях, но одновременно она будет возрастать
вследствие испускания излучения веществом.


Обозначим через $\kappa_{a\nu}$[$\text{м}^{-1}$] спектральный коэффициент поглощения,
равный доле падающего излучения, поглощенной веществом на единице длины
пути распространения излучения.
Приращение интенсивности излучения за счет поглощения равно
$(dI_\nu)_\text{погл} = -\kappa_{a\nu} I_\nu ds$, где $ds$ — элемент пути.
Отметим, что $1/\kappa_{a_\nu}$ есть средняя длина свободного пробега фотона до его поглощения
веществом [67, с. 281].


Для получения выражения для испускания излучения элементом объема
часто используется предположение о локальном термодинамическом равновесии.
Оно означает, что любой малый элемент объема среды находится в
локальном термодинамическом равновесии, вследствие чего состояние любой
точки может быть охарактеризовано локальной температурой $T(x)$.
Это предположение законно, когда столкновения атомов в веществе происходят столь
часто, что это приводит к локальному термодинамическому равновесию в каждой точке $x$ среды.
В этом случае испускание излучения элементом объема
можно описать с помощью функции Планка [68, с. 36].
Приращение интенсивности излучения за счет испускания равно
$(dI_\nu)_\text{исп}  = j_\nu ds$  $j_\nu$ -- коэффициент испускания.
В локальном термодинамическом равновесии справедлива формула [68, с. 36],
[67, с. 282] $j_\nu = \kappa_{a\nu} I_{b\nu}$, где
$I_{b\nu}$ — интенсивность излучения абсолютно черного тела.


Абсолютно черным называется тело, которое поглощает все падающее
со всех направлений излучение любой частоты без отражения, пропускания и
рассеяния. Из закона Кирхгофа следует, что абсолютно черное тело также излучает
максимальное количество энергии при данной температуре [68, с. 25], [67, с. 5].
Интенсивность излучения абсолютно черного тела при температуре $T$ равна

\[
    I_{b\nu}(T) = \frac{2h\nu^3n^2}{c^2_0(e^{h\nu/kT} - 1)},
\]

где $h$ -- постоянная Планка, $k$ - постоянная Больцмана, $c_0$ -- скорость света в вакууме,
$T$ -- абсолютная температура, $n$ -- показатель преломления.
Интегральная интенсивность злучения абсолютно черного тела $I_b(T)$ вычилсяется по формуле [68, с. 28], [67, с. 10]
\[
    I_b(T) = \int^{\infty}_0 I_{b\nu}(T) d\nu = \frac{n^2 \sigma T^4}{\pi},
\]
где $\sigma$ -- постоянная Стефана-Больцмана.


Рассеяние излучения учитывается так же, как поглощение, с той разницей,
что рассеянная энергия просто перенаправляется и возникает в приращении
интенсивности излучения в другом направлении.
Различают когерентное и некогерентное рассеяние.
Рассеяние называется когерентным, если рассеянное излучение имеет ту же самую частоту,
что и падающее излучение, и некогерентным, если частота рассеянного
излучения отличается от частоты падающего излучения.
В дальнейшем мы будем рассматривать только когерентное рассеяние.
Обозначим через $\kappa_{s\nu}$ [$\text{м}^{-1}$] спектральный коэффициент рассеяния, равный
доле падающего излучения, рассеянной веществом во всех направлениях на
единице длины пути распространения излучения.
Тогда приращение интенсивности излучения за счет «рассеяния вне» равно
$(dI_{nu})_\text{расс.вне} = - \kappa_{s\nu}I_{nuds}$.
Для описания "рассеяния в" вводится неотрицательная фазовая функция рассеяния
$P_{nu} = (\omega, \omega')$ такая, что $\frac{1}{4\pi}\int_S P_{nu} (\omega, \omega')d\omega = 1$.
Величина $\frac{1}{4\pi}\int_S P_{nu} (\omega, \omega')d\omega$
определяет вероятность того, что излучение частоты $\nu$,
падающее в направлении $\omega'$,
будет рассеяно в пределах элементарного телесного угла $d\omega$ в направлении $\omega$.
Случай $P_\nu \equiv 1$ соответствует изотропному рассеянию.
Тогда для того, чтобы получить приращение интенсивности излучения за счет «рассеяния в», нужно
проинтегрировать $I_\nu(\omega')P_\nu(\omega,\omega')/4\pi$ по всем входящим направлениям
$\omega'$ [67, с. 283]:
$(dI_\nu)_\text{расс.в} = ds \frac{\kappa_{s\nu}}{4\pi}\int_S I_\nu(\omega')P_{nu(\omega,\omega')d\omega'}$.
Учитывая приращения интенсивности излучения с учетом поглощения,
испускания и рассеяния, получим искомое уравнение переноса излучения
[68, с. 272], [67, с. 284]:
\[
    \begin{aligned}
        &\frac{1}{c} \frac{\partial I_v(x, \omega, t)}
        {\partial t}+\omega \cdot \nabla_x I_v(x, \omega, t)+\kappa_v I_v(x, \omega, t)=\\
        &=\mathrm{\kappa}_{a v} I_{b v}(T(x, t))+\frac{\mathrm{K}_{s v}}{4 \pi}
        \int_S I_v\left(x, \omega^{\prime}, t\right) P_v\left(\omega, \omega^{\prime}\right) d \omega^{\prime}.
    \end{aligned}
\]
Здесь $\kappa_{nu = \kappa_{a\nu} + \kappa_{s\nu}}$ -- полный спектральный коэффициент взаимодействия,
$c$ --  скорость света в среде.


Далее получим граничные условия для уравнения переноса излучения.
Будем считать, что граница области непрозрачна, испускает излучение диффузно
и отражает излучение диффузно и зеркально.
Степенью черноты поверхности $\varepsilon \nu(x)$ называется отношение количества энергии,
испускаемого данной поверхностью, к количеству энергии, испускаемому абсолютно черным телом при
той же температуре.
При диффузном испускании излучения степень черноты не зависит от направления и определяется формулой
$\varepsilon_\nu(x) = \frac{I_{\nu,\text{исп}}(x)}{I_{b\nu}(T(x))}$, где
$I_{\nu,\text{исп})}(x)$ -- интенсивность излучения, испускаемого
поверхностью при температуре $T(x)$ [68, с. 53].

При диффузном поглощении степень черноты равняется
поглотительной способности, которая равна доле поглощенного излучения [67, с. 66].
Также введем коэффициенты зеркального и диффузного отражения
$\rho^s_\nu(x), \rho^d_\nu(x)$ как части зеркально и диффузно отраженного излучения соответственно.
Отметим, что в случае непрозрачной поверхности $\varepsilon_\nu + \rho^s_\nu + \rho^d_{nu = 1}$.
Граничное условие имеет вид [69], [67, с. 289]
\[
    \begin{aligned}
        &I_v(x, \omega, t)=\varepsilon_v(x) I_{b v}(T(x, t))+\rho_v^s(x) I_v\left(x, \omega_R, t\right)+ \\
        &\quad+\frac{\rho_v^d(x)}{\pi} \int_{\omega^{\prime} \cdot \mathbf{n}>0} I_v\left(x, \omega^{\prime},
        t\right) \omega^{\prime} \cdot \mathbf{n} d \omega^{\prime}, \omega \cdot \mathbf{n}<0,
    \end{aligned}
\]
где $\mathbf{n}$ -- вектор внешней нормали к границе области, $\omega$ -- входящее направление,
$\omega_R$ -- направление отражения, определяемое из соотношения
$\omega + (-\omega_R) = 2 \text{cos}\theta = \omega \cdot \mathbf{n}$ косинус угла между
вектором нормали и направлением падающего излучения.
Таким образом, $\omega_R = \omega -2(\omega \cdot \mathbf{n})\mathbf{n}$.


Поле температуры описывается уравнением теплопроводности [67, с. 297]:
\[
    \rho c_p \frac{\partial T(x, t)}{\partial t}-k \Delta T(x, t)+\rho c_p
    \mathbf{v}(x, t) \cdot \nabla T(x, t)=-\div \mathbf{q}_r(x, t),
\]

где $T[K]$ -- температура, $\mathbf{v}$ [м/с] поле скоростей, $k$ [Вт/м $\cdot К$]
-- коэффициент теплопроводности, $c_p$ [Дж/(кг$\cdot K$)] -- удельная теплоёмкость при постоянном
давлении, $\rho$ [кг/$\text{м}^3$] -- плотность,
$\mathbf{q}_r$ -- вектор плотности потока излучения, определяемый формулой [67, с. 292]
$\mathbf{q}_r(x, t) = \int^\infty_0\int_S I_\nu(x,\omega,t)\omega d\omega d\nu$.
Дивергенция вектора плотности потока излучения $\div \mathbf{q}_r$
характеризует изменение в единицу времени энергии излучения,
заключенной в единице объема среды, по всему спектру частот вследствие испускания
излучения во всё сферическое пространство и поглощения падающего из него излучения [68, с. 274].
Для нахождения $\div \mathbf{q}_{r}$ проинтегрируем уравнение (1.1) по $\boldsymbol{\omega} \in S$, получим
\[
    \begin{gathered}
        \frac{1}{c} \frac{\partial}{\partial t} \int_{S} I_{v}(x, \boldsymbol{\omega}, t) d
        \boldsymbol{\omega}+\div \int_{S} I_{v}(x, \boldsymbol{\omega}, t) \boldsymbol{\omega} d
        \boldsymbol{\omega}+\kappa_{v} \int_{S} I_{v}(x, \boldsymbol{\omega}, t) d \boldsymbol{\omega}= \\
        =4 \pi \kappa_{a v} I_{b v}(T(x, t))+\frac{\kappa_{s v}}{4 \pi} \int_{S}
        \int_{S} I_{v}\left(x, \boldsymbol{\omega}^{\prime}, t\right) P_{v}\left(\boldsymbol{\omega},
        \boldsymbol{\omega}^{\prime}\right) d \boldsymbol{\omega}^{\prime} d \boldsymbol{\omega}.
    \end{gathered}
\]


Поменяем порядок интегрирования во втором слагаемом в правой части:
\[
    \begin{aligned}
        \int_{S} \int_{S} I_{v}\left(x, \omega^{\prime}, t\right) P_{v} &\left(\boldsymbol{\omega},
        \boldsymbol{\omega}^{\prime}\right) d \boldsymbol{\omega}^{\prime} d \boldsymbol{\omega}=\\
        &=\int_{S} I_{v}\left(x, \boldsymbol{\omega}^{\prime}, t\right) \int_{S} P_{v}\left(\boldsymbol{\omega},
        \omega^{\prime}\right) d \boldsymbol{\omega} d \boldsymbol{\omega}^{\prime}=4 \pi \int_{S} I_{v}\left(x,
        \boldsymbol{\omega}^{\prime}, t\right) d \boldsymbol{\omega}^{\prime}.
    \end{aligned}
\]


Обозначим через $G_{v}(x, t)=\int_{S} I_{v}(x, \omega, t) d \omega$
пространственную плотность падающего излучения.
Тогда
\[
    \frac{1}{c} \frac{\partial G_{v}(x, t)}{\partial t}+\div \int_{S} I_{v}(x, \boldsymbol{\omega}, t)
    \boldsymbol{\omega} d \omega+\mathrm{\kappa}_{v} G_{v}(x, t)=4 \pi
    \mathrm{\kappa}_{a v} I_{b v}(T(x, t))+\mathrm{\kappa}_{s v} G_{v}(x, t),
\]
отсюда
\[
    \begin{gathered}
        \div \int_{S} I_{v}(x, \omega, t) \omega \boldsymbol{\omega} d
        \boldsymbol{\omega}=4 \pi \mathrm{\kappa}_{a v} I_{b v}(T(x, t))-\mathrm{\kappa}_{a v}
        G_{v}(x, t)-\frac{1}{c} \frac{\partial G_{v}(x, t)}{\partial t}, \\
        \div \mathbf{q}_{r}(x, t)=\int_{0}^{\infty} \mathrm{\kappa}_{a v}
        \left(4 \pi I_{b v}(T(x, t))-G_{v}(x, t)\right) d v-\frac{1}{c}
        \frac{\partial}{\partial t} \int_{0}^{\infty} G_{v}(x, t) d v.
    \end{gathered}
\]
Таким образом, уравнение теплопроводности принимает вид
\[
    \begin{aligned}
        &\rho c_{p} \frac{\partial T(x, t)}{\partial t}-k \Delta T(x, t)+\rho c_{p} \mathbf{v}(x, t) \cdot \nabla T(x, t)= \\
        &=-\int_{0}^{\infty} \int_{S} \kappa_{a v}\left(I_{b v}(T(x, t))-I_{v}(x, \omega, t)\right) d \omega d v+\frac{1}{c} \frac{\partial}{\partial t} \int_{0}^{\infty} \int_{S} I_{v}(x, \omega, t) d \omega d v
    \end{aligned}
\]

Получим граничные условия для уравнения теплопроводности из закона Ньютона-Рихмана.
Согласно этому закону, плотность теплового потока пропорциональна разности температур
поверхности тела $T$ и окружающей среды $T_{b}$ : $q=h\left(T-T_{b}\right)$.
Здесь $h\left[\right.$ Вт $\left./\left(\mathrm{м}^{2} \cdot \mathrm{K}\right)\right]$ - коэффициент
теплоотдачи, характеризующий интенсивность теплообмена между поверхностью тела и окружающей средой.
Численно он равен количеству тепла, отдаваемому (воспринимаемому) единицей поверхности в единицу времени
при разности температур между поверхностью и средой в $1 \mathrm{~K}[70]$. Отметим, что непосредственно
на поверхности контакта тела с окружающей средой $T=T_{b}$, однако мы считаем, что температура $T$ на
границе поверхности - это температура за пределами пограничного слоя [71].
Рассматривая граничное условие для уравнения переноса излучения (1.2),
будем считать, что поверхностное излучение происходит из пограничного слоя,
поэтому в качестве аргумента функции $I_{b v}(T)$ будем использовать $T_{b}$.
По закону сохранения энергии количество тепла, отводимое с единицы поверхности
вследствие теплоотдачи, должно равняться теплу, подводимому к единице поверхности
вследствие теплопроводности из внутренних объемов тела, тогда
$h\left(T-T_{b}\right)=\mathbf{q} \cdot \mathbf{n}=-k \nabla T
\cdot \mathbf{n}=-k \frac{\partial T}{\partial n}$.
Таким образом, граничное условие имеет вид:

\[
    k \frac{\partial T(x, t)}{\partial n}+h(x)\left(T(x, t)-T_{b}(x, t)\right)=0 .
\]

Следует отметить, что условия третьего рода для температуры обычно ставятся на твердой стенке,
где $\mathbf{v} \cdot \mathbf{n}=0$.
В данном случае постановка условий третьего рода на всей границе и, в частности,
на участке втекания моделирует процесс теплообмена при малых значениях нормальной компоненты скорости.

В дальнейшем мы будем рассматривать случай «серой» среды,
когда $\kappa_{a v}$ и $\mathrm{K}_{s v}$ не зависят от частоты $v$,
так что $\mathrm{K}_{a v}=\mathrm{K}_{a}, \mathrm{~K}_{s v}=\mathrm{K}_{s}$.
Граница области также предполагается «серой».
В этом случае уравнения и граничные условия принимают вид (ср. [69]):
\[
    \begin{aligned}
        & \frac{1}{c} \frac{\partial I(x, \boldsymbol{\omega}, t)}{\partial t}+\boldsymbol{\omega} \cdot \nabla_{x} I(x, \boldsymbol{\omega}, t)+\kappa I(x, \boldsymbol{\omega}, t)= \\
        & =\frac{\kappa_{s}}{4 \pi} \int_{S} P\left(\omega, \omega^{\prime}\right) I\left(x, \omega^{\prime}, t\right) d \omega^{\prime}+\kappa_{a} \frac{\sigma n^{2} T^{4}(x, t)}{\pi}, \\
        & \rho c_{p} \frac{\partial T(x, t)}{\partial t}-k \Delta T(x, t)+\rho c_{p} \mathbf{v}(x, t) \cdot \nabla T(x, t)= \\
        & =-\mathrm{\kappa}_{a}\left(4 \sigma n^{2} T^{4}(x, t)-\int_{S} I(x, \boldsymbol{\omega}, t) d \boldsymbol{\omega}\right)+\frac{1}{c} \frac{\partial}{\partial t} \int_{S} I(x, \boldsymbol{\omega}, t) d \boldsymbol{\omega}, \\
        & \text { гр.у.: } I(x, \boldsymbol{\omega}, t)=\varepsilon(x) \frac{\sigma n^{2}}{\pi} T_{b}^{4}(x, t)+\rho^{s}(x) I\left(x, \boldsymbol{\omega}_{R}, t\right)+ \\
        & +\frac{\rho^{d}(x)}{\pi} \int_{\omega^{\prime} \cdot \mathbf{n}>0} I\left(x, \omega^{\prime}, t\right) \omega^{\prime} \cdot \mathbf{n} d \omega^{\prime}, \omega \cdot \mathbf{n}<0, \\
        & \text { гр.у.: } k \frac{\partial T(x, t)}{\partial n}+h(x)\left(T(x, t)-T_{b}(x, t)\right)=0 \text {. } \\
        & \text { Здесь } I=\int_{0}^{\infty} I_{v} d v \text {. }
    \end{aligned}
\]
Поставим также начальные условия:
\[
    I(x, \boldsymbol{\omega}, 0)=I_{0}(x, \boldsymbol{\omega}), \quad T(x, 0)=T_{0}(x) .
\]
Соотношения (1.3)-(1.7) представляют собой модель сложного теплообмена с полным уравнением переноса излучения.


Перейдем к безразмерным величинам. Обозначим
\[
    I(x, \omega, t)=\left(\frac{\sigma n^{2}}{\pi}
    T_{\max }^{4}\right) I^{*}(x, \boldsymbol{\omega}, t),
    \quad T(x, t)=T_{\max } \theta(x, t),
\]
Здесь $I^{*}-$ нормализованная интенсивность излучения, $\theta-$ нормализованная температура, $T_{\max }$ - максимальная температура в ненормализованной модели. Подставив (1.8) в уравнения $(1.3),(1.4)$, получим
\[
    \begin{aligned}
        \frac{1}{c} \frac{\partial I^{*}(x, \omega, t)}{\partial t}+\omega \cdot \nabla_{x} I^{*}(x, \omega, t) &+\kappa I^{*}(x, \omega, t)=\\
        =& \frac{\kappa_{s}}{4 \pi} \int_{S} P\left(\omega, \omega^{\prime}\right) I^{*}\left(x, \omega^{\prime}, t\right) d \omega^{\prime}+\kappa_{a} \theta^{4}(x, t),
    \end{aligned}
\]
\[
    \begin{aligned}
        \frac{\partial \theta(x, t)}{\partial t} &-a \Delta \theta(x, t)+\mathbf{v}(x, t) \cdot \nabla \theta(x, t)=\\
        &=-b \kappa_{a}\left(\theta^{4}(x, t)-\frac{1}{4 \pi} \int_{S} I^{*}(x, \omega, t) d \omega\right)+\frac{b}{4 \pi c} \frac{\partial}{\partial t} \int_{S} I^{*}(x, \omega, t) d \boldsymbol{\omega},
    \end{aligned}
\]
где $a=\frac{k}{\rho c_{p}}, b=\frac{4 \sigma n^{2} T_{\max }^{3}}{\rho c_{p}}$.
Подставляя (1.8) в граничные условия (1.5),

и полагая $T_{b}=T_{\max } \theta_{b}$, получим
\[
    \begin{aligned}
        & \text { гр.у.: } I^{*}(x, \boldsymbol{\omega}, t)=\varepsilon(x) \theta_{b}^{4}(x, t)+\rho^{s}(x) I^{*}\left(x, \omega_{R}, t\right)+ \\
        & +\frac{\rho^{d}(x)}{\pi} \int_{\omega^{\prime} \cdot \mathbf{n}>0} I^{*}\left(x, \omega^{\prime}, t\right) \omega^{\prime} \cdot \mathbf{n} d \omega^{\prime}, \omega \cdot \mathbf{n}<0, \\
        & \text { гр.у.: } a \frac{\partial \theta(x, t)}{\partial n}+\beta(x)\left(\theta(x, t)-\theta_{b}(x, t)\right)=0 \text {, }
    \end{aligned}
\]
где $\beta=\frac{h}{\rho c_{p}}$. Аналогично получаем начальные условия:
\[
    I^{*}(x, \boldsymbol{\omega}, 0)=I_{0}^{*}(x, \boldsymbol{\omega}), \quad \theta(x, 0)=\theta_{0}(x),
\]
где $I_{0}^{*}(x, \boldsymbol{\omega})=\left(\frac{\sigma n^{2}}{\pi}
T_{\max }^{4}\right)^{-1} I_{0}(x, \boldsymbol{\omega}), \quad \theta_{0}(x)=\frac{T_{0}(x)}{T_{\max }}$.


\section{$1.2$ Диффузионное $P_{1}$ приближение}
$P_{1}$ приближение уравнения переноса излучения является частным случаем метода сферических гармоник $\left(P_{N}\right)$. Идея $P_{N}$ приближений состоит в том, что функцию интенсивности излучения $I(x, \omega)$ раскладывают в ряд Фурье по сферическим гармоникам $\mathcal{Y}_{l}^{m}(\boldsymbol{\omega})$ [67, с. 496]:
\[
    I(x, \omega)=\sum_{l=0}^{\infty} \sum_{m=-l}^{l} I_{l}^{m}(x) \mathcal{Y}_{l}^{m}(\omega),
\]
где $I_{l}^{m}(x)$ - коэффициенты, зависящие от $x$.
Также в ряд раскладывают фазовую функцию $P\left(\omega, \omega^{\prime}\right)$.
Тогда решение уравнения переноса излучения ищется в виде отрезка ряда Фурье для $l \leqslant N$.
При подстановке указанной конечной суммы в исходное уравнение интегро-дифференциальное уравнение
переноса излучения относительно $I(x, \omega)$ сводится к $(N+1)^{2}$ дифференциальным уравнениям
относительно $I_{l}^{m}(x)$.


В $P_{1}$ приближении используется линейное приближение для интенсивности излучения и фазовой функции:
\[
    \begin{gathered}
        I^{*}(x, \omega, t)=\varphi(x, t)+\boldsymbol{\Phi}(x, t) \cdot \omega, \\
        P\left(\omega, \omega^{\prime}\right)=1+A \omega \cdot \omega^{\prime} .
    \end{gathered}
\]
Для фазовой функции выполняется условие нормировки:
\[
    \frac{1}{4 \pi} \int_{S} P\left(\omega, \omega^{\prime}\right) d \omega=1+\frac{A}{4 \pi}
    \int_{S} \omega \cdot \omega^{\prime} d \omega=1,
\]
вычисление интеграла см. ниже. Коэффициент $A \in[-1,1]$ описывает анизотропию рассеяния,
а величина $A / 3$ имеет смысл среднего косинуса угла рассеяния, поскольку
\[
    \frac{1}{4 \pi} \int_{S}\left(\omega \cdot \omega^{\prime}\right)
    P\left(\omega, \omega^{\prime}\right) d \omega=\frac{1}{4 \pi}
    \int_{S} \omega \cdot \omega^{\prime} d \omega+\frac{A}{4 \pi}
    \int_{S}\left(\omega \cdot \omega^{\prime}\right)\left(\omega \cdot \omega^{\prime}\right) d \omega=\frac{A}{3},
\]
вычисление интегралов см. ниже. Случай $A=0$ соответствует изотропному рассеянию.
Диапазон допустимых значений величины $A \in[-1,1]$ обусловлен тем, что при $|A|>1$
фазовая функция может принимать отрицательные значения.

Отметим, что если функция $I^{*}$ ищется в виде (1.14), то [67, с. 502]
\[
    G(x, t)=\int_{S} I^{*}(x, \omega, t) d \omega=4 \pi \varphi(x, t),
    \quad \mathbf{q}_{r}(x, t)=\int_{S} I^{*}(x, \boldsymbol{\omega}, t)
    \boldsymbol{\omega} d \boldsymbol{\omega}=\frac{4 \pi}{3} \boldsymbol{\Phi}(x, t),
\]
поэтому
\[
    \varphi(x, t)=\frac{1}{4 \pi} G(x, t), \quad \Phi(x, t)=\frac{3}{4 \pi} \mathbf{q}_{r}(x, t),
\]
где $G$ - аппроксимация пространственной плотности падающего излучения,
$\mathbf{q}_{r}-$ аппроксимация плотности потока излучения.
Следовательно, функция $\varphi(x, t)$ имеет физический смысл нормализованной интенсивности излучения в
точке $x$ в момент времени $t$, усредненной по всем направлениям.

Лемма 1. Справедливы равенства:
\[
    \begin{gathered}
        \int_{S} 1 \cdot d \omega=4 \pi, \quad \int_{S} \mathbf{a} \cdot \boldsymbol{\omega} d \omega=0,
        \quad \int_{S}(\mathbf{a} \cdot \boldsymbol{\omega})(\mathbf{b} \cdot \boldsymbol{\omega}) d
        \omega=\frac{4 \pi}{3} \mathbf{a} \cdot \mathbf{b}, \\
        \int_{S} \omega d \omega=0, \quad \int_{S}(\mathbf{a} \cdot \omega) \omega d
        \omega=\frac{4 \pi}{3} \mathbf{a}, \\
        \int_{\boldsymbol{\omega} \cdot \mathbf{a}>0} \mathbf{a} \cdot \boldsymbol{\omega} d
        \boldsymbol{\omega}=\pi, \quad \int_{\omega \cdot \mathbf{a}>0}(\mathbf{a} \cdot
        \boldsymbol{\omega})(\mathbf{b} \cdot \boldsymbol{\omega}) d \omega=\frac{2 \pi}{3} \mathbf{a} \cdot \mathbf{b},
    \end{gathered}
\]
\includegraphics[width=\textwidth]{2022_10_02_2aab7a2952871da15492g-17}

\textbf{Доказательство.}
Первое равенство вытекает из определения поверхностного интеграла и представляет собой выражение для площади
поверхности единичной сферы.


Для вычисления остальных интегралов воспользуемся формулой перехода от поверхностного интеграла
к двойному [72, с. 143, теорема 5.3]:
\[
    \int_{S} f(\omega) d \omega=\int_{D} f\left(\omega_{1}(u, v), \omega_{2}(u, v),
    \omega_{3}(u, v)\right)\left|\boldsymbol{\omega}_{u} \times \boldsymbol{\omega}_{v}\right| d u d v
\]
где $D=\left\{(u, v): 0 \leqslant u \leqslant 2 \pi,-\frac{\pi}{2} \leqslant v \leqslant \frac{\pi}{2}\right\},
\omega_{1}(u, v)=\cos u \cos v, \omega_{2}(u, v)=$ $\sin u \cos v, \omega_{3}(u, v)=\sin v,\left|\omega_{u}
\times \omega_{v}\right| d u d v=\cos v d u d v-$ элемент площади поверхности единичной сферы.
Тогда для вычисления остальных интегралов воспользуемся формулой перехода от поверхностного интеграла
к двойному [72, с. 143, теорема 5.3]:
\[
    \int_{S} f(\omega) d \omega=\int_{D} f\left(\omega_{1}(u, v), \omega_{2}(u, v),
    \omega_{3}(u, v)\right)\left|\boldsymbol{\omega}_{u} \times \boldsymbol{\omega}_{v}\right| d u d v
\]
где $D=\left\{(u, v): 0 \leqslant u \leqslant 2 \pi,-\frac{\pi}{2} \leqslant v \leqslant
\frac{\pi}{2}\right\}, \omega_{1}(u, v)=\cos u \cos v, \omega_{2}(u, v)=$ $\sin u \cos v,
\omega_{3}(u, v)=\sin v,\left|\omega_{u} \times \omega_{v}\right| d u d v=\cos v d u d v-$
элемент площади поверхности единичной сферы.
Тогда
\[
    \int_{S} f(\omega) d \omega=\int_{D} f\left(\omega_{1}(u, v), \omega_{2}(u, v),
    \omega_{3}(u, v)\right)\left|\boldsymbol{\omega}_{u} \times \boldsymbol{\omega}_{v}\right| d u d v
\]



Для вычисления второго интеграла положим
$f(\boldsymbol{\omega})=\mathbf{a} \cdot \boldsymbol{\omega}=\sum_{i=1}^{3} a_{i} \boldsymbol{\omega}_{i}$,

\[
    \begin{aligned}
        & \int_{S} \mathbf{a} \cdot \boldsymbol{\omega} d \boldsymbol{\omega}=
        \int_{-\pi / 2}^{\pi / 2}
        \int_{0}^{2 \pi}\left(a_{1} \cos u \cos v+a_{2} \sin u \cos v+a_{3} \sin v\right) \cos v d u d v=0. \\
        & \text { получим } \\
        & \text { В третьем интеграле положим } f(\boldsymbol{\omega})=(\mathbf{a} \cdot \boldsymbol{\omega})(\mathbf{b} \cdot \boldsymbol{\omega})=\sum_{i, j=1}^{3} a_{i} b_{j} \boldsymbol{\omega}_{i} \boldsymbol{\omega}_{j} \text {, } \\
        & \int_{S}(\mathbf{a} \cdot \boldsymbol{\omega})(\mathbf{b} \cdot \boldsymbol{w}) d \boldsymbol{w}= \\
        & =\int_{-\pi / 2}^{\pi / 2} \int_{0}^{2 \pi} \mathbf{a}^{T}\left(\begin{array}{ccc}
                                                                              \cos ^{2} u \cos ^{2} v & \sin u \cos u \cos ^{2} v & \cos u \sin v \cos v \\\sin u \cos u \cos ^{2} v & \sin ^{2} u \cos ^{2} v & \sin u \sin v \cos v \\\cos u \sin v \cos v & \sin u \sin v \cos v & \sin ^{2} v
        \end{array}\right) \mathbf{b} \cos v d u d v= \\
        & =\int_{-\pi / 2}^{\pi / 2} \mathbf{a}^{T}\left(\begin{array}{ccc}
                                                             \pi \cos ^{2} v & 0 & 0 \\0 & \pi \cos ^{2} v & 0 \\0 & 0 & 2 \pi \sin ^{2} v
        \end{array}\right) \mathbf{b} \cos v d v=\frac{4 \pi}{3} \mathbf{a} \cdot \mathbf{b}
    \end{aligned}
\]
здесь $\mathbf{a}, \mathbf{b}$ - векторы-столбцы.


Равенства во второй строке получаются из доказанных равенств:
\[
    \begin{gathered}
        \int_{S} \omega d \boldsymbol{\omega}=\sum_{i=1}^{3} \mathbf{e}_{i}
        \int_{S}\left(\boldsymbol{\omega} \cdot \mathbf{e}_{i}\right) d \boldsymbol{\omega}=0, \\
        \int_{S}(\mathbf{a} \cdot \boldsymbol{\omega}) \omega d
        \boldsymbol{\omega}=\sum_{i=1}^{3} \mathbf{e}_{i}
        \int_{S}(\mathbf{a} \cdot \boldsymbol{\omega})\left(\boldsymbol{\omega} \cdot \mathbf{e}_{i}\right) d
        \boldsymbol{\omega}=\frac{4 \pi}{3} \sum_{i=1}^{3}\left(\mathbf{a} \cdot \mathbf{e}_{i}\right)
        \mathbf{e}_{i}=\frac{4 \pi}{3} \mathbf{a}.
    \end{gathered}
\]


Для доказательства первого равенства в третьей строке введем систему координат так, чтобы ось $O z$ была сонаправлена с вектором а. Воспользуемся формулой (1.16), в которой вместо $S$ следует взять верхнюю полусферу, $D=$ $\left\{(u, v): 0 \leqslant u \leqslant 2 \pi, 0 \leqslant v \leqslant \frac{\pi}{2}\right\}$. Заметим, что $f(\omega)=\mathbf{a} \cdot \boldsymbol{\omega}=|\mathbf{a}| \sin v$. Таким образом,
\[
    \int_{\omega \cdot \mathbf{a}>0} \mathbf{a} \cdot \omega d \omega=|\mathbf{a}|
    \int_{0}^{\pi / 2} \int_{0}^{2 \pi} \sin v \cos v d u d v=2 \pi \int_{0}^{\pi / 2} \sin v \cos v d v=\pi
\]
Для доказательства второго равенства в третьей строке заметим, что
\[
    \begin{gathered}
        \int_{S}(\mathbf{a} \cdot \boldsymbol{\omega})(\mathbf{b} \cdot \boldsymbol{\omega}) d
        \boldsymbol{\omega}=\int_{\boldsymbol{\omega} \cdot
        \mathbf{a}>0}(\mathbf{a} \cdot \boldsymbol{\omega})(\mathbf{b} \cdot \mathbf{\omega}) d
        \boldsymbol{\omega}+\int_{\boldsymbol{\omega} \cdot \mathbf{a}<0}(\mathbf{a} \cdot \boldsymbol{\omega})(\mathbf{b}
        \cdot \boldsymbol{\omega}) d \boldsymbol{\omega}, \\
        \int_{\omega \cdot \mathbf{a}>0}(\mathbf{a} \cdot \boldsymbol{\omega})(\mathbf{b} \cdot \boldsymbol{\omega}) d
        \boldsymbol{\omega}=\int_{\boldsymbol{\omega} \cdot \mathbf{a}<0}(\mathbf{a} \cdot \boldsymbol{\omega})(\mathbf{b}
        \cdot \boldsymbol{\omega}) d \boldsymbol{\omega},
    \end{gathered}
\]
следовательно,
\[
    \int_{\boldsymbol{\omega} \cdot \mathbf{a}>0}(\mathbf{a} \cdot \boldsymbol{\omega})(\mathbf{b}
    \cdot \boldsymbol{\omega}) d \boldsymbol{\omega}=\frac{1}{2} \int_{S}(\mathbf{a} \cdot \boldsymbol{\omega})(\mathbf{b}
    \cdot \boldsymbol{\omega}) d \boldsymbol{\omega}=\frac{2 \pi}{3} \mathbf{a} \cdot \mathbf{b}
\]


Подставляя $(1.14),(1.15)$ в (1.9), получаем
\[
    \begin{gathered}
        \frac{1}{c}\left(\frac{\partial \varphi(x, t)}{\partial t}+\boldsymbol{\omega} \cdot
        \frac{\partial \Phi(x, t)}{\partial t}\right)+\boldsymbol{\omega} \cdot \nabla
        \varphi(x, t)+\boldsymbol{\omega} \cdot \nabla_{x}(\Phi(x, t) \cdot \boldsymbol{\omega})+\kappa
        \varphi(x, t)+\mathrm{\kappa} \Phi(x, t) \cdot \boldsymbol{\omega}= \\
        =\frac{\kappa_{s}}{4 \pi} \int_{S}\left(1+A \omega \cdot
        \boldsymbol{\omega}^{\prime}\right)\left(\varphi(x, t)+\boldsymbol{\Phi}(x, t)
        \cdot \boldsymbol{\omega}^{\prime}\right) d \boldsymbol{\omega}^{\prime}+\mathrm{\kappa}_{a} \theta^{4}(x, t).
    \end{gathered}
\]
С учетом равенств
\[
    \int_{S} \Phi(x, t) \cdot \omega^{\prime} d \omega^{\prime}=0, \quad \int_{S} \omega \cdot \omega^{\prime} d
    \omega^{\prime}=0, \quad \int_{S}\left(\Phi(x, t) \cdot \omega^{\prime}\right)\left(\omega \cdot
    \omega^{\prime}\right) d \omega^{\prime}=\frac{4 \pi}{3} \Phi(x, t) \cdot \omega
\]
имеем
\[
    \begin{gathered}
        \frac{1}{c}\left(\frac{\partial \varphi(x, t)}{\partial t}+\omega \cdot
        \frac{\partial \Phi(x, t)}{\partial t}\right)+\boldsymbol{\omega} \cdot
        \nabla \varphi(x, t)+\boldsymbol{\omega} \cdot \nabla_{x}(\boldsymbol{\Phi}(x, t)
        \cdot \boldsymbol{\omega})+\boldsymbol{\kappa} \varphi(x, t)+\mathrm{\kappa}
        \boldsymbol{\Phi}(x, t) \cdot \boldsymbol{\omega}= \\
        =\mathrm{k}_{s}\left(\varphi(x, t)+\frac{A}{3} \boldsymbol{\Phi}(x, t) \cdot
        \boldsymbol{\omega}\right)+\mathrm{\kappa}_{a} \theta^{4}(x, t),
    \end{gathered}
\]
или
\begin{gather*}
    \frac{1}{c}\left(\frac{\partial \varphi(x, t)}{\partial t}+\omega \cdot
    \frac{\partial \Phi(x, t)}{\partial t}\right)+\omega \cdot \nabla
    \varphi(x, t)+\omega \cdot \nabla_{x}(\Phi(x, t) \cdot \omega)+\\
    +\mathrm{\kappa}_{a} \varphi(x, t)+\left(\mathrm{\kappa}_{a}+\mathrm{\kappa}_{s}^{\prime}\right)
    \Phi(x, t) \cdot \omega=\mathrm{\kappa}_{a} \theta^{4}(x, t),\\
\end{gather*}
где $\kappa_{s}^{\prime}=\kappa_{s}(1-A / 3)$ - приведенный коэффициент рассеяния.


Проинтегрируем уравнение (1.17) по $\omega \in S$.
Получим
\[
    \frac{1}{c} \frac{\partial \varphi(x, t)}{\partial t}+\frac{1}{3} \div \Phi(x, t)+\mathrm{\kappa}_{a}
    \varphi(x, t)=\mathrm{\kappa}_{a} \theta^{4}(x, t),
\]
так как
\[
    \begin{gathered}
        \int_{S} \omega \cdot \nabla_{x}(\Phi(x, t) \cdot \boldsymbol{\omega}) d \boldsymbol{\omega}=\sum_{i=1}^{3}
        \int_{S}\left(\boldsymbol{\omega} \cdot \mathbf{e}_{i}\right)\left(\omega \cdot
        \frac{\partial \Phi(x, t)}{\partial x_{i}}\right) d \boldsymbol{\omega}= \\
        =\frac{4 \pi}{3} \sum_{i=1}^{3} \frac{\partial \Phi(x, t)}{\partial x_{i}} \cdot
        \mathbf{e}_{i}=\frac{4 \pi}{3} \sum_{i=1}^{3} \frac{\partial \Phi_{i}(x, t)}{\partial x_{i}}=
        \frac{4 \pi}{3} \div \boldsymbol{\Phi}(x, t).
    \end{gathered}
\]

$\mathrm{Y}_{\text {множим уравнение }}(1.17)$ на $\omega:$

\[
    \begin{aligned}
        \frac{1}{c} \frac{\partial \varphi(x, t)}{\partial t} \omega+&
        \frac{1}{c}\left(\omega \cdot \frac{\partial \Phi(x, t)}{\partial t}\right)
        \omega+(\omega \cdot \nabla \varphi(x, t)) \omega+\left(\omega \cdot \nabla_{x}(\Phi(x, t)
        \cdot \omega)\right) \omega+\\
        &+\kappa_{a} \varphi(x, t) \omega+\left(\kappa_{a}+\kappa_{s}^{\prime}\right)(\Phi(x, t)
        \cdot \omega) \omega=\kappa_{a} \theta^{4}(x, t) \omega
    \end{aligned}
\]
и проинтегрируем полученное равенство по $\omega \in S$.


Для вычисления четвертого слагаемого представим интеграл по единичной сфере $S$
как сумму интегралов по верхней $S_{1}$ и нижней $S_{2}$ полусферам и воспользуемся тем, что
\[
    \int_{S_{2}}\left(\omega \cdot \nabla_{x}(\Phi(x, t) \cdot \omega)\right) \omega d
    \omega=-\int_{S_{1}}\left(\omega \cdot \nabla_{x}(\Phi(x, t) \cdot \omega)\right) \omega d \omega,
\]
следовательно, интеграл равен 0. Таким образом,
\[
    \frac{1}{c} \frac{\partial \Phi(x, t)}{\partial t}+\left(\kappa_{a}+\kappa_{s}^{\prime}\right)
    \Phi(x, t)+\nabla \varphi(x, t)=0.
\]
Итак, уравнения (1.18), (1.19) представляют собой $P_{1}$ приближение для уравнения переноса излучения.
Дальнейшие преобразования основываются на предположении, что выполняется закон Фика:
\[
    \Phi(x, t)=-3 \alpha \nabla \varphi(x, t),
\]
где $\alpha=\frac{1}{3\left(\kappa_{a}+\kappa_{s}^{\prime}\right)}=\frac{1}{3 \kappa-A \kappa_{s}}$.
Фактически мы пренебрегаем производной $\frac{\partial \Phi}{\partial t}$ в уравнении (1.19).
Подставив (1.20) в (1.18), получим
\[
    \frac{1}{c} \frac{\partial \varphi(x, t)}{\partial t}-\alpha \Delta
    \varphi(x, t)+\kappa_{a}\left(\varphi(x, t)-\theta^{4}(x, t)\right)=0 .
\]


Чтобы получить уравнение для температуры, подставим (1.14) в (1.10).
Получим
\[
    \frac{\partial \theta(x, t)}{\partial t}-a \Delta \theta(x, t)+\mathbf{v}(x, t) \cdot
    \nabla \theta(x, t)+b \kappa_{a}\left(\theta^{4}(x, t)-\varphi(x, t)\right)=\frac{b}{c}
    \frac{\partial \varphi(x, t)}{\partial t}.
\]
Учитывая (1.21), уравнение (1.22) можно записать в виде с кросс-диффузией:
\[
    \frac{\partial \theta(x, t)}{\partial t}-a \Delta \theta(x, t)+\mathbf{v}(x, t) \cdot
    \nabla \theta(x, t)=b \alpha \Delta \varphi(x, t).
\]

В дальнейшем вместо уравнения (1.22) будем использовать уравнение с нулевой правой частью (см., например, [73])
\[
    \frac{\partial \theta(x, t)}{\partial t}-a \Delta \theta(x, t)+\mathbf{v}(x, t) \cdot \nabla
    \theta(x, t)+b \kappa_{a}\left(\theta^{4}(x, t)-\varphi(x, t)\right)=0.
\]

Далее выведем граничные условия типа Маршака для $P_{1}$ приближения (см. [74]).
Для этого подставим (1.14) в граничное условие (1.11):
\[
    \begin{gathered}
        \varphi(x, t)+\boldsymbol{\Phi}(x, t) \cdot \boldsymbol{\omega}=\varepsilon(x)
        \theta_{b}^{4}(x, t)+\rho^{s}(x)\left(\varphi(x, t)+\boldsymbol{\Phi}(x, t)
        \cdot \boldsymbol{\omega}_{R}\right)+ \\
        +\frac{\rho^{d}(x)}{\pi} \int_{\omega^{\prime} \cdot
        \mathbf{n}>0}\left(\varphi(x, t)+\boldsymbol{\Phi}(x, t) \cdot
        \boldsymbol{\omega}^{\prime}\right) \omega^{\prime} \cdot \mathbf{n} d \omega^{\prime},
        \quad \boldsymbol{\omega} \cdot \mathbf{n}<0, \quad
        \boldsymbol{\omega}_{R}=\boldsymbol{\omega}-2(\boldsymbol{\omega} \cdot
        \mathbf{n}) \mathbf{n}.
    \end{gathered}
\]


Для вычисления интеграла применим лемму 1 :
\[
    \begin{aligned}
        \varphi(x, t)+\boldsymbol{\Phi}(x, t) \cdot \boldsymbol{\omega} &=\varepsilon(x) \theta_{b}^{4}(x, t)+\rho^{s}(x)[\varphi(x, t)+\boldsymbol{\Phi}(x, t) \cdot \boldsymbol{\omega}-2(\boldsymbol{\omega} \cdot \mathbf{n})(\boldsymbol{\Phi}(x, t) \cdot \mathbf{n})]+\\
        &+\rho^{d}(x)\left(\varphi(x, t)+\frac{2}{3} \boldsymbol{\Phi}(x, t) \cdot \mathbf{n}\right), \quad \boldsymbol{\omega} \cdot \mathbf{n}<0 .
    \end{aligned}
\]
Умножим данное равенство на $\boldsymbol{\omega} \cdot \mathbf{n}$ и проинтегрируем по множеству входящих направлений, для которых $\boldsymbol{\omega} \cdot \mathbf{n}<0$. Получим
\[
    \begin{gathered}
        -\pi \varphi(x, t)+\frac{2 \pi}{3} \boldsymbol{\Phi}(x, t) \cdot \mathbf{n}=
        -\pi \varepsilon(x) \theta_{b}^{4}(x, t)-\pi \rho^{s}(x) \varphi(x, t)+\frac{2 \pi \rho^{s}(x)}{3}
        \boldsymbol{\Phi}(x, t) \cdot \mathbf{n}- \\
        -\frac{4 \pi \rho^{s}(x)}{3} \boldsymbol{\Phi}(x, t) \cdot \mathbf{n}-\pi
        \rho^{d}(x)\left(\varphi(x, t)+\frac{2}{3} \boldsymbol{\Phi}(x, t) \cdot \mathbf{n}\right),
    \end{gathered}
\]
ИЛИ
\[
    \varepsilon(x) \varphi(x, t)=\varepsilon(x) \theta_{b}^{4}(x, t)+
    \frac{2(2-\varepsilon(x))}{3} \boldsymbol{\Phi}(x, t) \cdot \mathbf{n}.
\]
Воспользуемся равенством (1.20), будем иметь
\[
    \alpha \frac{\partial \varphi(x, t)}{\partial n}+\gamma(x)\left(\varphi(x, t)-\theta_{b}^{4}(x, t)\right)=0,
\]
где $\gamma=\frac{\varepsilon}{2(2-\varepsilon)}$.
Отметим, что на участках втекания и вытекания среды можно принять $\gamma=1 / 2[75]$.


Дополним полученные соотношения граничным условием для температуры (1.12):
\[
    a \frac{\partial \theta(x, t)}{\partial n}+\beta(x)\left(\theta(x, t)-\theta_{b}(x, t)\right)=0
\]
и начальными условиями
\[
    \theta(x, 0)=\theta_{0}(x), \quad \varphi(x, 0)=\varphi_{0}(x) .
\]
Соотношения (1.21), (1.23), (1.24)-(1.26) образуют диффузионную модель сложного теплообмена.

Укажем возможные пути обоснования закона Фика (1.20).
В [76, с. 136], [77, с. 222$]$, [78, с. 96] указано, что в уравнении (1.19) можно пренебречь
слагаемым $\frac{1}{c} \frac{\partial \Phi}{\partial t}$, если
\[
    \frac{1}{|\Phi|} \frac{\partial|\Phi|}{\partial t} \ll c\left(\kappa_{a}+\kappa_{s}^{\prime}\right).
\]
Это предположение означает, что относительное изменение плотности потока излучения во времени много
меньше частоты столкновений фотонов, так как величина
$\frac{1}{\mathrm{~K}_{a}+\mathrm{K}_{s}^{\prime}}$ есть средняя длина свободного пробега (transport mean free path) [78]

В диффузионном приближении предполагается, что среда имеет большое альбедо $\left(\kappa_{a} \ll \kappa_{s}\right)$ и
излучение почти изотропно [78, с. 88].
В [78, с. 97] указано, что предположения о почти изотропности излучения
(направленное расширение) и о малом относительном изменении плотности потока излучения (временное расширение потока
фотонов по отношению к среднему времени свободного пробега) выполняются при большом числе рассеяний фотонов,
так что оба приближения можно свести к предположению $\kappa_{s}^{\prime} \gg \kappa_{a}$.
Кроме того, необходимо, чтобы точка наблюдения находилась достаточно далеко от источников и от границ.

В [79, 80] делается предположение $3 \omega_{0} \alpha \ll c$, где $\omega_{0}$ --
частота синусоидально модулированного источника.
Авторы [79,81] сначала выводят из (1.18), (1.19) уравнение второго порядка по времени,
а затем отбрасывают некоторые слагаемые, которые можно считать малыми в силу указанного предположения.
Применив к уравнению (1.18) операцию дифференцирования по $t$, а к уравнению (1.19) операцию дивергенции, получим
\[
    \begin{gathered}
        \frac{1}{c} \frac{\partial^{2} \varphi}{\partial t^{2}}+\frac{1}{3}
        \frac{\partial}{\partial t} \div \Phi+\kappa_{a} \frac{\partial \varphi}{\partial t}=\kappa_{a}
        \frac{\partial\left(\theta^{4}\right)}{\partial t} \\
        \frac{1}{c} \frac{\partial}{\partial t} \div
        \Phi+\left(\kappa_{a}+\kappa_{s}^{\prime}\right) \div \Phi+\Delta \varphi=0
    \end{gathered}
\]
Умножим второе уравнение на $c/3$ и вычтем из первого уравнения.
Учитывая (1.18), будем иметь
\[
    \frac{1}{c} \frac{\partial \varphi}{\partial t}-\alpha \Delta
    \varphi+\kappa_{a}\left(\varphi-\theta^{4}\right)+\frac{3 \alpha \kappa_{a}}{c}
    \frac{\partial \varphi}{\partial t}+\frac{3 \alpha}{c^{2}}
    \frac{\partial^{2} \varphi}{\partial t^{2}}=\frac{3 \alpha \kappa_{a}}{c}
    \frac{\partial\left(\theta^{4}\right)}{\partial t}.
\]
Подчеркнутые слагаемые отбрасываем, принимая во внимание,
что $3 \alpha \kappa_{a}=\frac{\kappa_{a}}{\kappa_{a}+\kappa_{s}^{\prime}} \ll 1$.
Более точные оценки с переходом в частотную область указаны в $[79,80]$.
Однако их применение для нашей задачи требует дополнительных оценок правой части (1.27), содержащей $\theta$.


Также автор [67, с. 509] отмечает, что $P_{1}$ приближение может давать ошибочный результат
в оптически тонкой среде со слишком анизотропным распределением интенсивности, в частности,
в многомерных областях с длинными узкими конфигурациями и/или когда излучение поверхности
преобладает над излучением среды.
Среда называется оптически толстой, если средняя длина
свободного пробега фотона мала по сравнению с ее характерным размером [68, с. 343].
Авторы $[82$, с. 228] также указывают, что для применения диффузионного $P_{1}$ приближения альбедо
$\mathrm{k}_{s} / \mathrm{k}$ должно быть близко к единице и среда должна быть оптически толстой.
В [79, с. 8] указано, что фазовая функция не должна быть слишком анизотропной
$(\|A / 3\| \text{ не слишком близко к } 1)$.

Таким образом, благоприятными условиями для применения $P_{1}$ приближения являются:
1) $\kappa_{a} \ll \kappa_{s}^{\prime}$;

2) оптически толстая среда;

3) удаление от границ области.
