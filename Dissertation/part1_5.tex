\section{Математический аппарат моделирования сложного теплообмена}
\label{sec:ch1/sec5}
Предполагается, что $\Omega \subset \mathbb{R}^3$ — ограниченная
область с липшицевой границей $\Gamma$~\cite[с.~232]{Zeidler1986},
$Q = \Omega \times (0, T)$, $\Sigma = \Gamma \times (0, T)$.
Вектор внешней нормали к границе области обозначается через $n$.

Введем следующие обозначения: $|\Omega|$ — объем области $\Omega$,
$|\Gamma|$ — площадь границы $\Gamma$, $\mu(E)$ — мера множества $E$;
$L^p(\Omega)$, $L^p(\Gamma)$, $1 \le p \le \infty$ — пространства Лебега;
$H^s(\Omega)$ — пространство Соболева $W^{2s}(\Omega)$,
$H^1_0(\Omega) = \{v \in H^1(\Omega) : v|_\Gamma = 0\}$,
$C^\infty_0(\Omega)$ — пространство функций класса $C^\infty$
с компактным носителем в $\Omega$; аналогично определяются пространства
вектор-функций $L^p_s(\Omega)$, $H^s(\Omega)$, $H^1_0(\Omega)$,
$C^\infty_0(\Omega)$; $L^p(0, T; X)$ — пространство Лебега функций
со значениями в банаховом пространстве $X$, $C([0, T]; X)$ — пространство
функций, непрерывных на $[0, T]$, со значениями в $X$, $X'$ — пространство,
сопряженное с пространством $X$.

Если $y \in L^p(0, T; X)$, то обозначаем $y' = \frac{dy}{dt}$.

Обозначим $\mathcal{H} = L^2(\Omega)$, $\mathcal{V} = H^1(\Omega)$.
Пространство $\mathcal{H}$ отождествляем с пространством $\mathcal{H}'$,
так что $\mathcal{V} \subset \mathcal{H} = \mathcal{H}' \subset \mathcal{V}'$.
Через $\| \cdot \|$ будем обозначать норму в $\mathcal{H}$,
а через $(f, v)$ — значение функционала $f \in \mathcal{V}'$ на элементе
$v \in \mathcal{V}$, совпадающее со скалярным произведением в $\mathcal{H}$,
если $f \in \mathcal{H}$.
Определим пространство $\mathcal{W} = \{ y \in L^2(0, T; \mathcal{V}):
y' \in L^2(0, T, \mathcal{V}') \}$.

Скалярное произведение в $\mathcal{H}$ и нормы в $\mathcal{H}$,
$\mathcal{V}$, $\mathcal{V}'$ определяются формулами
\begin{align*}
(f, g)
    &= \int_\Omega f(x)g(x) dx, \| f \|^2 = (f, f), \\
    \| f \|^2_{\mathcal{V}} = \| f \|^2 + \| \nabla f \|^2, \\
    & \| f \|_{\mathcal{V}'} = \sup \{ (f, v):
    v \in \mathcal{V}, \|v\|_{\mathcal{V}} = 1 \}.
\end{align*}


\begin{lemma}[неравенство Гёльдера]
    \label{lemma:hoelder}~\cite[с.~35]{Zeidler1990b}
    Пусть $p, q \in [1, \infty]$, $\frac{1}{p} + \frac{1}{q} = 1$.
    Тогда для любых $u \in L^p(\Omega)$, $v \in L^q(\Omega)$
    выполнено неравенство:
    \[
        \left| \int_\Omega uv \, dx \right|
        \le \|u\|_{L^p(\Omega)} \|v\|_{L^q(\Omega)}.
    \]
\end{lemma}

\begin{lemma}[неравенство Юнга]
    \label{lemma:young}~\cite[с.~38]{Zeidler1990a}
    Пусть $p, q > 1$, $\frac{1}{p} + \frac{1}{q} = 1$.
    Тогда для любых $a, b \ge 0$, $\epsilon > 0$
    справедливо неравенство:
    \[
        ab \le \frac{\epsilon^p a^p}{p} + \frac{b^q}{q\epsilon}.
    \]
\end{lemma}

\begin{lemma}
    \label{lemma:reflexive}~\cite[с.~254]{Zeidler1990a}
    Всякое гильбертово пространство рефлексивно.
\end{lemma}

\begin{lemma}
    \label{lemma:reflexive_weak}~\cite[с.~255]{Zeidler1990a}
    Пусть $X$ — рефлексивное банахово пространство.
    Если последовательность $x_n \in X$ ограничена,
    то в ней существует слабо сходящаяся подпоследовательность.
\end{lemma}

\begin{lemma}
    \label{lemma:separable_star_weak}~\cite[260]{Zeidler1990a}
    Пусть $X$ — сепарабельное банахово пространство.
    Если последовательность $f_n \in X'$ ограничена,
    то в ней существует *-слабо сходящаяся подпоследовательность.
\end{lemma}

\begin{lemma}
    \label{lemma:weak_limit}~\cite[258]{Zeidler1990a}
    Пусть $X$ — банахово пространство.
    Если $x_n \in X$, $x_n \rightharpoonup x$ слабо в $X$,
    то последовательность $x_n$ ограничена и
    $\|x\| \le \lim\limits_{n \to \infty} \|x_n\|$.
\end{lemma}

\begin{lemma}
    \label{lemma:convex_closed}~\cite[47]{Troeltzsch2010}
    Пусть $U$ — выпуклое, замкнутое множество в банаховом пространстве
    $X$, $x_n \in U$, $x_n \rightharpoonup x$ слабо в $X$.
    Тогда $x \in U$.
\end{lemma}

\begin{definition}
    \label{def:compact_set}~\cite[с.~48]{Lusternik1982}
    Множество $M$ банахова пространства $X$ называется компактным,
    если из всякого бесконечного подмножества множества $M$ можно
    выделить подпоследовательность,
    сходящуюся к некоторой точке этого множества.
    Множество называется относительно компактным,
    если его замыкание компактно.
\end{definition}

\begin{definition}
    \label{def:compact_operator}~\cite[с.~190]{Lusternik1982}
    Пусть $X, Y$ — банаховы пространства.
    Оператор $A : M \subset X \to Y$ называется компактным,
    если он переводит всякое ограниченное подмножество множества
    $M$ в относительно компактное множество пространства $Y$.
    Если, кроме того, оператор $A$ непрерывен,
    то он называется вполне непрерывным.
\end{definition}

\begin{theorem}[Принцип Шаудера]
    \label{thm:schauder}~\cite[с.~193]{Lusternik1982}
    Пусть $X$ — банахово пространство, $M \subset X$ — выпуклое
    замкнутое множество, $A : M \to M$ — вполне непрерывный оператор.
    Тогда оператор $A$ имеет неподвижную точку $x \in M$, т.е. $Ax = x$.
\end{theorem}

\begin{lemma}
    \label{lem:embedding}~\cite[с.~365]{Zeidler1990a}
    Пусть $X, Y, Z$ — банаховы пространства и $X \subset Y \subset Z$,
    причем вложение $X \subset Y$ компактно,
    а вложение $Y \subset Z$ непрерывно.
    Тогда для любого $\varepsilon > 0$ существует постоянная
    $C_\varepsilon > 0$ такая, что
    $\|u\|_Y \leq \varepsilon \|u\|_X
    + C_\varepsilon \|u\|_Z \ \forall u \in X$.
\end{lemma}

\begin{theorem}
    \label{thm:compactness}~\cite[теорема 3]{Simon1986}
    Пусть $F$ — ограниченное множество в $L^2(0, T; V)$, и
    \[
        \int_0^{T-h} \|f(t+h) - f(t)\|^2 dt \to 0
    \]
    при $h \to 0$ равномерно относительно $f \in F$.
    Тогда $F$ относительно компактно в $L^2(0, T; H)$.
\end{theorem}

\begin{theorem}[Гильберт, Шмидт]
    \label{thm:hilbert-schmidt}~\cite[с. 263]{Kolmogorov2004}
    Пусть $A$ — линейный, самосопряженный, компактный оператор
    в гильбертовом пространстве $X$.
    Тогда множество собственных элементов оператора $A$ образует
    ортогональный базис в $X$.
\end{theorem}

\begin{theorem}[Лакс, Мильграм]
    \label{thm:lax-milgram}~\cite[с. 40]{Oben1977}
    Пусть $X$ — гильбертово пространство,
    $B : X \times X \to \mathbb{R}$ — непрерывная билинейная форма,
    $B(x, x) \geq C\|x\|^2_X$, $C > 0$.
    Тогда для любого $f \in X'$ задача
    $B(x, z) = (f, z) \quad \forall z \in X$
    имеет единственное решение $x \in X$.
\end{theorem}

\begin{definition}
    \label{def:weakly_semicontinuous}~\cite[с. 47]{Troeltzsch2010}
    Пусть $X$ — банахово пространство.
    Функционал $J : X \to \mathbb{R}$ называется слабо полунепрерывным
    снизу, если для любой последовательности $x_n \in X$ такой,
    что $x_n \rightharpoonup x$ слабо, выполняется неравенство
    $J(x) \leq \lim_{n \to \infty} J(x_n)$.
\end{definition}

\begin{lemma}
    \label{lem:weakly_semicontinuous_functional}~\cite[с. 47]{Troeltzsch2010}
    Пусть $X$ — банахово пространство.
    Если функционал $J : X \to \mathbb{R}$ непрерывный и выпуклый,
    то он слабо полунепрерывен снизу.
\end{lemma}

\begin{corollary}
    \label{cor:weakly_semicontinuous_norm}
    Пусть $X$ — банахово пространство, $a \in X$.
    Тогда функционал $J : X \to \mathbb{R}$, $J(x) = \|x - a\|^2_X$
    слабо полунепрерывен снизу.
\end{corollary}

\begin{lemma}
    \label{lem:embedding_Lp_Ls}~\cite[с. 37]{Zeidler1990a}
    Вложение $L^p(\Omega) \subset L^s(\Omega)$ непрерывно при
    $1 \leq s \leq p \leq \infty$.
\end{lemma}

\begin{lemma}
    \label{lem:Lp_separability_reflexivity}~\cite[с. 1020]{Zeidler1990b}
    Пространство $L^p(\Omega)$ сепарабельно при $1 \leq p < \infty$
    и рефлексивно при $1 < p < \infty$.
\end{lemma}

\begin{lemma}
    \label{lem:convergence_max_function}
    Пусть $f_k \to f$ в $L^p(\Omega)$, $1 \leq p < \infty$.
    Тогда $g_k \to g$ в $L^p(\Omega)$,
    где $g_k = \max\{f_k, 0\}$, $g = \max\{f, 0\}$.
\end{lemma}

\begin{proof}
    Так как $|g_k - g| \leq |f_k - f|$, то
    \begin{equation*}
        \|g_k - g\|_{L^p(\Omega)} =
        \int_{\Omega} |g_k - g|^p \mathrm{d}x
        \leq \int_{\Omega} |f_k - f|^p \mathrm{d}x =
        \|f_k - f\|_{L^p(\Omega)} \to 0.
    \end{equation*}
\end{proof}

\begin{lemma}
    \label{lem:convergence_positive_functions}
    Пусть $f_k \to f$ в $L^p(\Omega)$, $1 \leq p < \infty$,
    $f_k \geq 0$ почти всюду в $\Omega$.
    Тогда $f \geq 0$ почти всюду в $\Omega$.
\end{lemma}

\begin{proof}
    По лемме~\ref{lem:convergence_max_function} получаем,
    что $f_k = \max\{f_k, 0\} \to \max\{f, 0\}$ в $L^p(\Omega)$.
    Следовательно, $f = \max\{f, 0\}$,
    поэтому $f \geq 0$ почти всюду в $\Omega$.
\end{proof}

\begin{lemma}
    \label{lem:convergence_power_functions}
    Пусть $f_k, f \geq 0$, $f_k \to f$ в $L^p(\Omega)$, $1 < p < \infty$.
    Тогда $f_k^s \to f^s$ в $L^1(\Omega)$, если $1 < s < p$.
\end{lemma}

\begin{proof}
    По формуле Лагранжа
    \[
        |f_k^s - f^s| \leq s \xi^{s-1} |f_k - f|
        \leq s(f_k^{s-1} + f^{s-1}) |f_k - f|,
    \]
    где $\xi(x)$ находится между $f_k(x)$ и $f(x)$.
    Применяя неравенство Гёльдера с показателями
    $p, q = \frac{p}{p-1}$, получаем, что
    \begin{equation}
        \label{eq:1_5:23}
        \begin{split}
            \|f_k^s - f^s\|_{L^1(\Omega)} =
            &\int_{\Omega} |f_k^s - f^s| \, dx
            \leq s \int_{\Omega} (f_k^{s-1} + f^{s-1}) |f_k - f| \, dx \\
            &\leq \|f_k - f\|_{L^p(\Omega)} \left( \|f_k^{s-1}\|_{L^q(\Omega)}
            + \|f^{s-1}\|_{L^q(\Omega)} \right).
        \end{split}
    \end{equation}

    Применяя неравенство Гёльдера с показателями $\frac{p-1}{p-s}$
    и $\frac{p-1}{s-1}$, получаем, что
    \[
        \|f_k^{s-1}\|_{L^q(\Omega)} = \left(\int_{\Omega} f_k^{(s-1)p}\,
        dx\right)^{\frac{1}{p}} \leq |\Omega|^{\frac{p-s}{p}}
        \left(\int_{\Omega} f_k^p\, dx\right)^{\frac{s-1}{p}} =
        |\Omega|^{\frac{s}{p}-1} \|f_k\|_{L^p(\Omega)}^{s-1}.
    \]

    Следовательно, второй сомножитель в\eqref{eq:1_5:23} ограничен,
    поэтому $\|f_k^s - f^s\|_{L^1(\Omega)} \to 0$.
\end{proof}

\begin{theorem}[Лебег]
    \label{th:lebeg}~\cite[321]{Kolmogorov2004}
    Пусть $\Omega \subset \mathbb{R}^N$, $f_n \to f$ п.в. в $G$,
    $\forall n : |f_n(x)| \leq \varphi(x)$ п.в. в $G$,
    где $\varphi \in L^1(\Omega)$.
    Тогда $f \in L^1(\Omega)$, $\int_{\Omega} f_n(x) \, dx \to \int_{\Omega} f(x) \, dx$.
\end{theorem}

\begin{theorem}[Леви]
    \label{th:levi}~\cite[322]{Kolmogorov2004}
    Пусть $\Omega \subset \mathbb{R}^N$, последовательность $f_n \in L^1(\Omega)$
    не убывает и $\forall n : \int_{\Omega} f_n(x) \, dx \leq K$.
    Тогда $f_n \to f$ п.в. в $G$, где $f \in L^1(\Omega)$,
    $\int_{\Omega} f_n(x) \, dx \to \int_{\Omega} f(x) \, dx$.
\end{theorem}

\begin{lemma}
    \label{lm:1_5:15}\cite[47]{Ziemer1989}
    Пусть $u \in H^{1}(\Omega)$, $\nabla u = 0$ n.в. в $\Omega$.
    Тогда $u = \text{const}\, \theta \Omega$.
\end{lemma}

\begin{lemma}
    \label{lm:1_5:16}\cite[47]{Kinderlehrer1983}
    Пусть $u \in H^{1}(\Omega)$, $u^{+} = \max \{u, 0\}$,
    $u^{-} = \min \{u, 0\}$.
    Тогда $u^{+}, u^{-} \in H^{1}(\Omega)$ и
    \[
        \nabla u^{+} =
        \begin{cases}
            \nabla u, & \text{если } u > 0, \\
            0, & \text{если } u \leq 0,
        \end{cases}
        \qquad
        \nabla u^{-} =
        \begin{cases}
            \nabla u, & \text{если } u < 0, \\
            0, & \text{если } u \geq 0.
        \end{cases}
    \]
\end{lemma}

\begin{lemma}
    \label{lm:1_5:lipshitz}\cite[50]{Kinderlehrer1983}
    Пусть $f(t), t \in \mathbb{R}$ - липшицева функция,
    производная которой существует всюду, за исключением,
    быть может, множества $\left\{a_{1}, \ldots, a_{M}\right\}$,
    $u \in H^{1}(\Omega)$, $G \subset \mathbb{R}^{N}$.
    Тогда $f(u) \in H^{1}(\Omega)$,
    $\frac{\partial f(u)}{\partial x_{i}} = f^{\prime}(u)
    \frac{\partial u}{\partial x_{i}}$,
    где обе части этого равенства считаются равными нулю,
    когда $x \in \bigcup_{j=1}^{M}\left\{y: u(y)=a_{j}\right\}$.
\end{lemma}

\begin{lemma}
    \label{lm:1_5:embedding}\cite[1026]{Zeidler1990b}
    Вложение $H^{1}(\Omega) \subset L^{p}(\Omega)$ непрерывно
    при $1 \leq p \leq 6$ и компактно при $1 \leq p<6$.
\end{lemma}

%TODO: refactor further (limit reached)

\begin{lemma}
    \label{lm:1_5:19}\cite[239]{Zeidler1990a}
    Оператор следа $\gamma: H^{1}(\Omega) \rightarrow L^{2}(\Gamma)$ непрерывен.
\end{lemma}

\begin{lemma}
    \label{lm:1_5:20}\cite[4]{girault1979finite}
    Образ оператора следа $\gamma: H^{1}(\Omega) \rightarrow L^{2}(\Gamma)$
    -- плотное подпространство пространства $L^{2}(\Gamma)$.
\end{lemma}

\begin{lemma}
    \label{lm:1_5:21}\cite{berninger2009non}
    Пусть $f(t), t \in \mathbb{R}$ - липшицева функция.
    Для любой функции $u \in H^{1}(\Omega)$ справедливо равенство
    $f(\gamma(u))=\gamma(f(u))$, где
    $\gamma: H^{1}(\Omega) \rightarrow L^{2}(\Gamma)$ - оператор следа.
\end{lemma}

\begin{lemma}
    \label{lm:1_5:22}
    Пусть $u \in H^{1}(\Omega), u \geq 0$ n.в. в $\Omega$.
    Тогда $\left.u\right|_{\Gamma} \geq 0$ п.в. на $\Gamma$.
\end{lemma}

\begin{proof}
    Применим лемму~\ref{lm:1_5:21} для функции $f(t)=\max \{t, 0\}$:

    \[
        \max \left\{\left.u\right|_{\Gamma},
        0\right\}=\left.\max \{u, 0\}\right|_{\Gamma}=\left.u\right|_{\Gamma}
    \]


    Следовательно, $\left.u\right|_{\Gamma} \geq 0$ п.в. на $\Gamma$.
\end{proof}

\begin{lemma}
    \label{lm:1_5:23}\cite[41]{grisvard1985elliptic}
    Для любого $\varepsilon>0$ существует постоянная $C_{\varepsilon}>0$ такая,
    что для любой функции $u \in H^{1}(\Omega)$ выполняется неравенство

    \[
        \|u\|_{L^{2}(\Gamma)}^{2} \leq \varepsilon\|\nabla u\|^{2}
        +C_{\varepsilon}\|u\|^{2}.
    \]

\end{lemma}

\begin{lemma}[Гронуолл]
    \label{lm:1_5:24}\cite[191]{Gaevskii1978}
    Пусть $f:[0, T] \rightarrow \mathbb{R}$ -- непрерывная функция $b \geq 0$.
    Если
    \[
        f(t) \leq a+b \int_{0}^{t} f(\tau) d \tau \quad \forall t \in[0, T]
    \]
    $m o f(t) \leq a e^{b t} \forall t \in[0, T]$
\end{lemma}

\begin{lemma}
    \label{lm:1_5:25}
    Оператор $A: V \rightarrow V^{\prime}$, определяемый формулой
    $(A u, v)=$ $a(\nabla u, \nabla v)+\int_{\Gamma} b u v d \Gamma$,
    где $a>0, b \in L^{\infty}(\Gamma), b \geq b_{0}>0, b_{0}=$
    const, непрерывен.
\end{lemma}

\begin{proof}
    \[
        \begin{array}{r}
            \|A u\|_{V^{\prime}}=\sup _{\|v\|_{V}=1}(A u, v)
            \leq \sup _{\|v\|_{V}=1}\left(a\|\nabla u\|\|\nabla v\|
            +\|b\|_{L^{\infty}(\Gamma)}\|u\|_{L^{2}(\Gamma)}\|v\|_{L^{2}(\Gamma)}\right) \leq \\
            \leq a\|u\|_{V}+\|b\|_{L^{\infty}(\Gamma)} C_{1}^{2}\|u\|_{V}=C\|u\|_{V},
        \end{array}
    \]
    где $C_{1}$ - норма оператора следа
    $\gamma: H^{1}(\Omega) \rightarrow L^{2}(\Gamma)$.
\end{proof}

\begin{lemma}
    \label{lm:1_5:26}
    Пусть $\mathbf{v} \in L^{\infty} \left(0, T; \mathbf{H}^{1}(\Omega)\right)$,
    оператор $B(t): V \rightarrow V^{\prime}$ oпpeделяется формулой
    $(B(t) u, v)=(\mathbf{v} \cdot \nabla u, v)$.
    Тогда $\exists C>0:|(B(t) u, w)|
    \leq$ $C\|u\|_{V}\|w\|_{V} \forall u, w \in V, t \in(0, T)$.
\end{lemma}

\begin{proof}
    Применяя неравенство Гёльдера с показателями 4, 2, 4, получаем, что
    \[
        \begin{aligned}
            & |(B(t) u, w)|=|(\mathbf{v} \cdot \nabla u, w)|
            \leq\|\mathbf{v}(t)\|_{\mathbf{L}^{4}(\Omega)}
            \|\nabla u\|\|w\|_{L^{4}(\Omega)} \leq \\
            & \leq C_{1}^{2}\|\mathbf{v}\|_{L^{\infty}\left(0, T; \mathbf{H}^{1}
            (\Omega)\right)}\|u\|_{V}\|w\|_{V}
        \end{aligned}
    \]
    где $C_{1}-$ норма оператора вложения $H^{1}(\Omega) \subset L^{4}(\Omega)$
\end{proof}

\begin{lemma}
    \label{lm:1_5:27}\cite[238]{Zeidler1990a}
    Функционал $f(v)=(A v, v)^{1 / 2}$,
    где $A$ -- оператор из леммы 1.25,
    определяет норму в пространстве $V$,
    эквивалентную стандартной норме в $V$.
\end{lemma}

\begin{lemma}
    \label{lm:1_5:28}~\cite[411]{Zeidler1990a}
    Если $X$ - рефлексивное и сепарабельное банахово пространство,
    то $L^{p}(0, T ; X), 1<p<\infty$ -- рефлексивное
    и сепарабельное банахово пространство.
\end{lemma}

\begin{lemma}
    \label{lm:1_5:29}\cite[449]{Zeidler1990a}
    Если $X$ - рефлексивное и сепарабельное банахово пространство,
    то $L^{1}(0, T ; X)$ - сепарабельное банахово пространство.
\end{lemma}

\begin{lemma}
    \label{lm:1_5:30}
    Пусть $f: \mathbb{R} \rightarrow \mathbb{R},\left|f\left(x_{1}\right)
    -f\left(x_{2}\right)\right| \leq C\left|x_{1}-x_{2}\right|$.
    Тогда отображение $y \mapsto f(y)$ непрерывно
    отображает пространство $C([0, T] ; H)$ в себя.
\end{lemma}

\begin{proof}
    Пусть $y \in C([0, T] ; H)$.
    Тогда $z=f(y) \in C([0, T] ; H)$,
    так как $\left\|z\left(t_{1}\right)-z\left(t_{2}\right)\right\|
    \leq C\left\|y\left(t_{1}\right)-y\left(t_{2}\right)\right\|$.
    Поскольку $\left\|z_{1}(t)-z_{2}(t)\right\|
    \leq C \| y\left(t_{1}\right)-$ $y\left(t_{2}\right) \|, \mathrm{TO}$

    \[
        \begin{array}{r}
            \left\|z_{1}-z_{2}\right\|_{C([0, T] ; H)}=
            \max _{t \in[0, T]}\left\|z_{1}(t)
            -z_{2}(t)\right\| \leq C \max _{t \in[0, T]}\left\|y_{1}(t)
            -y_{2}(t)\right\| = \\
            =C\left\|y_{1}-y_{2}\right\|_{C([0, T] ; H)}.
        \end{array}
    \]
    Лемма доказана.
\end{proof}

\begin{lemma}
    \label{lm:1_5:31}\cite[423]{Zeidler1990a}
    Для любой функции и $\in W$ справедлива формуЛа
    \[
        \|u(t)\|^{2}-\|u(0)\|^{2}=2 \int_{0}^{t}\left(u^{\prime}(\tau),
        u(\tau)\right) d \tau, \quad 0 \leq t \leq T
    \]
\end{lemma}

\begin{lemma}
    \label{lm:1_5:32}
    Для любой функции и $\in W$ справедливо равенство
    \[
        \frac{d\|u(t)\|^{2}}{d t}=2\left(u^{\prime}(t),
        u(t)\right) \quad \text { n.8. нa }(0, T)
    \]
\end{lemma}

\begin{proof}
    Продифференцируем (1.24) по $t$ согласно~\cite[356]{Kolmogorov2004}
\end{proof}

\begin{lemma}
    \label{lm:1_5:33}
    Пусть $y \in W, k \geq 0$.
    Тогда $z_{1}=\max \{y-k, 0\} \in L^{2}(0, T ; V)
    \cap$ $C([0, T] ; H), z_{2}=
    \min \{y+k, 0\} \in L^{2}(0, T ; V) \cap C([0, T] ; H)$
    и справедливы равенства
    \[
        2 \int_{0}^{t}\left(y^{\prime}(\tau), z_{k}(\tau)\right) d \tau=
        \left\|z_{k}(t)\right\|^{2}
        -\left\|z_{k}(0)\right\|^{2}, \quad 0 \leq t \leq T, \quad k=1,2
    \]
\end{lemma}

\begin{proof}
    Пространство $H^{1}(Q)$ плотно в $W$~\cite[423]{Zeidler1990a} поэтому
    существует последовательность $y_{j} \in H^{1}(Q), y_{j} \rightarrow y$ в $W$.
    В силу непрерывности вложения $W \subset C([0, T] ; H)$
    имеем $y_{j} \rightarrow y$ в $C([0, T] ; H)$.

    Положим $z_{1 j}=\max \left\{y_{j}-k, 0\right\}$.
    Отображение $y \mapsto \max \{y-k, 0\}$ непрерывно отображает
    пространство $C([0, T] ; H)$ в себя (лемма 1.30$)$, поэтому
    $z_{1 j} \rightarrow z_{1}$ в $C([0, T] ; H)$.

    Поскольку $\left\|z_{1 j}(t)\right\| \leq\left\|y_{j}(t)\right\|$
    и $\left\|\nabla z_{1 j}(t)\right\| \leq\left\|\nabla y_{j}(t)\right\|$,
    то $\left\|z_{1 j}(t)\right\|_{V} \leq$ $\left\|y_{j}(t)\right\|_{V}$,
    следовательно, последовательность $z_{1 j}$ ограничена в $L^{2}(0, T ; V)$,
    поэтому в ней существует подпоследовательность
    $z_{1 j} \rightarrow z_{1}$ слабо в $L^{2}(0, T ; V)$.

    По лемме 1.16 получаем, что

    \[
        z_{1 j}^{\prime}=
        \left\{\begin{array}{l}
                   y_{j}^{\prime}, \quad \text { если }
                   y_{j}>k\left(z_{1 j}>0\right), \\
                   0, \quad \text { если }
                   y_{j} \leq k\left(z_{1 j}=0\right).
        \end{array}\right.
    \]
    Рассмотрим интеграл

    \[
        \begin{aligned}
            & 2 \int_{0}^{t}\left(y_{j}^{\prime}, z_{1 j}\right) d \tau=
            2 \int_{0}^{t}\left(z_{1 j}^{\prime}, z_{1 j}\right) d \tau=
            \int_{\Omega} \int_{0}^{t}
            \frac{d\left(z_{1 j}^{2}\right)}{d t} d \tau d x= \\
            &=\left\|z_{1 j}(t)\right\|^{2}-\left\|z_{1 j}(0)\right\|^{2}.
        \end{aligned}
    \]
    Перейдем к пределу в (1.25) при $j \rightarrow \infty$.
    Примем во внимание, что $y_{j}^{\prime} \rightarrow y^{\prime}$
    сильно в $L^{2}\left(0, T ; V^{\prime}\right)$ и
    $z_{1 j} \rightarrow z_{1}$ слабо в $L^{2}(0, T ; V)$,
    а также $z_{1 j} \rightarrow z_{1}$ в $C([0, T] ; H)$,
    поэтому $\left\|z_{1 j}(t)\right\| \rightarrow\left\|z_{1}(t)\right\|, t \in[0, T]$.
    Таким образом, получаем утверждение леммы для $z_{1}$.
    Утверждение для $z_{2}$ доказывается аналогично.
\end{proof}

\begin{theorem}
    \label{th:1_5:7}\cite[426]{Zeidler1990a}
    Пусть $A(t): V \rightarrow V^{\prime}$ -- линейный оператор,
    $\forall u, v \in V$ функция $t \mapsto(A(t) u, v)$
    измерима на $(0, T), \exists C, c>0, d \geq 0:
    \forall u, v \in$ $V, t \in(0, T):|(A(t) u, v)|
    \leq C\|u\|_{V}\|v\|_{V},(A(t) u, u)
    \geq c\|u\|_{V}^{2}-d\|u\|_{H}^{2},
    f \in$ $L^{2}\left(0, T ; V^{\prime}\right), u_{0} \in H$.
    Тогда задача
    \[
        \begin{gathered}
            u^{\prime}(t)+A(t) u(t)=f(t) \quad \text { n.в. на }(0, T),\\
            u(0)=u_{0}, \quad u \in W
        \end{gathered}
    \]
    имеет единственное решение.
\end{theorem}
