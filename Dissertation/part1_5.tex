\section{Квазистационарная модель сложного теплообмена}
\label{sec:ch1/sec4}
Квазистационарная модель - это тип математической модели,
который описывает систему, претерпевающую медленные изменения
со временем или имеющую относительно длительный период стабильности
по сравнению с интересующим масштабом времени.
Такие модели часто используются, когда изучаемая система находится
в равновесии или близка к нему, но при этом может испытывать небольшие,
медленные колебания со временем.
Термин `квази` означает,
что система не совсем стационарна (то есть, фиксирована или неизменна),
а скорее, ее состояние меняется настолько медленно, что его можно считать
почти стационарным для определенных анализов или целей.

В контексте теплообмена, излучения или других физических процессов
квазистационарные модели могут использоваться для описания сценариев,
когда параметры и свойства системы меняются очень медленно по сравнению
с масштабом времени конкретного изучаемого явления.
Такие модели могут упростить анализ и снизить вычислительную сложность,
позволяя исследователям сосредоточиться на основных аспектах проблемы.


Квазистационарный радиационный и диффузионный теплообмен в ограниченной
области $\Omega \subset \mathbb{R}^3$ с границей
$\Gamma = \partial\Omega$ моделируется в рамках приближения $P_1$
для уравнения радиационного теплообмена следующей
начально-краевой задачей:
\begin{align}
    \frac{\partial\theta}{\partial t} - a\Delta\theta
    + b\kappa_a (|\theta|\theta^3 - \phi) &= 0, \\
    - \alpha\Delta\phi + \kappa_a (\phi - |\theta|\theta^3 ) &= 0,
    \quad x \in \Omega, \quad 0 < t < T ; \label{eq:1_5:1} \\
    a \frac{\partial \theta}{\partial n}
    +\left.\beta\left(\theta-\theta_{b}\right)\right|_{\Gamma}&=0,
    \quad \alpha \frac{\partial \varphi}{\partial \mathbf{n}} + \gamma
    (\varphi-\theta_0^4)|_{\Gamma} = 0, \text{ на } \Gamma; \label{eq:1_5:2} \\
    \theta|_{t=0} &= \theta_0. \label{eq:1_5:3}
\end{align}

Данная модель описывает систему связанных уравнений в частных производных,
моделирующих квазистационарный радиационный и диффузионный теплообмен в
ограниченной области $\Omega \subset \mathbb{R}^3$ в трехмерном пространстве.
Используется приближение $P_1$ для уравнения радиационного теплообмена,
что является упрощенным подходом к решению задач радиационного теплообмена.

Задача состоит из начально-краевой задачи с тремя компонентами:


1.\ Первое уравнение представляет баланс энергии из-за радиационного
теплообмена $\left(\frac{\partial \theta}{\partial t}\right)$ и проводящего
теплообмена $(a \Delta \theta)$ с термом источника тепла
$\left(b \kappa_a \left(|\theta| \theta^3 - \phi \right)\right)$
в ограниченной области $\Omega$ на промежутке времени $0 < t < T$.
Радиационный теплообмен моделируется с использованием приближения $P_1$,
которое упрощает уравнение радиационного теплообмена.

2.\ Второе уравнение представляет баланс энергии из-за диффузионного
теплообмена $(\alpha \Delta \phi)$ с термом источника тепла
$\left(\kappa_a \left(\phi - |\theta| \theta^3 \right)\right)$
в той же области $\Omega$ на промежутке времени $0 < t < T$.
Это уравнение связано с первым уравнением через термы источника тепла.

3.\ Граничные условия задаются уравнениями,
$a(\partial_n \theta +\left.\beta\left(\theta-\theta_{b}\right)\right|_{\Gamma}=0$
и $\alpha \frac{\partial \varphi}{\partial \mathbf{n}}
+ \gamma (\varphi-\theta_0^4)|_{\Gamma} = 0$
на границе $\Gamma$, которые являются условиями Робина,
представляющими смесь условий Дирихле
(фиксированное значение) и Неймана (фиксированный градиент).

4.\ Наконец, начальное условие предоставляется четвертым уравнением,
$\theta|_{t=0} = \theta_0$, которое дает начальное
распределение температуры в области.


%TODO: Пассаж об однозначной разрешимости
%Приведем некоторые известные результаты.
%Будем далее предполагать, что выполняется следующее предположение.
%
%\textbf{Предположение 1.}
%Пусть $\beta, \gamma \in L^{\infty}(\Gamma), \beta \geq \beta_{0}>0, \gamma \geq \gamma_{0}>0$,
%$\beta_{0}, \gamma_{0}=$ константные функции,
%$\theta_{b} \in L^{\infty}(\Sigma)$, где $\Sigma \coloneqq \Gamma \times(0, T)$.

Доказательство однозначной разрешимости задачи~\eqref{eq:1_5:1}--\eqref{eq:1_5:3}
представлено в работе~\cite{JMAA-19}.

Приведём здесь анализ задачи, где $\beta=\gamma=1, \theta_b = r$.
Обозначим $ u = \theta_b^4$.

Также будем далее предполагать

(j) $a, b, \alpha, \kappa_{a} =$ Const $>0$,

(jj) $\theta_{b}, q_{b} \in U, r
=a\left(\theta_{b}+q_{b}\right) \in L^{5}(\Sigma) \theta_{0} \in L^{5}(\Omega)$.


Здесь под $U$ мы обозначаем пространство $L^{2}(\Sigma)$ с нормой
\[
    \|u\|_{\Sigma}=\left(\int_{\Sigma} u^{2} d \Gamma d t\right)^{1/2}.
\]

Используя следующие равенства, определим операторы $A: V \rightarrow V^{\prime}, B: U \rightarrow V^{\prime}$,
которые выполняются для любых $y, z \in V, w \in L^{2}(\Gamma)$:

\[
    (A y, z)=(\nabla y, \nabla z)+\int_{\Gamma} y z d \Gamma, \quad(B w, z)=\int_{\Gamma} w z d \Gamma.
\]
Билинейная форма $(A y, z)$ определяет скалярное произведение в пространстве $V$,
и соответствующая норма $\|z\|_{V}=\sqrt{(A z, z)}$ эквивалентна стандартной норме в $V$.
Следовательно, определен непрерывный обратный оператор
$A^{-1}: V^{\prime} \mapsto V$.
Заметим, что для любых $v \in V, w \in L^{2}(\Gamma), g \in V^{\prime}$
выполняются следующие неравенства:

\[
    \|v\|^{2} \leq C_{0}\|v\|_{V}^{2},\|v\|_{V^{\prime}} \leq C_{0}\|v\|_{V},
    \|B w\|_{V^{\prime}} \leq\|w\|_{\Gamma},\left\|A^{-1} g\right\|_{V} \leq\|g\|_{V^{\prime}}.
\]
Константа $C_0$ зависит только от пространства $\Omega$.

\begin{definition}
    Пара $\theta \in W, \varphi \in L^{2}(0, T ; V)$
    называется слабым решением задачи~\eqref{eq:1_5:1}--\eqref{eq:1_5:3}
    если
    \begin{equation}
        \label{eq:1_5:weak}
        \theta^{\prime}+a A \theta+b \kappa_{a}\left([\theta]^{4}-\varphi\right)=B r,
        \quad \theta(0)=\theta_{0}, \quad \alpha A \varphi+\kappa_{a}\left(\varphi-[\theta]^{4}\right)=B u
    \end{equation}
\end{definition}
Здесь и далее будем обозначать через
$[\theta]^s \coloneqq |\theta|^s \mathrm{sign}\,s  \in \mathbb{R}$.
\begin{lemma}
    \label{lm:1_5:1}
    Пусть выполняются условия (j), (jj), $u \in U$.
    Тогда существует уникальное слабое решение задачи~\eqref{eq:1_5:1}--\eqref{eq:1_5:3} и справедливо
    \[
        \psi=[\theta]^{5 / 2} \in L^{\infty}(0, T ; H) \cap L^{2}(0, T ; V),
        \quad[\theta]^{4} \in L^{2}(0, T ; H).
    \]
\end{lemma}

\begin{proof}

    Выразим $\varphi$ из последнего уравнения~\eqref{eq:1_5:weak} и подставим его в первое.
    В результате получаем следующую задачу Коши для уравнения с операторными коэффициентами:

    \begin{equation}
        \label{eq:1_5:4}
        \theta^{\prime}+a A \theta+L[\theta]^{4}=B r+f, \quad \theta(0)=\theta_{0}.
    \end{equation}

    Здесь
    \[
        L=\alpha b \kappa_{a} A\left(\alpha A+\kappa_{a} I\right)^{-1}:
        V^{\prime} \rightarrow V^{\prime},
        f=b \kappa_{a}\left(\alpha A+\kappa_{a} I\right)^{-1} B u \in L^{2}(0, T ; V).
    \]

    Получим априорные оценки решения задачи~\eqref{eq:1_5:4},
    на основании которых стандартным образом выводится разрешимость этой задачи.
    Пусть $[\zeta, \eta]=\left(\left(\alpha I+\kappa_{a} A^{-1}\right) \zeta,
    \eta\right), \zeta \in V^{\prime}, \eta \in V$.
    Обратите внимание, что выражение $[[\eta]]=\sqrt{[\eta, \eta]}$
    определяет норму в $H$, эквивалентную стандартной.

    Скалярно умножив уравнение~\eqref{eq:1_5:4} в смысле пространства $H$,
    на $\left(\alpha I+\kappa_{a} A^{-1}\right) \theta$, получаем
    \begin{equation}
        \label{eq:1_5:5}
        \frac{1}{2} \frac{d}{d t}[[\theta]]^{2}+a \alpha(A \theta, \theta)
        +a \kappa_{a}\|\theta\|^{2}
        +\alpha b \kappa_{a}\|\theta\|_{L^{5}(\Omega)}^{5}=[B r, \theta]+[f, \theta].
    \end{equation}

    Равенство~\eqref{eq:1_5:5} влечет оценку
    \[
        \|\theta\|_{L^{\infty}(0, T ; H)}+\|\theta\|_{L^{2}(0, T ; V)}+\|\theta\|_{L^{5}(Q)} \leq C_{1},
    \]
    где $C_{1}$ зависит только от
    $a, b, \alpha, \kappa_{a},\|f\|_{L^{2}(0, T ; H)},\left\|\theta_{0}\right\|,\|r\|_{L^{2}(\Sigma)}$.

    Далее, пусть $\psi=[\theta]^{5 / 2}$.
    Отметим, что

    \[
        \left(\theta^{\prime},[\theta]^{4}\right)
        =\frac{1}{5} \frac{d}{d t}\|\psi\|^{2}, \quad\left(A \theta,[\theta]^{4}\right)
        =\frac{16}{25}\|\nabla \psi\|^{2}+\|\psi\|_{L^{2}(\Gamma)}^{2}.
    \]
    Умножая в смысле скалярного произведения $H$ уравнение~\eqref{eq:1_5:4}
    на $[\theta]^{4}=[\psi]^{8 / 5}$, получаем

    \begin{equation}
        \label{eq:1_5:6}
        \frac{1}{5} \frac{d}{d t}\|\psi\|^{2}
        +a\left(\frac{16}{25}\|\nabla \psi\|^{2}+\|\psi\|_{L^{2}(\Gamma)}^{2}\right)
        +\left(L[\psi]^{8 / 5},[\psi]^{8 / 5}\right)
        =\left(B r+f,[\psi]^{8 / 5}\right).
    \end{equation}


    Равенство~\eqref{eq:1_5:6} влечет оценку
    \begin{equation}
        \label{eq:1_5:7}
        \|\psi\|_{L^{\infty}(0, T ; H)}+\|\psi\|_{L^{2}(0, T ; V)}
        +\left\|[\theta]^{4}\right\|_{L^{2}(0, T ; H)} \leq C_{2},
    \end{equation}

    где $C_2$ зависит только от
    $a, b, \alpha, \kappa_{a},\|f\|_{L^{2}(0, T ; H)},
    \left\|\theta_{0}\right\|_{L^{5}(\Omega)},\|r\|_{L^{5}(\Sigma)}$.
    Далее оценим $\left\|\theta^{\prime}\right\|_{L^{2}\left(0, T ; V^{\prime}\right)}$
    с учетом $\theta^{\prime}=B r+f-a A \theta-L[\theta]^{4}$.
    Из-за ограничений на начальные данне, верно утверждение,
    что $B r, f \in L^{2}\left(0, T ; V^{\prime}\right)$.

    Так как $\theta \in L^{2}(0, T ; V)$,
    следовательно $A \theta \in L^{2}\left(0, T ; V^{\prime}\right)$.
    Пусть $\zeta=L[\theta]^{4}$, тогда
    \[
        \alpha \zeta+\kappa_{a} A^{-1} \zeta=\alpha b \kappa_{a}[\theta]^{4}.
    \]

    Умножая в смысле скалярного произведения $H$ последнее равенство на $\zeta$, получаем

    \[
        \alpha\|\zeta\|^{2}+\kappa_{a}\left(A^{-1} \zeta, \zeta\right)
        =\alpha b \kappa_{a}\left([\theta]^{4}, \zeta\right)
        \leq \alpha\left(\|\zeta\|^{2}
        +\frac{\left(b \kappa_{a}\right)^{2}}{4}\left\|[\theta]^{4}\right\|^{2}\right).
    \]

    Следовательно, $\|\zeta\|_{V^{\prime}}^{2}=\left(A^{-1} \zeta,
    \zeta\right) \leq \frac{\alpha \kappa_{a} b^{2}}{4}\left\|[\theta]^{4}\right\|^{2}$
    и в силу оценок~\eqref{eq:1_5:5},~\eqref{eq:1_5:7} получаем

    \begin{equation}
        \label{eq:1_5:8}
        \left\|\theta^{\prime}\right\|_{L^{2}\left(0, T ; V^{\prime}\right)}
        \leq\|B r+f\|_{L^{2}\left(0, T ; V^{\prime}\right)}
        +a C_{1}+\sqrt{\alpha \kappa_{a}} b C_{2}.
    \end{equation}
    Оценок~\eqref{eq:1_5:5}--\eqref{eq:1_5:8} достаточно, для доказательства разрешгимости задачи.
    Пусть $\theta_{1,2}$ является решением задачи~\eqref{eq:1_5:4}, $\eta=\theta_{1}-\theta_{2}$.
    Тогда
    \[
        \eta^{\prime}+a A \eta+L\left(\left[\theta_{1}\right]^{4}-
        \left[\theta_{1}\right]^{4}\right)=0, \quad \eta(0)=0.
    \]

    Умножая в смысле скалярного произведения $H$
    последнее уравнение на $\left(\alpha I+\kappa_{a} A^{-1}\right) \eta$,
    получаем
    \[
        \frac{1}{2} \frac{d}{d t}[[\eta]]^{2}
        +a \alpha(A \eta, \eta)
        +a \kappa_{a}\|\eta\|^{2}
        +\alpha b \kappa_{a}\left(\left[\theta_{1}\right]^{4}
        -\left[\theta_{1}\right]^{4}, \theta_{1}-\theta_{2}\right)=0.
    \]
    Последний член в левой части неотрицательный,
    поэтому, интегрируя полученное равенство по времени,
    получаем $\eta=\theta_{1}-\theta_{2}=0$, что означает единственность решения.
    Лемма доказана.

\end{proof}

%
%Далее представим функцию $p_{\lambda}(s)=|s|^{\lambda} \operatorname{sign} s,
%\lambda>0, s \in \mathbb{R}$, которая служит для работы с нелинейностью.
%Отметим, что $p_{\lambda}^{\prime}(s)=\lambda|s|^{\lambda-1}$
%и $p_{\lambda}\left(p_{\mu}(s)\right)=p_{\lambda \mu}(s)$.
%
%Определим операторы $A_{1,2}: V \rightarrow V^{\prime}$
%и функционалы $g_{1,2} \in V^{\prime}$ через следующие уравнения, для произвольных
%функций $\theta, \varphi, v \in V$:
%
%\begin{equation*}
%\begin{gathered}
%\left(A_{1} \theta, v\right)=a(\nabla \theta, \nabla v)
%+\int_{\Gamma} \beta \theta v d \Gamma, \quad\left(A_{2} \varphi, v\right)
%=\alpha(\nabla \varphi, \nabla v)+\int_{\Gamma} \gamma \varphi v d \Gamma, \\
%\left(g_{1}, v\right)=\int_{\Gamma} \beta \theta_{b} v d \Gamma,
%\quad\left(g_{2}, v\right)
%=\int_{\Gamma} \gamma p_{4}\left(\theta_{b}\right) v d \Gamma.
%\end{gathered}
%\end{equation*}
%
%Билинейные формы $\left(A_{1} u, v\right)$ and $\left(A_{2} u, v\right)$
%определяют нормы в пространстве $V$ и эквивалентны стандартной норме $V$.
%Таким образом, операторы $A_{1,2}^{-1}: V^{\prime} \mapsto V$
%определены и ограничены.
%Далее мы  будем использовать энергитическую норму $\left(A_{1} v, v\right)^{1 / 2}$
%как норму пространства $V$.
%In particular, the following inequalities hold due to Sobolev's embedding theorem:
%Следующие неравенства выполняются согласно теореме вложенности Соболева:
%
%\begin{equation*}
%\|v\|^{2} \leq K\|v\|_{V}^{2}, \quad\|v\|_{L^{6}(\Omega)}^{2} \leq K_{1}\|v\|_{V}^{2},
%\quad\|v\|_{V}^{2} \leq K_{2}\left(A_{2} v, v\right)
%\end{equation*}
%
%Здесь константы $K, K_{1}, K_{2}$ не зависят от $v \in V$.


%
%Здесь $t$ - время, $x$ - положение, а $\omega$ - направление излучения.
%Условие $\mathbf{n} \cdot \omega < 0$ означает, что излучение движется в направлении,
%противоположном нормали к границе области, то есть входит в область $\Omega$.
%
%Система уравнений квазилинейна и связана, что делает ее решение сложным.
%Можно использовать численные методы, такие как методы конечных элементов
%или конечных разностей, для поиска приближенных решений этой задачи.
%Результаты могут предоставить информацию о поведении процессов теплообмена
%в данной области и могут быть использованы в различных приложениях,
%таких как изучение теплообмена в материалах или теплового управления
%в инженерных системах.
%
%где $S^{d-1}$ обозначает единичную сферу в $\mathbb{R}^{d}$.
%Для того чтобы получить корректно поставленную задачу, определим
%следующие граничные условия.
%Входящее излучение задается прозрачным условием на границе,
%и его интенсивность определяется как
%
%\[
%    I(t, x, \omega) = a u^{4}, \quad n \cdot \omega < 0,
%    \quad x \in \partial \Omega.
%\]
%
%Температура предполагается подчинённой граничным условиям типа Робина,
%которые представляют закон охлаждения Ньютона.
%Эти условия определяются следующим образом:
%
%\[
%    n \cdot \nabla T = \frac{h}{\varepsilon k}(u - T),
%    \quad x \in \partial \Omega.
%\]
%
%Здесь $n$ -- нормаль к границе области, $\nabla T$ -- градиент температуры,
%$h$ -- коэффициент теплоотдачи, $\varepsilon$ -- теплопроводность,
%$k$ -- коэффициент теплопроводности, $u$ -- температура окружающей среды,
%а $T$ -- температура внутри области.
%Эти условия описывают теплообмен между областью $\Omega$ и окружающей средой.
%
%Вместе эти граничные условия определяют взаимодействие радиационного
%теплообмена и температуры с окружающей средой.
%Таким образом, включение этих условий в начально-краевую задачу
%позволяет получить корректно поставленную задачу, которую можно
%решить с использованием численных методов.
%
%В начальный момент времени $t = 0$ температура равна $T(0, x) = T_{0}(x)$.
%В этих уравнениях $I(t, x, \omega)$ обозначает спектральную интенсивность
%излучения в точке $x \in \Omega$, движущегося в направлении $\omega \in S^{d-1}$,
%в момент времени $t \geq 0$.
%Внешнее излучение $I_{b} = a u^{4}$ предполагается известным для входящих
%направлений (то есть, $n \cdot \omega < 0$) на границе.
%Обозначим нормаль к внешней стороне $\partial \Omega$ через $n$.
%Кроме того, $T(t, x)$ обозначает температуру материала, а $u$ -- внешнюю
%температуру на границе, которая действует как управляющая переменная.
%
%Уравнения содержат параметры оптической плотности $\kappa$,
%теплопроводности $k$ и коэффициента конвективного теплообмена $h$,
%которые предполагаются положительными константами.
%Масштабированная оптическая толщина обозначена через $\varepsilon$.
%В уравнениях вводится константа $a$ для удобства обозначений, которая
%связана с постоянной Стефана-Больцмана через $a = \sigma / \pi$.
%Отметим, что согласно закону Стефана,
%полное тепловое излучение равно $B(T) = a T^{4}$.
%
%Поскольку данная модель имеет высокую размерность фазового пространства
%из-за зависимости от направления $\omega \in S^{d-1}$, ее численная сложность
%слишком велика для оптимизационных задач,
%где нелинейная система состояний должна быть решена несколько раз.
%Вместо этого мы используем диффузионные аппроксимации типа $S P_{N}$~\cite{Larsen2002, }$[6,9]$
%для уравнений радиационного теплообмена.
%Эти аппроксимации были разработаны недавно и широко протестированы
%для различных задач радиационного переноса,
%где они оказались достаточно точными~\cite{seaid2004efficient}.
%
%$S P_{1}$-аппроксимация для уравнений радиационного теплообмена представлена системой
%
%\begin{align}
%    \partial_{t} T & =k \Delta T+\frac{1}{3 \kappa} \Delta \rho, \label{eq:1_5:4}\\
%    0 & =-\varepsilon^{2} \frac{1}{3 \kappa} \Delta \rho+\kappa \rho-\kappa 4 \pi a|T|^{3} T,
%    \label{eq:1_5:5}
%\end{align}
%
%с граничными условиями
%
%
%\begin{align}
%        n \cdot \nabla T & =\frac{h}{\varepsilon k}(u-T), \label{eq:1_5:6}\\
%        n \cdot \nabla \rho & =\frac{3 \kappa}{2 \varepsilon}
%        \left(4 \pi a|u|^{3} u-\rho\right),\label{eq:1_5:7}.
%\end{align}
%и дополнен начальным условием $T(0, x) = T_{0}(x)$.
%Здесь $\rho$ - радиационный поток,
%а заданная температура на границе обозначена через $u$.
%
%Используя эту аппроксимацию, мы снижаем численную сложность модели,
%что делает ее подходящей для оптимизационных задач.
%\textit{Замечание} Радиационный поток для полной модели определяется
%как $\rho = \int_{S^{d-1}} I d \omega$.
%Мы заменили нелинейную функцию $u^{4}$ на $|u|^{3} u$ для обеспечения монотонности.
%
%В~\cite{JMAA-16} вводится и численно исследуется оптимальная краевая задача управления
%с функционалами затрат типа слежения, например:
%
%\[
%    J(T, u) = \frac{1}{2}\left\|T - T_{d}\right\|_{L^{2}\left(0,1 ; L^{2}(\Omega)\right)}^{2}
%    + \frac{\delta}{2}\left\|u - u_{d}\right\|_{H^{1}(0,1 ; \mathbb{R})}^{2}
%\]
%
%
%Решение этой задачи определяет оптимальные параметры
%управления температурой для радиационного теплообмена.
%где $(T, \rho)$ есть решение~\eqref{eq:1_5:4}--\eqref{eq:1_5:5}.
%Здесь $T_{d} = T_{d}(t, x)$ - заданный температурный профиль,
%$u_{d} = u_{d}(t)$ - заданный контроль окружающей температуры,
%который должен быть улучшен.
%Кроме того, положительная константа $\delta$ позволяет регулировать вес штрафного слагаемого.
%Краевая задача управления
%
%\[
%    \min J(T, u) \text { с учётом }(T, \rho, u),
%    \text { в соответствии с системой~\eqref{eq:1_5:4}--\eqref{eq:1_5:5}}.
%\]
%
%
%Эта оптимальная задача управления рассматривается как задача условной оптимизации,
%а сопряженные переменные используются для построения подходящего численного алгоритма.
%%В этой статье мы предоставляем анализ этого подхода.
%%Мы доказываем существование оптимального управления $u$ и уникальную разрешимость системы состояний,
%%что существенно для введения сокращенного функционала затрат.
%%Затем показывается уникальная разрешимость линеаризованной системы состояний
%%и определяются сопряженные уравнения.
