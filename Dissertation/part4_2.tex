\section{Численные алгоритмы минимизации функционалов}\label{sec:ch4/sec2}
В данной главе мы опишем градиентный спуск и его модификации,
а также некоторые стохастические методы.

\subsection{Градиентный спуск и модификации}
\label{subsec:ch4/sec2/grad}
\textit{Градиентный спуск} - это способ минимизации целевой функции
$J(\boldsymbol{\theta})$, параметризованной параметрами модели
$\boldsymbol{\theta} \in \mathbb{R}^{d}$, путем обновления параметров
в направлении, противоположном градиенту целевой функции
$\nabla_{\boldsymbol{\theta}} J(\boldsymbol{\theta})$
относительно параметров.
Скорость обучения $\eta$ определяет размер шагов, которые мы делаем,
чтобы достичь (локального) минимума.
Иными словами, мы следуем направлению наклона поверхности,
созданной целевой функцией, вниз до тех пор,
пока не достигнем локального минимума.

Известны различные способы модификации градиентного спуска.
Перечислим некоторые из них.

\textbf{Градиентный спуск с проекцией}

Метод градиентного спуска широко используется для минимизации дифференцируемой
функции $f(\mathbf{x})$, итеративно двигаясь в направлении наибольшего убывания.
Алгоритм осуществляется путем обновления параметров
$\mathbf{x} \in \mathbb{R}^n$ в направлении, противоположном
градиенту функции $\nabla f(\mathbf{x})$ с величиной шага
(скорость обучения) $\eta$.
Этот простой, но мощный алгоритм оптимизации используется во
множестве приложений, таких как машинное обучение, компьютерное
зрение и обработка сигналов.
В некоторых случаях задача оптимизации
имеет дополнительные ограничения, которые могут быть включены в
алгоритм градиентного спуска с использованием проекции.
В этом отчете представлен обзор метода градиентного спуска с проекцией,
а также обсуждаются его основные особенности и приложения.


Цель алгоритма градиентного спуска - минимизировать функцию
$f(\mathbf{x})$, итеративно обновляя параметры следующим образом:

\[ x_{k+1} = x_k - \eta \nabla f(x_k), \]

где $\mathbf{x}_k$ - вектор параметров на итерации $k$,
$\eta$ - величина шага, и $\nabla f(\mathbf{x}_k)$ - градиент
функции на итерации $k$.
Когда задача оптимизации включает ограничения, алгоритм
градиентного спуска должен быть изменен, чтобы учесть эти ограничения.
Один из распространенных подходов состоит в использовании проекции
на множество ограничений.


Пусть $\mathcal{C}$ - множество ограничений.
Алгоритм градиентного спуска с проекцией
можно описать следующим образом:
\[ x_{k+1} = \mathcal{P}_{\mathcal{C}}(x_k - \eta \nabla f(x_k)), \]

где $\mathcal{P}_{\mathcal{C}}(\cdot)$ - оператор проекции
на множество ограничений $\mathcal{C}$.
Проекция обеспечивает сохранение обновленного
вектора параметров $\mathbf{x}_{k+1}$ в пределах множества
ограничений, таким образом, удовлетворяя ограничения задачи.

\textbf{Оператор проекции}
Оператор проекции $\mathcal{P}_{\mathcal{C}}(\cdot)$
проецирует заданную точку на множество ограничений $\mathcal{C}$.
Проекция точки $\mathbf{y}$ на множество $\mathcal{C}$ определяется как:
\[
    \mathcal{P}_{\mathcal{C}}(y) =
    \arg \min_{x \in \mathcal{C}} \|y - x\|^2,
\]

где $\|\cdot\|$ обозначает евклидову норму.
Оператор проекции находит точку в множестве ограничений
$\mathcal{C}$, которая ближе всего к заданной точке $\mathbf{y}$.

\textbf{Примеры множеств ограничений}

Множество ограничений может иметь различные формы
в зависимости от задачи оптимизации.
Некоторые распространенные множества ограничений включают:
\begin{itemize}
    \item \textbf{Ограничения-коробки:} Множество ограничений представляет
    собой коробку, определенную как
    $\mathcal{C} = \{\mathbf{x} \in \mathbb{R}^n , | , a_i
    \leq x_i \leq b_i, , i=1,\dots,n\}$.
    В этом случае оператор проекции можно вычислить поэлементно:
    \[
        (\mathcal{P}_{\mathcal{C}}(y))_i =
        \min(\max(y_i, a_i), b_i), \quad i=1,\dots,n.
    \]

    \item \textbf{Ограничения-шары:} Множество ограничений представляет
    собой закрытый шар с радиусом $r$ и центром $\mathbf{c}$,
    определенный как $\mathcal{C} = \{\mathbf{x} \in \mathbb{R}^n , | ,
    \|\mathbf{x} - \mathbf{c}\| \leq r\}$.
    Оператор проекции для этого множества ограничений:
    \[
        \mathcal{P}_{\mathcal{C}}(y) = c
        + \min\left(1, \frac{r}{\|y-c\|}\right)(y-c).
    \]
    \item \textbf{Ограничения-симплексы:} Множество ограничений
    представляет собой симплекс, определенный как
    $\mathcal{C} = \{\mathbf{x} \in \mathbb{R}^n ,
    | , \mathbf{x} \geq \mathbf{0}, , \sum_{i=1}^n x_i = 1\}$.
    Оператор проекции для этого множества ограничений включает
    более сложный алгоритм, такой как тот,
    который представлен, например~\cite{Duchi2011}.
\end{itemize}

Градиентный спуск с проекцией нашел различные применения
в разных областях, включая:

\begin{itemize}
    \item \textbf{Разреженная оптимизация:}
    В машинном обучении разреженность является желаемым
    свойством для интерпретируемости модели и отбора признаков.
    Добавление ограничения $\ell_1$ к задаче оптимизации
    обеспечивает разреженность решения.
    Алгоритм градиентного спуска с проекцией может
    быть использован для решения таких задач.

    \item \textbf{Обработка изображений:} В обработке изображений
    минимизация полной вариации является популярным подходом
    для устранения шума и восстановления изображений.
    Задача оптимизации заключается в минимизации негладкой целевой
    функции с ограничениями на изображение.
    Градиентный спуск с проекцией может быть
    использован для решения этих типов задач.

    \item \textbf{Сжатое ощущение:} Сжатое ощущение
    (Compressed Sensing) - это метод, используемый в обработке
    сигналов для восстановления разреженного сигнала из
    небольшого количества линейных измерений.
    Задача восстановления формулируется как задача оптимизации
    с ограничениями на разреженность.
    Градиентный спуск с проекцией может быть использован для
    решения задачи восстановления сжатого ощущения.
\end{itemize}

Градиентный спуск с проекцией является универсальным алгоритмом
оптимизации, который позволяет включать ограничения в
стандартный метод градиентного спуска.
Оператор проекции обеспечивает выполнение
ограничений задачи при обновлении параметров.
Алгоритм успешно применяется в различных областях, таких как
машинное обучение, обработка изображений и обработка сигналов.
