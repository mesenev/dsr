\section{Численные алгоритмы минимизации функционалов}\label{sec:ch4/sec2}
Градиентный спуск и его модификации, стохастические методы.
Приведём работу Алексеева~\cite{Alekseev2019Simulation}, где в которой
задача управления решается методом роя частиц.

\subsection{Градиентный спуск}\label{subsec:ch4/sec2/grad}

\subsection{Стохастические методы}\label{subsec:ch4/sec2/stokhastic}

\subsection{Метод Монте-Карло нахождения решения прямой задачи}
\label{subsec:ch4/sec2/montekarlo}
%TODO: rework it!
Дадим описание вычислительного алгоритма, основанного на
методе Монте-Карло, решения прямой задачи для векторного
уравнения переноса.
Для его обоснования докажем ряд
вспомогательных утверждений, в итоге дающих представление задачи
$(\ref{2-1})$, $(\ref{2-2})$ в виде ряда Неймана.

Пусть  $\widetilde{\mu}(r)$ -- некоторая функция из $C_b (G_0)$,
удовлетворяющая неравенству $\widetilde{\mu}(r) \geq \mu(r)$, $r
\in G$. В силу леммы  \ref{lemma:l-5} будет справедливо следующее
утверждение.

\begin{lemma}
    \label{lemma:1-9}%2
    Функции, определенные формулами
    \[
        \widetilde{\tau}(r, \omega, t) = \int \limits_0^t \widetilde{\mu}
        (r - t'\omega) dt', \quad \widetilde{\tau}(r, \omega) = \int
        \limits_0^{d(r,-\omega)} \widetilde{\mu} (r - t\omega) dt,
    \]
    принадлежат \;соответственно \;пространствам \;$C_b (G_0 \times
    \Omega \times [0, d(r,-\omega)])$ и $C_b (G_0 \times \Omega)$.
\end{lemma}
Определим интегральные операторы $\widetilde{A}: K \rightarrow K$,
$S: K \rightarrow K \cap C^{(4)}(G_0 \times \Omega)$ и
$\widetilde{S}: K \rightarrow K$, так что
\[
    (\widetilde{A}\varphi)(r,\omega) = \int \limits_0^{d(r,-\omega)}
    \exp(-\widetilde{\tau}(r,\omega,t)) \varphi(r-t\omega, \omega) dt,
\]
\vskip0.1cm
\[
    (S\varphi)(r,\omega) = \mu_s(r)\int \limits_\Omega
    P(r,\omega,\omega') \varphi(r, \omega') d\omega' ,
\]
\vskip0.1cm
\[
    (\widetilde{S}\varphi)(r,\omega) = \mu_s(r)\int \limits_\Omega
    P(r,\omega,\omega') \varphi(r, \omega') d\omega' +
    (\widetilde{\mu}(r)-\mu(r))\varphi(r, \omega).
\]
Пусть
\[
    \overline{\mu} = \sup_{r \in G}\widetilde{\mu}(r), \quad
    \overline{\lambda} = \sup_{r \in G}\frac{\widetilde{\mu}(r) -
    \mu(r) + \mu_s(r)}{\widetilde{\mu}(r)} .
\]
\vskip0.3cm
\noindent Заметим, что для $f(r,\omega) \in K$
\[
    \| ((\widetilde{A}\widetilde{S})f(r,\omega))_1 \|  \leq
\]
\[
    \sup_{(r,\omega) \in G_0 \times \Omega_0}  \Bigg| \int
    \limits_0^{d(r,-\omega)} \exp(-\widetilde{\tau}(r,\omega,t))(
    (Sf)_1(r-t\omega, \omega) +
\]
\[
    (\widetilde{\mu}(r-t\omega) - \mu(r-t\omega)) f_1(r-t\omega,
    \omega)) dt \Bigg| \leq
\]
\vskip0.1cm
\[
    \sup_{(r,\omega) \in G_0 \times \Omega_0} \Bigg| \int
    \limits_0^{d(r,-\omega)} \exp(-\widetilde{\tau}(r,\omega,t))
    \widetilde{\mu}(r-t\omega)\times
\]
\[
    \times \Bigg( \frac{\widetilde{\mu}(r-t\omega) - \mu(r-t\omega) +
    \mu_s(r-t\omega)}{\widetilde{\mu}(r-t\omega)} \Bigg) dt \Bigg| \|
    f_1 \| \leq
\]
%\vskip0.1cm
\begin{equation}
    \label{2-39}
    \overline{\lambda}\left(1 - e^{-\overline{\mu} d} \right)\| f_1\|
    .
\end{equation}

%\vskip0.1cm
Докажем следующее утверждение.

\begin{lemma}
    \label{lemma:1-10}%3
    Функция $f$ является решением краевой задачи $(\ref{2-1})$,
    $(\ref{2-2})$ тогда и только тогда, когда $f$ удовлетворяет
    уравнению
    \begin{equation}
        \label{2-40}
        f(r,\omega) = h(r - d(r, - \omega) \omega, \omega) \exp (-
        \widetilde{\tau} (r, \omega)) + (\widetilde{A}J)(r,\omega) +
        (\widetilde{A}\widetilde{S}f)(r,\omega)
    \end{equation}
    в пространстве $C_b^{(4)}(G_0 \times \Omega_0)$.
\end{lemma}

\begin{proof}[необходимость]
%\emph{Необходимость}.
Пусть функция $f(r,\omega)$ является решением прямой задачи, тогда
$f(r,\omega) \in D(G_0 \times \Omega_0)$ и для любых точек
$(r,\omega) \in G_0 \times \Omega_0$
\begin{equation}
    \label{2-41}
    \widetilde{l}f(r - t \omega , \omega)  = F(r - t \omega, \omega),
\end{equation}
\[
    \widetilde{l}f(r, \omega) = \omega \cdot \nabla_r f(r, \omega) +
    \widetilde{\mu} (r) f(r,\omega), \quad F(r, \omega) =
    (\widetilde{S}f)(r, \omega) + J(r, \omega)
\]
почти везде на множестве  $\{ r-t \omega  : t \in [0,d(r,-\omega)]
\}$. Так как функция $f(r - t \omega, \omega)$ абсолютно
непрерывна на множестве $\{ r - t \omega : t \in [0,d(r,-\omega)]
\}$, то выражение $\widetilde{l}f(r - t \omega , \omega)$ является
интегрируемой функцией по $t$ для любых $(r,\omega) \in G_0 \times
\Omega_0$. Умножая равенство (\ref{2-41}) слева и справа на
функцию $\exp(-\widetilde{\tau}(r,\omega,t))$ и интегрируя по $t
\in [0,d(r,-\omega)]$, получаем
\begin{multline}
    \label{2-42}
    \int \limits^{d(r,-\omega)}_{0} \exp (- \widetilde{\tau} (r,
    \omega,t)) \widetilde{l}f(r - t \omega, \omega) dt = \\ \int
    \limits^{d(r,-\omega)}_{0} \exp (- \widetilde{\tau} (r, \omega,t))
    F(r - t \omega, \omega)dt .
\end{multline}
Так как функция $\widetilde{l}f(r, \omega) \in C_b^{(4)}(G_0
\times \Omega_0)$, а функция $\widetilde{\tau}(r,\omega,t)$
абсолютно непрерывна по переменной $t$, то к левой части равенства
(\ref{2-42}) применима формула интегрирования по частям:

\begin{multline}
    \label{2-43}
    \int \limits^{d(r,-\omega)}_{0} \exp (- \widetilde{\tau} (r,
    \omega,t))\widetilde{l}f(r - t \omega, \omega) dt = f(r,\omega)
    -\\
    f(r - d(r,-\omega)\omega, \omega) \exp (-
    \widetilde{\tau}(r,\omega)) - \int \limits^{d(r,-\omega)}_{0} \exp
    (- \widetilde{\tau} (r, \omega,t))\times \\ \times \big(-
    \widetilde{\mu} (r - t \omega) f(r - t \omega , \omega) +
    \widetilde{\mu} (r - t \omega) f(r - t \omega, \omega)\big) dt .
\end{multline}
Учитывая граничное условие (\ref{2-2}), из (\ref{2-43}) получаем
\begin{multline*}
    \int \limits^{d(r,-\omega)}_{0} \exp (- \widetilde{\tau} (r,
    \omega,t))F(r - t \omega, \omega) dt =\\ f(r,\omega) - h(r -
    d(r,-\omega)\omega, \omega) \exp( - \widetilde{\tau} (r, \omega)).
\end{multline*}
Отсюда и из свойств интегральных операторов $\widetilde{A}$ и
$\widetilde{S}$ непосредственно вытекает, что функция
$f(r,\omega)$ из пространства $C_b^{(4)}(G_0 \times \Omega_0)$ для
любых $(r,\omega) \in G_0 \times \Omega_0$ удовлетворяет уравнению
(\ref{2-40}). Необходимость доказана.
\end{proof}

\begin{proof}[Достаточность]
%\emph{Достаточность}.
Пусть функция $f(r,\omega) \in D(G_0 \times
\Omega_0)$ удовлетворяет уравнению (\ref{2-40}). Тогда для любых
$(\xi,\omega) \in \Gamma^-$ из уравнения  (\ref{2-40}), учитывая,
что $d(\xi + t \omega, - \omega) = t$, имеем
\begin{multline}
    \label{2-44}
    f(\xi + t \omega, \omega) = \exp (- \widetilde{\tau}
    (\xi,-\omega,t))\times \\ \left( h(\xi, \omega) + \int
    \limits^{t}_{0} \exp ( \widetilde{\tau} (\xi,-\omega,t'))F(\xi + t
    ' \omega, \omega) dt' \right).
\end{multline}

Так как неопределенный интеграл от интегрируемой по $t$ функции
является абсолютно непрерывной функцией, а сумма и произведение
абсолютно непрерывных функций есть также абсолютно непрерывная
функция, то из (\ref{2-44}) заключаем, что функция
$f(\xi + \omega t, \omega)$ абсолютно непрерывна на
$\{\xi+ t \omega :t \in [0,d(\xi,\omega)] \}$ для любых $(\xi,\omega) \in \Gamma^-$.
Непосредственно убеждаемся, что при $t = 0$ функция $f(\xi +
\omega t, \omega)$ удовлетворяет краевому условию (\ref{2-2}).

Далее, подействовав оператором $\widetilde{l}$ на $f(\xi +
t\omega,\omega)$ и воспользовавшись равенством (\ref{2-44}),
получим при почти всех $t \in [0,d(\xi,\omega)]$
\begin{multline*}
    \widetilde{l}f(\xi + t\omega,\omega) = - \widetilde{\mu} (\xi +
    t\omega) \exp (- \widetilde{\tau}(\xi,-\omega,t)) \times \\
    \left(h(\xi,\omega) + \int \limits^{t}_{0} \exp (\widetilde{\tau}
    (\xi,-\omega,t')) F(\xi + t' \omega,\omega)dt' \right) +
    \\
    \widetilde{\mu} (\xi + \omega t) \exp (-\widetilde{\tau}
    (\xi,-\omega,t)) \times \\ \left( h(\xi,\omega) + \int
    \limits^{t}_{0} \exp (\widetilde{\tau} (\xi,-\omega,t'))F(\xi + t'
    \omega,\omega) dt' \right) +
    \\
    \exp (-\widetilde{\tau} (\xi,-\omega,t)) \exp (\widetilde{\tau}
    (\xi,-\omega,t))F(\xi + t\omega,\omega) = F(\xi + t\omega,\omega).
\end{multline*}
Таким образом, функция $f(r,\omega)$ из класса $D(G_0 \times
\Omega_0)$, являющаяся решением уравнения (\ref{2-40}),
удовлетворяет всем условиям в определении краевой задачи
(\ref{2-1}), (\ref{2-2}). Лемма доказана.
\end{proof}

Обозначим
\[
    \widetilde{f}_0(r,\omega) = h(r - d(r, - \omega) \omega, \omega)
    \exp (- \widetilde{\tau} (r, \omega)) +
    (\widetilde{A}J)(r,\omega).
\]
Теперь докажем окончательное утверждение о корректности прямой
задачи (\ref{2-1}), (\ref{2-2}).

\begin{theorem}
    \label{thm:1-6}%1
    Пусть $\widetilde{h}(r,\omega) \in K, \, J(r,\omega) \in
    C_b^{(4)}(G_0 \times \Omega) \cap K$ и справедливы условия
    $(\ref{2-11})$, $(\ref{2-12})$, тогда в конусе $K$ решение задачи
    $(\ref{2-1})$, $(\ref{2-2})$ существует, единственно и представимо
    в виде ряда Неймана
    \begin{equation}
        \label{2-45}
        f(r,\omega) = \widetilde{f}_0(r,\omega) + \sum
        \limits_{n=1}^{\infty} (\widetilde{A}\widetilde{S})^n
        \widetilde{f}_0(r,\omega),
    \end{equation}
    сходящегося в норме пространства $C^{(4)}_b(G_0 \times \Omega_0)$.
\end{theorem}

\begin{proof}
По лемме \ref{lemma:1-10}
достаточно доказать разрешимость уравнения (\ref{2-40}). Заметим,
что конус $K$ является замкнутым подмножеством банахова
пространства $C_b^{(4)}(G_0 \times \Omega_0)$, следовательно $K$
--- полное пространство. Поэтому достаточно доказать, что оператор
$\widetilde{A}\widetilde{S}$ переводит $K$ в $K$, функция
$\widetilde{f}_0(r,\omega) \in K$ и $\|
\widetilde{A}\widetilde{S}f \| < \| f \|$ для всех $f \in K$.

Так как оператор $\widetilde{S}: K \rightarrow   K$ и оператор
$\widetilde{A}: K \rightarrow K$, то $\widetilde{A}\widetilde{S}:
K \rightarrow K$. Поскольку $ \widetilde{h}(r,\omega) \in K$,
$\widetilde{\tau}(r,\omega) \in C_b(G_0 \times \Omega)$ и
$J(r,\omega) \in  K \cap C_b^{(4)}(G_0 \times \Omega)$, то
$\widetilde{f}_0(r, \omega) \in K$.

Используя неравенство (\ref{2-39}),  для всех $f  \in K$ имеем
\begin{multline*}
    \| \widetilde{A}\widetilde{S}f \|_4 = \max \limits_{1 \leq i \leq
    4} \|(\widetilde{A}\widetilde{S}f)_i\|=\|
    (\widetilde{A}\widetilde{S}f)_1 \| \leq \\
    \overline{\lambda}\left( 1 - e^{-\overline{\mu} d} \right) \| f_1
    \| = \overline{\lambda}\left( 1 - e^{-\overline{\mu} d} \right) \|
    f \|_4 .
\end{multline*}
И поскольку $\mu(r) \geq \mu_s(r)$, $r \in G$, то
$\overline{\lambda} \leq 1$ и
\[
    \| \widetilde{A}\widetilde{S}f \|_4 \leq \overline{\lambda} \left(
    1 - e^{-\overline{\mu} d} \right) \| f \|_4 < \| f \|_4.
\]
Следовательно, утверждение теоремы вытекает из принципа сжимающих
отображений.
\end{proof}

Отметим, что в случае $\widetilde{\mu}(r) = \mu(r)$ обоснование
корректности прямой задачи совпадает с аналогичными рассуждениями,
проведенными в \S 3.1. Необходимость введения функции
$\widetilde{\mu}(r)$ вызвана потребностью обоснования
вычислительного алгоритма решения прямой задачи, рассмотренного
далее.


Опишем алгоритм вычисления вектор-функции $f$, основанный на
методе Монте-Карло. Пусть для любых $r\in G $ выполняется
неравенство $\mu(r)\leq \overline{\mu}$, где $\overline{\mu}$ --
некоторая константа. Тогда, полагая $\widetilde{\mu}(r) =
\overline{\mu}$, по теореме \ref{thm:1-6} получаем представление
решения в виде равномерно сходящегося ряда  (\ref{2-45}). Добавку
$(\overline{\mu}-\mu(r))\varphi(r,\omega)$ в выражении для
интегрального оператора $\widetilde{S}$ можно трактовать как
некоторое фиктивное рассеяние без изменения направления
распространения фотона. Данный метод носит название метода
максимального сечения и позволяет более просто разыгрывать длину
свободного пробега частицы даже в областях со сложной структурой.
Подобный подход дает неплохие результаты в случае, когда изменение
коэффициента полного взаимодействия в среде невелико. Пусть $m$ --
учитываемое нами число членов ряда Неймана, а $n$ -- число
моделируемых траекторий, тогда приближенное значение функции
$f(r,\omega)$ можно вычислить по формуле:
\begin{equation}
    \label{2-53}
    f(r,\omega)\approx \overline{f}_n(r,\omega) = \frac{1}{n} \sum
    \limits_{i=1}^{n} s_i(r,\omega),
\end{equation}
\[
    s_i(r,\omega) = \widetilde{f_0}(r,\omega)+ \sum \limits_{j=1}^{m}
    \prod \limits_{k=1}^{j} \frac{1-\exp(-\overline{\mu}
    d(r^{i,k-1},-\omega^{i,k-1}))} {\overline{\mu}}\times
\]
\[
    \times(\overline{\mu}-\mu(r^{i,k})+\mu_s(r^{i,k}))
    Q(\omega^{i,k-1},\omega^{i,k})
    \widetilde{f_0}(r^{i,j},\omega^{i,j}).
\]


Для генерации точек траекторий $r^{i,k}$ и направлений
$\omega^{i,k}$, необходимых для реализации алгоритма, используем
начальные значения: $r^{i,0} = r$, $\omega^{i,0} = \omega$.

На каждом последующем шаге вычисление точки $r^{i,k}$
осуществляется по формуле:
\begin{equation}
    \label{2-54}
    r^{i,k}=r^{i,k-1}-\omega^{i,k-1} t_{i,k},
\end{equation}
где $t_{i,k}$ -- независимая реализация случайной величины,
распределенной на $[0, d(r^{i,k-1}, -\omega^{i,k-1})]$ с
плотностью
\begin{equation}
    \label{2-55}
    \rho(t) = \frac{ \overline{\mu} \exp (-\overline{\mu} t )}{1-\exp
        (\overline{\mu} d(r^{i,k-1},-\omega^{i,k-1}))}.
\end{equation}
Далее разыгрываем реализацию $\alpha_{i,k}$ равномерно
распределенной на $[0,1]$ случайной величины. При
\begin{equation}
    \label{2-56}
    \alpha_{i,k} \geq
    \frac{\overline{\mu}-\mu(r^{i,k})}{\overline{\mu}-\mu(r^{i,k})+\mu_s(r^{i,k})}
\end{equation}
для определения величины $\omega^{i,k}$  и матрицы
$Q(\omega^{i,k-1},\omega^{i,k})$ используем следующие расчетные
формулы:
\begin{equation}
    \label{2-57}
    \omega^{i,k}_1=\sqrt{1-\nu^{2}_{i,k}}\cos \gamma_{i,k},
\end{equation}
\begin{equation}
    \label{2-58}
    \omega^{i,k}_2=\sqrt{1-\nu^{2}_{i,k}}\sin \gamma_{i,k},
\end{equation}
\begin{equation}
    \label{2-59}
    \omega^{i,k}_3=\nu_{i,k},
\end{equation}
\begin{equation}
    \label{2-60}
    Q(\omega^{i,k-1},\omega^{i,k}) = 4\pi
    P(\omega^{i,k-1},\omega^{i,k}),
\end{equation}
а при \[ \alpha_{i,k} <
\frac{\overline{\mu}-\mu(r^{i,k})}{\overline{\mu}-\mu(r^{i,k})+\mu_s(r^{i,k})}
\]
следующие:
\begin{equation}
    \label{2-61}
    \omega^{i,k} = \omega^{i,k-1}, \quad
    Q(\omega^{i,k-1},\omega^{i,k}) = E.
\end{equation}
Здесь $\nu_{i,k}$, $\gamma_{i,k}$ есть независимые реализации
равномерно распределенных на соответствующих промежутках $[-1,1]$
и $[0,2\pi)$ случайных величин, а $E$ есть единичная матрица \;$4
\times 4$.

Таким образом, алгоритм нахождения решения прямой задачи выглядит
следующим образом.

0. Пусть $i=1$.

1. Берем $i$-ую траекторию и полагаем $k=1$.

2. Для $k$-ого звена в траектории разыгрываем длину свободного
пробега $t_{i,k}$ с плотностью вероятности, задаваемой формулой
(\ref{2-55}).

3. Вычисляем координаты новой точки столкновения $r^{i,k}$ по
формуле (\ref{2-54})

4. Разыгрываем реализацию $\alpha_{i,k}$ равномерно распределенной
на [0,1] случайной величины.

5. Если выполняется условие (\ref{2-56}), то новое направление
$\omega^{i,k}$ находится из \;(\ref{2-57})-(\ref{2-59}),\; а\,
матрица\;\; $Q(\omega^{i,k-1},\omega^{i,k})$\;\; по\, формуле\;
(\ref{2-60}).\; Если \;же условие не выполняется, то согласно
(\ref{2-61}) направление не меняется, а в качестве матрицы
$Q(\omega^{i,k-1},\omega^{i,k})$ берется единичная.

6. Формируем слагаемые ряда Неймана и накапливаем сумму в
(\ref{2-53}).


7. Если учтены не все члены ряда Неймана (при $k<m$), то переходим
к пункту 2, увеличивая $k$ на единицу.

8. Если учтены не все траектории  (при $i<n$), то переходим к
пункту 1, увеличивая переменную $i$ на единицу.

Приведенный здесь алгоритм решения прямой задачи (\ref{2-1}),
(\eqref{2-2}) был программно реализован и далее использовался для
нахождения выходящего излучения, необходимого для решения задачи
компьютерной томографии.

