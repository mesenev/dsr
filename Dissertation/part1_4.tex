\section{Стационарная модель сложного теплообмена}\label{sec:ch1/sec3}

\subsection{Постановка краевой задачи}
\label{subsec:ch1/sec3/state}

Стационарная нормализованная диффузионная модель, описывающая
радиационный, кондуктивный и конвективный теплообмен в
ограниченной области $\Omega \subset \mathbb{R}^3$,
имеет следующий вид~\cite{modest2013radiative}:

\begin{equation}
    \label{eq:1_4:4-1}
    -a \Delta \theta + \textbf{v} \cdot \nabla \theta
    + b \kappa_a \theta^4 =  b \kappa_a \varphi,
\end{equation}

\begin{equation}
    \label{eq:1_4:4-2}
    - \alpha \Delta \varphi + \kappa_a \varphi = \kappa_a \theta^4.
\end{equation}

Здесь $\theta$ -- нормализованная температура, $\varphi$ --
нормализованная интенсивность излучения, усредненная по всем
направлениям, $\textbf{v}$ -- заданное поле скоростей, $\kappa_a$ --
коэффициент поглощения.
Постоянные $a$, $b$ и $\alpha$
определяются следующим образом:
\[
    a=\frac{k}{\rho c_v},\quad b = \frac{4\sigma n^2 T_{\max}^3}{\rho c_v},
    \quad \alpha=\frac{1}{3\kappa - A \kappa_s},
\]
где $k$ -- теплопроводность, $c_v$ -- удельная теплоемкость, $\rho$ --
плотность, $\sigma$ -- постоянная Стефана-Больцмана, $n$ --
показатель преломления, $T_{\max}$ -- максимальная температура в
ненормализованной модели, $\kappa = \kappa_s + \kappa_a$ -- коэффициент
полного взаимодействия, $\kappa_s$ -- коэффициент рассеяния.
Коэффициент $A \in [-1, 1]$ описывает анизотропию рассеяния, случай
$A=0$ соответствует изотропному рассеянию.

Будем предполагать, что функции $\theta$ и $\varphi$, описывающие
процесс сложного теплообмена, удовлетворяют следующим условиям на
границе $\Gamma = \partial \Omega$:
\begin{equation}
    \label{eq:1_4:4-3}
    a \frac{\partial \theta}{\partial \mathbf{n}}
    +\left.\beta\left(\theta-\theta_{b}\right)\right|_{\Gamma}=0,
\end{equation}
\begin{equation}
    \label{eq:1_4:4-4}
    \alpha \frac{\partial \varphi}{\partial \mathbf{n}} + \gamma
    (\varphi-\theta_b^4)|_{\Gamma} = 0.
\end{equation}
Здесь через $\partial/\partial \mathbf{n}$ обозначаем производную
в направлении внешней нормали.
Неотрицательные функции $\theta_{b}, \gamma$, определенные на $\Gamma$, и функция $\beta$,
описывающая, в частности, отражающие свойства границы $\Gamma$, являются заданными.

\subsection{Слабое решение краевой задачи}\label{subsec:ch1/sec3/weak}
% Далее CNSNS_2014
Пусть $\Omega$ -- липшицева ограниченная область, граница $\Gamma$
которой состоит из конечного числа гладких
кусков, а исходные данные удовлетворяют условиям: \\
(i) $\mathbf{v} \in H^{1}(\Omega) \cap L^{\infty}(\Omega), \quad \nabla \cdot \mathbf{v}=0$; \\
(ii) $\theta_{0}, \beta, \gamma \in L^{\infty}(\Gamma),
0 \leqslant \Theta_{0} \leqslant M,
\beta \geqslant \beta_{0}>0, \gamma \geqslant \gamma_{0}>0$; \\
(iii) $\gamma+(\mathbf{v} \cdot \mathbf{n}) \geqslant c_{0}>0$ на части границы,
где $(\mathbf{v} \cdot \mathbf{n})<0.$ \\

Здесь $M, \beta_{0}, \gamma_{0}$, и $c_{0}$ положительные постоянные.


\begin{definition}
    Пара $\{\theta, \varphi\} \in H^1(\Omega) \times H^1(\Omega)$ называется
    слабым решением задачи $(\ref{eq:1_4:4-1})$-$(\ref{eq:1_4:4-4})$, если для
    любых $\eta, \psi \in H^1(\Omega)$
    выполняются равенства
    \begin{equation}
        \label{eq:1_4:4-5}
            a(\nabla \theta, \nabla \eta) + \left(\mathbf{v} \cdot \nabla \theta
            + b \kappa_{a}\left(|\theta| \theta^{3} - \varphi\right), \eta\right) \\
            + \int_{\Gamma} \beta\left(\theta - \theta_{0}\right) \eta d \Gamma=0,
    \end{equation}
    \begin{equation}
        \label{eq:1_4:4-6}
        \alpha(\nabla \varphi, \nabla \psi)+\kappa_{a}\left(\varphi-|\theta| \theta^{3},
        \psi\right)+\int_{\Gamma} \gamma\left(\varphi-\theta_{0}^{4}\right) \psi d \Gamma=0.
    \end{equation}
\end{definition}

Отметим, что в силу вложения $H^1(\Omega) \subset L^6(\Omega)$ выражение
$(\theta^4, \eta)$  имеет смысл для любой функции $\eta \in H^1(\Omega)$.

\begin{theorem}
    \label{th:1_4:weakExist}~\cite[Th. 2]{CNSNS-15}
    Пусть выполняются условия (i)-(iii).
    Тогда существует единственное слабое
    решение задачи $(\ref{eq:1_4:4-1})$-$(\ref{eq:1_4:4-4})$,
    удовлетворяющее неравенствам
    \begin{align}
        & a\|\nabla \theta\|^{2} \leqslant b \kappa_{a} M^{5}|\Omega|
        + \|\gamma\|_{L^{\infty}(\Gamma)} M^{2}|\Gamma|,\\
        & \alpha\|\nabla \varphi\|^{2} \leqslant \kappa_{a} M^{8}|\Omega|
        + \|\beta\|_{L^{\infty}(\Gamma)} M^{8}|\Gamma|,\\
        & 0 \leqslant \theta \leqslant M, \quad 0 \leqslant \varphi \leqslant M^{4}.
    \end{align}
\end{theorem}
