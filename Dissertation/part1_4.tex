\section{Квазистационарная модель сложного теплообмена}
\label{sec:ch1/sec4}
Квазистационарная модель - это тип математической модели,
который описывает систему, претерпевающую медленные изменения
со временем или имеющую относительно длительный период стабильности
по сравнению с интересующим масштабом времени.
Такие модели часто используются, когда изучаемая система находится
в равновесии или близка к нему, но при этом может испытывать небольшие,
медленные колебания со временем.
Термин `квази` означает,
что система не совсем стационарна (то есть, фиксирована или неизменна),
а скорее, ее состояние меняется настолько медленно, что его можно считать
почти стационарным для определенных анализов или целей.

В контексте теплообмена, излучения или других физических процессов
квазистационарные модели могут использоваться для описания сценариев,
когда параметры и свойства системы меняются очень медленно по сравнению
с масштабом времени конкретного изучаемого явления.
Такие модели могут упростить анализ и снизить вычислительную сложность,
позволяя исследователям сосредоточиться на основных аспектах проблемы.


Квазистационарный радиационный и диффузионный теплообмен в ограниченной
области $\Omega \subset \mathbb{R}^3$ с границей
$\Gamma = \partial\Omega$ моделируется в рамках приближения P1
для уравнения радиационного теплообмена следующей
начально-краевой задачей [16, 23]:
\begin{align*}
    \frac{\partial\theta}{\partial t} - a\Delta\theta
    + b\kappa_a (|\theta|\theta^3 - \phi) &= 0, \\
    - \alpha\Delta\phi + \kappa_a (\phi - |\theta|\theta^3 ) &= 0,
    \quad x \in \Omega, \quad 0 < t < T ; \tag{1} \\
    a(\partial_n \theta + \theta) &= r,
    \quad \alpha(\partial_n \phi + \phi) = u
    \quad \text{на} \quad \Gamma ; \tag{2} \\
    \theta|_{t=0} &= \theta_0. \tag{3}
\end{align*}

Данная модель описывает систему связанных уравнений в частных производных (УЧП),
моделирующих квазистационарный радиационный и диффузионный теплообмен в
ограниченной области $\Omega \subset \mathbb{R}^3$ в трехмерном пространстве.
Используется приближение P1 для уравнения радиационного теплообмена,
что является упрощенным подходом к решению задач радиационного теплообмена.

Задача состоит из начально-краевой задачи с тремя компонентами:

\begin{enumerate}
    \item Первое уравнение представляет баланс энергии из-за радиационного
    теплообмена $\left(\frac{\partial \theta}{\partial t}\right)$ и проводящего
    теплообмена $(a \Delta \theta)$ с термом источника тепла
    $\left(b \kappa_a \left(|\theta| \theta^3 - \phi \right)\right)$
    в ограниченной области $\Omega$ на промежутке времени $0 < t < T$.
    Радиационный теплообмен моделируется с использованием приближения P1,
    которое упрощает уравнение радиационного теплообмена.

    \item Второе уравнение представляет баланс энергии из-за диффузионного
    теплообмена $(\alpha \Delta \phi)$ с термом источника тепла
    $\left(\kappa_a \left(\phi - |\theta| \theta^3 \right)\right)$
    в той же области $\Omega$ на промежутке времени $0 < t < T$.
    Это уравнение связано с первым уравнением через термы источника тепла.

    \item Граничные условия задаются третьим уравнением,
    $a(\partial_n \theta + \theta) = r$ и $\alpha(\partial_n \phi + \phi) = u$
    на границе $\Gamma$, которые являются условиями Робина,
    представляющими смесь условий Дирихле
    (фиксированное значение) и Неймана (фиксированный градиент).

    \item Наконец, начальное условие предоставляется четвертым уравнением,
    $\theta|_{t=0} = \theta_0$, которое дает начальное
    распределение температуры в области.

\end{enumerate}

Здесь $t$ - время, $x$ - положение, а $\omega$ - направление излучения.
Условие $n \cdot \omega < 0$ означает, что излучение движется в направлении,
противоположном нормали к границе области, то есть входит в область $\Omega$.

Система уравнений квазилинейна и связана, что делает ее решение сложным.
Можно использовать численные методы, такие как методы конечных элементов
или конечных разностей, для поиска приближенных решений этой задачи.
Результаты могут предоставить информацию о поведении процессов теплообмена
в данной области и могут быть использованы в различных приложениях,
таких как изучение теплообмена в материалах или теплового управления
в инженерных системах.

где $S^{d-1}$ обозначает единичную сферу в $\mathbb{R}^{d}$.
Для того чтобы получить корректно поставленную задачу, определим
следующие граничные условия.
Входящее излучение задается прозрачным условием на границе,
и его интенсивность определяется как

\[
    I(t, x, \omega) = a u^{4}, \quad n \cdot \omega < 0,
    \quad x \in \partial \Omega.
\]

Температура предполагается подчиняться граничным условиям типа Робина,
которые представляют закон охлаждения Ньютона.
Эти условия определяются следующим образом:

\[
    n \cdot \nabla T = \frac{h}{\varepsilon k}(u - T),
    \quad x \in \partial \Omega.
\]

Здесь $n$ -- нормаль к границе области, $\nabla T$ -- градиент температуры,
$h$ -- коэффициент теплоотдачи, $\varepsilon$ -- теплопроводность,
$k$ -- коэффициент теплопроводности, $u$ -- температура окружающей среды,
а $T$ -- температура внутри области.
Эти условия описывают теплообмен между областью $\Omega$ и окружающей средой.

Вместе эти граничные условия определяют взаимодействие радиационного
теплообмена и температуры с окружающей средой.
Таким образом, включение этих условий в начально-краевую задачу
позволяет получить корректно поставленную задачу, которую можно
решить с использованием численных методов.

В начальный момент времени $t = 0$ температура равна $T(0, x) = T_{0}(x)$.
В этих уравнениях $I(t, x, \omega)$ обозначает спектральную интенсивность
излучения в точке $x \in \Omega$, движущегося в направлении $\omega \in S^{d-1}$,
в момент времени $t \geq 0$.
Внешнее излучение $I_{b} = a u^{4}$ предполагается известным для входящих
направлений (то есть, $n \cdot \omega < 0$) на границе.
Обозначим нормаль к внешней стороне $\partial \Omega$ через $n$.
Кроме того, $T(t, x)$ обозначает температуру материала, а $u$ -- внешнюю
температуру на границе, которая действует как управляющая переменная.

Уравнения содержат параметры оптической плотности $\kappa$,
теплопроводности $k$ и коэффициента конвективного теплообмена $h$,
которые предполагаются положительными константами.
Масштабированная оптическая толщина обозначена через $\varepsilon$.
В уравнениях вводится константа $a$ для удобства обозначений, которая
связана с постоянной Стефана-Больцмана через $a = \sigma / \pi$.
Отметим, что согласно закону Стефана,
полное тепловое излучение равно $B(T) = a T^{4}$.

Поскольку данная модель имеет высокую размерность фазового пространства
из-за зависимости от направления $\omega \in S^{d-1}$, ее численная сложность
слишком велика для оптимизационных задач,
где нелинейная система состояний должна быть решена несколько раз.
Вместо этого мы используем диффузионные аппроксимации типа $S P_{N}$ $[6,9]$
для уравнений радиационного теплообмена.
Эти аппроксимации были разработаны недавно и широко протестированы
для различных задач радиационного переноса,
где они оказались достаточно точными [16].

$S P_{1}$-аппроксимация для уравнений радиационного теплообмена представлена системой

\[
    \begin{aligned}
        \partial_{t} T & =k \Delta T+\frac{1}{3 \kappa} \Delta \rho, \\
        0 & =-\varepsilon^{2} \frac{1}{3 \kappa} \Delta \rho+\kappa \rho-\kappa 4 \pi a|T|^{3} T,
    \end{aligned}
\]

с граничными условиями

\[
    \begin{aligned}
        n \cdot \nabla T & =\frac{h}{\varepsilon k}(u-T), \\
        n \cdot \nabla \rho & =\frac{3 \kappa}{2 \varepsilon}
        \left(4 \pi a|u|^{3} u-\rho\right),
    \end{aligned}
\]


Используя эту аппроксимацию, мы снижаем численную сложность модели,
что делает ее подходящей для оптимизационных задач.
и дополнен начальным условием $T(0, x) = T_{0}(x)$.
Здесь $\rho$ - радиационный поток,
а заданная температура на границе обозначена через $u$.

Замечание 1.1. Радиационный поток для полной модели (1.1b) определяется
как $\rho = \int_{S^{d-1}} I d \omega$.
Мы заменили нелинейную функцию $z^{4}$ на $|z|^{3} z$ для обеспечения монотонности.

В [12] вводится и численно исследуется оптимальная краевая задача управления
с функционалами затрат типа слежения, например,

\[
    J(T, u) = \frac{1}{2}\left\|T - T_{d}\right\|_{L^{2}\left(0,1 ; L^{2}(\Omega)\right)}^{2}
    + \frac{\delta}{2}\left\|u - u_{d}\right\|_{H^{1}(0,1 ; \mathbb{R})}^{2}
\]


Решение этой задачи определяет оптимальные параметры
управления температурой для радиационного теплообмена.
где $(T, \rho)$ есть решение (1.2).
Здесь $T_{d} = T_{d}(t, x)$ - заданный температурный профиль,
$u_{d} = u_{d}(t)$ - заданный контроль окружающей температуры,
который должен быть улучшен.
Кроме того, положительная константа $\delta$ позволяет регулировать вес штрафного слагаемого.
Rраевая задача управления

\[
    \begin{aligned}
        & \min J(T, u) \text { с учётом }(T, \rho, u), \\
        & \text { в соответствии с системой }(1.2).
    \end{aligned}
\]


Эта оптимальная задача управления рассматривается как задача условной оптимизации,
а сопряженные переменные используются для построения подходящего численного алгоритма [12].
В этой статье мы предоставляем анализ этого подхода.
Мы доказываем существование оптимального управления $u$ и уникальную разрешимость системы состояний,
что существенно для введения сокращенного функционала затрат.
Затем показывается уникальная разрешимость линеаризованной системы состояний
и определяются сопряженные уравнения.
