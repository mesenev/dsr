\section{Квазистационарная модель сложного теплообмена}\label{sec:ch1/sec4}
Квазистационарная модель - это тип математической модели,
который описывает систему, претерпевающую медленные изменения
со временем или имеющую относительно длительный период стабильности
по сравнению с интересующим масштабом времени.
Такие модели часто используются, когда изучаемая система находится
в равновесии или близка к нему, но при этом может испытывать небольшие,
медленные колебания со временем.
Термин `квази` означает,
что система не совсем стационарна (то есть, фиксирована или неизменна),
а скорее, ее состояние меняется настолько медленно, что его можно считать
почти стационарным для определенных анализов или целей.

В контексте теплообмена, излучения или других физических процессов
квазистационарные модели могут использоваться для описания сценариев,
когда параметры и свойства системы меняются очень медленно по сравнению
с масштабом времени конкретного изучаемого явления.
Такие модели могут упростить анализ и снизить вычислительную сложность,
позволяя исследователям сосредоточиться на основных аспектах проблемы.
