\section{Анализ оптимизационного метода решения
задачи сложного теплообмена с граничными условиями типа
Коши}\label{sec:ch2/sec2}

\subsection{Постановка задачи, не содержащей краевых условий для интенсивности излучения}\label{subsec:ch2/sec2/subsec1}
Рассмотрим систему эллиптических уравнений~\eqref{eq:2_1:initial}
моделирующую стационарный радиационный и диффузиозный теплообмен в
ограниченной области $\Omega \subset \mathbb{R}^3$.
\begin{equation}
    \label{eq:2_2:eq1}
    - a\Delta\theta + b\kappa_a(|\theta|\theta^3- \varphi)=0, \quad
    -\alpha \Delta \varphi
    + \kappa_a(\varphi-|\theta|\theta^3)=0,\; x\in\Omega.
\end{equation}
Будем предполагать, что на границе $\Gamma = \partial \Omega$ известно температурное поле,
\begin{equation}
    \label{eq:2_2:bc1} \theta = \theta_b.
\end{equation}


Для задания краевого условия для интенсивности излучения
требуется знать функцию, описывающую отражающие свойства границы~\cite{JVM-14}.
В том случае, если указанная функция неизвестна, естественно вместо
краевого условия для интенсивности излучения задавать тепловые потоки на границе
\begin{equation}
    \label{eq:2_2:bc2}
    \partial_n\theta = q_b.
\end{equation}
Здесь через $\partial_n$ обозначаем производную в направлении
внешней нормали $\mathbf n$.

Нелокальная разрешимость нестационарной и
стационарной краевых задач для уравнений сложного теплообмена
без краевых условий на интенсивность излучения и
с условиями~\eqref{eq:2_2:bc1},\eqref{eq:2_2:bc2}
для температуры доказана в~\cite{Chebotarev2019Problem,CMMP20}.

Данный раздел посвящен анализу предлагаемого
оптимизационного метода решения краевой задачи~\eqref{eq:2_2:eq1}-\eqref{eq:2_2:bc2}
с условиями типа Коши для температуры.
Указанный метод заключается в рассмотрении задачи
граничного оптимального управления для
системы ~\eqref{eq:2_2:eq1} с ``искусственными`` краевыми условиями
\begin{equation}
    \label{eq:2_2:bc3}
    a(\partial_n\theta+\theta) = r,\;\;
    \alpha(\partial_n\varphi+\varphi) = u \text{ на }\Gamma.
\end{equation}
Функция $r(x),\, x\in\Gamma$ является заданной, а неизвестная функция $u(x),\, x\in\Gamma$
играет роль управления.
Экстремальная задача заключается в отыскании
тройки $\{\theta_\lambda,\varphi_\lambda,u_\lambda\}$
такой, что
\begin{equation}
    \label{eq:2_2:cost}
    J_\lambda(\theta, u) = \frac{1}{2}\int\limits_\Gamma (\theta - \theta_b)^2 d\Gamma
    + \frac{\lambda}{2}\int\limits_\Gamma u^2 d\Gamma \rightarrow\inf
\end{equation}
на решениях краевой задачи~\eqref{eq:2_2:eq1},\eqref{eq:2_2:bc3}.
Функция $\theta_b(x),\, x\in\Gamma$  и параметр регуляризации $\lambda>0$ заданы.

Как будет показано ниже, задача оптимального
управления~\eqref{eq:2_2:eq1}, \eqref{eq:2_2:bc3},\eqref{eq:2_2:cost}, если
$r \coloneqq a(\theta_b+q_b)$, где $q_b$ -- заданная на $\Gamma$ функция,
является при малых $\lambda$ аппроксимацией краевой
задачи~\eqref{eq:2_2:eq1}-\eqref{eq:2_2:bc2}.

\subsection{Формализация задачи управления}\label{subsec:ch2/sec2/subsec2}
В дальнейшем считаем, что $\Omega\subset \mathbb{R}^3$~--- ограниченная строго липшицева
область, граница $\Gamma$ которой состоит из конечного числа
гладких кусков.
Через $L^p$, $1 \leq p \leq \infty$ обозначаем
пространство Лебега, а через $H^s$ -- пространство Соболева $W^s_2$.
Пусть $H = L^2(\Omega), \; V = H^1(\Omega)$, через $V'$ обозначаем
пространство, сопряженное с пространством $V$.
Пространство $H$ отождествляем с пространством $H'$ так, что $V \subset H = H' \subset V'$.
Обозначим через $\|\cdot\|$ стандартную норму в $H$, а через
$(f,v)$ -- значение функционала $f\in V'$ на элементе $v\in V$,
совпадающее со скалярным произведением в $H$, если $f\in H$.
Через $U$ обозначаем пространство $L^2(\Gamma)$ с нормой
$\|u\|_\Gamma=\left(\int_\Gamma u^2 d\Gamma\right)^{1/2}.$



Будем предполагать, что

$(j) \;\; a,b,\alpha,\kappa_a, \lambda ={\textrm Const}> 0,$

$(jj) \;\, \theta_b, \,q_b \in U,\;\; r=a(\theta_b+q_b).$


Определим операторы $A\colon V \to V'$, $B\colon U \to V'$, используя
следующие равенства, справедливые для любых $y,z \in V$, $w\in U$:
\[
    (Ay,z) = (\nabla y, \nabla z) +
    \int\limits_{\Gamma}yz d\Gamma, \;\;\; (Bw, z)
    = \int\limits_{\Gamma}wz d\Gamma.
\]
Билинейная форма $(Ay,z)$ определяет скалярное произведение
в пространстве $V$, а соответствующая норма $\|z\|_V=\sqrt{(Az,z)}$ эквивалентна
стандартной норме $V$.
Поэтому определен непрерывный обратный оператор
$A^{-1}:\,V'\mapsto V.$ Отметим, что для любых
$v\in V$, $w\in U$, $g\in V'$ справедливы неравенства
\begin{equation}
    \label{eq:2_2:e}
    \|v\|^2\leq C_0\|v\|^2_V,\; \|v\|_{V'}\leq C_0\|v\|_V,\; \|Bw\|_{V'}\leq \|w\|_\Gamma,\;
    \|A^{-1}g\|_{V}\leq \|g\|_{V'}.
\end{equation}
Здесь постоянная $C_0>0$ зависит только от области $\Omega.$


Далее используем следующее обозначение
$[h]^s \coloneqq |h|^s \mathrm{sign}\, h$,
$s > 0$, $h \in \mathbb R$ для монотонной степенной функции.


\begin{definition}
    Пара $\theta, \varphi\in V$
    называется \textit{слабым решением} задачи~\eqref{eq:2_2:eq1},\eqref{eq:2_2:bc3}, если
    \begin{equation}
        \label{eq:2_2:w1}
        a A \theta + b \kappa_a ([\theta]^4 - \varphi ) = Br,\quad
        \alpha A \varphi + \kappa_a (\varphi - [\theta]^4)  = Bu.
    \end{equation}
\end{definition}

Для формулировки задачи оптимального управления определим оператор
ограничений $F(\theta, \varphi, u) : V \times V \times U \rightarrow V' \times V'$,
\[
    F(\theta, \varphi, u) = \{ aA\theta + b \kappa_a ( [\theta]^4- \varphi) - Br,\;
    \alpha A \varphi + \kappa_a (\varphi -[\theta]^4) - Bu\}.
\]


\textbf{Задача $(CP)$.} Найти тройку $\{\theta, \varphi, u \} \in V \times V \times U$
такую, что
\begin{equation}
    \label{eq:2_2:cp}
    J_\lambda(\theta, u) \equiv \frac{1}{2}\|\theta -\theta_b\|^2_\Gamma
    + \frac{\lambda}{2}\|u\|^2_\Gamma \rightarrow \inf,\;\; F(\theta, \varphi, u)=0.
\end{equation}

\subsection{Разрешимость задачи $(CP)$}\label{subsec:ch2/sec2/subsec3}


Докажем предварительно однозначную разрешимость краевой задачи~\eqref{eq:2_2:eq1},\eqref{eq:2_2:bc3}.

\begin{lemma}
    \label{lm:2_2:1}
    Пусть выполняются условия (j),(jj), $u\in U$.
    Тогда существует единственное слабое решение
    задачи~\eqref{eq:2_2:eq1},\eqref{eq:2_2:bc3} и при этом
    \begin{equation}
        \label{eq:2_2:e1}
        \begin{aligned}
            a\|\theta\|_V \leq \|r\|_\Gamma + \frac{C_0\kappa_a}{\alpha}\|r+bu\|_\Gamma, \\
            \alpha b \|\varphi\|_V \leq \|r\|_\Gamma +
            \left(\frac{C_0\kappa_a}{\alpha} + 1\right)\|r+bu\|_\Gamma.
        \end{aligned}
    \end{equation}
\end{lemma}

\begin{proof}
    Если второе уравнение в~\eqref{eq:2_2:w1} умножить на $b$ и сложить с первым, то получим равенства
    \begin{gather*}
        A \left( a \theta + \alpha b \varphi \right) = B(r + bu),\;
        a\theta + \alpha b \varphi = A^{-1}B(r + bu),\\
        \varphi = \frac{1}{\alpha b}(A^{-1}B(r +bu) - a\theta).
    \end{gather*}
    Поэтому $\theta \in V$ является решением следующего уравнения:
    \begin{equation}
        \label{eq:2_2:lemma-1-1}
        a A \theta + \frac{\kappa_a}{\alpha} \theta + b\kappa_a [\theta]^4 = g.
    \end{equation}
    Здесь \[ g = Br + \frac{\kappa_a}{\alpha}A^{-1}B(r+bu) \in V'. \]
    Однозначная разрешимость уравнения~\eqref{eq:2_2:lemma-1-1} с монотонной нелинейностью
    хорошо известна (см.\ например~\cite{Kufner}).
    Следовательно задача~\eqref{eq:2_2:w1} однозначно разрешима.

    Для получения оценок~\eqref{eq:2_2:e1} умножим скалярно~\eqref{eq:2_2:lemma-1-1}
    на $\theta \in V$ и отбросим неотрицательные
    слагаемые в левой части.
    Тогда
    \[
        a \|\theta\|^2_V \leq (g, \theta) \leq \|g\|_{V'}\|\theta\|_V,
        \quad a\|\theta\|_V \leq \|g\|_{V'}.
    \]
    Неравенства~\eqref{eq:2_2:e} позволяют оценить $\|g\|_{V'}$ и $\|\varphi\|_V $,
    \[
        \|g\|_{V'} \leq \|r\|_\Gamma + \frac{C_0\kappa_a}{\alpha}\|r + bu\|_\Gamma, \quad
        \|\varphi\|_V \leq \frac{1}{\alpha b} \|r + bu\|_\Gamma + \frac{a}{\alpha b} \|\theta\|_V.
    \]
    В результате получаем оценки~\eqref{eq:2_2:e1}.
\end{proof}

Полученные оценки решения управляемой системы позволяют доказать
разрешимость задачи оптимального управления.

\begin{theorem}
    \label{th:2_2:1}
    Пусть выполняются условия $(j), (jj)$.
    Тогда существует решение задачи $(CP)$.
\end{theorem}

\begin{proof}
    Пусть $j_\lambda = \inf J_\lambda$ на множестве $u \in U$, $F(\theta, \varphi, u)=0.$
    Выберем минимизирующую последовательность
    $u_m \in U, \; \theta_m \in V, \;\varphi_m\in V$,
    \[
        J_\lambda(\theta_m, u_m) \rightarrow j_\lambda,
    \]
    \begin{equation}
        \label{eq:2_2:ms}
        a A \theta_m +b \kappa_a([\theta]^4 - \varphi_m) = Br, \;\;
        \alpha A \varphi_m + \kappa_a (\varphi_m - [\theta]^4) = B u_m.
    \end{equation}
    Из ограниченности последовательности $u_m$ в пространстве $U$ следуют, на основании
    леммы~\ref{lm:2_2:1}, оценки
    \[
        \|\theta_m\|_V \leq C,\;\;
        \|\varphi_m\|_V \leq C,\;\;\|\theta_m\|_{L^6(\Omega)} \leq C.
    \]
    Здесь через $C>0$ обозначена наибольшая из постоянных, ограничивающих
    соответствующие нормы и не зависящих от $m$.
    Переходя при необходимости к подпоследовательностям, заключаем, что
    существует тройка $\{ \hat{u}, \hat{\theta}, \hat{\varphi} \} \in U \times V \times V,$
    \begin{equation}
        \label{eq:2_2:l}
        u_m \rightarrow \hat{u} \text{  слабо в } U, \;\;
        \theta_m, \varphi_m \rightarrow \hat{\theta}, \hat{\varphi} \text{
            слабо в } V, \text{
            сильно в } L^4(\Omega).
    \end{equation}
    Заметим также, что $\forall v \in V$
    \begin{equation}
        \label{eq:2_2:l1}
        |( [\theta_m]^4 - [\hat{\theta}]^4, v)
        \leq 2 \| \theta_m - \hat{\theta}\|_{L^4(\Omega)} \|v\|_{L^4(\Omega)}
        \left( \| \theta_m \|^3_{L_6(\Omega)} + \| \hat{\theta} \|^3_{L_6(\Omega)}\right).
    \end{equation}
    Результаты о сходимости~\eqref{eq:2_2:l},\eqref{eq:2_2:l1} позволяют перейти
    к пределу в~\eqref{eq:2_2:ms}.
    Поэтому
    \[
        a A \hat{\theta} + b \kappa_a ([\hat{\theta}]^4 - \hat{\varphi}) = Br, \;
        \alpha A \hat{\varphi} + \kappa_a (\hat{\varphi} -[\hat{\theta}]^4) = B \hat{u},
    \]
    и при этом $j_\lambda \leq J_\lambda(\hat{\theta},
    \hat{u}) \leq \varliminf J_\lambda(\theta_m, u_m) = j_\lambda$.
    Следовательно, тройка $\{\hat{\theta}, \hat{\varphi}, \hat{u} \}$ есть
    решение задачи $(CP).$
\end{proof}

\subsection{Условия оптимальности}\label{subsec:ch2/sec2/subsec4}


Для получения системы оптимальности достаточно использовать
принцип Лагранжа для гладко-выпуклых экстремальных задач~\cite{11,10}.
Проверим справедливость ключевого условия, что образ производной
оператора ограничений $F(y, u)$, где $y=\{\theta,\varphi\}\in V\times V$,
совпадает с пространством $V'\times V'.$ Именно это условие гарантирует
невырожденность условий оптимальности.
Напомним, что
\[
    F(y, u) = \{ aA\theta + b \kappa_a ( [\theta]^4- \varphi) - Br,\;
    \alpha A \varphi + \kappa_a (\varphi -[\theta]^4) - Bu\}.
\]

\begin{lemma}
    \label{lm:2_2:2}
    Пусть выполняются условия (j),(jj).
    Для любой пары $\hat{y} \in V \times V, \hat{u} \in U$ справедливо равенство
    \[
        \texttt{Im}F_y'(y, u) = V' \times V'.
    \]
\end{lemma}


\begin{proof}
    Достаточно проверить, что задача
    \[
        aA \xi + b \kappa_a (4|\hat{\theta}|^3 \xi - \eta) = f_1, \; \;
        \alpha A \eta + \kappa_a (\eta - 4|\hat{\theta}|^3 \xi) = f_2
    \]
    разрешима для всех $f_{1,2}\in V'.$ Данная задача равносильна системе
    \[
        aA\xi + \kappa_a\left(4b|\theta|^3 + \frac{a}{\alpha}\right) \xi = f_1
        +\frac{\kappa_a}{\alpha}f_3, \; \;
        \eta =\frac{1}{\alpha b}( f_3-a\xi).
    \]
    Здесь $f_3=A^{-1}(f_1+bf_2)\in V.$ Разрешимость первого уравнения указанной
    системы очевидным образом следует из леммы Лакса-Мильграма.


    В соответствии с леммой~\ref{lm:2_2:2}, лагранжиан задачи $(CP)$ имеет вид
    \begin{gather*}
        L(\theta, \varphi, u, p_1, p_2) = J_\lambda(\theta, u)
        + (aA\theta + b\kappa_a([\theta]^4 - \varphi) - Br, p_1) \\
        + (\alpha A \varphi + \kappa_a(\varphi - [\theta]^4) - Bu, p_2).
    \end{gather*}
    Здесь $p=\{p_1,p_2\}\in V\times V$ -- сопряженное состояние.
    Если $\{\hat{\theta}, \hat{\varphi}, \hat{u} \}$ -- решение задачи $(CP)$, то
    в силу принципа Лагранжа~\cite[Гл. 2, теорема 1.5]{10} справедливы вариационные равенства
    $\forall v\in V,\, w\in U$
    \begin{equation}
        \label{eq:2_2:oc1}
        \begin{gathered}
        (\hat{\theta} -\theta_b, v)
            _\Gamma + (aAv + 4 b\kappa_a |\hat{\theta}|^3 v, p_1)
            - \kappa_a ( 4 |\hat{\theta}|^3 v, p_2) = 0,\\
            b \kappa_a (v, p_1)+ (\alpha A v + \kappa_a v, p_2) = 0,
        \end{gathered}
    \end{equation}
    \begin{equation}
        \label{eq:2_2:oc2}
        \lambda(\hat{u},w)_\Gamma - (Bw, p_2) = 0.
    \end{equation}
    Таким образом, из условий~\eqref{eq:2_2:oc1},\eqref{eq:2_2:oc2}
    получаем следующий результат, который вместе с уравнениями~\eqref{eq:2_2:w1}
    для оптимальной тройки определяет систему оптимальности задачи $(CP)$.
\end{proof}

\begin{theorem}
    \label{th:2_2:2}
    Пусть выполняются условия (j),(jj).
    Если $\{\hat{\theta}, \hat{\varphi}, \hat{u}\}$ -- решение задачи $(CP)$,
    то существует единственная пара $\{p_1, p_2 \} \in V\times V$ такая, что
    \begin{equation}
        \label{eq:2_2:as}
        aAp_1 +4|\hat{\theta}|^3 \kappa_a(bp_1 - p_2) = B(\theta_b - \hat{\theta}), \;\;
        \alpha A p_2 + \kappa_a (p_2 - b p_1)=0
    \end{equation}
    и при этом $\lambda\hat{u} = p_2$.
\end{theorem}

Полученные условия оптимальности используются в главе~\ref{sec:ch4/sec4} при разработке алгоритма
нахождения квазирешения.

\subsection{Аппроксимация задачи с условиями типа Коши}
\label{subsec:ch2/sec2/approximation}

Рассмотрим краевую задачу~\eqref{eq:2_2:eq1}--\eqref{eq:2_2:bc2} для
уравнений сложного теплообмена, в которой нет краевых условий
на интенсивность излучения.
Существование $\theta,\varphi\in H^2(\Omega)$,
удовлетворяющих~\eqref{eq:2_2:eq1}--\eqref{eq:2_2:bc2} для достаточно гладких
$\theta_b,\, q_b$ и достаточные условия единственности решения
установлены в~\cite{CMMP20}.
Покажем, что решения задачи $(CP)$ при $\lambda\to+0$
аппроксимируют решение задачи~\eqref{eq:2_2:eq1}-\eqref{eq:2_2:bc2}.


\begin{theorem}
    \label{th:2_2:3}
    Пусть выполняются условия (j),(jj) и существует решение
    задачи~\eqref{eq:2_2:eq1}--\eqref{eq:2_2:bc2}.
    Если $\{\theta_\lambda,\varphi_\lambda,u_\lambda\}$ -- решение
    задачи $(CP)$ для $\lambda>0$, то существует последовательность $\lambda\to +0$
    такая, что
    \[
        \theta_\lambda\rightarrow\theta_*, \;\; \varphi_\lambda\rightarrow\varphi_*
        \text{ слабо в }V,\text{ сильно в }H,
    \]
    где $\theta_*,\varphi_*$ -- решение задачи~\eqref{eq:2_2:eq1}-\eqref{eq:2_2:bc2}.
\end{theorem}

\begin{proof}
    Пусть $\theta,\varphi\in H^2(\Omega)$ -- решение
    задачи~\eqref{eq:2_2:eq1}--\eqref{eq:2_2:bc2},
    $u=\alpha(\partial_n\varphi + \varphi)\in U$.
    Тогда
    \[
        a A \theta + b \kappa_a ([\theta]^4 - \varphi ) = Br,\quad
        \alpha A \varphi + \kappa_a (\varphi - [\theta]^4) = Bu,
    \]
    где $r \coloneqq a(\theta_b+q_b)$.
    Поэтому, с учетом того, что $\theta|_\Gamma=\theta_b$,
    \[
        J_\lambda(\theta_\lambda, u_\lambda)
        = \frac{1}{2}\|\theta_\lambda -\theta_b\|^2_\Gamma
        + \frac{\lambda}{2}\|u_\lambda\|^2_\Gamma
        \leq J_\lambda(\theta, u) = \frac{\lambda}{2}\|u\|^2_\Gamma.
    \]
    Следовательно,
    \[
        \|u_\lambda\|^2_\Gamma\leq C,\;\; \|\theta_\lambda
        -\theta_b\|^2_\Gamma\to 0,\; \lambda\to + 0.
    \]
    Здесь и далее $C>0$ не зависит от $\lambda.$
    Из ограниченности последовательности $u_\lambda$ в пространстве $U$ следуют, на основании
    леммы~\ref{lm:2_4:1}, оценки
    \[
        \|\theta_\lambda\|_V \leq C,\;\;
        \|\varphi\|_\lambda \leq C.
    \]
    Поэтому можно выбрать последовательность $\lambda\to+0$ такую, что
    \begin{equation}
        \label{eq:2_2:LL}
        u_\lambda \rightarrow u_* \text{  слабо в } U, \;\;
        \theta_\lambda, \varphi_\lambda \rightarrow \theta_*,\varphi_*
        \text{ слабо в } V, \text{ сильно в } L^4(\Omega).
    \end{equation}
    Результаты~\eqref{eq:2_2:LL} позволяют перейти к пределу при $\lambda\to+0$
    в уравнениях для $\theta_\lambda,\varphi_\lambda,u_\lambda$ и тогда
    \begin{equation}
        \label{eq:2_2:CC}
        a A \theta_* + b \kappa_a ([\theta_*]^4 - \varphi_* ) = Br,\quad
        \alpha A \varphi_* + \kappa_a (\varphi_* - [\theta_*]^4)  = Bu_*.
    \end{equation}
    При этом $\theta_*|_\Gamma=\theta_b$.
    Из первого уравнения в~\eqref{eq:2_2:CC}, с учетом, что $r = a(\theta_b + q_b)$,
    выводим
    \[
        - a\Delta\theta_* + b\kappa_a([\theta_*]^4- \varphi_*) = 0 \text{ п.\ в. в }\Omega,
        \quad \theta_*=\theta_b,\quad \partial_n\theta = q_b \text{ п.\ в. на }\Gamma.
    \]
    Из второго уравнения в~\eqref{eq:2_2:CC} следует, что
    $-\alpha \Delta \varphi + \kappa_a(\varphi-[\theta]^4) = 0$ почти всюду в $\Omega$.
    Таким образом, пара $\theta_*,\varphi_*$
    -- решение задачи~\eqref{eq:2_2:eq1}--\eqref{eq:2_2:bc2}.
\end{proof}

\begin{remark}
    Из ограниченности последовательности $u_\lambda$
    в пространстве $U$ следует
    ее слабая относительная компактность и существование последовательности
    (возможно не единственной) $\lambda\to+0$ такой, что
    $u_\lambda \rightarrow u_*$ слабо в $U$.
\end{remark}

Для практического решения задачи~\eqref{eq:2_2:eq1}-\eqref{eq:2_2:bc2} важно то,
что \textit{для любой последовательности} $\lambda\to+0$ справедлива оценка
$\|\theta_\lambda -\theta_b\|^2_\Gamma\leq C\lambda$,
а поскольку $\partial_n\theta_\lambda=\theta_b+q_b-\theta_\lambda$,
то также $\|\partial_n\theta_\lambda-q_b\|^2_\Gamma\leq C\lambda$.
Указанные неравенства гарантируют, что граничные значения
$\theta_\lambda,\,\partial_n\theta_\lambda$ при малых $\lambda$
аппроксимируют краевые условия задачи~\eqref{eq:2_2:eq1}-\eqref{eq:2_2:bc2}.

В главе~\ref{subsec:ch4/sec4/subsec1} рассмотрен численный метод решения задачи
и приведены численные примеры, демонстрирующие его эффективность.
