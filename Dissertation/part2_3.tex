\section{Анализ оптимизационного метода для квазистационарной модели}
\label{sec:ch2/sec3}
%paper03
В данном параграфе рассмотрен нестационарный аналог задачи
с данными Коши для температуры на границе области.

\subsection{Постановка задачи оптимального управления}
\label{subsec:ch2/sec3/state}
Квазистационарный радиационный и диффузионный теплообмен в ограниченной области
$\Omega \subset \mathbb{R}^{3}$ с границей $\Gamma=\partial \Omega$ моделируем
следующей начально-краевой задачей:
\begin{equation}
    \label{eq:2_3:1}
    \begin{split}
        & \frac{\partial \theta}{\partial t} - a \Delta \theta
        + b \kappa_{a} \left(|\theta| \theta^{3}-\varphi\right) = 0,\\
        & - \alpha \Delta \varphi
        + \kappa_{a} \left(\varphi-|\theta| \theta^{3}\right) = 0,
        \quad x \in \Omega, \quad 0 < t < T;
    \end{split}
\end{equation}
\begin{align}
    a \left(\partial_{n} \theta+\theta\right)=r,
    & \quad \alpha\left(\partial_{n} \varphi
    + \varphi\right) = u \text { на } \Gamma;  \label{eq:2_3:2}\\
    & \left.\theta\right|_{t=0} = \theta_{0}. \label{eq:2_3:3}
\end{align}

Здесь $\theta$ — нормализованная температура,
$\varphi$ — нормализованная интенсивность излучения,
усредненная по всем направлениям.
Положительные параметры $a, b, \kappa_{a}$ и $\alpha$,
описывающие свойства среды,
определяются стандартным образом~\cite{Kovtanyuk2015}.
Функция $r(x, t), x \in \Gamma, t \in(0, T)$,
а неизвестная функция $u(x, t), x \in \Gamma, t \in$ $(0, T)$ — управление.


Экстремальная задача состоит в том, чтобы найти тройку
$\left\{\theta_{\lambda}, \varphi_{\lambda}, u_{\lambda}\right\}$ такую, что
\begin{equation}
    \label{eq:2_3:4}
    J_{\lambda}(\theta, u)=\frac{1}{2} \int_{0}^{T}
    \int_{\Gamma}\left(\theta-\theta_{b}\right)^{2} d \Gamma d t+\frac{\lambda}{2}
    \int_{0}^{T} \int_{\Gamma} u^{2} d \Gamma d t \rightarrow \inf
\end{equation}
на решениях задачи~\eqref{eq:2_3:1}--\eqref{eq:2_3:3}.
Функция $\theta_{b}(x, t), x \in \Gamma, t \in(0, T)$,
а также регуляризирующий параметр $\lambda>0$ являются заданными.

Задача оптимального управления~\eqref{eq:2_3:1}--\eqref{eq:2_3:4},
если $r \coloneqq a\left(\theta_{b}+q_{b}\right)$,
где $q_{b}$ — заданная функция на $\Sigma =\Gamma \times(0, T)$,
является при малых значениях $\lambda$ аппроксимацией краевой
задачи для уравнения~\eqref{eq:2_3:1},
для которого неизвестны граничные условия для интенсивности излучения $\varphi$.
Вместо них задаются граничные температура и нормальная производная
\begin{equation}
    \label{eq:2_3:5}
    \left.\theta\right|_{\Gamma}=\theta_{b},
    \left.\quad \partial_{n} \theta\right|_{\Gamma}=q_{b}.
\end{equation}

Основные результаты раздела заключаются в использовании априорных
оценок решения задачи~\eqref{eq:2_3:1},~\eqref{eq:2_3:2} полученных
в разделе~\ref{sec:ch1/sec4},
на основании которых доказывается разрешимость задачи
оптимального управления~\eqref{eq:2_3:1}--\eqref{eq:2_3:4},
и выводится система оптимальности.
Показано, что последовательность
$\left\{\theta_{\lambda}, \varphi_{\lambda}, u_{\lambda}\right\}$
решений экстремальной задачи~\eqref{eq:2_3:1}--\eqref{eq:2_3:4},
при $ \lambda \rightarrow+0$
сходится к решению начально-краевой задачи~\eqref{eq:2_3:1},~\eqref{eq:2_3:5},
с условиями типа Коши для температуры.

\subsection{Формализация задачи управления}\label{subsec:ch2/sec3/formalization}
В дальнейшем, как и ранее предполагается,
что $\Omega \subset \mathbb{R}^{3}$ — ограниченная строго липшицева область,
граница которой $\Gamma$ состоит из конечного числа гладких кусков.
Через $L^{s}, 1 \leq s \leq$ $\infty$
обозначается пространство Лебега, а через $H^{s}$
пространство Соболева $W_{2}^{s}$.
Пусть $H=$ $L^{2}(\Omega), V=H^{1}(\Omega)$.
Анализ начально-краевой задачи проведён в параграфе~\ref{sec:ch1/sec4}.
Для удобства напомним основные условия и определения.

Через $U$ обозначим пространство $L^{2}(\Sigma)$ с нормой
$ \|u\|_{\Sigma}=\left(\int_{\Sigma} u^{2} d \Gamma d t\right)^{1 / 2}$.
Также будем использовать пространство
\[
    W=\left\{y \in L^{2}(0, T ; V):y^{\prime}
    \in L^{2}\left(0, T, V^{\prime}\right)\right\}
\]
где $y^{\prime}=d y / d t$.

Будем считать, что

$(k)\; a, b, \alpha, \kappa_{a}, \lambda=$ Const $>0$,

$(kk)\; \theta_{b}, q_{b} \in U, r=a\left(\theta_{b}+q_{b}\right)
\in L^{5}(\Sigma), \; \theta_{0} \in L^{5}(\Omega)$.


Определим операторы $A: V \rightarrow V^{\prime}, B: U \rightarrow V^{\prime}$,
используя следующие равенства, справедливые для любых
$y, z \in V, w \in L^{2}(\Gamma)$:
\[
    (A y, z)=(\nabla y, \nabla z)+\int_{\Gamma} y z d \Gamma,
    \quad(B w, z)=\int_{\Gamma} w z d \Gamma.
\]
\begin{definition}
    Пара $\theta \in W, \varphi \in L^{2}(0, T ; V)$ называется слабым
    решением задачи~\eqref{eq:2_3:1}--\eqref{eq:2_3:3}, если
    \begin{equation}
        \label{eq:2_3:6}
        \theta^{\prime}+a A \theta+b \kappa_{a}\left([\theta]^{4}-\varphi\right)=B r,
        \quad \theta(0)=\theta_{0}, \quad \alpha A \varphi
        + \kappa_{a}\left(\varphi-[\theta]^{4}\right)=B u.
    \end{equation}
\end{definition}

Для формулировки задачи оптимального управления определим оператор ограничения
$F(\theta, \varphi, u): W \times L^{2}(0, T; V) \times U
\rightarrow L^{2} \left(0, T, V^{\prime}\right)
\times L^{2}\left(0, T, V^{\prime}\right) \times H$ таким, что
\[
    \begin{aligned}
        F(\theta, \varphi, u)= & \{\theta^{\prime}+a A
        \theta+b \kappa_{a}\left([\theta]^{4}-\varphi\right)-B r, \\
        & \alpha A \varphi+\kappa_{a}\left(\varphi-[\theta]^{4}\right)-B u,
        \theta(0)-\theta_{0}\}.
    \end{aligned}
\]

Таким образом, задача оптимального управления $OC$ заключается в отыскании тройки
$\{\theta, \varphi, u\} \in W \times L^{2}(0, T ; V) \times U$ такой, что
\[
    J_{\lambda}(\theta, u) \equiv \frac{1}{2}\left\|\theta-
    \theta_{b}\right\|_{\Sigma}^{2}+
    \frac{\lambda}{2}\|u\|_{\Sigma}^{2}
    \rightarrow \inf, \quad F(\theta, \varphi, u)=0.
\]

\subsection{Разрешимость задачи $OC$}
\label{subsec:ch2/sec3/subsec3}

\begin{theorem}
    \label{th:2_3:1}
    Пусть выполняются условия $(k), (kk)$.
    Тогда существует решение задачи $OC$.
\end{theorem}


\begin{proof}
    Пусть $j_{\lambda}=\inf \{ J_{\lambda} : u \in U, F(\theta, \varphi, u)=0\}$.
    Выберем минимизирующую последовательность
    $u_{m} \in U, \theta_{m} \in W, \varphi_{m} \in L^{2}(0, T; V),
    J_{\lambda}\left(\theta_{m}, u_{m}\right) \rightarrow j_{\lambda}$,
    \begin{equation}
        \label{eq:2_3:13}
        \begin{gathered}
            \theta_{m}^{\prime}+a A \theta_{m}
            + b \kappa_{a}\left(\left[\theta_{m}\right]^{4}
            - \varphi_{m}\right)=B r, \theta_{m}(0)=\theta_{0}, \\
            \alpha A \varphi_{m}+\kappa_{a}\left(\varphi_{m}-
            \left[\theta_{m}\right]^{4}\right)=B u_{m}.
        \end{gathered}
    \end{equation}

    В силу леммы~\ref{lm:1_5:1}, ограниченность последовательности $u_{m}$
    в пространстве $U$ влечет оценки
    \[
        \begin{gathered}
            \left\|\theta_{m}\right\|_{L^{2}(0, T ; V)} \leq C,
            \quad\left\|\theta_{m}\right\|_{L^{\infty}\left(0, T; L^{5}(\Omega)\right)} \leq C,
            \left\|\theta_{m}^{\prime}\right\|_{L^{2}\left(0, T; V^{\prime}\right)} \leq C, \\
            \int_{0}^{T} \int_{\Omega}\left|\theta_{m}\right|^{8} d x d t \leq C,
            \quad\left\|\varphi_{m}\right\|_{L^{2}(0, T ; V)} \leq C.
        \end{gathered}
    \]

    Здесь $C>0$ обозначает наибольшую из констант,
    ограничивающих соответствующие нормы и не зависящих от $m$.
    Переходя, если необходимо, к подпоследовательностям, заключаем, что существует тройка
    $\{\widehat{u}, \widehat{\theta}, \widehat{\varphi}\}
    \in U \times W \times L^{2}(0, T; V)$,
    \[
        \begin{gathered}
            u_{m} \rightarrow \widehat{u} \text { слабо в } U \\
            \theta_{m} \rightarrow \widehat{\theta}
            \text{ слабо в } L^{2}(0, T; V), \text { сильно в } L^{2}(Q), \\
            \varphi_{m} \rightarrow \widehat{\varphi}
            \text{ слабо в }  L^{2}(0, T ; V).
        \end{gathered}
    \]
    Более того, $\widehat{\theta} \in L^{8}(Q) \cap L^{\infty}\left(0, T ; L^{5}(\Omega)\right)$.


    Результаты о сходимости позволяют перейти к пределу в~\eqref{eq:2_3:13}.
    В этом случае предельный переход в
    нелинейной части следует из следующего неравенства,
    справедливого при $\xi \in C^{\infty}(\bar{Q})$:
    \[
        \begin{aligned}
            & \int_{0}^{T}\left|\left(\left[\theta_{m}\right]^{4}
            -[\widehat{\theta}]^{4}, \xi\right)\right| d t \leq \\
            & \quad 2 \max _{\bar{Q}}|\xi|
            \left(\left\|\theta_{m}\right\|_{L^{5}(\Omega)}^{5 / 3}\left\|
            \theta_{m}\right\|_{L^{8}(\Omega)}^{4 / 3}
            + \|\widehat{\theta}\|_{L^{5}(\Omega)}^{5 / 3}
            \|\widehat{\theta}\|_{L^{8}(\Omega)}^{4 / 3}\right)\left\|\theta_{m}
            - \widehat{\theta}\right\|_{L^{2}(Q)}.
        \end{aligned}
    \]


    В результате получаем, что
    \[
        \widehat{\theta}^{\prime}+a A \widehat{\theta}+b
        \kappa_{a}\left([\widehat{\theta}]^{4}-\widehat{\varphi}\right)=B r,
        \quad \widehat{\theta}(0)=\theta_{0},
        \quad \alpha A \widehat{\varphi}+\kappa_{a}\left(\widehat{\varphi}-
        [\widehat{\theta}]^{4}\right)=B \widehat{u},
    \]
    где
    \[
        j_{\lambda} \leq J_{\lambda}(\widehat{\theta},
        \widehat{u}) \leq \underline{\lim }
        J_{\lambda}\left(\theta_{m}, u_{m}\right)=j_{\lambda}.
    \]

    Таким образом, тройка $\{\widehat{\theta}, \widehat{\varphi}, \widehat{u}\}$
    есть решение задачи $OC$.
\end{proof}

\subsection{Условия оптимальности}\label{subsec:ch2/sec3/subsec4}
Для получения системы оптимальности достаточно использовать принцип Лагранжа
для гладко-выпуклых экстремальных задач~\cite{10, 11}.
Проверим выполнение ключевого условия,
что образ производной оператора связи $F_{y}^{\prime}(y, u)$,
где $y=\{\theta, \varphi\} \text{ в }  W \times L^{2}(0, T ; V)$,
совпадает с пространством
$L^{2}\left(0, T; V^{\prime}\right) \times L^{2} \left(0, T ; V^{\prime}\right) \times H$.

Напомним, что
\begin{equation*}
    \begin{aligned}
        F(\theta, \varphi, u) &=\{\theta^{\prime}+
        a A \theta+b \kappa_{a}\left([\theta]^{4}-\varphi\right)-B r, \\
        &\alpha A \varphi+\kappa_{a}\left(\varphi-[\theta]^{4}\right)-B u,
        \theta(0)-\theta_{0}\}.
    \end{aligned}
\end{equation*}
\begin{lemma}
    \label{lm:2_3:2}
    Пусть выполнены условия $(k), (kk)$.
    Если $\widehat{y} \in W \times L^{2}(0, T ; V), \widehat{u} \in U$
    является решением задачи $OC$, то справедливо равенство:
    \[
        \operatorname{Im} F_{y}^{\prime}
        (\widehat{y}, \widehat{u})=L^{2}\left(0, T; V^{\prime}\right)
        \times L^{2}\left(0, T; V^{\prime}\right) \times H.
    \]
\end{lemma}

\begin{proof}
    Достаточно проверить, что задача
    \[
        \xi^{\prime}+a A \xi+b \kappa_{a}\left(4|\widehat{\theta}|^{3}
        \xi-\eta\right)=f_{1}, \quad \xi(0)=\xi_{0},
        \quad \alpha A \eta+\kappa_{a}\left(\eta-4|\widehat{\theta}|^{3} \xi\right)=f_{2}
    \]
    разрешима для всех $f_{1,2} \in L^{2}\left(0, T; V^{\prime}\right), \xi_{0} \in H$.
    Выразим $\eta$ из последнего уравнения и подставим его в первое.
    В результате получаем следующую задачу:
    \begin{equation}
        \label{eq:2_3:14}
        \xi^{\prime}+a A \xi+4 L\left(|\widehat{\theta}|^{3}
        \xi\right)=f_{1}+b \kappa_{a}\left(\alpha A+\kappa_{a}
        I\right)^{-1} f_{2}, \xi(0)=\xi_{0}.
    \end{equation}
    Однозначная разрешимость линейной задачи~\eqref{eq:2_3:14}
    доказывается аналогично лемме~\ref{lm:1_5:1}.
\end{proof}


Согласно лемме~\ref{lm:2_3:2} лагранжиан задачи $OC$ имеет вид
\[
    \begin{aligned}
        & L\left(\theta, \varphi, u, p_{1}, p_{2}, q\right)=
        J_{\lambda}(\theta, u) +\int_{0}^{T}\left(\theta^{\prime}
        + a A \theta+b \kappa_{a}\left([\theta]^{4}-\varphi\right)
        - B r, p_{1}\right) d t \\
        & + \int_{0}^{T}\left(\alpha A \varphi
        + \kappa_{a}\left(\varphi-[\theta]^{4}\right)
        -B u, p_{2}\right) d t+\left(q, \theta(0)-\theta_{0}\right).
    \end{aligned}
\]

Здесь $p=\left\{p_{1}, p_{2}\right\} \in L^{2}(0, T; V) \times L^{2}(0, T; V) $
— сопряженное состояние, $q \in H$ — множитель Лагранжа для начального условия.
Если $\{\widehat{\theta}, \widehat{\varphi}, \widehat{u}\}$
является решением задачи $OC$, то в силу принципа
Лагранжа~\cite[гл. 2, теорема 1.5]{10} выполняются вариационные равенства
$\forall \zeta \in L^{2}(0, T ; V), v \in U$:
\[
    \begin{gathered}
        \int_{0}^{T}\left(\left(B\left(\widehat{\theta}-\theta_{b}\right),
        \zeta\right)+\left(\zeta^{\prime}
        +a A \zeta+4 b \kappa_{a}|\widehat{\theta}|^{3} \zeta, p_{1}\right)
        - \kappa_{a}\left(4|\widehat{\theta}|^{3} \zeta,
        p_{2}\right)\right) d t \\
        +(q, \zeta(0))=0, \\
        \int_{0}^{T}\left(\left(\alpha A \zeta
        +\kappa_{a} \zeta, p_{2}\right)
        -b \kappa_{a}\left(\zeta, p_{1}\right)\right) d t=0,
        \int_{0}^{T}\left(\lambda(\widehat{u}, v)_{\Gamma}
        -\left(B v, p_{2}\right)\right) d t=0.
    \end{gathered}
\]

Таким образом, из полученных условий получаем следующий результат.

\begin{theorem}
    \label{th:2_3:2}
    Пусть выполнены условия $(k), (kk)$.
    Если $\{\widehat{\theta}, \widehat{\varphi}, \widehat{u}\}$ — решение задачи $OC$,
    то существует единственная пара $\left\{p_ {1}, p_{2}\right\} \in W \times W$ такая, что
    \begin{equation}
        \label{eq:2_3:15}
        \begin{aligned}
            -p_{1}^{\prime}+a A p_{1}+4|\widehat{\theta}|^{3} \kappa_{a}\left(b p_{1}
            -p_{2}\right)&=B\left(\theta_{b}-\widehat{\theta}\right),
            p_{1}(T)=0, \\
            \alpha A p_{2}+\kappa_{a}\left(p_{2}-b p_{1}\right)&=0,
        \end{aligned}
    \end{equation}
    а также $\lambda \widehat{u}=\left.p_{2}\right|_{\Sigma}$.
\end{theorem}

\subsection{Аппроксимация задачи с граничными условиями типа Коши}
\label{subsec:ch2/sec3/approximation}

Рассмотрим начально-краевую задачу для уравнений сложного
теплообмена, в которой отсутствуют граничные условия на интенсивность излучения
\begin{equation}
    \label{eq:2_3:16}
    \frac{\partial \theta}{\partial t}-a \Delta \theta
    + b \kappa_{a}\left([\theta]^{4}
    - \varphi\right)=0, \quad-\alpha \Delta \varphi
    + \kappa_{a}\left(\varphi-[\theta]^{4}\right) = 0,
    \quad(x, t) \in Q,
\end{equation}
\begin{equation}
    \label{eq:2_3:17}
    \theta=\theta_{b}, \quad \partial_{n} \theta= q_{b} \text { на } \Sigma,
    \left.\quad \theta\right|_{t=0} = \theta_{0}.
\end{equation}


Существование и единственность функций $\theta \in L^{2}\left(0, T; H^{2}(\Omega)\right),
\varphi, \Delta \varphi \in L^{2}( Q)$, удовлетворяющие~\eqref{eq:2_3:16},~\eqref{eq:2_3:17}
для достаточно гладких $\theta_{b}, q_{b}$, доказаны в~\cite{Chebotarev2019Problem}.
Покажем, что решения задачи $OC$ при $\lambda \rightarrow+0$
аппроксимируют решение задачи~\eqref{eq:2_3:16},~\eqref{eq:2_3:17}.
\begin{theorem}
    \label{th:2_3:3}
    Пусть выполняются условия $(k), (kk)$ и существует решение
    $\theta, \varphi \in$ $L^{2}\left(0, T ; H^{2}(\Omega) \right)$
    задачи~\eqref{eq:2_3:16},~\eqref{eq:2_3:17}.
    Если $\left\{\theta_{\lambda}, \varphi_{\lambda}, u_{\lambda}\right\}$
    — решение задачи $OC$ при $\lambda>0$, то при $\lambda\rightarrow+0$
    \[
        \begin{gathered}
            \theta_{\lambda} \rightarrow \theta \text { слабо в } L^{2}(0, T ; V),
            \text { сильно в } L^{2}(Q), \\
            \varphi_{\lambda} \rightarrow \varphi \text { слабо в } L^{2}(0, T ; V).
        \end{gathered}
    \]
\end{theorem}

\begin{proof}
    Пусть $\theta, \varphi \in L^{2}\left(0, T ; H^{2}(\Omega)\right)$
    — решение задачи~\eqref{eq:2_3:16},~\eqref{eq:2_3:17},
    $u=\alpha\left(\partial_{n} \varphi+\varphi\right) \in U$.
    Тогда
    \[
        \theta^{\prime}+a A \theta+b \kappa_{a}\left([\theta]^{4}-\varphi\right)=B r,
        \quad \theta(0)=\theta_{0}, \quad \alpha A \varphi
        + \kappa_{a}\left(\varphi-[\theta]^{4}\right)=B u,
    \]
    где $r\coloneqq a\left(\theta_{b}+q_{b}\right)$.
    Следовательно, принимая во внимание,
    что $\left.\theta\right|_{\Gamma}=\theta_{b}$,
    \[
        J_{\lambda}\left(\theta_{\lambda},
        u_{\lambda}\right)=\frac{1}{2}\left\|\theta_{\lambda}-\theta_{b}\right\|_{\Sigma}^{2}
        +\frac{\lambda}{2}\left\|u_{\lambda}\right\|_{\Sigma}^{2}
        \leq J_{\lambda}(\theta, u)=\frac{\lambda}{2}\|u\|_{\Sigma}^{2}.
    \]
    Таким образом,
    \[
        \left\|u_{\lambda}\right\|_{\Sigma}^{2} \leq C, \quad\left\|\theta_{\lambda}
        -\theta_{b}\right\|_{\Sigma}^{2} \rightarrow 0, \lambda \rightarrow+0.
    \]


    Здесь и далее $C>0$ не зависит от $\lambda$.
    Ограниченность последовательности $u_{\lambda}$
    в пространстве $U$ влечет по лемме~\ref{lm:1_5:1} оценки:
    \[
        \begin{gathered}
            \left\|\theta_{\lambda}\right\|_{L^{2}(0, T ; V)} \leq C,
            \quad\left\|\theta_{\lambda}\right\|_{L^{\infty}
            \left(0, T ; L^{5}(\Omega)\right)} \leq C,
            \quad\left\|\theta_{\lambda}^{\prime}\right\|_{L^{2}
            \left(0, T ; V^{\prime}\right)} \leq C, \\
            \int_{0}^{T} \int_{\Omega}
            \left|\theta_{\lambda}\right|^{8} d x d t \leq C,
            \quad\left\|\varphi_{\lambda}\right\|_{L^{2}(0, T ; V)} \leq C.
        \end{gathered}
    \]
    Следовательно, можно выбрать последовательность $\lambda\rightarrow+0$ такую, что
    \[
        \begin{gathered}
            u_{\lambda} \rightarrow u_{*} \text { слабо в } U \\
            \theta_{\lambda} \rightarrow \theta_{*}
            \text { слабо в } L^{2}(0, T; V) \text { сильно в } L^{2}(Q), \\
            \varphi_{\lambda} \rightarrow \varphi_{*}
            \text { слабо в } L^{2}(0, T; V).
        \end{gathered}
    \]

    Полученные результаты о сходимости позволяют, как и в теореме~\ref{th:2_3:1},
    перейти к пределу при $\lambda \rightarrow+0$ в уравнениях для $\theta_{\lambda},
    \varphi_{\lambda}, u_{\lambda}$.
    Тогда
    \begin{equation}
        \label{eq:2_3:18}
        \begin{aligned}
            \theta_{*}^{\prime}+a A \theta_{*}
            + b \kappa_{a}\left(\left[\theta_{*}\right]^{4}
            - \varphi_{*}\right)&=B r, \quad \theta_{*}(0)=\theta_{0}, \\
            \alpha A \varphi_{*}+\kappa_{a}\left(\varphi_{*}
            - \left[\theta_{*}\right]^{4}\right)&=B u_{*}.
        \end{aligned}
    \end{equation}
    При этом $\left.\theta_{*}\right|_{\Gamma}=\theta_{b}$.
    Из первого уравнения в~\eqref{eq:2_3:18}, учитывая,
    что $r=a\left(\theta_{b}+q_{b}\right)$, получаем
    $
    \frac{\partial \theta_{*}}{\partial t}-a \Delta \theta_{*}
    +b \kappa_{a}\left(\left[\theta_{*}\right]^{4}-\varphi_{*}\right)=0$
    $\text { почти всюду в } Q, \quad \theta_{*}=\theta_{b},
    \quad \partial_{n} \theta=q_{b} \text { почти всюду в } \Sigma.
    $

    Из второго уравнения в~\eqref{eq:2_3:18} следует, что
    $-\alpha \Delta \varphi+\kappa_{a}\left(\varphi-[\theta]^{4}\right)=0$ почти всюду в $Q$.
    Таким образом, пара $\theta_{*}, \varphi_{*}$
    является решением задачи~\eqref{eq:2_3:16},~\eqref{eq:2_3:17}.
    Поскольку решение этой задачи единственно~\cite{Chebotarev2019Problem},
    то $\theta_{*}=\theta, \varphi_{*}=\varphi$.
\end{proof}
