\section{Анализ оптимизационного метода для квазистационарной модели}
\label{sec:ch2/sec3}
%paper03

\subsection{Постановка задачи оптимального управления}
\label{subsec:ch2/sec3/state}
Квазистационарный радиационный и диффузионный теплообмен в ограниченной области
$\Omega \subset \mathbb{R}^{3}$ с границей $\Gamma=\partial \Omega$ моделируем
со следующей начально-краевой задачей:

\begin{equation}
    \label{eq:2_3:1}
    \begin{split}
        & \frac{\partial \theta}{\partial t} - a \Delta \theta
        + b \kappa_{a} \left(|\theta| \theta^{3}-\varphi\right) = 0,\\
        & - \alpha \Delta \varphi
        + \kappa_{a} \left(\varphi-|\theta| \theta^{3}\right) = 0,
        \quad x \in \Omega, \quad 0 < t < T;
    \end{split}
\end{equation}
\begin{align}
    a \left(\partial_{n} \theta+\theta\right)=r,
    & \quad \alpha\left(\partial_{n} \varphi
    + \varphi\right) = u \text { на } \Gamma;  \label{eq:2_3:2}\\
    & \left.\theta\right|_{t=0} = \theta_{0}. \label{eq:2_3:3}
\end{align}

Здесь $\theta$ — нормированная температура,
$\varphi$ — нормированная интенсивность излучения,
усредненная по всем направлениям.
Положительные параметры $a, b, \kappa_{a}$ и $\alpha$,
описывающие свойства среды,
определяются стандартным образом~\cite{Kovtanyuk2015}.
Функция $r(x, t), x \in \Gamma, t \in(0, T)$,
а неизвестная функция $u(x, t), x \in \Gamma, t \in$ $(0, T)$ — управление.


Экстремальная задача состоит в том, чтобы найти тройку
$\left\{\theta_{\lambda}, \varphi_{\lambda}, u_{\lambda}\right\}$ такую, что

\begin{equation}
    \label{eq:2_3:4}
    J_{\lambda}(\theta, u)=\frac{1}{2} \int_{0}^{T}
    \int_{\Gamma}\left(\theta-\theta_{b}\right)^{2} d \Gamma d t+\frac{\lambda}{2}
    \int_{0}^{T} \int_{\Gamma} u^{2} d \Gamma d t \rightarrow \inf
\end{equation}
на решениях задачи~\eqref{eq:2_3:1}--\eqref{eq:2_3:3}.

Функция $\theta_{b}(x, t), x \in \Gamma, t \in(0, T)$,
а также регуляризирующий параметр $\lambda>0$ являются заданными.

Задача оптимального управления~\eqref{eq:2_3:1}--\eqref{eq:2_3:4},
если $r \coloneqq a\left(\theta_{b}+q_{b}\right)$,
где $q_{b}$ — заданная функция на $\Sigma =\Gamma \times(0, T)$,
является при малых значениях $\lambda$ аппроксимацией краевой
задачи для уравнения~\eqref{eq:2_3:1},
для которого неизвестны граничные условия для интенсивности излучения $\varphi$.
Вместо них задаются граничные температура и нормальная производная
\begin{equation}
    \label{eq:2_3:5}
    \left.\theta\right|_{\Gamma}=\theta_{b},
    \left.\quad \partial_{n} \theta\right|_{\Gamma}=q_{b}.
\end{equation}

Основные результаты главы заключаются в получении априорных
оценок решения задачи~\eqref{eq:2_3:1},~\eqref{eq:2_3:2},
на основании которых доказывается разрешимость задачи
оптимального управления~\eqref{eq:2_3:1}--\eqref{eq:2_3:4},
и оптимальность система является производной.
Показано, что последовательность
$\left\{\theta_{\lambda}, \varphi_{\lambda}, u_{\lambda}\right\}$
решений экстремальной задачи~\eqref{eq:2_3:1}--\eqref{eq:2_3:4},
при $ \lambda \rightarrow+0$
сходится к решению начально-краевой задачи~\eqref{eq:2_3:1},~\eqref{eq:2_3:5},
с условиями типа Коши для температуры.

\subsection{Формализация задачи управления}\label{subsec:ch2/sec3/formalization}
В дальнейшем как и ранее предполагается,
что $\Omega \subset \mathbb{R}^{3}$ — ограниченная строго липшицева область,
граница которой $\Gamma$ состоит из конечного числа гладких кусков.
Через $L^{s}, 1 \leq s \leq$ $\infty$
обозначается пространство Лебега, а через $H^{s}$
пространство Соболева $W_{2}^{s}$.
Пусть $H=$ $L^{2}(\Omega), V=H^{1}(\Omega)$.
Через $V^{\prime}$ обозначим двойственное к $V$ пространство,
а через $L^{s}(0, T; X)$
пространство Лебега функций из $L^{s}$, определенных
на $(0, T)$ со значениями в пространстве $X$.
Пространство $H$ отождествляется с пространством
$H^{\prime}$ так, что $V \subset H=H^{\prime} \subset V^{\prime}$.
Обозначим через $\|\cdot\|$ стандартную норму в $H$, а через $(f, v)$
значение функционала $f \in V^{\prime}$ в элементе $v \in V$,
что совпадает со скалярным произведением в $H$, если $f \in H$.

Через $U$ обозначим пространство $L^{2}(\Sigma)$ с нормой

\[ \|u\|_{\Sigma}=\left(\int_{\Sigma} u^{2} d \Gamma d t\right)^{1 / 2}. \]


Также будем использовать пространство
\[
    W=\left\{y \in L^{2}(0, T ; V): \quad y^{\prime}
    \in L^{2}\left(0, T, V^{\prime}\right)\right\},
\]
где $y^{\prime}=d y / d t$.

Будем считать, что \\
$(k) a, b, \alpha, \kappa_{a}, \lambda=$ Const $>0$, \\
$(kk) \theta_{b}, q_{b} \in U, r=a\left(\theta_{b}+q_{b}\right)
\in L^{5}(\Sigma), \; \theta_{0} \in L^{5}(\Omega)$.


Определим операторы $A: V \rightarrow V^{\prime}, B: U \rightarrow V^{\prime}$,
используя следующие равенства, справедливые для любых
$y, z \in V, w \in L^{2}(\Gamma)$:

\[
    (A y, z)=(\nabla y, \nabla z)+\int_{\Gamma} y z d \Gamma,
    \quad(B w, z)=\int_{\Gamma} w z d \Gamma.
\]

Билинейная форма $(A y, z)$ определяет скалярное произведение в пространстве $V$,
и соответствующая норма $\|z\|_{V}=\sqrt{(A z, z)}$ эквивалентна к стандартной норме в $V$.
Следовательно, определен непрерывный обратный оператор $A^{-1}: V^{\prime} \mapsto V$.
Заметим, что для любых $v \in V, w \in L^{2}(\Gamma), g \in V^{\prime}$
выполняются следующие неравенства:

\[
    \|v\|^{2} \leq C_{0}\|v\|_{V}^{2},\|v\|_{V^{\prime}} \leq C_{0}\|v\|_{V},\|B w\|_{V^{\prime}}
    \leq\|w\|_{\Gamma},\left\|A^{-1} g\right\|_{V} \leq\|g\|_{V^{\prime}}.
\]

Здесь константа $C_{0}>0$ зависит только от домена $\Omega$.

В дальнейшем будем использовать следующие обозначение
$[h]^{s} \coloneqq |h|^{s} \operatorname{sign} h, s>0, h \in \mathbf{R}$
для монотонной степенной функции.


\begin{definition}
    Пара $\theta \in W, \varphi \in L^{2}(0, T ; V)$ называется слабым
    решением задачи~\eqref{eq:2_3:1}--\eqref{eq:2_3:3}, если
    \begin{equation}
        \label{eq:2_3:6}
        \theta^{\prime}+a A \theta+b \kappa_{a}\left([\theta]^{4}-\varphi\right)=B r,
        \quad \theta(0)=\theta_{0}, \quad \alpha A \varphi
        + \kappa_{a}\left(\varphi-[\theta]^{4}\right)=B u.
    \end{equation}
\end{definition}

Для формулировки задачи оптимального управления определим оператор ограничения
$F(\theta, \varphi, u): W \times L^{2}(0, T; V) \times U
\rightarrow L^{2} \left(0, T, V^{\prime}\right)
\times L^{2}\left(0, T, V^{\prime}\right) \times H$ таким, что

\[
    F(\theta, \varphi, u)=\left\{\theta^{\prime}+a A
    \theta+b \kappa_{a}\left([\theta]^{4}-\varphi\right)-B r,
    \alpha A \varphi+\kappa_{a}\left(\varphi-[\theta]^{4}\right)-B u,
    \theta(0)-\theta_{0}\right\}.
\]

Таким образом задача $(\mathbf{OC})$ заключается в отыскании тройки
$\{\theta, \varphi, u\} \in W \times L^{2}(0, T ; V) \times U$ такой, что

\[
    J_{\lambda}(\theta, u) \equiv \frac{1}{2}\left\|\theta-
    \theta_{b}\right\|_{\Sigma}^{2}+
    \frac{\lambda}{2}\|u\|_{\Sigma}^{2}
    \rightarrow \inf, \quad F(\theta, \varphi, u)=0.
\]

\subsection{Разрешимость задачи $(\mathbf{OC})$}
\label{subsec:ch2/sec3/subsec3}
Докажем сначала однозначную разрешимость
задачи~\eqref{eq:2_3:1}--\eqref{eq:2_3:3}.

\begin{lemma}
    \label{lm:2_3:1}
    Пусть выполняются условия $(k), (kk)$, $u \in U$.
    Тогда существует единственное слабое решение
    задачи~\eqref{eq:2_3:1}--\eqref{eq:2_3:3} и, кроме того,

    \[
        \psi=[\theta]^{5 / 2} \in L^{\infty}(0, T; H) \cap L^{2}(0, T ; V),
        \quad[\theta]^{4} \in L^{2}(0, T ; H).
    \]
\end{lemma}

\begin{proof}
    Выразим $\varphi$ из последнего уравнения~\eqref{eq:2_3:6} и подставим его в первое.
    В результате получаем следующую задачу Коши для уравнения с операторными коэффициентами:

    \begin{equation}
        \label{eq:2_3:7}
        \theta^{\prime}+a A \theta+L[\theta]^{4}=B r+f, \quad \theta(0)=\theta_{0}.
    \end{equation}

    Здесь

    \[
        L=\alpha b \kappa_{a} A\left(\alpha A+\kappa_{a} I\right)^{-1}: V^{\prime}
        \rightarrow V^{\prime}, f=b \kappa_{a}\left(\alpha A
        +\kappa_{a} I\right)^{-1} B u \in L^{2}(0, T ; V).
    \]

    Получим априорные оценки решения задачи~\eqref{eq:2_3:7},
    на основании которых стандартным образом выводится
    разрешимость этой задачи.
    Пусть $[\zeta, \eta]=\left(\left(\alpha I+\kappa_{a} A^{-1}\right) \zeta,
    \eta\right), \zeta \in V^{\prime}, \eta \in V$.
    Отметим, что выражение $[[\eta]]=\sqrt{[\eta, \eta]}$ определяет норму в $H$,
    эквивалентную стандартной.

    Умножая скалярно в $H$ уравнение в~\eqref{eq:2_3:7}
    на $\left(\alpha I+\kappa_{a} A^{-1}\right) \theta$, получаем

    \begin{equation}
        \label{eq:2_3:8}
        \frac{1}{2} \frac{d}{d t}[[\theta]]^{2}+a
        \alpha(A \theta, \theta)+a \kappa_{a}\|\theta\|^{2}+\alpha
        b \kappa_{a}\|\theta\|_{L^{5}(\Omega)}^{5}=[B r, \theta] + [f, \theta].
    \end{equation}

    Из равенства ~\eqref{eq:2_3:8} следует оценка

    \begin{equation}
        \label{eq:2_3:9}
        \|\theta\|_{L^{\infty}(0, T ; H)}+\|\theta\|_{L^{2}(0, T ; V)}
        +\|\theta\|_{L^{5}(Q)} \leq C_{1},
    \end{equation}


    где $C_{1}$ зависит только от $a, b, \alpha, \kappa_{a},\|f\|_{L^{2}(0, T ; H)},
    \left\|\ theta_{0}\right\|,\|r\|_{L^{2}(\Sigma)}$.

    Далее положим $\psi=[\theta]^{5 / 2}$.
    В таком случае

    \[
        \left(\theta^{\prime},[\theta]^{4}\right)=\frac{1}{5} \frac{d}{d t}\|\psi\|^{2},
        \quad\left(A \theta,[\theta]^{4}\right)=\frac{16}{25}\|\nabla \psi\|^{2}
        +\|\psi\|_{L^{2}(\Gamma)}^{2}.
    \]

    Умножая скалярно в $H$ уравнение в~\eqref{eq:2_3:7}
    на $[\theta]^{4}=[\psi]^{8 / 5}$, получаем

    \begin{equation}
        \label{eq:2_3:10}
        \frac{1}{5} \frac{d}{d t}\|\psi\|^{2}+a\left(\frac{16}{25}\|\nabla \psi\|^{2}
        + \|\psi\|_{L^{2}(\Gamma)}^{2}\right)+
        \left(L[\psi]^{8 / 5},[\psi]^{8 / 5}\right)=\left(B r+f,[\psi]^{8 / 5}\right).
    \end{equation}

    Равенство~\eqref{eq:2_3:10} влечет оценку

    \begin{equation}
        \label{eq:2_3:11}
        \|\psi\|_{L^{\infty}(0, T ; H)}+\|\psi\|_{L^{2}(0, T ; V)}+
        \left\|[\theta]^{4}\right\|_{L^{2}(0, T ; H)} \leq C_{2},
    \end{equation}
    где $C_{2}$ зависит только от
    $a, b, \alpha, \kappa_{a},\|f\|_{L^{2}(0, T ; H)},
    \left\|\theta_{0}\right\|_{L^{5}(\Omega)},\|r\|_{L^{5}(\Sigma)}$.
    Получим оценку
    $\left\|\theta^{\prime}\right\|_{L^{2}\left(0, T ; V^{\prime}\right)}$
    с учетом $ \theta^{\prime}=B r+f-a A \theta-L[\theta]^{4}$.
    В силу условий на исходные данные
    $B r, f \in L^{2}\left(0, T ; V^{\prime}\right)$.
    Поскольку $\theta \in L^{2}(0, T ; V)$,
    то $A \theta \in L^{2}\left(0, T ; V^{\prime}\right)$.
    Пусть $\zeta=L[\theta]^{4}$.
    Таким образом

    \[
        \alpha \zeta+\kappa_{a} A^{-1} \zeta=\alpha b \kappa_{a}[\theta]^{4}.
    \]
    Умножая в смысле скалярного произведения $H$
    последнее равенство на $\zeta$, получаем

    \[
        \alpha\|\zeta\|^{2}+\kappa_{a}\left(A^{-1} \zeta,
        \zeta\right)=\alpha b \kappa_{a}\left([\theta]^{4},
        \zeta\right) \leq \alpha\left(\|\zeta\|^{2}
        +\frac{\left(b \kappa_{a}\right)^{2}}{4}\left\|
        [\theta]^{4}\right\|^{2}\right).
    \]


    Следовательно, $\|\zeta\|_{V^{\prime}}^{2}=\left(A^{-1} \zeta,
    \zeta\right) \leq \frac{\alpha \kappa_{ a} b^{2}}{4}\left\|[\theta]^{4}\right\|^{2}$
    и в силу оценок~\eqref{eq:2_3:9},~\eqref{eq:2_3:11} получаем

    \begin{equation}
        \label{eq:2_3:12}
        \left\|\theta^{\prime}\right\|_{L^{2}\left(0, T; V^{\prime}\right)}
        \leq\|B r+f\|_{L^{2}\left(0, T ; V^{\prime}\right)}+a C_{1}+\sqrt{\alpha \kappa_{a}} b C_{2}.
    \end{equation}


    Оценок~\eqref{eq:2_3:9}—\eqref{eq:2_3:12} достаточно
    для доказательства разрешимости задачи.

    Пусть $\theta_{1,2}$ — решения задачи~\eqref{eq:2_3:7},
    $\eta=\theta_{1}-\theta_{2}$.
    Затем
    \[
        \eta^{\prime}+a A \eta+L\left(\left[\theta_{1}\right]^{4}-
        \left[\theta_{1}\right]^{4}\right)=0, \quad \eta(0)=0.
    \]


    Умножая в смысле скалярного произведения $H$ последнее уравнение на
    $\left(\alpha I+\kappa_{a} A^{-1}\right) \eta$, получаем

    \[
        \frac{1}{2} \frac{d}{d t}[[\eta]]^{2}+a \alpha(A \eta, \eta)
        + a \kappa_{a}\|\eta\|^{2}+\alpha b
        \kappa_{a}\left(\left[\theta_{1}\right]^{4}
        -\left[\theta_{1}\right]^{4}, \theta_{1}-\theta_{2}\right) = 0.
    \]


    Последний член в левой части неотрицательный,
    поэтому, интегрируя полученное равенство по времени,
    получаем $\eta=\theta_{1}-\theta_{2}=0$, что означает единственность решение.
    Лемма доказана.
\end{proof}

\begin{theorem}
    \label{th:2_3:1}
    Пусть выполняются условия $(k), (kk)$.
    Тогда сществует решение задачи $(OC)$.
\end{theorem}


\begin{proof}
    Пусть $j_{\lambda}=\inf J_{\lambda}$ on the set $u \in U, F(\theta, \varphi, u)=0$.
    Выберем минимизирующую последовательность
    $u_{m} \in U, \theta_{m} \in W, \varphi_{m} \in L^{2}(0, T; V),
    J_{\lambda}\left(\theta_{m}, u_{m}\right) \rightarrow j_{\lambda}$,

    \begin{equation}
        \label{eq:2_3:13}
        \begin{aligned}
            \theta_{m}^{\prime}+a A \theta_{m}
            + b \kappa_{a}\left(\left[\theta_{m}\right]^{4}
            - \varphi_{m}\right)=B r, & \theta_{m}(0)=\theta_{0}, \\
            & \alpha A \varphi_{m}+\kappa_{a}\left(\varphi_{m}-
            \left[\theta_{m}\right]^{4}\right)=B u_{m}.
        \end{aligned}
    \end{equation}

    Ограниченность последовательности $u_{m}$
    в пространстве $U$ влечет по~\ref{lm:2_3:1} оценки

    \[
        \begin{gathered}
            \left\|\theta_{m}\right\|_{L^{2}(0, T ; V)} \leq C,
            \quad\left\|\theta_{m}\right\|_{L^{\infty}\left(0, T; L^{5}(\Omega)\right)} \leq C,
            \left\|\theta_{m}^{\prime}\right\|_{L^{2}\left(0, T; V^{\prime}\right)} \leq C, \\
            \int_{0}^{T} \int_{\Omega}\left|\theta_{m}\right|^{8} d x d t \leq C,
            \quad\left\|\varphi_{m}\right\|_{L^{2}(0, T ; V)} \leq C.
        \end{gathered}
    \]

    Здесь $C>0$ обозначает наибольшую из констант,
    ограничивающих соответствующие нормы и не зависящих от $m$.
    Переходя, если необходимо, к подпоследовательностям, заключаем, что существует тройка
    $\{\widehat{u}, \widehat{\theta}, \widehat{\varphi}\}
    \in U \times W \times L^{2}(0, T; V)$,

    \[
        \begin{gathered}
            u_{m} \rightarrow \widehat{u} \text { слабо в } U \\
            \theta_{m} \rightarrow \widehat{\theta}
            \text{ слабо в } L^{2}(0, T; V), \text { сильно в } L^{2}(Q), \\
            \varphi_{m} \rightarrow \widehat{\varphi}
            \text{ слабо в }  L^{2}(0, T ; V).
        \end{gathered}
    \]


    Более того, $\widehat{\theta} \in L^{8}(Q)
    \cap L^{\infty}\left(0, T ; L^{5}(\Omega)\right)$.

    Результаты сходимости позволяют перейти к пределу в~\eqref{eq:2_3:13}.
    В этом случае предельный переход в
    нелинейной части следует из следующего неравенства,
    справедливого при $\xi \in C^{\infty}(\bar{Q})$:

    \[
        \begin{aligned}
            & \int_{0}^{T}\left|\left(\left[\theta_{m}\right]^{4}
            -[\widehat{\theta}]^{4}, \xi\right)\right| d t \leq \\
            & \quad 2 \max _{\bar{Q}}|\xi|
            \left(\left\|\theta_{m}\right\|_{L^{5}(\Omega)}^{5 / 3}\left\|
            \theta_{m}\right\|_{L^{8}(\Omega)}^{4 / 3}
            + \|\widehat{\theta}\|_{L^{5}(\Omega)}^{5 / 3}
            \|\widehat{\theta}\|_{L^{8}(\Omega)}^{4 / 3}\right)\left\|\theta_{m}
            - \widehat{\theta}\right\|_{L^{2}(Q)}.
        \end{aligned}
    \]


    В результате получаем, что

    \[
        \widehat{\theta}^{\prime}+a A \widehat{\theta}+b
        \kappa_{a}\left([\widehat{\theta}]^{4}-\widehat{\varphi}\right)=B r,
        \quad \widehat{\theta}(0)=\theta_{0},
        \quad \alpha A \widehat{\varphi}+\kappa_{a}\left(\widehat{\varphi}-
        [\widehat{\theta}]^{4}\right)=B \widehat{u},
    \]

    где
    \[
        j_{\lambda} \leq J_{\lambda}(\widehat{\theta},
        \widehat{u}) \leq \underline{\lim }
        J_{\lambda}\left(\theta_{m}, u_{m}\right)=j_{\lambda}.
    \]

    Таким образом, тройка $\{\widehat{\theta}, \widehat{\varphi}, \widehat{u}\}$
    есть решение задачи $(OC)$.
\end{proof}

\subsection{Условия оптимальности}\label{subsec:ch2/sec3/subsec4}
Для получения системы оптимальности достаточно использовать принцип Лагранжа
для гладко-выпуклых экстремальных задач~\cite{10, 11}.
Проверим выполнение ключевого условия,
что образ производной оператора связи $F_{y}^{\prime}(y, u)$,
где $y=\{\theta, \varphi\} \text{ в }  W \times L^{2}(0, T ; V)$,
совпадает с пространством
$L^{2}\left(0, T; V^{\prime}\right) \times L^{2} \left(0, T ; V^{\prime}\right) \times H$.

Напомним, что
\begin{equation*}
    \begin{aligned}
        F(\theta, \varphi, u) &=\{\theta^{\prime}+
        a A \theta+b \kappa_{a}\left([\theta]^{4}-\varphi\right)-B r, \\
        &\alpha A \varphi+\kappa_{a}\left(\varphi-[\theta]^{4}\right)-B u,
        \theta(0)-\theta_{0}\}.
    \end{aligned}
\end{equation*}
\begin{lemma}
    \label{lm:2_3:2}
    Пусть выполнены условия $(k), (kk)$.
    Если $\widehat{y} \in W \times L^{2}(0, T ; V), \widehat{u} \in U$
    является решением задачи $(OC)$, то справедливо равенство:

    \[
        \operatorname{Im} F_{y}^{\prime}
        (\widehat{y}, \widehat{u})=L^{2}\left(0, T; V^{\prime}\right)
        \times L^{2}\left(0, T; V^{\prime}\right) \times H.
    \]
\end{lemma}

\begin{proof}
    Достаточно проверить, что задача

    \[
        \xi^{\prime}+a A \xi+b \kappa_{a}\left(4|\widehat{\theta}|^{3}
        \xi-\eta\right)=f_{1}, \quad \xi(0)=\xi_{0},
        \quad \alpha A \eta+\kappa_{a}\left(\eta-4|\widehat{\theta}|^{3} \xi\right)=f_{2}
    \]
    разрешима для всех $f_{1,2} \in L^{2}\left(0, T; V^{\prime}\right), \xi_{0} \in H$.
    Выразим $\eta$ из последнего уравнения и подставим его в первое.
    В результате получаем следующую задачу:
    \begin{equation}
        \label{eq:2_3:14}
        \xi^{\prime}+a A \xi+4 L\left(|\widehat{\theta}|^{3}
        \xi\right)=f_{1}+b \kappa_{a}\left(\alpha A+\kappa_{a}
        I\right)^{-1} f_{2}, \xi(0)=\xi_{0}.
    \end{equation}
    Однозначная разрешимость линейной задачи~\eqref{eq:2_3:14}
    доказывается аналогично лемме~\ref{lm:2_3:1}.
\end{proof}


Согласно лемме~\ref{lm:2_3:2} лагранжиан задачи $(OC)$ имеет вид

\[
    \begin{aligned}
        & L\left(\theta, \varphi, u, p_{1}, p_{2}, q\right)=
        J_{\lambda}(\theta, u) +\int_{0}^{T}\left(\theta^{\prime}
        + a A \theta+b \kappa_{a}\left([\theta]^{4}-\varphi\right)
        - B r, p_{1}\right) d t \\
        & + \int_{0}^{T}\left(\alpha A \varphi
        + \kappa_{a}\left(\varphi-[\theta]^{4}\right)
        -B u, p_{2}\right) d t+\left(q, \theta(0)-\theta_{0}\right).
    \end{aligned}
\]

Здесь $p=\left\{p_{1}, p_{2}\right\} \in L^{2}(0, T; V) \times L^{2}(0, T; V) $
— сопряженное состояние, $q \in H$ — множитель Лагранжа для начального условия.
Если $\{\widehat{\theta}, \widehat{\varphi}, \widehat{u}\}$
является решением задачи $(OC)$, то в силу принципа
Лагранжа~\cite[гл. 2, теорема 1.5]{10} выполняются вариационные равенства
$\forall \zeta \in L^{2}(0, T ; V), v \in U$

\[
    \begin{gathered}
        \int_{0}^{T}\left(\left(B\left(\widehat{\theta}-\theta_{b}\right),
        \zeta\right)+\left(\zeta^{\prime}
        +a A \zeta+4 b \kappa_{a}|\widehat{\theta}|^{3} \zeta, p_{1}\right)
        -\kappa_{a}\left(4|\widehat{\theta}|^{3} \zeta,
        p_{2}\right)\right) d t+(q, \zeta(0))=0, \\
        \int_{0}^{T}\left(\left(\alpha A \zeta
        +\kappa_{a} \zeta, p_{2}\right)
        -b \kappa_{a}\left(\zeta, p_{1}\right)\right) d t=0,
        \int_{0}^{T}\left(\lambda(\widehat{u}, v)_{\Gamma}
        -\left(B v, p_{2}\right)\right) d t=0.
    \end{gathered}
\]

Таким образом, из полученных условий получаем следующий результат.

\begin{theorem}
    \label{th:2_3:2}
    Пусть выполнены условия $(k), (kk)$.
    Если $\{\widehat{\theta}, \widehat{\varphi}, \widehat{u}\}$ — решение задачи $(OC)$,
    то существует единственная пара $\left\{p_ {1}, p_{2}\right\} \in W \times W$ такая, что

    \begin{equation}
        \label{eq:2_3:15}
        \begin{aligned}
            -p_{1}^{\prime}+a A p_{1}+4|\widehat{\theta}|^{3} \kappa_{a}\left(b p_{1}
            -p_{2}\right)&=B\left(\theta_{b}-\widehat{\theta}\right),
            p_{1}(T)=0, \\
            \alpha A p_{2}+\kappa_{a}\left(p_{2}-b p_{1}\right)&=0,
        \end{aligned}
    \end{equation}
    а также $\lambda \widehat{u}=\left.p_{2}\right|_{\Sigma}$.
\end{theorem}

\subsection{Аппроксимация задачи с граничными условиями типа Коши}
\label{subsec:ch2/sec3/approximation}
Рассмотрим начально-краевую задачу для уравнений сложного
теплообмена, в которой отсутствуют граничные условия на интенсивность излучения:

\begin{equation}
    \label{eq:2_3:16}
    \frac{\partial \theta}{\partial t}-a \Delta \theta
    + b \kappa_{a}\left([\theta]^{4}
    - \varphi\right)=0, \quad-\alpha \Delta \varphi
    + \kappa_{a}\left(\varphi-[\theta]^{4}\right) = 0,
    \quad(x, t) \in Q,
\end{equation}
\begin{equation}
    \label{eq:2_3:17}
    \theta=\theta_{b}, \quad \partial_{n} \theta= q_{b} \text { на } \Sigma,
    \left.\quad \theta\right|_{t=0} = \theta_{0}.
\end{equation}


Существование и единственность функций $\theta \in L^{2}\left(0, T; H^{2}(\Omega)\right),
\varphi, \Delta \varphi \in L^{2}( Q)$, удовлетворяющие~\eqref{eq:2_3:16},\eqref{eq:2_3:17}
для достаточно гладких $\theta_{b}, q_{b}$, доказаны в~\cite{Chebotarev2019Problem}.
Покажем, что решения задачи $(OC)$ при $\lambda \rightarrow+0$
аппроксимируют решение задачи~\eqref{eq:2_3:16},\eqref{eq:2_3:17}

\begin{theorem}
    \label{th:2_3:3}
    Пусть выполняются условия $(k), (kk)$ и существует решение
    $\theta, \varphi \in$ $L^{2}\left(0, T ; H^{2}(\Omega) \right)$
    задачи~\eqref{eq:2_3:16},\eqref{eq:2_3:17}.
    Если $\left\{\theta_{\lambda}, \varphi_{\lambda}, u_{\lambda}\right\}$
    — решение задачи $(OC)$ при $\lambda>0$, то при $\lambda\rightarrow+0$

    \[
        \begin{gathered}
            \theta_{\lambda} \rightarrow \theta \text { слабо в } L^{2}(0, T ; V),
            \text { сильно в } L^{2}(Q), \\
            \varphi_{\lambda} \rightarrow \varphi \text { слабо в } L^{2}(0, T ; V).
        \end{gathered}
    \]
\end{theorem}

\begin{proof}
    Пусть $\theta, \varphi \in L^{2}\left(0, T ; H^{2}(\Omega)\right)$
    — решение задачи~\eqref{eq:2_3:16},\eqref{eq:2_3:17},
    $u=\alpha\left(\partial_{n} \varphi+\varphi\right) \in U$.
    Тогда

    \[
        \theta^{\prime}+a A \theta+b \kappa_{a}\left([\theta]^{4}-\varphi\right)=B r,
        \quad \theta(0)=\theta_{0}, \quad \alpha A \varphi
        + \kappa_{a}\left(\varphi-[\theta]^{4}\right)=B u,
    \]
    где $r\coloneqq a\left(\theta_{b}+q_{b}\right)$.
    Следовательно, принимая во внимание,
    что $\left.\theta\right|_{\Gamma}=\theta_{b}$,
    \[
        J_{\lambda}\left(\theta_{\lambda},
        u_{\lambda}\right)=\frac{1}{2}\left\|\theta_{\lambda}-\theta_{b}\right\|_{\Sigma}^{2}
        +\frac{\lambda}{2}\left\|u_{\lambda}\right\|_{\Sigma}^{2}
        \leq J_{\lambda}(\theta, u)=\frac{\lambda}{2}\|u\|_{\Sigma}^{2}.
    \]

    Таким образом
    \[
        \left\|u_{\lambda}\right\|_{\Sigma}^{2} \leq C, \quad\left\|\theta_{\lambda}
        -\theta_{b}\right\|_{\Sigma}^{2} \rightarrow 0, \lambda \rightarrow+0.
    \]


    Здесь и далее $C>0$ не зависит от $\lambda$.
    Ограниченность последовательности $u_{\lambda}$
    в пространстве $U$ влечет по лемме~\ref{lm:2_3:1} оценки

    \[
        \begin{gathered}
            \left\|\theta_{\lambda}\right\|_{L^{2}(0, T ; V)} \leq C,
            \quad\left\|\theta_{\lambda}\right\|_{L^{\infty}
            \left(0, T ; L^{5}(\Omega)\right)} \leq C,
            \quad\left\|\theta_{\lambda}^{\prime}\right\|_{L^{2}
            \left(0, T ; V^{\prime}\right)} \leq C, \\
            \int_{0}^{T} \int_{\Omega}
            \left|\theta_{\lambda}\right|^{8} d x d t \leq C,
            \quad\left\|\varphi_{\lambda}\right\|_{L^{2}(0, T ; V)} \leq C.
        \end{gathered}
    \]
    Следовательно, можно выбрать последовательность $\lambda\rightarrow+0$ такую, что
    \[
        \begin{gathered}
            u_{\lambda} \rightarrow u_{*} \text { слабо в } U \\
            \theta_{\lambda} \rightarrow \theta_{*}
            \text { слабо в } L^{2}(0, T; V) \text { сильно в } L^{2}(Q), \\
            \varphi_{\lambda} \rightarrow \varphi_{*}
            \text { слабо в } L^{2}(0, T; V).
        \end{gathered}
    \]

    Полученные результаты о сходимости позволяют, как и в теореме~\ref{th:2_3:1},
    перейти к пределу при $\lambda \rightarrow+0$ в уравнениях для $\theta_{\lambda},
    \varphi_{\lambda}, u_{\lambda}$.
    Тогда
    \begin{equation}
        \label{eq:2_3:18}
        \begin{aligned}
            \theta_{*}^{\prime}+a A \theta_{*}
            + b \kappa_{a}\left(\left[\theta_{*}\right]^{4}
            - \varphi_{*}\right)&=B r, \quad \theta_{*}(0)=\theta_{0}, \\
            \alpha A \varphi_{*}+\kappa_{a}\left(\varphi_{*}
            - \left[\theta_{*}\right]^{4}\right)&=B u_{*}.
        \end{aligned}
    \end{equation}
    При этом $\left.\theta_{*}\right|_{\Gamma}=\theta_{b}$.
    Из первого уравнения в~\eqref{eq:2_3:18}, учитывая,
    что $r=a\left(\theta_{b}+q_{b}\right)$, получаем
    $
    \frac{\partial \theta_{*}}{\partial t}-a \Delta \theta_{*}
    +b \kappa_{a}\left(\left[\theta_{*}\right]^{4}-\varphi_{*}\right)=0$
    $\text { почти всюду в } Q, \quad \theta_{*}=\theta_{b},
    \quad \partial_{n} \theta=q_{b} \text { почти всюду в } \Sigma.
    $

    Из второго уравнения в~\eqref{eq:2_3:18} следует, что
    $-\alpha \Delta \varphi+\kappa_{a}\left(\varphi-[\theta]^{4}\right)=0$ почти всюду в $Q$.
    Таким образом, пара $\theta_{*}, \varphi_{*}$
    является решением задачи~\eqref{eq:2_3:16},~\eqref{eq:2_3:17}.
    Поскольку решение этой задачи единственно~\cite{Chebotarev2019Problem},
    то $\theta_{*}=\theta, \varphi_{*}=\varphi$.
\end{proof}
