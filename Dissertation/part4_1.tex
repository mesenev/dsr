\section{Численные алгоритмы решения прямых задач сложного теплообмена}
\label{sec:ch4/sec1}

\subsection{Стационарная модель}
\label{subsec:ch4/sec1/stationary}
Стационарная модель сложного теплообмена
в $P_1$ приближении записывается как
\[
    \begin{gathered}
        -a \Delta \theta+b \kappa_{a}| \theta|^{3} \theta =
        b \kappa_{a} \varphi \\
        -\alpha \Delta \varphi+\kappa_{a} \varphi =
        \kappa_{a}|\theta|^{3} \theta \\
        a \frac{\partial \theta}{\partial n}
        +\left.\beta\left(\theta-\theta_{b}\right)\right|_{\Gamma}=0,
        \quad \alpha \frac{\partial \varphi}{\partial n}
        +\left.\gamma\left(\varphi-\theta_{b}^{4}\right)\right|_{\Gamma}=0.
    \end{gathered}
\]

Основную сложность в решении прямой задачи представляет нелинойность
по температурному полю, которое входит в систему
дифференциальных уравнений в четвёртой степени.

Метод простой итераций применяется для решения задач в двумерной области.
Он был использован в работах~\cite{Kovtanyuk2015,astrakhantseva2014numerical}.
Однако, в трёхмерной области этот метод демонстрирует недостаточную сходимость,
что приводит к долгим итерационным процессам и увеличению вычислительных затрат.

Вместо метода простой итераций для решения сложных задач оптимального управления
температурой может быть применен метод Ньютона.
Метод Ньютона является одним из наиболее эффективных методов оптимизации
и решения нелинейных уравнений.
Он использует локальную аппроксимацию функции в виде касательной плоскости
и обновляет решение на основе градиента и гессиана функции.
Этот метод обычно обеспечивает быструю сходимость и может быть
применен для решения сложных задач оптимального
управления температурой в трехмерных областях.


Метод Ньютона является усовершенствованным вариантом метода простой итерации,
в котором нелинейное слагаемое $|\theta|^3 \theta$ аппроксимируется
выражением $\widetilde{\theta}^4+4 \widetilde{\theta^3}(\theta-\widetilde{\theta})$,
где $\widetilde{\theta}$ - приближение для температуры на предыдущей итерации.
Эта аппроксимация обеспечивает более точное решение и быструю сходимость,
что делает метод Ньютона более эффективным для
решения задач с высокой степенью нелинейности.
В результате модифицированная система уравнений примет следующий вид:

\[
    \begin{gathered}
        -a \Delta \theta+b \kappa_{a}\left(\left(4 \widetilde{\theta}^{3}
        \theta-3 \widetilde{\theta}^{4}\right)-\varphi\right)=0,
        \quad-\alpha \Delta \varphi
        +\kappa_{a}\left(\varphi
        -\left(4 \widetilde{\theta}^{3}
        \theta-3 \widetilde{\theta}^{4}\right)\right)=0, \\
        a \frac{\partial \theta}{\partial n}
        +\left.\beta\left(\theta-\theta_{b}\right)\right|_{\Gamma}=0,
        \quad \alpha \frac{\partial \varphi}{\partial n}
        +\left.\gamma\left(\varphi-\theta_{b}^{4}\right)\right|_{\Gamma}=0.
    \end{gathered}
\]

Монотонная сходимость метода Ньютона является важным свойством,
которое позволяет обеспечить стабильность итерационного процесса
и успешное решение эллиптических уравнений
с монотонным и выпуклым нелинейным слагаемым.
В литературе было проведено множество исследований, направленных
на изучение сходимости метода Ньютона для таких уравнений.

В частности, в работах~\cite{Mukhamadiev1971, Schryer1971} были представлены
результаты анализа монотонной сходимости метода Ньютона для эллиптического
уравнения с монотонным и выпуклым нелинейным слагаемым.
Результаты этих исследований показывают, что метод Ньютона обладает
хорошими свойствами сходимости и может быть применен для решения
сложных задач оптимального управления в моделях сложного теплообмена.

\subsection{Квазистационарные и квазилинейные модели}
\label{subsec:ch4/sec1/quasi}

\textbf{Квазистационарная модель} радиационного и диффузионного теплообмена в ограниченной области
$\Omega \subset \mathbb{R}^{3}$ с границей $\Gamma=\partial \Omega$ в разделе~\ref{sec:ch2/sec3} представлена следующим образом

\begin{align*}
    & \frac{\partial \theta}{\partial t} - a \Delta \theta
    + b \kappa_{a} \left(|\theta| \theta^{3}-\varphi\right) = 0,\\
    & - \alpha \Delta \varphi
    + \kappa_{a} \left(\varphi-|\theta| \theta^{3}\right) = 0, \\
    \quad x \in \Omega, \quad 0 < t < T;
    a \left(\partial_{n} \theta+\theta\right)=r, \\
    \quad \alpha\left(\partial_{n} \varphi
    + \varphi\right) = u \text { на } \Gamma;
    \left.\theta\right|_{t=0} = \theta_{0}.
\end{align*}



\textbf{Квазистационарная модель} радиационного и диффузионного теплообмена в ограниченной области
$\Omega \subset \mathbb{R}^{3}$ с границей $\Gamma=\partial \Omega$ в разделе~\ref{sec:ch3:sec3}
\begin{gather*}
    \sigma \partial \theta / \partial t-\operatorname{div}(k(\theta) \nabla \theta)
    -\beta \varphi=u_{1} \chi, \\
    \quad-\operatorname{div}(\alpha \nabla \varphi)+\beta \varphi= u_{2} \chi, \\
    k(\theta) \partial_{n} \theta+\left.\gamma
    \left(\theta-\theta_{b}\right)\right|_{\Gamma}=0,
    \quad \alpha \partial_{n} \varphi +
    \left.0.5 \varphi\right|_{\Gamma}=0,\left.\quad \theta\right|_{t=0}=\theta_{0}.
%        \quad x \in \Omega, \quad 0<t<T, \\
\end{gather*}
