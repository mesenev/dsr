\section{ГРАНИЧНАЯ ОБРАТНАЯ ЗАДАЧА СЛОЖНОГО ТЕПЛООБМЕНА}

\subsection{Inverse problem}
В липшицевой ограниченной области $\Omega \subset \mathbb{R}^{3}$ с границей
$\Gamma:=\partial \Omega$ рассмотрим следующую систему нелинейных уравнений [1]:
\[
    -a \Delta \theta+b \kappa_{a}\left(|\theta| \theta^{3}-\varphi\right)=0,
    \quad-\alpha \Delta \varphi+\kappa_{a}\left(\varphi-|\theta| \theta^{3}\right)=0, x \in \Omega.
\]
Здесь $\theta$ — нормированная температура, $\varphi$ — нормированная усредненная интенсивность излучения.
Приведены положительные параметры $a, b, \kappa_{a}$ и $\alpha$, описывающие внутренние свойства среды [1].


Предположим, что $\Gamma=\bar{\Gamma}_{1} \cup \bar{\Gamma}_{2}$ такое,
что $\Gamma_{1} \cap \Gamma_{2}=\emptyset$.
На границе $\Gamma$ задаем тепловой поток $q_{b}$,
\[
    a \partial_{n} \theta=q_{b}, \quad x \in \Gamma.
\]
Граничное условие для интенсивности излучения при $\Gamma_{2}$ не задано.
В качестве условия паратерминации на границе $\Gamma_{1}$ зададим следующие условия:
\[
    \alpha \partial_{n} \varphi+\gamma\left(\varphi-\theta_{b}^{4}\right)=0,
    \quad \theta=\theta_{b} \quad x \in \Gamma_{ 1} .
\]

Оптимизационный метод решения задачи (1)-(3) заключается в рассмотрении задачи
граничного оптимального управления для эквивалентной системы эллиптических уравнений.

Определим новую неизвестную функцию $\psi=a \theta+\alpha b \varphi$.
Складывая первое уравнение в (1) со вторым, умноженным на $b$, заключаем,
что $\psi$ является гармонической функцией. Исключая $\varphi$ из первого
уравнения (1) и используя граничные условия (2) и (3), получаем краевую задачу
\[
    \begin{gathered}
        \quad a \Delta \theta+g(\theta)=\frac{\kappa_{a}}{\alpha} \psi, \quad \Delta \psi=0, \quad x \in \Omega, \\
        a \partial_{n} \theta=q_{b}, \quad \text { on } \Gamma, \quad \alpha \partial_{n} \psi+\gamma \psi=r, \quad \theta=\theta_{ b} \text { on } \Gamma_{1} .
    \end{gathered}
\]
Здесь $g(\theta)=b \kappa_{a}|\theta| \theta^{3}+a \kappa_{a} \theta / \alpha,
r=\alpha b \gamma \theta_{b}^{4}+\alpha q_{b}+a \gamma \theta_{b }$.


Задача оптимального управления, аппроксимирующая задачу (4), (5),
заключается в нахождении тройки $\left\{\theta_{\lambda}, \psi_{\lambda}, u_{\lambda}\right\}$
такой, что

\[
    \begin{gathered}
        J_{\lambda}(\theta, u)=\frac{1}{2} \int_{\Gamma_{1}}\left(\theta-\theta_{b}\right)^{2} d
        \Gamma+\frac{\lambda}{2} \int_{\Gamma_{2}} u^{2} d \Gamma \rightarrow \text{ inf } \\
        -a \Delta \theta+g(\theta)=\frac{\kappa}{\alpha} \psi, \quad \Delta \psi=0, \quad x \in \Omega \\
        a \partial_{n} \theta+s \theta=q_{b}+s \theta_{b},
        \quad \alpha \partial_{n} \psi+\gamma \psi=r \text{ on } \Gamma_{1},
        \quad a \partial_{n} \theta=q_{b}, \quad \alpha \partial_{n} \psi=u \text{ on } \Gamma_{2}.
    \end{gathered}
\]


Здесь $\lambda, s>0$ — регуляризующие параметры.

\subsection{Solvability of the control problem}
Пусть $U=L^{2}\left(\Gamma_{2}\right)$ — пространство управлений,
$H=L^{2}(\Omega), V=H^{1}(\Omega )=W_{2}^{1}(\Omega)$, а $V^{\prime}$ двойственно к $V$.
Тогда мы отождествим $H$ с его двойственным пространством $H^{\prime}$ таким,
что $V \subset H=H^{\prime} \subset V^{\prime}$, и обозначим через $\|\cdot \|$ норму в $H$,
а через $(f, v)$ значение функционала $f \in V^{\prime}$ на элементе $v \in V$,
совпадающее со скалярным произведением в $ H$, если $f \in H$.


Предположим, что выполняются следующие условия:

(i) $a, b, \alpha, \kappa_{a}, \lambda, s=$ Const $>0$,

(ii) $0<\gamma_{0} \leq \gamma \in L^{\infty}\left(\Gamma_{1}\right), \quad \theta_{b},
r \in L^{2}\left(\Gamma_{1}\right), \quad q_{b} \in L^{2}(\Gamma)$.

Пусть $A_{1,2}: V \rightarrow V^{\prime}, B_{1}: L^{2}\left(\Gamma_{1}\right) \rightarrow V^{\prime},
B_{2}: U \rightarrow V^{\prime}$ такое, что для любого
$y, z \in V$, $f, v \in L^{2}\left(\Gamma_{1}\right)$, и $h, w \in U$

\[
    \begin{gathered}
        \left(A_{1} y, z\right)=a(\nabla y, \nabla z)+s \int_{\Gamma_{1}} y z d \Gamma,
        \quad\left(A_{2} y, z\right)=\alpha(\nabla y, \nabla z)+\int_{\Gamma_{1}} \gamma y z d \Gamma, \\
        \left(B_{1} f, v\right)=\int_{\Gamma_{1}} f v d \Gamma, \quad\left(B_{2} h, w\right)=\int_{\Gamma_{2}} h w d \Gamma
    \end{gathered}
\]

Слабая постановка краевой задачи на
решения, функционал (1) которых минимизируется, имеет вид:

\[
    A_{1} \theta+g(\theta)=\frac{\kappa_{a}}{\alpha} \psi+f_{1}, \quad A_{2} \psi=f_{2}+B_{2} u
\]
где $f_{1}=B_{1}\left(q_{b}+s \theta_{b}\right)+B_{2} q_{b}$ и $f_{2}=B_{1} r$.

Определим оператор ограничения $F(\theta, \psi, u): V \times V \times U \rightarrow V^{\prime} \times V^{\prime}$,

\[
    F(\theta, \psi, u)=\left\{A_{1} \theta+g(\theta)
    - \frac{\kappa_{a}}{\alpha} \psi-f_{1}, A_{2} \psi-f_{2}-B_{2} u\right\}
\]

Задача $\left(P_{\lambda}\right)$.
Найдём тройку $\left\{\theta_{\lambda}, \psi_{\lambda}, u_{\lambda}\right\} \in V \times V \times U$
такую, что
\[
    J_{\lambda}(\theta, u)=\frac{1}{2}\left\|\theta-\theta_{b}\right\|_{L^{2}\left(\Gamma_{1}\right)}^{2}
    + \frac{\lambda}{2}\|u\|_{U}^{2} \rightarrow \inf , \quad F(\theta, \psi, u)=0.
\]

\textbf{Теорема 1.}
Пусть выполнены условия (i), (ii).
Тогда существует хотя бы одно решение $\operatorname{задачи}\left(P_{\lambda}\right)$

\subsection{Optimality conditions}
В силу принципа Лагранжа для гладких выпуклых экстремальных задач [2]
невырожденность условий оптимальности гарантируется условием
что образ производной оператора
$F(y, u)$, где $y=\{\theta, \psi\} \in V \times V$, совпадает с $V^{\prime} \times V^{\prime}$.
Это означает, что система


\[
    A_{1} \xi+g^{\prime}(\theta) \xi-\frac{\kappa_{a}}{\alpha} \eta=q_{1}, \quad A_{2} \eta=q_{2}
\]
решаемо для всех $\theta \in V, q_{1}, q_{2} \in V^{\prime}$.
Здесь $g^{\prime}(\theta)=4b\kappa_{a}|\theta|^{3}+\kappa_{a} / \alpha$.
Из второго уравнения получаем $\eta=A_{2}^{-1} q_{2}$.
Разрешимость первого уравнения при известном $\eta \in V$ очевидным образом следует из леммы Лакса-Мильграма.
Справедливость остальных условий принципа Лагранжа очевидна.

Лагранжиан задачи $\left(P_{\lambda}\right)$ имеет вид

\[
    L\left(\theta, \psi, u, p_{1}, p_{2}\right)=J_{\lambda}(\theta, u)
    + \left(A_{1} \theta+g(\theta)-\frac{\kappa_{a}}{\alpha} \psi-f_{1},
    p_{1}\right)+\left(A_{2} \psi-f_{2}-B_{2} u, p_{2}\right),
\]
где $p=\left\{p_{1}, p_{2}\right\} \in V \times V$ — сопряженное состояние.

Пусть $\{\widehat{\theta}, \widehat{\varphi}, \widehat{u}\}$
будет решением задачи $\left(P_{\lambda}\right)$.
По принципу Лагранжа $[2$, Th. 1,5],
получаем следующие равенства $\forall v \in V, w \in U$:
\[
    \begin{gathered}
        \left(B_{1}\left(\widehat{\theta}-\theta_{b}\right), v\right)+\left(A_{1} v
        +g^{\prime}(\widehat{\theta}) v, p_{1}\right)=0,
        -\frac{\kappa_{a}}{\alpha}\left(v, p_{1}\right)+\left(A_{2} v, p_{2}\right)=0, \\
        \lambda\left(B_{2} \widehat{u}, w\right)-\left(B_{2} w, p_{2}\right)=0 .
    \end{gathered}
\]

Из (9), (10) следуют уравнения для сопряженного состояния.

\textbf{Theorem 2.}
Пусть выполняются условия (i), (ii) и
$\{\widehat{\theta}, \widehat{\psi}, \widehat{u}\}$
есть решение $\left(P_{\lambda}\right)$,
тогда есть единственная пара $\left\{p_{1}, p_{2}\right\} \in V \times V$ такая, что
\[
    A_{1} p_{1}+g^{\prime}(\widehat{\theta}) p_{1}=-B_{1}\left(\widehat{\theta}-\theta_{b}\right),
    \quad A_{2} p_{2}=\frac{\kappa_{a}}{\alpha} p_{1}, \quad \lambda \widehat{u}=\left.p_{2}\right|_{\Gamma_{2}}.
\]

\subsection{Аппроксимация обратной задачи}
Докажем, что если пара $\{\theta, \varphi\} \in V \times V$ является решением задачи (1)-(3) и,
кроме того, $q=\left.a \partial_ {n} \varphi\right|_{\Gamma_{2}} \in L^{2}\left(\Gamma_{2}\right)$,
то решения задачи $\left(P_{\lambda} \right)$ как $\lambda \rightarrow+0$ аппроксимирует решение задачи (1)-(3).
Заметим, что пара для всех $v \in V$ удовлетворяет равенствам
\[
    \begin{gathered}
        a(\nabla \theta, \nabla v)+b \kappa_{a}\left(|\theta| \theta^{3}-\varphi, v\right)=\int_{\Gamma} q_{b} v d \Gamma, \\
        \alpha(\nabla \varphi, \nabla v)+\int_{\Gamma_{1}} \gamma \varphi v d
        \Gamma+\kappa_{a}\left(\varphi-|\theta| \theta^{3}, v\right)=\int_{\Gamma_{1}} \gamma \theta_{b}^{4} v d
        \Gamma+\int_{\Gamma_{2}} q v d \Gamma,
    \end{gathered}
\]
и при этом $\left.\theta\right|_{\Gamma_{1}}=\theta_{b}$.

\textbf{Theorem 3.}
Пусть выполнены условия (i), (ii) и существует решение задачи (1)-(3), удовлетворяющее равенствам (12) и (13).
Если $\left\{\theta_{\lambda}, \psi_{\lambda}, u_{\lambda}\right\}$
является решением задачи $\left(P_{\lambda}\right)$ при $\lambda>0$,
тогда существует последовательность $\lambda \rightarrow+0$
такая, что $\theta_{\lambda} \rightarrow \theta_{*},
\left(\psi_{\lambda}-a \theta_{\lambda}\right) / \alpha b \rightarrow \varphi_{*}$
слабо в $V$, сильно в $H$, где $\theta_{*}, \varphi_{*}$ есть решение задачи (1)-(3).
