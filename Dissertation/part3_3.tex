\section{Метод штрафных функций для решения задачи оптимального управления}
\label{sec:ch3:sec3}
%Chebotarev_Park_Mesenev_Kovtanyuk_22.pdf
%Penalty method to solve an optimal control problem for a qasilinear parabolic equation

%
%\subsection{Введение}\label{subsec:ch3:sec3:subsec1}
%
%В настоящей работе рассматривается задача оптимального управления для модели
%эндовенозной лазерной абляции в ограниченной трехмерной области
%$\Omega$ с отражающей границей $\Gamma=\partial\Omega$.
%Проблема состоит в том, чтобы свести к минимуму функционал
%\[
%    J(\theta)=\int_{G_{d}}\left(\left.\theta\right|_{t=T}
%    -\theta_{d}\right)^{2} d x \rightarrow \inf
%\]
%на решениях начально-краевой задачи
%\[
%    \begin{gathered}
%        \sigma \partial \theta / \partial t-\operatorname{div}(k(\theta) \nabla \theta)
%        -\beta \varphi=u_{1} \chi,
%        \quad-\operatorname{div}(\alpha \nabla \varphi)+\beta \varphi=
%        u_{2} \chi, \quad x \in \Omega, \quad 0<t<T, \\
%        k(\theta) \partial_{n} \theta+\left.\gamma
%        \left(\theta-\theta_{b}\right)\right|_{\Gamma}=0,
%        \quad \alpha \partial_{n} \varphi +
%        \left.0.5 \varphi\right|_{\Gamma}=0,\left.\quad \theta\right|_{t=0}=\theta_{0}.
%    \end{gathered}
%\]
%В этом случае устанавливаются следующие ограничения:
%\[
%    u_{1,2} \geq 0, \quad u_{1}+u_{2} \leq P,\left.\quad \theta\right|_{G_{b}}
%     \leq \theta_{*}.
%\]
%Здесь $\theta$ - температура, $\varphi$ - интенсивность излучения, усредненная по всем направлениям,
%$\alpha$ - коэффициент диффузии оптического излучения, $\mu_{a}$ - коэффициент поглощения,
%$k(\theta)$ - коэффициент теплопроводности, $\sigma(x, t)$ произведение удельной теплоемкости
%и объемной плотности, $u_{1}$ описывает мощность источника, затрачиваемую
%на нагрев наконечника волокна, $u_{2}$ - мощность источника, затрачиваемая на излучение,
%$\chi$ равно характеристической функции части среды, в которой расположен наконечник волокна,
%деленной на объем наконечника волокна.
%Функции $\theta_{b}, \theta_{0}$ определяют граничное и начальное распределения температуры.
%Мы обозначим через $\partial_{n}$ производную в направлении внешней нормали $\mathbf{n}$ к границе $\Gamma$.
%Требуется обеспечить близость распределения температуры к желаемому температурному полю $\theta_{d}$
%в конечный момент времени $t=T$ в поддомене $G_{d}$, при этом температура в поддомене $G_{b}$
%не превышает постоянного критического значения значение $\theta_{*}$.

\subsection{Формализация задачи оптимального управления}\label{subsec:ch3:sec3:subsec2}

В дальнейшем мы предполагаем, что $\Omega $ является ограниченной областью Липшица,
$\Gamma =\partial \Omega, Q =\Omega \times(0, T)$, $\Sigma=\Gamma \times(0, T)$.
Обозначим через $L ^ {p}, 1 \leq p \leq \infty$ пространство Лебега и через $H^{1}$
пространство Соболева $W_{2}^{1}$.
Пространство $L ^ {p}(0, T ; X)$ (соответственно, $C([0, T] ; X)$ ) состоит из
$p$-интегрируемых по $(0, T)$ (соответственно, непрерывных по $[0, T])$ функции со значениями
в банаховом пространстве $X$.
Обозначим $H=L ^{2}(\Omega), V=H^{1} (\Omega)$ и $V ^ {\prime}$ двойственное значение $V$.
Затем мы отождествляем $H $ с его двойным пространством $H ^ {\prime}$ таким,
что $V \subset H = H ^ {\prime} \subset V ^ {\prime}$ и обозначаем
через $ \|\cdot \|$ норму в $H$, и на $(h, v)$ значение функционала $h \in V ^ {\prime}$
на элементе $v  \in V$, совпадающее с внутренним произведением в $H$, если $h \in H$.


Пусть выполняются следующие условия:

(i) $0<\sigma_{0} \leq \sigma \leq \sigma_{1},
\quad|\partial \sigma / \partial t| \leq \sigma_{2}, \quad \sigma_{j}=$ Const.

(ii) $0<k_{0} \leq k(s) \leq k_{1}, \quad\left|k^{\prime}(s)\right|
\leq k_{2}, s \in \mathbb{R}, k_{j}=$ Const.

(iii) $\theta_{0} \in H, \gamma \in L^{\infty}(\Gamma),
\gamma \geq \gamma_{0}=$ Const $>0, \quad \theta_{b} \in L^{\infty}(\Sigma),
\quad \theta_{d} \in G_{d}$.

(iv) $0<\alpha_{0} \leq \alpha(x) \leq \alpha_{1}, \quad 0<\beta_{0}
\leq \beta(x) \leq \beta_{1}, \quad x \in \Omega$
Мы определяем нелинейный оператор $A:V\rightarrow V^{\prime}$
и линейный оператор $B:V\rightarrow V ^ {\prime}$,
используя следующее равенство, действительное для любого $\theta, v, \varphi, w \in V$:
\[
    (A(\theta), v)=(k(\theta) \nabla \theta, \nabla v)+\int_{\Gamma} \gamma \theta v d
    \Gamma=(\nabla h(\theta), \nabla v)+\int_{\Gamma} \gamma \theta v d \Gamma,
\]

где%в пизде
\[
    h(s)=\int_{0}^{s} k(r) d r ; \quad(B \varphi, w) =
    (\alpha \nabla \varphi, \nabla w)
    + (\beta \varphi, w)+\frac{1}{2} \int_{\Gamma} \varphi w d \Gamma
\]
Далее, с помощью следующей билинейной формы, мы определяем внутреннее произведение в$V$ :
\[
    (u, v)_{V}=(\nabla u, \nabla v)+\int_{\Gamma} u v d \Gamma.
\]
Соответствующая норма эквивалентна стандартной норме пространства $V$.

\textbf{Определение 1.} Пусть $u_{1,2} \in L^{2}(0, T)$.
Пара $\theta, \varphi \in L^{2}(0, T ; V)$ слабое решение задачи (1), (2) если
$\sigma \theta^{\prime} \in L^{2}\left(0, T ; V^{\prime}\right)$ и
\[
    \sigma \theta^{\prime}+A(\theta)-\beta \varphi=g+u_{1} \chi,
    \quad \theta(0)=\theta_{0}, \quad B \varphi=u_{2} \chi,
\]
где
\[
    \theta^{\prime}=d \theta / d t, \quad g \in L^{\infty}\left(0, T ; V^{\prime}\right),
    \quad(g, v)=\int_{\Gamma} \gamma \theta_{b} v d \Gamma
\]
\textit{Замечание 1.} Так как $(\sigma \theta)^{\prime}=\sigma \theta^{\prime}+\theta
\partial \sigma / \partial t \in L^{2}\left(0, T ; V^{\prime}\right)$ and $\sigma
\theta \in L^{2}(0, T ; V)$, then $\sigma \theta \in$ $C([0, T] ; H)$,
и поэтому начальные условия имеют физические основания.

Из леммы Лакса-Мильграма следует, что для любой функции $g \in H$ существует единственное
решение уравнения $B\varphi= g$.
Более того, обратный оператор $B ^{-1}: H \rightarrow V$ является непрерывным.
Следовательно, мы можем исключить интенсивность излучения $\varphi=u_{2} B ^ {-1} \chi$
и сформулировать задачу оптимального управления следующим образом.
Проблема (CP)
\[
    \begin{gathered}
        J(\theta)=\int_{G_{d}}\left(\left.\theta\right|_{t=T}
        - \theta_{d}\right)^{2} d x \rightarrow \inf,
        \quad \sigma \theta^{\prime}+A(\theta)=g+u, \quad \theta(0)=\theta_{0}, \\
        \left.\theta\right|_{G_{b}} \leq \theta_{*}, \quad u \in U_{a d}.
    \end{gathered}
\]
Здесь
\[
    U_{a d}=\left\{u=u_{1} \chi+u_{2} \beta B^{-1} \chi: u_{1,2} \in L^{2}(0, T), u_{1,2}
    \geq 0, u_{1}+u_{2} \leq P\right\}
\]

\subsection{Предварительные результаты}\label{subsec:ch3:sec3:subsec3}
В статье [5] получен следующий результат.

\textit{Лемма 1.}
Пусть условия (i) - (iv) выполняются и $u \in L^{2}\left(0, T ; V^{\prime}\right)$.
Тогда есть решение проблемы
\[
    \sigma \theta^{\prime}+A(\theta)=g+u, \quad \theta(0)=\theta_{0},
\]
такое что $\theta \in L^{\infty}(0, T ; H)$, а также верна следующая оценка:
\[
    \|\theta(t)\|^{2}+\|\theta\|_{L^{2}\left(0, T ; V^{\prime}\right)}^{2}
    \leq C\left(\left\|\theta_{0}\right\|^{2}+\|g+u\|_{L^{2}
    \left(0, T; V^{\prime}\right)}^{2}\right),
\]

где $C>0$ не зависит от $\theta_{0}, g$, и $u$.

\textit{Lemma 2.}
Пусть условия (i) - (iv) выполняются, $u=0, \theta_{0} \leq \theta_{*}$ другими словами,
в $\Omega, \theta_{b} \leq \theta_{*}$ то есть $\Sigma$, и $\theta$ будут решением задачи (4).
Тогда $\theta \leq \theta_{*}$ в $\Omega \times(0, T)$.

\textit{Доказательство.}
Умножая в смысле внутреннего произведения в $H$ первое уравнение в (4) на
$v=\max \left\{\theta-\theta_{*}, 0\right\}\in L^{2}(0, T; V)$, мы получаем
\[
    \left(\sigma v^{\prime}, v\right)+(k(\theta) \nabla v, \nabla v)
    + \int_{\Gamma} \gamma \theta v d \Gamma=0.
\]

Отбрасывая неотрицательные второе и третье слагаемые, мы приходим к оценке
\[
    \frac{d}{d t}(\sigma v, v) \leq\left(\sigma_{t} v, v\right) \leq \sigma_{2}\|v\|^{2}.
\]

Интегрируя последнее неравенство по времени и принимая во внимание,
что $\left.v\right|_{t=0}=0$, мы получаем
\[
    \sigma_{0}\|v(t)\|^{2} \leq(\sigma v(t), v(t))
    \leq \sigma_{2} \int_{0}^{t}\|v(\tau)\|^{2} d \tau
\]
Основываясь на лемме Гронуолла, мы приходим к выводу, что $v=0$ и, следовательно,
$\theta \leq \theta_{*}$ в $\Omega \times(0, T)$

Леммы 1 и 2 подразумевают непустое множество допустимых пар задачи (CP)
и ограниченность минимизирующей последовательности допустимых пар
$\left\{\theta_{m}, u_{m}\right\} \in L^{2}(0, T ; V) \times$
$U_{a d}$ так, что $J\left(\theta_{m}\right) \rightarrow j=\inf J$, где
\[
    \sigma \theta_{m}^{\prime}+A\left(\theta_{m}\right)=g+u_{m},
    \quad \theta_{m}(0)=\theta_{0},\left.\quad \theta_{m}\right|_{G_{b}} \leq \theta_{*}.
\]
Аналогично [4], переходя к пределу в системе (5), можно установить разрешимость задачи (CP).

\textbf{Theorem 1.}
Пусть условия (i)-(iv) выполняются,
$\theta_{0} \leq \theta_{*}$ a.e. в $\Omega, \theta_{b} \leq \theta_{*}$ a.e. в $\Sigma$.
Тогда решение проблемы (CP) существует.

\subsection{Метод штрафных функций}\label{subsec:ch3:sec3:subsec4}
Рассмотрим следующую задачу оптимального управления с параметром $\varepsilon>0$,
решения которой аппроксимируют решение задачи (CP) как $\varepsilon \rightarrow+0$.
Problem $\left(\mathrm{CP}_{\varepsilon}\right)$
\[
    \begin{gathered}
        J_{\varepsilon}(\theta)=\int_{G_{d}}
        \left(\left.\theta\right|_{t=T}-\theta_{d}\right)^{2} d x
        + \frac{1}{\varepsilon} \int_{0}^{T} \int_{G_{b}} F(\theta) d x d t \rightarrow \inf \\
        \sigma \theta^{\prime}+A(\theta)=g+u, \quad \theta(0)=\theta_{0}, \quad u \in U_{a d}
    \end{gathered}
\]
Здесь,
\[
    F(\theta)=
    \begin{cases}
        0, & \text { if } \theta \leq \theta_{*}, \\
        \left(\theta-\theta_{*}\right)^{2}, & \text { if } \theta>\theta_{*}
    \end{cases}
\]
Оценки, представленные в лемме 1, позволяют, аналогично
доказательству теоремы 1, доказать разрешимость задачи со штрафом.
\textbf{Теорема 2.} Пусть выполняются условия (i)-(iv).
Тогда существует решение проблемы $\left(C P_{\varepsilon}\right)$.

Рассмотрим аппроксимативные свойства решений задачи со штрафом.
Пусть $\left\{\theta_{\varepsilon}, u_{\varepsilon}\right\}$ будет решением проблемы
$\left(\mathrm{CP}_{\varepsilon}\right)$ и $\{\theta, u\}$ будет решением проблемы(CP).
Тогда,
\[
    \sigma \theta_{\varepsilon}^{\prime}+A\left(\theta_{\varepsilon}\right)=g+u_{\varepsilon},
    \quad \theta_{\varepsilon}(0)=\theta_{0}.
\]

так как $\left.\theta\right|_{G_{b}} \leq \theta_{*}$, верны следующие неравенства:
\[
    \int_{G_{d}}\left(\left.\theta_{\varepsilon}\right|_{t=T}-\theta_{d}\right)^{2} d x \leq J(\theta),
    \quad \int_{0}^{T} \int_{G_{b}} F\left(\theta_{\varepsilon}\right) d x d t \leq \varepsilon J(\theta).
\]

Из полученных оценок, используя при необходимости подпоследовательности в качестве
$\varepsilon \rightarrow+0$, аналогично, как и в доказательстве теоремы 1,
мы можем доказать существование функций
$\widehat{u} \in U_{a d}, \widehat{\theta} \in L^{2}(0, T ; V)$ таких, что

$u_{\varepsilon} \rightarrow \widehat{u}$ слабо в
$L^{2}(0, T ; H), \theta_{\varepsilon} \rightarrow \widehat{\theta}$ слабо в $L^{2}(0, T ; V)$,
сильно в $L^{2}(0, T ; H)$;
\[
    \int_{0}^{T} \int_{G_{b}} F\left(\theta_{\varepsilon}\right) d x d t \rightarrow \int_{0}^{T}
    \int_{G_{b}} F(\widehat{\theta}) dx dt \quad \text { и } \quad \int_{0}^{T} \int_{G_{b}}
    F\left(\theta_{\varepsilon}\right) dx dt \rightarrow 0, \text { как } \varepsilon \rightarrow+0
\]

Следовательно, $F(\widehat{\theta})=0$ и $\left.\widehat{\theta}\right|_{G_{b}} \leq \theta_{*}$.
Результатов сходимости достаточно, чтобы перейти к пределу как $\varepsilon \rightarrow+0$
в системе состояний (6) и доказать, что предельная пара
$\{\widehat{\theta}, \widehat{u}\} \in$ $L^{2}(0, T ; V) \times U_{a d}$
является приемлемым для проблемы(CP).
Поскольку функционал $J$ является слабо полунепрерывным снизу, то есть
\[
    j \leq J(\widehat{\theta}) \leq \liminf J\left(\theta_{\varepsilon}\right) \leq J(\theta)=j=\inf J
\]
Тогда пара $\{\widehat{\theta}, \widehat{u}\}$ это решение проблемы $(\mathrm{CP})$.
\textbf{Теорема 3.} Пусть выполняются условия (i)-(iv),
$\theta_{0} \leq \theta_{*}$ a.e. в $\Omega, \theta_{b} \leq \theta_{*}$ a.e.
в
$\Sigma$.
If $\left\{\theta_{\varepsilon}, u_{\varepsilon}\right\}$  решения проблемы
$\left(C P_{\varepsilon}\right)$ for $\varepsilon>0$, тогда существует последовательность вида
$\varepsilon \rightarrow+0$ $u_{\varepsilon} \rightarrow \widehat{u}$ слабо в
$L^{2}(0, T ; H), \quad \theta_{\varepsilon} \rightarrow \widehat{\theta}$
слабо в $L^{2}(0, T ; V)$,
сильно в $L^{2}(0, T ; H)$, where $\{\widehat{\theta}, \widehat{u}\}$ есть решение проблемы (CP).
