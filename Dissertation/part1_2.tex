\section{Диффузионное $P_{1}$ приближение уравнения переноса излучения}
\label{sec:ch1/sec2}

$P_{1}$ приближение уравнения переноса излучения является частным случаем метода
сферических гармоник $\left(P_{N}\right)$.
Идея $P_{N}$ приближений состоит в том,
что функцию интенсивности излучения $I(x, \omega)$
раскладывают в ряд Фурье по сферическим гармоникам
$\mathcal{Y}_{l}^{m}(\boldsymbol{\omega})$~\cite[496]{modest2013radiative}:
\[
    I(x, \omega)=\sum_{l=0}^{\infty} \sum_{m=-l}^{l} I_{l}^{m}(x)
    \mathcal{Y}_{l}^{m}(\omega),
\]
где $I_{l}^{m}(x)$ - коэффициенты, зависящие от $x$.
Также в ряд раскладывают фазовую функцию $P\left(\omega, \omega^{\prime}\right)$.
Тогда решение уравнения переноса излучения ищется в виде отрезка ряда Фурье для $l \leqslant N$.
При подстановке указанной конечной суммы в исходное
уравнение интегро-дифференциальное уравнение
переноса излучения относительно $I(x, \omega)$ сводится
к $(N+1)^{2}$ дифференциальным уравнениям
относительно $I_{l}^{m}(x)$.


В $P_{1}$ приближении используется линейное приближение
для интенсивности излучения и фазовой функции:
\begin{gather}
    I^{*}(x, \omega, t) = \varphi(x, t)
    +\boldsymbol{\Phi}(x, t) \cdot \omega, \label{eq:1_2:14}\\
    P\left(\omega, \omega^{\prime}\right)= 1
    + A \omega \cdot \omega^{\prime}. \label{eq:1_2:15}
\end{gather}
Для фазовой функции выполняется условие нормировки:
\[
    \frac{1}{4 \pi} \int_{S} P\left(\omega, \omega^{\prime}\right) d \omega=1+\frac{A}{4 \pi}
    \int_{S} \omega \cdot \omega^{\prime} d \omega=1,
\]
вычисление интеграла см.\ ниже.
Коэффициент $A \in[-1,1]$ описывает анизотропию рассеяния,
а величина $A / 3$ имеет смысл среднего косинуса угла рассеяния, поскольку
\[
    \frac{1}{4 \pi} \int_{S}\left(\omega \cdot \omega^{\prime}\right)
    P\left(\omega, \omega^{\prime}\right) d \omega=\frac{1}{4 \pi}
    \int_{S} \omega \cdot \omega^{\prime} d \omega+\frac{A}{4 \pi}
    \int_{S}\left(\omega \cdot \omega^{\prime}\right)
    \left(\omega \cdot \omega^{\prime}\right) d \omega=\frac{A}{3},
\]
вычисление интегралов см.\ ниже.
Случай $A=0$ соответствует изотропному рассеянию.
Диапазон допустимых значений величины $A \in[-1,1]$ обусловлен тем, что при $|A|>1$
фазовая функция может принимать отрицательные значения.

Отметим, что если функция $I^*$ ищется в виде~\eqref{eq:1_2:14},
то~\cite[502]{modest2013radiative}:
\begin{gather*}
    G(x, t)=\int_{S} I^{*}(x, \omega, t) d \omega=4 \pi \varphi(x, t), \\
    \mathbf{q}_{r}(x, t)=\int_{S} I^{*}(x, \boldsymbol{\omega}, t)
    \boldsymbol{\omega} d \boldsymbol{\omega}=\frac{4 \pi}{3} \boldsymbol{\Phi}(x, t),
\end{gather*}
поэтому
\[
    \varphi(x, t)=\frac{1}{4 \pi} G(x, t),
    \quad \Phi(x, t)=\frac{3}{4 \pi} \mathbf{q}_{r}(x, t),
\]
где $G$ - аппроксимация пространственной плотности падающего излучения,
$\mathbf{q}_{r}-$ аппроксимация плотности потока излучения.
Следовательно, функция $\varphi(x, t)$ имеет физический смысл
нормализованной интенсивности излучения в
точке $x$ в момент времени $t$, усредненной по всем направлениям.

\begin{lemma}
    \label{lm:1:1}
    Справедливы равенства:
    \[
        \begin{gathered}
            \int_{S} 1 \cdot d \omega=4 \pi, \quad \int_{S} \mathbf{a}
            \cdot \boldsymbol{\omega} d \omega=0,
            \quad \int_{S}(\mathbf{a} \cdot \boldsymbol{\omega})(\mathbf{b} \cdot \boldsymbol{\omega}) d
            \omega=\frac{4 \pi}{3} \mathbf{a} \cdot \mathbf{b}, \\
            \int_{S} \omega d \omega=0, \quad \int_{S}(\mathbf{a} \cdot \omega) \omega d
            \omega=\frac{4 \pi}{3} \mathbf{a}, \\
            \int_{\boldsymbol{\omega} \cdot \mathbf{a}>0} \mathbf{a} \cdot \boldsymbol{\omega} d
            \boldsymbol{\omega}=\pi, \quad \int_{\omega \cdot \mathbf{a}>0}(\mathbf{a} \cdot
            \boldsymbol{\omega})(\mathbf{b} \cdot \boldsymbol{\omega}) d \omega
            =\frac{2 \pi}{3} \mathbf{a} \cdot \mathbf{b},
        \end{gathered}
    \]
    где $a, b$ -- любые векторы.
\end{lemma}
\begin{proof}
    Первое равенство вытекает из определения поверхностного интеграла
    и представляет собой выражение для площади
    поверхности единичной сферы.


    Для вычисления остальных интегралов воспользуемся
    формулой перехода от поверхностного интеграла
    к двойному
    \begin{equation}
        \label{eq:1_2:16}
        \int_{S} f(\omega) d \omega=\int_{D} f
        \left(\omega_{1}(u, v), \omega_{2}(u, v),
        \omega_{3}(u, v)\right)\left|\boldsymbol{\omega}_{u}
        \times \boldsymbol{\omega}_{v}\right| d u d v,
    \end{equation}

    где $D=\left\{(u, v): 0 \leqslant u \leqslant 2 \pi,-\frac{\pi}{2}
    \leqslant v \leqslant \frac{\pi}{2}\right\},
    \omega_{1}(u, v)=\cos u \cos v, \omega_{2}(u, v)=$ $\sin u \cos v,
    \omega_{3}(u, v)=\sin v,\left|\omega_{u}
    \times \omega_{v}\right| d u d v=\cos v d u d v-$ элемент площади поверхности единичной сферы.
    Тогда для вычисления остальных интегралов воспользуемся
    формулой перехода от поверхностного интеграла
    к двойному:
    \[
        \int_{S} f(\omega) d \omega=\int_{D} f\left(\omega_{1}(u, v), \omega_{2}(u, v),
        \omega_{3}(u, v)\right)\left|\boldsymbol{\omega}_{u} \times \boldsymbol{\omega}_{v}\right| d u d v
    \]
    где $D=\left\{(u, v): 0 \leqslant u \leqslant 2 \pi,-\frac{\pi}{2} \leqslant v \leqslant
    \frac{\pi}{2}\right\}, \omega_{1}(u, v)=\cos u \cos v, \omega_{2}(u, v)=$ $\sin u \cos v,
    \omega_{3}(u, v)=\sin v,\left|\omega_{u} \times \omega_{v}\right| d u d v=\cos v d u d v$
    -- элемент площади поверхности единичной сферы.
    Тогда
    \[
        \int_{S} f(\omega) d \omega=\int_{D} f\left(\omega_{1}(u, v), \omega_{2}(u, v),
        \omega_{3}(u, v)\right)\left|\boldsymbol{\omega}_{u}
        \times \boldsymbol{\omega}_{v}\right| d u d v.
    \]

    Для вычисления второго интеграла положим
    $f(\boldsymbol{\omega})=\mathbf{a}
    \cdot \boldsymbol{\omega}=\sum_{i=1}^{3} a_{i} \boldsymbol{\omega}_{i}$,

    \[
        \begin{aligned}
            & \int_{S} \mathbf{a} \cdot \boldsymbol{\omega} d \boldsymbol{\omega}=
            \int_{-\pi / 2}^{\pi / 2}
            \int_{0}^{2 \pi}\left(a_{1} \cos u \cos v+
            a_{2} \sin u \cos v+a_{3} \sin v\right) \cos v d u d v=0. \\
            & \text { получим } \\
            & \text { В третьем интеграле положим }
            f(\boldsymbol{\omega})=(\mathbf{a} \cdot \boldsymbol{\omega})(\mathbf{b}
            \cdot \boldsymbol{\omega})=\sum_{i, j=1}^{3} a_{i} b_{j}
            \boldsymbol{\omega}_{i} \boldsymbol{\omega}_{j} \text {, } \\
            & \int_{S}(\mathbf{a} \cdot \boldsymbol{\omega})(\mathbf{b}
            \cdot \boldsymbol{w}) d \boldsymbol{w}= \\
            & =\int_{-\pi / 2}^{\pi / 2} \int_{0}^{2 \pi}
            \mathbf{a}^{T}\left(\begin{array}{ccc}
                                    \cos ^{2} u \cos ^{2} v & \sin u \cos u \cos ^{2} v
                                    & \cos u \sin v \cos v \\\sin u \cos u \cos ^{2} v
                                    & \sin ^{2} u \cos ^{2} v
                                    & \sin u \sin v \cos v \\\cos u \sin v \cos v
                                    & \sin u \sin v \cos v & \sin ^{2} v
            \end{array}\right) \mathbf{b} \cos v d u d v= \\
            & =\int_{-\pi / 2}^{\pi / 2} \mathbf{a}^{T}\left(
            \begin{array}{ccc}
                \pi \cos ^{2} v & 0 & 0 \\0 & \pi \cos ^{2} v & 0 \\0 & 0 & 2 \pi \sin ^{2} v
            \end{array}\right) \mathbf{b} \cos v d v=\frac{4 \pi}{3} \mathbf{a} \cdot \mathbf{b}
        \end{aligned}
    \]
    здесь $\mathbf{a}, \mathbf{b}$ - векторы-столбцы.


    Равенства во второй строке получаются из доказанных равенств:
    \[
        \begin{gathered}
            \int_{S} \omega d \boldsymbol{\omega}=\sum_{i=1}^{3} \mathbf{e}_{i}
            \int_{S}\left(\boldsymbol{\omega}
            \cdot \mathbf{e}_{i}\right) d \boldsymbol{\omega}=0, \\
            \int_{S}(\mathbf{a} \cdot \boldsymbol{\omega}) \omega d
            \boldsymbol{\omega}=\sum_{i=1}^{3} \mathbf{e}_{i}
            \int_{S}(\mathbf{a} \cdot \boldsymbol{\omega})\left(\boldsymbol{\omega}
            \cdot \mathbf{e}_{i}\right) d
            \boldsymbol{\omega}=\frac{4 \pi}{3} \sum_{i=1}^{3}\left(\mathbf{a}
            \cdot \mathbf{e}_{i}\right)
            \mathbf{e}_{i}=\frac{4 \pi}{3} \mathbf{a}.
        \end{gathered}
    \]


    Для доказательства первого равенства в третьей строке введем систему
    координат так, чтобы ось $O z$ была сонаправлена с вектором а.
    Воспользуемся формулой~\eqref{eq:1_2:16}, в которой вместо $S$ следует
    взять верхнюю полусферу, $D=$ $\left\{(u, v): 0 \leqslant u
    \leqslant 2 \pi, 0 \leqslant v \leqslant \frac{\pi}{2}\right\}$.
    Заметим, что $f(\omega)=\mathbf{a} \cdot \boldsymbol{\omega}=|\mathbf{a}| \sin v$.

    Таким образом,
    \[
        \int_{\omega \cdot \mathbf{a}>0} \mathbf{a} \cdot \omega d \omega=|\mathbf{a}|
        \int_{0}^{\pi / 2} \int_{0}^{2 \pi} \sin v \cos v d u d v =
        2 \pi \int_{0}^{\pi / 2} \sin v \cos v d v=\pi.
    \]
    Для доказательства второго равенства в третьей строке заметим, что
    \[
        \begin{gathered}
            \int_{S}(\mathbf{a} \cdot \boldsymbol{\omega})(\mathbf{b}
            \cdot \boldsymbol{\omega}) d
            \boldsymbol{\omega}=\int_{\boldsymbol{\omega} \cdot
            \mathbf{a}>0}(\mathbf{a} \cdot \boldsymbol{\omega})(\mathbf{b}
            \cdot \mathbf{\omega}) d
            \boldsymbol{\omega}+
            \int_{\boldsymbol{\omega} \cdot \mathbf{a}<0}(\mathbf{a}
            \cdot \boldsymbol{\omega})(\mathbf{b}
            \cdot \boldsymbol{\omega}) d \boldsymbol{\omega}, \\
            \int_{\omega \cdot \mathbf{a}>0}(\mathbf{a}
            \cdot \boldsymbol{\omega})(\mathbf{b}
            \cdot \boldsymbol{\omega}) d
            \boldsymbol{\omega}=\int_{\boldsymbol{\omega}
            \cdot \mathbf{a}<0}(\mathbf{a}
            \cdot \boldsymbol{\omega})(\mathbf{b}
            \cdot \boldsymbol{\omega}) d \boldsymbol{\omega},
        \end{gathered}
    \]
    следовательно,
    $
    \int_{\boldsymbol{\omega} \cdot \mathbf{a}>0}(\mathbf{a}
    \cdot \boldsymbol{\omega})(\mathbf{b}
    \cdot \boldsymbol{\omega}) d \boldsymbol{\omega}=
    \frac{1}{2} \int_{S}(\mathbf{a}
    \cdot \boldsymbol{\omega})(\mathbf{b}
    \cdot \boldsymbol{\omega}) d \boldsymbol{\omega}=
    \frac{2 \pi}{3} \mathbf{a} \cdot \mathbf{b}.
    $
    Лемма доказана.
\end{proof}

Подставляя~\eqref{eq:1_2:14}\eqref{eq:1_2:15} в~\eqref{eq:1_1:9}, получаем
\begin{gather*}
    \frac{1}{c}\left(\frac{\partial \varphi(x, t)}{\partial t}
    + \boldsymbol{\omega} \cdot \frac{\partial \Phi(x, t)}{\partial t}\right)
    + \boldsymbol{\omega} \cdot \nabla \varphi(x, t) \\
    + \boldsymbol{\omega} \cdot \nabla_{x}(\Phi(x, t) \cdot \boldsymbol{\omega})
    + \kappa \varphi(x, t)
    + \mathrm{\kappa} \Phi(x, t) \cdot \boldsymbol{\omega}= \\
    \frac{\kappa_{s}}{4 \pi} \int_{S}\left(1+A \omega \cdot
    \boldsymbol{\omega}^{\prime}\right)\left(\varphi(x, t)+\boldsymbol{\Phi}(x, t)
    \cdot \boldsymbol{\omega}^{\prime}\right) d \boldsymbol{\omega}^{\prime}
    + \mathrm{\kappa}_{a} \theta^{4}(x, t).
\end{gather*}

С учетом равенств
\begin{gather*}
    \int_{S} \Phi(x, t) \cdot \omega^{\prime} d \omega^{\prime} = 0,
    \quad \int_{S} \omega \cdot \omega^{\prime} d \omega^{\prime} = 0, \\
    \quad \int_{S}\left(\Phi(x, t) \cdot \omega^{\prime}\right)\left(\omega \cdot
    \omega^{\prime}\right) d \omega^{\prime}=\frac{4 \pi}{3} \Phi(x, t) \cdot \omega
\end{gather*}

имеем

\begin{gather*}
    \frac{1}{c}\left(\frac{\partial \varphi(x, t)}{\partial t}
    + \omega \cdot \frac{\partial \Phi(x, t)}{\partial t}\right) \\
    + \boldsymbol{\omega} \cdot \nabla \varphi(x, t)
    + \boldsymbol{\omega} \cdot \nabla_{x}
    (\boldsymbol{\Phi}(x, t) \cdot \boldsymbol{\omega})
    + \boldsymbol{\kappa} \varphi(x, t)+\mathrm{\kappa}
    \boldsymbol{\Phi}(x, t) \cdot \boldsymbol{\omega} = \\
    = \mathrm{k}_{s}\left(\varphi(x, t)+\frac{A}{3} \boldsymbol{\Phi}(x, t) \cdot
    \boldsymbol{\omega}\right)+\mathrm{\kappa}_{a} \theta^{4}(x, t),
\end{gather*}

или

\begin{equation}
    \label{eq:1_2:17}
    \begin{aligned}
        &\frac{1}{c}\left(\frac{\partial \varphi(x, t)}{\partial t}
        + \omega \cdot \frac{\partial \Phi(x, t)}{\partial t}\right)
        + \omega \cdot \nabla \varphi(x, t)
        + \omega \cdot \nabla_{x}(\Phi(x, t) \cdot \omega) \\
        &+ \mathrm{\kappa}_{a} \varphi(x, t)+\left(\mathrm{\kappa}_{a}
        + \mathrm{\kappa}_{s}^{\prime}\right)
        \Phi(x, t) \cdot \omega = \mathrm{\kappa}_{a} \theta^{4}(x, t),
    \end{aligned}
\end{equation}

где $\kappa_{s}^{\prime}=\kappa_{s}(1-A / 3)$ - приведенный коэффициент рассеяния.


Проинтегрируем уравнение~\eqref{eq:1_2:17} по $\omega \in S$.
Получим
\begin{equation}
    \label{eq:1_2:18}
    \frac{1}{c} \frac{\partial \varphi(x, t)}{\partial t}
    + \frac{1}{3} \operatorname{div} \Phi(x, t)
    + \mathrm{\kappa}_{a} \varphi(x, t)
    = \mathrm{\kappa}_{a} \theta^{4}(x, t),
\end{equation}
так как
\[
    \begin{gathered}
        \int_{S} \omega \cdot \nabla_{x}(\Phi(x, t)
        \cdot \boldsymbol{\omega}) d \boldsymbol{\omega}=\sum_{i=1}^{3}
        \int_{S}\left(\boldsymbol{\omega}
        \cdot \mathbf{e}_{i}\right)\left(\omega \cdot
        \frac{\partial \Phi(x, t)}{\partial x_{i}}\right) d \boldsymbol{\omega} = \\
        = \frac{4 \pi}{3} \sum_{i=1}^{3} \frac{\partial \Phi(x, t)}{\partial x_{i}} \cdot
        \mathbf{e}_{i}=\frac{4 \pi}{3} \sum_{i=1}^{3}
        \frac{\partial \Phi_{i}(x, t)}{\partial x_{i}}=
        \frac{4 \pi}{3} \operatorname{div} \boldsymbol{\Phi}(x, t).
    \end{gathered}
\]

Умножим уравнение~\eqref{eq:1_2:17} на $\omega:$

\[
    \begin{aligned}
        \frac{1}{c} \frac{\partial \varphi(x, t)}{\partial t} \omega+&
        \frac{1}{c}\left(\omega \cdot \frac{\partial \Phi(x, t)}{\partial t}\right)
        \omega+(\omega \cdot \nabla \varphi(x, t)) \omega
        +\left(\omega \cdot \nabla_{x}(\Phi(x, t)
        \cdot \omega)\right) \omega+\\
        &+\kappa_{a} \varphi(x, t) \omega+\left(\kappa_{a}
        +\kappa_{s}^{\prime}\right)(\Phi(x, t)
        \cdot \omega) \omega=\kappa_{a} \theta^{4}(x, t) \omega
    \end{aligned}
\]
и проинтегрируем полученное равенство по $\omega \in S$.
Для вычисления четвертого слагаемого представим интеграл по единичной сфере $S$
как сумму интегралов по верхней $S_{1}$ и нижней $S_{2}$ полусферам и воспользуемся тем, что
\[
    \int_{S_{2}}\left(\omega \cdot \nabla_{x}(\Phi(x, t)
    \cdot \omega)\right) \omega d
    \omega=-\int_{S_{1}}\left(\omega \cdot \nabla_{x}(\Phi(x, t)
    \cdot \omega)\right) \omega d \omega,
\]
следовательно, интеграл равен 0.
Таким образом,
\begin{equation}
    \label{eq:1_2:19}
    \frac{1}{c} \frac{\partial \Phi(x, t)}{\partial t}
    + \left(\kappa_{a}+\kappa_{s}^{\prime}\right)
    \Phi(x, t)+\nabla \varphi(x, t)=0.
\end{equation}
Итак, уравнения~\eqref{eq:1_2:18}\eqref{eq:1_2:19}
представляют собой $P_{1}$ приближение
для уравнения переноса излучения.
Дальнейшие преобразования основываются на предположении, что выполняется закон Фика:
\begin{equation}
    \label{eq:1_2:20}
    \Phi(x, t)=-3 \alpha \nabla \varphi(x, t),
\end{equation}
где $\alpha=\frac{1}{3\left(\kappa_{a}+\kappa_{s}^{\prime}\right)}
=\frac{1}{3 \kappa-A \kappa_{s}}$.
Фактически мы пренебрегаем производной
$\frac{\partial \Phi}{\partial t}$ в уравнении~\eqref{eq:1_2:19}.
Подставив~\eqref{eq:1_2:20} в~\eqref{eq:1_2:18}, получим
\begin{equation}
    \label{eq:1_2:21}
    \frac{1}{c} \frac{\partial \varphi(x, t)}{\partial t}-\alpha \Delta
    \varphi(x, t)+\kappa_{a}\left(\varphi(x, t)-\theta^{4}(x, t)\right)=0.
\end{equation}


Чтобы получить уравнение для температуры, подставим~\eqref{eq:1_2:14} в~\eqref{eq:1_1:10}.
Получим
\begin{equation}
    \label{eq:1_2:22}
    \frac{\partial \theta(x, t)}{\partial t}-a \Delta \theta(x, t)+\mathbf{v}(x, t) \cdot
    \nabla \theta(x, t)+b \kappa_{a}\left(\theta^{4}(x, t)-\varphi(x, t)\right)=\frac{b}{c}
    \frac{\partial \varphi(x, t)}{\partial t}.
\end{equation}
Учитывая~\eqref{eq:1_2:21}, уравнение~\eqref{eq:1_2:22} можно записать в виде с кросс-диффузией:
\[
    \frac{\partial \theta(x, t)}{\partial t}-a \Delta \theta(x, t)+\mathbf{v}(x, t) \cdot
    \nabla \theta(x, t)=b \alpha \Delta \varphi(x, t).
\]

В дальнейшем вместо уравнения~\eqref{eq:1_2:22} будем использовать уравнение
с нулевой правой частью (см., например~\cite{frank2010optimal})
\begin{equation}
    \label{eq:1_2:23}
    \frac{\partial \theta(x, t)}{\partial t}-a \Delta \theta(x, t)+\mathbf{v}(x, t) \cdot \nabla
    \theta(x, t)+b \kappa_{a}\left(\theta^{4}(x, t)-\varphi(x, t)\right)=0.
\end{equation}

Далее выведем граничные условия типа Маршака для $P_{1}$ приближения (см.\ \cite{Marshak1947}).
Для этого подставим~\eqref{eq:1_2:14} в граничное условие~\eqref{eq:1_1:11}:
\[
    \begin{gathered}
        \varphi(x, t)+\boldsymbol{\Phi}(x, t) \cdot \boldsymbol{\omega}=\varepsilon(x)
        \theta_{b}^{4}(x, t)+\rho^{s}(x)\left(\varphi(x, t)+\boldsymbol{\Phi}(x, t)
        \cdot \boldsymbol{\omega}_{R}\right)+ \\
        +\frac{\rho^{d}(x)}{\pi} \int_{\omega^{\prime} \cdot
        \mathbf{n}>0}\left(\varphi(x, t)+\boldsymbol{\Phi}(x, t) \cdot
        \boldsymbol{\omega}^{\prime}\right) \omega^{\prime} \cdot \mathbf{n} d \omega^{\prime}, \\
        \quad \boldsymbol{\omega} \cdot \mathbf{n}<0, \quad
        \boldsymbol{\omega}_{R}=\boldsymbol{\omega}-2(\boldsymbol{\omega} \cdot
        \mathbf{n}) \mathbf{n}.
    \end{gathered}
\]
Для вычисления интеграла применим лемму 1:
\[
    \begin{aligned}
        \varphi(x, t)+\boldsymbol{\Phi}(x, t) \cdot \boldsymbol{\omega} & =
        \varepsilon(x) \theta_{b}^{4}(x, t)+\rho^{s}(x)[\varphi(x, t) + \\
        &+\boldsymbol{\Phi}(x, t) \cdot \boldsymbol{\omega}-2(\boldsymbol{\omega}
        \cdot \mathbf{n})(\boldsymbol{\Phi}(x, t) \cdot \mathbf{n})]+\\
        &+\rho^{d}(x)\left(\varphi(x, t)+\frac{2}{3} \boldsymbol{\Phi}(x, t)
        \cdot \mathbf{n}\right), \quad \boldsymbol{\omega} \cdot \mathbf{n}<0.
    \end{aligned}
\]
Умножим данное равенство на $\boldsymbol{\omega} \cdot \mathbf{n}$ и проинтегрируем по множеству
входящих направлений, для которых $\boldsymbol{\omega} \cdot \mathbf{n}<0$.
Получим
\[
    \begin{gathered}
        -\pi \varphi(x, t)+\frac{2 \pi}{3} \boldsymbol{\Phi}(x, t) \cdot \mathbf{n}=
        -\pi \varepsilon(x) \theta_{b}^{4}(x, t)-\pi \rho^{s}(x) \varphi(x, t)+\frac{2 \pi \rho^{s}(x)}{3}
        \boldsymbol{\Phi}(x, t) \cdot \mathbf{n}- \\
        -\frac{4 \pi \rho^{s}(x)}{3} \boldsymbol{\Phi}(x, t) \cdot \mathbf{n}-\pi
        \rho^{d}(x)\left(\varphi(x, t)+\frac{2}{3} \boldsymbol{\Phi}(x, t) \cdot \mathbf{n}\right),
    \end{gathered}
\]
или
\[
    \varepsilon(x) \varphi(x, t)=\varepsilon(x) \theta_{b}^{4}(x, t)+
    \frac{2(2-\varepsilon(x))}{3} \boldsymbol{\Phi}(x, t) \cdot \mathbf{n}.
\]
Воспользуемся равенством~\eqref{eq:1_2:20}, будем иметь
\begin{equation}
    \label{eq:1_2:24}
    \alpha \frac{\partial \varphi(x, t)}{\partial n}+
    \gamma(x)\left(\varphi(x, t)-\theta_{b}^{4}(x, t)\right)=0,
\end{equation}
где $\gamma=\frac{\varepsilon}{2(2-\varepsilon)}$.
Отметим, что на участках втекания и вытекания среды
можно принять $\gamma=1/2$~\cite{JVM-14}.


Дополним полученные соотношения граничным условием для температуры~\eqref{eq:1_1:12}:
\begin{equation}
    \label{eq:1_2:25}
    a \frac{\partial \theta(x, t)}{\partial n}
    +\beta(x)\left(\theta(x, t)-\theta_{b}(x, t)\right)=0
\end{equation}
и начальными условиями
\begin{equation}
    \label{eq:1_2:26}
    \theta(x, 0)=\theta_{0}(x), \quad \varphi(x, 0)=\varphi_{0}(x).
\end{equation}
Соотношения~\eqref{eq:1_2:21}\eqref{eq:1_2:23}\eqref{eq:1_2:24}--\eqref{eq:1_2:26}
образуют диффузионную модель сложного теплообмена.

\FloatBarrier
