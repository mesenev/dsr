\section{Диффузионное $P_{1}$ приближение уравнения переноса излучения}
\label{sec:ch1/sec2}

$P_{1}$ приближение уравнения переноса излучения является частным случаем метода
сферических гармоник $\left(P_{N}\right)$.
Идея $P_{N}$ приближений состоит в том,
что функцию интенсивности излучения $I(x, \omega)$
раскладывают в ряд Фурье по сферическим гармоникам
$\mathcal{Y}_{l}^{m}(\boldsymbol{\omega})$~\cite[496]{modest2013radiative}
\[
    I(x, \omega)=\sum_{l=0}^{\infty} \sum_{m=-l}^{l} I_{l}^{m}(x)
    \mathcal{Y}_{l}^{m}(\omega),
\]
где $I_{l}^{m}(x)$ -- коэффициенты, зависящие от $x$.
Также в ряд раскладывают фазовую функцию $P\left(\omega, \omega^{\prime}\right)$.
Тогда решение уравнения переноса излучения ищется в виде отрезка ряда Фурье для $l \leqslant N$.
При подстановке указанной конечной суммы в исходное
уравнение интегро-дифференциальное уравнение
переноса излучения относительно $I(x, \omega)$ сводится
к $(N+1)^{2}$ дифференциальным уравнениям
относительно $I_{l}^{m}(x)$.


В $P_{1}$ приближении используется линейное приближение
для интенсивности излучения и фазовой функции:
\begin{gather}
    I^{*}(x, \omega, t) = \varphi(x, t)
    +\boldsymbol{\Phi}(x, t) \cdot \omega, \label{eq:1_2:14}\\
    P\left(\omega, \omega^{\prime}\right)= 1
    + A \omega \cdot \omega^{\prime}. \label{eq:1_2:15}
\end{gather}
Для фазовой функции выполняется условие нормировки
\[
    \frac{1}{4 \pi} \int_{S} P\left(\omega, \omega^{\prime}\right) d \omega=1+\frac{A}{4 \pi}
    \int_{S} \omega \cdot \omega^{\prime} d \omega=1.
\]
Коэффициент $A \in[-1,1]$ описывает анизотропию рассеяния,
а величина $A / 3$ имеет смысл среднего косинуса угла рассеяния, поскольку
\[
    \frac{1}{4 \pi} \int_{S}\left(\omega \cdot \omega^{\prime}\right)
    P\left(\omega, \omega^{\prime}\right) d \omega=\frac{1}{4 \pi}
    \int_{S} \omega \cdot \omega^{\prime} d \omega+\frac{A}{4 \pi}
    \int_{S}\left(\omega \cdot \omega^{\prime}\right)
    \left(\omega \cdot \omega^{\prime}\right) d \omega=\frac{A}{3}.
\]
Случай $A=0$ соответствует изотропному рассеянию.
Диапазон допустимых значений величины $A \in[-1,1]$ обусловлен тем, что при $|A|>1$
фазовая функция может принимать отрицательные значения.
Отметим, что если функция $I^*$ ищется в виде~\eqref{eq:1_2:14},
то~\cite[502]{modest2013radiative}:
\begin{gather*}
    G(x, t)=\int_{S} I^{*}(x, \omega, t) d \omega=4 \pi \varphi(x, t), \\
    \mathbf{q}_{r}(x, t)=\int_{S} I^{*}(x, \boldsymbol{\omega}, t)
    \boldsymbol{\omega} d \boldsymbol{\omega}=\frac{4 \pi}{3} \boldsymbol{\Phi}(x, t),
\end{gather*}
поэтому
\[
    \varphi(x, t)=\frac{1}{4 \pi} G(x, t),
    \quad \Phi(x, t)=\frac{3}{4 \pi} \mathbf{q}_{r}(x, t),
\]
где $G$ -- аппроксимация пространственной плотности падающего излучения,
$\mathbf{q}_{r}-$ аппроксимация плотности потока излучения.
Следовательно, функция $\varphi(x, t)$ имеет физический смысл
нормализованной интенсивности излучения в
точке $x$ в момент времени $t$, усредненной по всем направлениям.
С учётом полученных представлений и закона Фика:
\begin{equation}
    \label{eq:1_2:20}
    \Phi(x, t)=-3 \alpha \nabla \varphi(x, t),
\end{equation}
где $\alpha=\frac{1}{3\left(\kappa_{a}+\kappa_{s}^{\prime}\right)} =\frac{1}{3 \kappa-A \kappa_{s}}$,
и пренебрегая слагаемым $\frac{1}{c}\frac{\partial}{\partial t}I^*$ в~\eqref{eq:1_1:9}
можно получить~\cite{Kovtanyuk2014a}
\begin{equation}
    \label{eq:1_2:21}
    -\alpha \Delta \varphi(x, t)+\kappa_{a}\left(\varphi(x, t)-\theta^{4}(x, t)\right)=0.
\end{equation}

В~\cite{Kovtanyuk2014a} также выводятся краевые условия для $\varphi$:
\begin{equation}
    \label{eq:1_2:24}
    \alpha \frac{\partial \varphi(x, t)}{\partial n}+
    \gamma(x)\left(\varphi(x, t)-\theta_{b}^{4}(x, t)\right)=0.
\end{equation}
где $\gamma=\frac{\varepsilon}{2(2-\varepsilon)}$.
Отметим, что на участках втекания и вытекания среды
можно принять $\gamma=1/2$~\cite{JVM-14}.


Чтобы получить уравнение для температуры, подставим~\eqref{eq:1_2:14} в~\eqref{eq:1_1:10}.
Получим
\begin{equation}
    \label{eq:1_2:22}
    \frac{\partial \theta(x, t)}{\partial t} - a \Delta \theta(x, t)
    + b \kappa_{a}\left(\theta^{4}(x, t)-\varphi(x, t)\right)=0.
\end{equation}
Учитывая~\eqref{eq:1_2:21}, уравнение~\eqref{eq:1_2:22} можно записать в виде с кросс-диффузией:
\[
    \frac{\partial \theta(x, t)}{\partial t}-a \Delta \theta(x, t)
    + \nabla \theta(x, t) = b \alpha \Delta \varphi(x, t).
\]


Дополним полученные соотношения граничным условием для температуры~\eqref{eq:1_1:12}
\begin{equation}
    \label{eq:1_2:25}
    a \frac{\partial \theta(x, t)}{\partial n}
    +\beta(x)\left(\theta(x, t)-\theta_{b}(x, t)\right)=0,
\end{equation}
с начальными условиями
\begin{equation}
    \label{eq:1_2:26}
    \theta(x, 0)=\theta_{0}(x), \quad \varphi(x, 0)=\varphi_{0}(x).
\end{equation}
Соотношения~\eqref{eq:1_2:21}--\eqref{eq:1_2:26}
образуют диффузионную модель сложного теплообмена в рамках
$P_1$ приближения для уравнения переноса теплового излучения.

\FloatBarrier
