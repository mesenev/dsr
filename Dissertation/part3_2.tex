%! suppress = NonBreakingSpace


\section{Задачи оптимального управления с фазовыми ограничениями}
\label{sec:ch3:sec2}
%Chebotarev_2nd, chebotarev_park_mesenev_kovtanyuk.pdf
%Optimal control with phase constraints for a quasilinear endovenous laser ablation model
%
Рассмотрим задачу оптимального
управления для квазилинейных уравнений радиационно-кондуктивного
теплообмена, моделирующих процесс внутривенной
лазерной абляции в ограниченной области $\Omega$ с отражающей границей $\Gamma=\partial\Omega$.
Задача заключается в минимизации функционала
\[ J(\theta)=\int_{0}^{T} \int_{G_{1}}\left(\theta-\theta_{d}\right)^{2} dx dt \rightarrow \inf \]
на решениях начально-краевой задачи:
\begin{equation}
    \label{eq:3_2:1}
    \begin{gathered}
        \sigma \partial \theta / \partial t-\operatorname{div}(k(\theta)
        \nabla \theta)-\beta \varphi=u_{1} \chi \\
        -\operatorname{div}(\alpha \nabla \varphi)+\beta \varphi=u_{2}
        \chi, \quad x \in \Omega, \quad t \in(0, T),
    \end{gathered}
\end{equation}
\begin{equation}
    \label{eq:3_2:2}
    \theta=\left.0\right|_{\Gamma},
    \quad \alpha \partial_{n} \varphi
    +\left.2^{-1} \varphi\right|_{\Gamma}=0,
    \left.\quad \theta\right|_{t=0}=\theta_{0}.
\end{equation}
При этом учитываются ограничения:
\[ u_{1,2} \geq 0, \quad u_{1}+u_{2} \leq P, \left.\quad \theta\right|_{G_{2}} \leq \theta_{*} \]

Здесь $G_{1}$ и $G_{2}$ подмножества $\Omega, \theta$
представляют разницу между реальной температурой
и температурой на границе, которая является постоянной.
$\varphi$ является интенсивностью излучения усредненной по всем направлениям,
$\alpha$ -- коэффициент диффузии фотонов, $\beta$ -- коэффициент поглощения,
$k(\theta)$ является коэффициентом теплопроводности, $\sigma(x, t)$
является произведением удельной теплоемкости и плотностью среды, $u_{1}$
описывает мощность источника тепла, $u_{2}$ -- мощность источника теплового излучения.
$P$ -- максимальная мощность источника,
$\chi$ есть характеристическая функция той части среды,
в которой он расположен, деленная на объём.

\begin{remark}
    Задача оптимального управления заключается в достижении заданного распределения
    температурного поля $\theta_{d}$ в области $G_{1}$, при этом температура в области
    $G_{2}$ не должна превышать критических значений $\theta_{*}=\text{const}>0$.
\end{remark}

Рассмотренная постановка используется при
моделировании процедуры внутривенной лазерной абляции (ВВЛА),
которая учитывает кондуктивный теплообмен,
а также перенос излучения и поглощение с выделением тепла.
Поток пузырьков, образующихся на нагретом наконечнике оптического волокна,
вносит значительный вклад в температурное поле.
В~\cite{van2014optical, Some_Poluektova2014, Endovenous_Malskat2014},
основываясь на анализе экспериментальных данных,
теплопередача потоком пузырьков моделируется с использованием кусочно-постоянного
коэффициента теплопроводности, который зависит от температуры следующим образом:
когда температура в некоторой точке достигает $95 ^ {\circ} \mathrm{C}$,
коэффициент теплопроводности увеличивается в $200$ раз.

\subsection{Формализация задачи оптимального управления}
\label{subsec:ch3:sec2:subsec2}

Будем далее предполагать, что $\Omega$ является липшицевой ограниченной областью,
$\Gamma=\partial \Omega, Q=\Omega \times(0, T)$, $\Sigma=\Gamma \times(0, T)$.

Будем предполагать, что выполняются следующие условия:

$(c1)\; \sigma_{0} \leq \sigma \leq \sigma_{1},
\quad|\partial \sigma / \partial t| \leq \sigma_{2}$

$(c2)\; k_{0} \leq k(s) \leq k_{1}, \quad\left|k^{\prime}(s)\right| \leq k_{2},
\quad s \in \mathbb{R}$,

$(c3)\; \theta_{0} \in H$

$(c4)\; \alpha_{0} \leq \alpha(x) \leq \alpha_{1},
\beta_{0} \leq \beta(x) \leq \beta_{1}, \quad x \in \Omega$,

где $\sigma_{i}, k_{i}, \alpha_{i}$, и $\beta_{i}$ положительные константы.

Определим нелинейный оператор $A: V \rightarrow V^{\prime}$ и линейный оператор
$B: H^{1}(\Omega) \rightarrow\left(H^{1}(\Omega)\right)^{\prime}$
используя следующие равенства, справедливые для любого
$\theta, v \in V, \varphi, w \in$ $H^{1}(\Omega)$

\[
    \begin{aligned}
        &(A(\theta), v)=(k(\theta) \nabla \theta, \nabla v)=(\nabla h(\theta), \nabla v) \\
        &(B \varphi, w)=(\alpha \nabla \varphi, \nabla w)+(\beta \varphi, w)+2^{-1}
        \int_{\Gamma} \varphi w d \Gamma
    \end{aligned}
\]

где
\[
    h(s)=\int_{0}^{s} k(r) d r.
\]

\begin{definition}
    Пусть $u_{1,2} \in L^{2}(0, T)$.
    Пара функций $\theta \in L^{2}(0, T ; V), \varphi \in L^{2}\left(0, T ; H^{1}(\Omega)\right)$
    является слабым решением задачи~\eqref{eq:3_2:1},~\eqref{eq:3_2:2}
    если $\sigma \theta^{\prime} \in$ $L^{2}\left(0, T ; V^{\prime}\right)$ и

    $\sigma \theta^{\prime}+A(\theta)-\beta \varphi=u_{1} \chi, \quad \theta(0)=\theta_{0},
    \quad B \varphi=u_{2} \chi$ где $\theta^{\prime}=d \theta / d t$.
\end{definition}

Из леммы Лакса-Мильграма следует, что для любой функции $g \in H$
существует единственное решение уравнения $B \varphi=g$.
Более того, обратный оператор $B^{-1}: H \rightarrow H^{1}(\Omega)$ непрерывен.
Поэтому можно исключить интенсивность излучения $\varphi=u_{2} B^{-1} \chi$ и
сформулировать задачу оптимального управления следующим образом.

\begin{definition}
[Задача P]
    \label{subsec:ch3:sec2:subsec3:P}
    \[
        J(\theta)=\int_{0}^{T}
        \int_{G_{1}}\left(\theta-\theta_{d}\right)^{2} d x d t \rightarrow \inf
    \]

    \[
        \begin{aligned}
            & \sigma \theta^{\prime}+A(\theta)=u,
            \quad \theta(0)=\theta_{0},\left.\quad
            \theta\right|_{G_{2}} \leq \theta_{*},
            \quad u \in U_{a d},
        \end{aligned}
    \]

    где

    \[
        \begin{array}{r}
            U_{a d}=\left\{u=u_{1} \chi+u_{2} \beta B^{-1}
            \chi: u_{1,2} \in L^{2}(0, T),\right. \\
            \left.u_{1,2} \geq 0, u_{1}+u_{2} \leq P\right\}
        \end{array}
    \]
\end{definition}

\subsection{Предварительные результаты}
\label{subsec:ch3:sec2:subsec4}
Рассмотрим задачу
\begin{equation}
    \label{eq:3_2:3}
    \sigma \theta^{\prime}+A(\theta)=f, \quad \theta(0)=\theta_{0}.
\end{equation}
Справедлива следующая лемма

\begin{lemma}[\cite{Inverse_Kovtanyuk2021}]
    \label{lm:3_2:1}
    Пусть выполняются условия $(c1)$-$(c3)$ и
    $f \in$ $L^{2}\left(0, T ; V^{\prime}\right)$.
    Тогда существует решение задачи~\eqref{eq:3_2:3} такое, что
    $\theta \in L^{\infty}(0, T ; H)$ и справедливы следующие оценки:
    \[
        \begin{gathered}
            \|\theta(t)\|^{2} \leq \frac{K}{\sigma_{0}} \exp \frac{\sigma_{2} t}{\sigma_{0}}
            \quad \text { п.\ в. в }(0, T) \\
            \int_{0}^{T}\|\theta(t)\|_{V}^{2} d t \leq
            \frac{K}{k_{0}}\left(1+\frac{\sigma_{2} T}{\sigma_{0}} \exp
            \frac{\sigma_{2} T}{\sigma_{0}}\right)
        \end{gathered}
    \]

    где $K=\sigma_{1}\left\|\theta_{0}\right\|^{2}
    + k_{0}^{-1}\|f\|_{L^{2}\left(0, T; V^{\prime}\right)}^{2}$.
\end{lemma}


Следующий результат важен для установления непустоты
множества допустимых пар управляющих состояний.

\begin{lemma}
    \label{lm:3_2:2}
    Пусть выполняются условия $(c1)$-$(c3)$ и
    $f=0$, $\theta_{0} \leq \theta_{*}$ п.\ в. в $\Omega$,
    и $\theta$ решение задачи~\eqref{eq:3_2:3}.
    Тогда $\theta \leq \theta_{*}$ п.\ в. в $\Omega \times(0, T)$.
\end{lemma}

\begin{proof}
    Скалярно умножим в $H$ первое уравнение~\eqref{eq:3_2:3} на
    $v=\max \{\theta-$ $\left.\theta_{*}, 0\right\} \in L^{2}(0, T ; V)$, получим

    \[ \left(\sigma v^{\prime}, v\right)+(k(\theta) \nabla v, \nabla v)=0. \]

    Отбрасывая неотрицательный второй член, приходим к оценке
    \[ \frac{d}{d t}(\sigma v, v) \leq\left(\sigma_{t} v, v\right) \leq \sigma_{2}\|v\|^{2}. \]
    Учитывая, что $\left.v\right|_{t=0}=0$, проинтегрируем
    последнее неравенство по времени.
    Тогда
    \[
        \sigma_{0}\|v(t)\|^{2} \leq(\sigma v(t), v(t))
        \leq \sigma_{2} \int_{0}^{t}\|v(\tau)\|^{2} d \tau
    \]

    На основании леммы Гронуолла заключаем,
    что $v=0$ и, следовательно, $\theta \leq \theta_{*}$ почти всюду в $\Omega\times(0,T)$.
\end{proof}

\subsection{Разрешимость задачи оптимального управления}
\label{subsec:ch3:sec2:subsec5}

\begin{theorem}
    \label{th:3_2:1}
    Пусть выполняются условия $(c1)$-$(c3)$, и $\theta_{0} \leq \theta_{*}$ п.\ в. в $\Omega$.
    Тогда существует решение задачи P\@.
\end{theorem}

\begin{proof}
    Согласно леммам~\ref{lm:3_2:1} и~\ref{lm:3_2:2} множество допустимых пар непусто.
    Рассмотрим минимизирующую последовательность допустимых
    пар $\left\{\theta_{m}, u_{m}\right\} \in$ $L^{2}(0, T ; V) \times U_{a d}$
    такой, что $J\left(\theta_{m}\right) \rightarrow j=\inf J$, где
    \begin{equation}
        \label{eq:3_2:4}
        \sigma \theta_{m}^{\prime}+A\left(\theta_{m}\right)=u_{m},
        \quad \theta_{m}(0)=\theta_{0},\left.\quad \theta_{m}\right|_{G_{2}} \leq \theta_{*}.
    \end{equation}

    Ограниченность в $L^{2}(0, T; H)$ множества допустимых
    управлений $U_{a d}$ влечет по лемме~\ref{lm:3_2:1} оценки:

    \begin{equation}
        \label{eq:3_2:5}
        \begin{gathered}
            \left\|\theta_{m}\right\|_{L^{\infty}(0, T ; H)} \leq C,
            \quad\left\|\theta_{m}\right\|_{L^{2}(0, T ; V)} \leq C, \\
            \left\|h\left(\theta_{m}\right)\right\|_{L^{2}(0, T ; V)} \leq C.
        \end{gathered}
    \end{equation}

    Здесь и далее при доказательстве теоремы через $C$
    обозначаются константы, не зависящие от $m$.
    Оценки~\eqref{eq:3_2:5}, используя при необходимости подпоследовательности,
    приводят к существованию функций
    $u \in U_{a d}, \quad \theta \in L^{2}(0, T; V)$, $\chi  \in L^{2}(0, T; V)$
    такое, что
    \begin{equation}
        \label{eq:3_2:6}
        \begin{aligned}
            & u_{m} \rightarrow u \text { слабо в } L^{2}(0, T ; H), \\
            & \theta_{m} \rightarrow \theta \text { слабо в }
            L^{2}(0, T ; V) \text {, } \\
            & \text { *-слабо в } L^{\infty}(0, T ; H), \\
            & h\left(\theta_{m}\right) \rightarrow \chi
            \text { слабо в } L^{2}(0, T; V).
        \end{aligned}
    \end{equation}

    Результаты о сходимости~\eqref{eq:3_2:6} достаточны для предельного перехода
    при $m \rightarrow \infty$ в~\eqref{eq:3_2:4} и доказательства того,
    что предельная функция $\theta \in L^{2}(0, T ; V) $ такова,
    что $\sigma \theta^{\prime} \in L^{2}\left(0, T ; V^{\prime}\right)$
    удовлетворяет равенству
    \[ \left(\sigma \theta^{\prime}, v\right)+(\nabla \chi, \nabla v)=(u, v) \quad \forall v \in V \]
    и выполняется начальное условие.

    Следующая оценка гарантирует компактность последовательности $\theta_{m}$ в $L^{2}(Q)$:
    \begin{equation}
        \label{eq:3_2:7}
        \int_{0}^{T-\delta}\left\|\theta_{m}(s+\delta)-\theta_{m}(s)\right\|^{2} d s \leq C \delta.
    \end{equation}

    Из неравенства~\eqref{eq:3_2:7}, используя при необходимости подпоследовательности,
    получаем, что $\theta_{m} \rightarrow \theta$ in $L^{2}(Q)$.
    Следовательно, в силу неравенства

    \[
        \left|h\left(\theta_{m}\right)-h(\theta)\right|
        \leq k_{1}\left|\theta_{m}-\theta\right|,
    \]
    следует, что $h\left(\theta_{m}\right) \rightarrow h(\theta)$ in $L^{2}(Q)$
    и, следовательно $\chi=h(\theta)$.
    Кроме того, предельная функция $\theta$ удовлетворяет неравенству
    $\left.\theta\right|_{G_{2}} \leq \theta_{*}.$
    Следовательно, допустима предельная пара
    $\{\theta, u\} \in L^{2}(0, T ; V) \times U_{a d}.$


    Поскольку функционал $J$ слабо полунепрерывен снизу,
    \[ j \leq J(\theta) \leq \liminf J\left(\theta_{m}\right)=j, \]
    пара $\{\theta, u\}$ является решением задачи P\@.
\end{proof}

\subsection{Задача со штрафом}
\label{subsec:ch3/sec2/penalty}
Для численного решения задачи оптимального управления с фазовыми ограничениями
$\left.\theta\right|_{G_{2}} \leq \theta_{*}$, рассмотрим следующую задачу со штрафом.

Задача $P_{\varepsilon}$.
$J_{\varepsilon}(\theta) \rightarrow \inf$, где
\[
    \begin{aligned}
        & J_{\varepsilon}(\theta)=\int_{0}^{T}
        \int_{G_{1}}\left(\theta-\theta_{d}\right)^{2} d x d t \\
        & +\frac{1}{\varepsilon} \int_{0}^{T}
        \int_{G_{2}} F(\theta) d x d t, \\
        & \sigma \theta^{\prime}+A(\theta)=u,
        \quad \theta(0)=\theta_{0}, \quad u \in U_{a d}.
    \end{aligned}
\]
Здесь,
\[
    F(\theta)=
    \begin{cases}
        0, & \text { если } \theta \leq \theta_{*} \\
        \left(\theta-\theta_{*}\right)^{2},
        & \text { если } \theta>\theta_{*}
    \end{cases}
\]

Оценки, представленные в лемме~\ref{lm:3_2:1}, также позволяют доказать разрешимость
задачи со штрафом аналогично доказательству теоремы~\ref{th:3_2:1}.

\begin{theorem}
    \label{th:3_2:2}
    Пусть выполняются условия $(c1)$-$(c3)$.
    Тогда существует решение задачи $\left(P_{\varepsilon}\right)$.
\end{theorem}

Рассмотрим аппроксимационные свойства решений задачи со штрафом.
Пусть $\left\{\theta_{\varepsilon}, u_{\varepsilon}\right\}$ — решения задачи
$\left(P_{\varepsilon}\right)$ и $\{\theta, u\}$ — решение задачи P\@.
Тогда,

\begin{equation}
    \label{eq:3_2:8}
    \sigma \theta_{\varepsilon}^{\prime}
    +A\left(\theta_{\varepsilon}\right)=u_{\varepsilon},
    \quad \theta_{\varepsilon}(0)=\theta_{0}.
\end{equation}

Так как $\left.\theta\right|_{G_{2}} \leq \theta_{*}$, выполняется следующее неравенство:

\[
    \int_{0}^{T} \int_{G_{1}}\left(\theta_{\varepsilon}
    -\theta_{d}\right)^{2} d x d t+\frac{1}{\varepsilon}
    \int_{0}^{T} \int_{G_{2}}
    F\left(\theta_{\varepsilon}\right) d x d t
    \leq \int_{0}^{T} \int_{G_{1}}
    \left(\theta-\theta_{d}\right)^{2} d x d t=J(\theta).
\]
Следовательно,
\[
    \begin{aligned}
        &\int_{0}^{T} \int_{G_{1}}\left(\theta_{\varepsilon}
        -\theta_{d}\right)^{2} d x d t \leq J(\theta), \\
        &\int_{0}^{T} \int_{G_{2}} F
        \left(\theta_{\varepsilon}\right) d x d t \leq \varepsilon J(\theta).
    \end{aligned}
\]
Из полученных оценок, используя при необходимости подпоследовательности,
соответствующие $\varepsilon_{k} \rightarrow+0$,
аналогично доказательству теоремы~\ref{th:3_2:1}, устанавливаем существование
функций $\widehat{u} \in U_{a d}, \widehat{\theta} \in L^{2}(0, T ; V)$, таких, что
\[
    \begin{aligned}
        u_{\varepsilon}  \rightarrow \widehat{u} & \text { слабо в } L^{2}(0, T ; H); \\
        \theta_{\varepsilon}  \rightarrow \widehat{\theta} & \text { слабо в } L^{2}(0, T ; V), \\
        & \text { сильно в } L^{2}(0, T; H).
    \end{aligned}
\]
Заметим, что
\[
    \begin{aligned}
        &\int_{0}^{T} \int_{G_{2}} F\left(\theta_{\varepsilon}\right) d x d t
        \rightarrow \int_{0}^{T} \int_{G_{2}} F(\widehat{\theta}) d x d t, \\
        &\int_{0}^{T} \int_{G_{2}} F\left(\theta_{\varepsilon}\right) d x d t
        \rightarrow 0, \text { при } \varepsilon \rightarrow+0,
    \end{aligned}
\]
что гарантирует $F(\widehat{\theta})=0$ и
$\left.\widehat{\theta}\right|_{G_{2}} \leq \theta_{*}$.

Результаты сходимости достаточны для предельного
перехода по $\varepsilon \rightarrow+0$ в системе~\eqref{eq:3_2:8}
и доказательства того, что предельная пара
$\{\widehat{\theta}, \widehat{u}\} \in L ^{2}(0, T ; V) \times U_{a d}$
допустима для задачи P\@.
Поскольку функционал $J$ слабо полунепрерывен снизу,

\[
    j \leq J(\widehat{\theta})
    \leq \liminf J \left(\theta_{\varepsilon}\right)
    \leq J(\theta)=j=\inf J.
\]
Тогда пара $\{\widehat{\theta}, \widehat{u}\}$ есть решение задачи P\@.

\begin{theorem}
    \label{th:3_2:3}
    Пусть выполнены условия $(c1)$-$(c3)$, и $\theta_{0} \leq \theta_{*}$ п.\ в. в $\Omega$.
    Если $\left\{\theta_{\varepsilon}, u_{\varepsilon}\right\}$ есть решение задачи
    $\left(P_{\varepsilon}\right)$ для $\varepsilon>0$, тогда существует
    такая последовательность $\varepsilon \rightarrow+0$, что
    \[
        \begin{aligned}
            &u_{\varepsilon} \rightarrow \widehat{u} \text { слабо в } L^{2}(0, T ; H) \\
            &\theta_{\varepsilon} \rightarrow \widehat{\theta} \text { сильно в } L^{2}(0, T ; H),
        \end{aligned}
    \]
    где $\{\widehat{\theta}, \widehat{u}\}$ есть решение задачи $P$\@.
\end{theorem}

\begin{remark}
    Вопрос единственности решения задачи $P$ является открытым, что обусловлено невыпуклостью этой задачи.
    Тем не менее, представленные результаты гарантируют сходимость решений задачи со штрафом
    к решению задачи $P$.
\end{remark}
