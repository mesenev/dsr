\section{Квазирешение граничной обратной задачи}\label{sec:ch2/sec1}
% dvmg668

\subsection{Постановка обратной задачи}\label{subsec:ch2/sec1/subsec1}

Нормализованная стационарная модель, описывающая процесс радиационного теплопереноса в
области $\Omega \subset \mathbb{R}^3$ с липшицевой границей $\Gamma$
согласно~\eqref{eq:1_4:4-1},~\eqref{eq:1_4:4-2}, имеет следующий вид
\begin{equation}
    \label{eq:2_1:initial}
    \begin{aligned}
        - a \Delta \theta + b \kappa_a(\theta ^ 3 | \theta | - \varphi) = 0,  \\
        - \alpha \Delta \varphi + \kappa_a (\varphi - \theta ^3 | \theta |) = 0.
    \end{aligned}
\end{equation}

Здесь $\theta$ -- нормализованная температура, $\varphi$ -- нормализованная интенсивность излучения,
усреднённая по всем направлениям, $\kappa_a$ -- коэффициент поглощения.
Константы $a, b, \alpha, \gamma, \beta$ описываются следующим образом:
\[
    a = \frac{k}{\rho c_v}, \; b = \frac{4 \sigma n^2 T^3_{\max}}{\rho c_v}, \;
    \alpha = \frac{1}{3\kappa -A \kappa_s},
\]
где $k$ -- теплопроводность, $c_v$ -- удельная теплоёмкость, $\rho$ -- плотность,
$\sigma$ -- постоянная Стефана\,--\,Больцмана, $n$ -- индекс рефракции,
$T_{\max}$ -- максимальная температура,
$\kappa \coloneqq \kappa_s + \kappa_a$ -- коэффициент полного взаимодействия,
$\kappa_s$ -- коэффициент рассеяния.
Коэффициент $A \in [-1,1]$ описывает анизотропию рассеивания;
случай $A=0$ отвечает изотропному рассеиванию.

Уравнения~\eqref{eq:2_1:initial} дополняются граничными условиями на
$\Gamma \coloneqq \partial \Omega =\overline{\Gamma}_0 \cup \overline{\Gamma}_1 \cup \overline{\Gamma}_2$,
где части границы $\Gamma_0, \Gamma_1, \Gamma_2$ не имеют пересечений:
\begin{equation}
    \label{eq:2_1:initial-boundary}
    \begin{aligned}
        \Gamma &: \; a \partial_n \theta + \beta (\theta - \theta _b) = 0, \\
        \Gamma_0 \cup \Gamma_2 &: \; \alpha \partial_n \varphi
        + \gamma(\varphi - \theta_b ^4 ) = 0, \\
        \Gamma_1 &: \; \alpha \partial_n \varphi + u(\varphi - \theta_b ^4 ) = 0. \\
    \end{aligned}
\end{equation}
Функции $\gamma, \theta_b, \beta$ -- являются известными.
Функция $u$ характеризует отражающие свойства участка границы $\Gamma_1$.
Предполагается, что
\begin{equation}
    \label{eq:2_1:control_bounds}
    0 < u_1 \leq u \leq u_2,
\end{equation}
где $u_1$ и $u_2$ -- заданные ограниченные функции.


Обратная задача состоит в нахождении функций
\[
    u(x), x \in  \Gamma_1, \; \theta(x), \varphi(x), x \in \Omega,
\]
удовлетворяющих условиям~\eqref{eq:2_1:initial}--\eqref{eq:2_1:control_bounds},
а также дополнительному условию на участке границы $\Gamma_2$
\begin{equation}
    \label{eq:2_1:theta_gamma}
    \theta|_{\Gamma_2}=\theta_0,
\end{equation}
где $\theta_0$ известная функция.


Сформулированная обратная задача~\eqref{eq:2_1:initial}--\eqref{eq:2_1:theta_gamma}
сводится к экстремальной задаче,
состоящей в минимизации функционала
\begin{equation}
    \label{eq:2_1:quality}
    J(\theta) = \frac{1}{2} \int_{\Gamma_2} (\theta - \theta_0)^2 d\Gamma,
\end{equation}
на решениях краевой задачи~\eqref{eq:2_1:initial}--\eqref{eq:2_1:control_bounds}.

\subsection{Формализация задачи нахождения квазирешения}\label{subsec:ch2/sec1/subsec2}


Будем предполагать что исходные данные удовлетворяют условию

$\text{(i)}\;    \beta\in L^\infty(\Gamma); \gamma \in L^\infty(\Gamma_0\cup\Gamma_2);$
$u_1, u_2 \in L^\infty(\Gamma_1);$
$0 < \beta_0 \le \beta; 0 < \gamma_0 \le \gamma;\; \beta_0,\gamma_0=Const,$
$0 \le u_1 \le u_2$.

Пусть $H = L^2(\Omega), V = H^1(\Omega), Y = V \times V $.
Пространство $H$ отождествляем с сопряжённым пространством $H'$ так,
что $V \subset H = H' \subset V'$.
Определим $(f,v)$ как значение функционала $f \in V'$ на элементе $v \in V$,
совпадающее со скалярным произведением в $H$, если $f\in H, \|f\|^2 = (f,f)$.

Пространство $U = L^2(\Gamma_1)$ является пространством управлений;
$U_{ad} = \{u \in U, u_1 \le u \le u_2 \}$ --- множество допустимых управлений.

Пусть $v$ произвольный элемент множества $V$.
Определим операторы:
\begin{gather*}
    A_{1,2}\colon V \to V', \;\; F \colon V \times U \to V', \; f \in V', \; g \in V'.\\
    (A_1\theta,v) = a( \nabla \theta, \nabla v ) + \int_\Gamma \beta \theta v d\Gamma, \;
    (A_2 \varphi, v) = \alpha (\nabla \varphi,\nabla v) + \int_{\Gamma_0 \cup \Gamma_2}
    \gamma \varphi v d\Gamma,\\
    (f,v) = \int_\Gamma \beta \theta_b v d\Gamma, \; \;
    (g,v) = \int_{\Gamma_0 \cup \Gamma_2} \gamma \theta_b^4 v d\Gamma,\\
    (F(\varphi, u), v) = \int_{\Gamma_1} u (\varphi - \theta^4_b)v d\Gamma.\\
\end{gather*}

Пару $\{\theta, \varphi \} \in Y$ будем называть слабым решением
задачи~\eqref{eq:2_1:initial},~\eqref{eq:2_1:initial-boundary}, если
\begin{equation}
    \label{eq:2_1:weakOperational}
    A_1 \theta + b \kappa_a (| \theta | \theta^3 - \varphi ) =
    f, A_2 \varphi + \kappa_a (\varphi - |\theta|\theta^3) + F(\varphi, u) = g.
\end{equation}

Задача нахождения квазирешения состоит в минимизации функционала $J(\theta)$,
определённом на компоненте $\theta$ решения системы~\eqref{eq:2_1:weakOperational}.
Таким образом, получаем следующую задачу на экстремум:
\begin{equation}
    \label{eq:2_1:minimizationOperational}
    J(\theta) \to \inf, \; \{\theta, \varphi\}
    \text{ решение~\eqref{eq:2_1:weakOperational}, соответствующее функции } u \in U_{ad}.
\end{equation}

Пара $\{\hat{\theta}, \hat{\varphi} \}$ и $u \in U_{ad}$ соответствующие минимуму функционала $J$,
называются оптимальным состоянием и квазирешением обратной
задачи~\eqref{eq:2_1:initial}--\eqref{eq:2_1:theta_gamma}.

\subsection{Анализ экстремальной задачи}\label{subsec:ch2/sec1/subsec3}

Для доказательства разрешимости задачи~\eqref{eq:2_1:minimizationOperational}
установим некоторые свойства решения
задачи~\eqref{eq:2_1:initial},~\eqref{eq:2_1:initial-boundary}.

\begin{lemma}[\cite{Kovtanyuk2015}]
    Пусть выполняется условие (i).
    Тогда для каждого $ u \in U_{ad} $ существует единственное слабое решение
    $\{\theta, \varphi \}$ для задачи~\eqref{eq:2_1:initial},~\eqref{eq:2_1:initial-boundary}
    и справедливы оценки:
    \begin{equation}
        \label{eq:2_1:lemma_1}
        M_1 \le \theta \le M_2, \; M_1^4 \le \varphi \le M_2^4,
    \end{equation}
    \begin{equation}
        \label{eq:2_1:lemma_2}
        \| \nabla \varphi \|^2 \le C.
    \end{equation}
    Здесь $M_1 = \text{ess inf } \theta_b, M_2 = \text{ess sup } \theta_b$,
    и константа $C > 0$ зависит только от
    $a, b, \alpha, \kappa_a, \beta, \gamma, \|u\|_{L^\infty(\Gamma)}$ и области $\Omega$.
\end{lemma}

На основе оценок~\eqref{eq:2_1:lemma_1} и~\eqref{eq:2_1:lemma_2}
аналогично~\cite{Kovtanyuk2014TheoreticalAnalysis}
доказывается разрешимость экстремальной
задачи~\eqref{eq:2_1:minimizationOperational}.


\begin{theorem}
    \label{th:2_1:1}
    Пусть выполняется условие (i).
    Тогда существует хотя бы одно решение задачи~\eqref{eq:2_1:minimizationOperational}.
\end{theorem}

Для вывода системы оптимальности, покажем дифференцируемость функционала $J$.
\begin{lemma}
    \label{lm:2_1:freshet_diff}
    Функционал $J : V \rightarrow \mathbb{R}$ дифференцируем по Фреше.
\end{lemma}

\begin{proof}
    Покажем, что для произвольной функции $\theta \in V$ выполняется следующее равенство:
    \begin{equation}
        \label{eq:2_1:lemma_proof_1}
        J(\theta + h) = J(\theta) + J'(\theta)\langle h \rangle
        + r(\theta, h) \; \forall h \in V, \; \text{ где } \;
        J'(\theta)\langle h \rangle = \int_{\Gamma_2} (\theta - \theta_0)h d\Gamma
    \end{equation}
    и для остаточного члена $r(\theta,h)$ справедливо соотношение
    \begin{equation}
        \label{eq:2_1:lemma_proof_2}
        \frac{|r(\theta,h)|}{\|h\|_V} \rightarrow 0
        \quad \text{при} \quad \|h\|_V \rightarrow 0.
    \end{equation}
    Перепишем~\eqref{eq:2_1:lemma_proof_1} в виде
    \[
        \frac{1}{2} \|\theta + h - \theta_0\|^2_{L^2(\Gamma_2)} =
        \frac{1}{2} \| \theta - \theta_0 \|^2_{L^2(\Gamma_2)} +
        (\theta - \theta_0, h)_{L^2(\Gamma_2)} +
        \frac{1}{2}\| h \|^2_{L^2(\Gamma_2)}.
    \]
    Согласно теореме о следах $ \|h\|_{L^2(\Gamma_2)} \le C \|h\|_V $,
    где $C$ не зависит от $h$.
    Поэтому
    \[
        \frac{r(\theta,h)}{\| h \|_V} \leq
        \frac{1}{2} C^2 \| h \|_V \rightarrow 0 \quad \text{при } \| h \|_V \rightarrow 0.
    \]
    Лемма доказана.
\end{proof}

Вывод условий оптимальности основан на принципе множителей
Лагранжа для гладко-выпуклых задач минимизации~\cite{10}[Теорема 1.5].
\begin{theorem}
    \label{th:2_1:2}
    Пусть $\hat{y}=\{\hat{\theta},\hat{\varphi} \} \in Y, \hat{u} \in U_{ad}$
    --- решение экстремальной задачи~\eqref{eq:2_1:minimizationOperational}.
    Тогда существует пара $p = (p_1, p_2)$, $p \in Y$
    такая, что тройка $(\hat{y}, \hat{u}, p)$, удовлетворяет следующим условиям:
    \begin{equation}
        \label{eq:2_1:theorem_2_eq1}
        A_1 p_1 + 4 |\hat{\theta}|^3 \kappa_a(b p_1 - p_2) = f_c,
        \;\; (f_c,v) = - \int_{\Gamma_2} (\hat{\theta} - \theta_0) v d\Gamma,
    \end{equation}
    \begin{equation}
        \label{eq:2_1:theorem_2_eq2}
        A_2 p_2 + \kappa_a (p_2-b p_1) = g_c(p_2, \hat{u}),
        \;(g_c(p_2, \hat{u}), v) = -\int_{\Gamma_1} \hat{u} p_2 v d\Gamma,
    \end{equation}
    \begin{equation}
        \label{eq:2_1:theorem_2_eq3}
        \int_{\Gamma_1} p_2 (\hat{\varphi} - \theta_b^4)(u-w) d\Gamma
        \leq 0 \quad \forall w \in U_{ad}.
    \end{equation}
\end{theorem}

\begin{proof}
    Перепишем уравнения~\eqref{eq:2_1:weakOperational} следующим образом:
    \[ H(y,u) = 0,\;\; y = \{\theta,\varphi\} \in Y, \]
    где
    \begin{gather*}
        H:Y \times U \to Y',\\
        H(y,u) =\{A_1 \theta + b \kappa_a (| \theta | \theta^3 - \varphi ) - f,
        A_2 \varphi + \kappa_a (\varphi - |\theta|\theta^3) + F(\varphi, u) - g \}.\\
    \end{gather*}
    Заметим, что для всех $u \in U_{ad}$, отображение
    $y \to J(\theta) $ и $y \to H(y,u)$ непрерывно
    дифференцируемо в окрестности $\mathcal{O}(\hat{y})$ точки $\hat{y}$.
    Непрерывная дифференцируемость членов в $H$ следует из непрерывной дифференцируемости
    функции $t \in \mathbb{R} \to | t | t^3$,
    а также из непрерывности вложения  $V \subset L^6(\Omega)$.
    В дополнение, отображение $u \to H(y,u)$ непрерывно из $U \to Y'$ и афинно.
    В~\cite{Kovtanyuk2014TheoreticalAnalysis} показано,
    что $\operatorname{Im}H_y'(\hat{y}, \hat{u}) = Y$,
    что влечёт невырожденность условий оптимальности.

    Рассмотрим функцию Лагранжа
    $L(y,u,p) = J(\theta) + (H(y,u),p),$ где $y,p \in Y,\, u \in U_{ad}$.
    Согласно принципу Лагранжа~\cite[Гл.2, теорема 1.5]{11}
    существует пара $p = \{p_1,p_2\} \in Y$ такая, что
    \begin{equation}
        \label{eq:2_1:th2_proof_1}
        (L'_\theta,\zeta) =\int_{\Gamma_2}(\hat\theta -\theta_0) \zeta d\Gamma
        + (A_1 \zeta + 4b\kappa_a |\hat\theta|^3 \zeta,p_1)
        - 4\kappa_a(|\hat\theta|^3 \zeta,p_2) = 0 \; \forall \zeta \in V,
    \end{equation}
    \begin{equation}
        \label{eq:2_1:th2_proof_2}
        (L'_\varphi, \zeta) = (A_2 \zeta + \kappa_a \zeta, p_2)
        - b \kappa_a(\zeta,p_1)
        + \int_{\Gamma_1} \hat u \zeta p_2 d\Gamma = 0 \; \forall \zeta \in V,
    \end{equation}
    \begin{equation}
        \label{eq:2_1:th2_proof_3}
        (L'_u,\tau) = \int_{\Gamma_1} \tau (\varphi - \theta^4_b) p_2 d\Gamma  \leq 0,
        \; \tau \coloneqq \hat u - w \; \forall w \in U_{ad}.
    \end{equation}
    Сопряжённые уравнения~\eqref{eq:2_1:theorem_2_eq1},~\eqref{eq:2_1:theorem_2_eq2}
    являются прямым следствием
    вариационных равенств~\eqref{eq:2_1:th2_proof_1} и~\eqref{eq:2_1:th2_proof_2}.
\end{proof}

Численный алгоритм, основанный на полученной системе
оптимальности представлен в разделе~\ref{sec:ch4/sec3}.
\FloatBarrier
